\section{Spezielle Relativitätstheorie}

Bei der Beschreibung der Elektrodynamik mittels den Maxwell-Gleichungen ergibt
sich ein Problem mit der Klassischen Mechanik, da die Maxwell-Gleichungen nicht
forminvariant unter Galilei-Transformationen sind. Aus der Sicht der
Klassischen Mechanik kann daher die Maxwell Theorie nicht korrekt sein.

Auf der anderen Seite wissen wir aus dem Experiment, dass sich
elektromagnetische Wellen mit Lichtgeschwindigkeit ausbreiten,
\begin{align*}
c = 299792458 \mathrm{m/s}.
\end{align*}
Bei Abstrahlung in Ruhe breiten sie sich kugelförmig aus. Bewegt sich die
Quelle, ist nach der Klassischen Mechanik die Ausbreitung nicht mehr
kugelförmig, es kommt zum Dopplereffekt.
\begin{figure}[!htbp]
\centering
\begin{pspicture}(-0.1,-1.2)(2.3,1.2)
\psline{->}(0.74,-0.49046874)(0.74,0.9895313)
\psline{->}(0.62,-0.37046874)(2.08,-0.37046874)
\pscircle[linecolor=yellow](0.74,-0.39046875){0.24}
\pscircle[linecolor=yellow](0.74,-0.39046875){0.5}
\pscircle[linecolor=yellow](0.74,-0.39046875){0.74}

\rput(0.5878125,1.0795312){\color{gdarkgray}$y$}
\rput(2.0854688,-0.54046875){\color{gdarkgray}$x$}
\end{pspicture}
\begin{pspicture}(-0.1,-1.24)(4.1,1.24)
\psline{->}(0.2378125,-1.0)(0.2378125,0.94)
\psline{->}(0.0778125,-0.88)(3.7178125,-0.88)

\psdots[linecolor=darkblue](2.0378125,0.0)
\psbezier[linecolor=yellow](1.5178125,0.0)(1.5178125,-0.88)(3.0378125,-0.66)(3.0378125,0.0)(3.0378125,0.66)(1.5178125,0.88)(1.5178125,0.0)
\psbezier[linecolor=yellow](1.6978126,0.0)(1.6978126,-0.58)(2.7178125,-0.46)(2.7378125,0.0)(2.7578125,0.46)(1.6978126,0.58)(1.6978126,0.0)
\psbezier[linecolor=yellow](1.2978125,0.0)(1.3178124,-1.22)(3.3378124,-0.78)(3.3378124,0.0)(3.3378124,0.78)(1.2778125,1.22)(1.2978125,0.0)
\psline[linecolor=darkblue]{->}(1.1778125,-0.04)(0.3578125,-0.04)
\psline[linecolor=darkblue]{->}(3.4378126,-0.04)(3.9178126,-0.04)

\rput(0.085625,0.95){\color{gdarkgray}$y$}
\rput(3.6032813,-1.05){\color{gdarkgray}$x$}
\rput(0.7603125,0.23){\color{gdarkgray}$v+c$}
\rput(3.7,0.23){\color{gdarkgray}$v-c$}
\end{pspicture} 
\caption{Wellenausbreitung im ruhenden und bewegten Bezugssystem.}
\end{figure}

Da elektromagnetische Wellen eine endliche Ausbreitungsgeschwindigkeit haben,
müsste so die Lichtgeschwindigkeit vom Bezugssystem abhängen. Insbesondere
würde ein absolutes Ruhesystem existieren, in dem Licht sich mit $c_0$
ausbreitet. Es wurden zahlreiche Experimente durchgeführt, um die
Relativbewegung des Laborsystems zu diesem absoluten Bezugssystem
nachzuweisen, darunter auch das berühmte Michelson-Morley Experiment, welches
schließlich das experimentelle Ergebnis lieferte, dass kein absolutes
Bezugssystem existiert und die Lichtgeschwindigkeit vom Bezugssystem unabhängig
ist. Die Klassische Mechanik bricht also bei Lichtgeschwindigkeit zusammen.
 
Einstein löste diesen Widerspruch, indem er davon ausging, dass sowohl die
Klassische Mechanik als auch die Maxwell Theorie korrekt sind, jedoch eine
experiementell nicht verifizierbare Annahme, die absolute Gleichzeitigkeit
aufgab. In der Klassischen Mechanik gingen wir bisher von einer absoluten
Zeitskala aus, die in allen relativ zueinander bewegten Koordinatensystem verwendet
werden kann. Die Aussage, zwei Ereignisse finden gleichzeitig statt, hat so
absolute Bedeutung. Einstein forderte nun, dass jedem Bezugssystem eine
eigene von anderen Bezugssystem unabhängige Zeitskala zuzuordnen ist.

\begin{bsp}
Geht man von absoluter Gleichzeitigkeit aus, nehmen ruhender und bewegter
Beobachter dieselben Ereignisse als ``gleichzeitig'' wahr (Siehe
Abb. \ref{figure:5:AbsGleichzeitigkeit}).

Dies führt jedoch dazu, dass ruhender und bewegter Beobachter eine
unterschiedliche Lichtgeschwindigkeit wahrnehmen ($v_1\neq v_2$), was
im Wiederspruch zur Konstanz der Lichtgeschwindigkeit steht.

\begin{figure}[H]
\centering
\begin{pspicture}(0.2,-1.4)(5.6,2.035)
%\psgrid
\psline{->}(0.3,-0.495)(4.48,-0.475)
\psline{->}(2.14,-0.575)(2.14,1.305)
\psline[linecolor=yellow](0.32,1.025)(2.12,-0.475)
\psline[linecolor=yellow](3.94,1.005)(2.12,-0.475)
\psline[linecolor=yellow]{<-}(3.44,0.585)(2.12,-0.475)

\psbezier(2.2,-0.595)(2.34,-1.095)(2.94,-0.715)(3.08,-1.095)

\psdots[linecolor=darkblue,dotsize=0.2](2.14,-0.495)
\psline[linecolor=darkblue]{->}(1.16,-1.995)(3.4,1.425)
\psbezier(3.48,1.485)(3.66,1.805)(4.04,1.325)(4.44,1.665)

\psline[dotsep=0.06cm]{<->}(0.84,0.665)(2.82,0.665)
\psline[dotsep=0.06cm]{<->}(2.96,0.665)(3.48,0.665)

\psbezier(3.58,0.625)(3.72,0.125)(4.32,0.505)(4.46,0.125)

\rput(0.76671875,0.175){\small\color{gdarkgray}Licht}
\rput(3.73,-1.285){\small\color{gdarkgray}ruhender Beobachter}
\rput(4.1,1.855){\small\color{gdarkgray}bewegter Beobachter}
\rput(4.3,-0.0050){\small\color{gdarkgray}Ereignis}

\rput(1.8909374,0.855){\small\color{gdarkgray}$v_2$}
\rput(3.2964063,0.855){\small\color{gdarkgray}$v_1$}

\rput(1.9685937,1.415){\small\color{gdarkgray}$t$}
\rput(4.445469,-0.645){\small\color{gdarkgray}$x$}
\end{pspicture} 
\begin{pspicture}(0.2,-1.4)(5.6,2.035)
%\psgrid
\psline{->}(0.3,-0.495)(4.48,-0.475)
\psline{->}(2.14,-0.575)(2.14,1.305)
\psline[linecolor=yellow](0.32,1.025)(2.12,-0.475)
\psline[linecolor=yellow](3.94,1.005)(2.12,-0.475)
\psline[linecolor=yellow]{<-}(3.44,0.585)(2.12,-0.475)

\psbezier(2.2,-0.595)(2.34,-1.095)(2.94,-0.715)(3.08,-1.095)

\psdots[linecolor=darkblue,dotsize=0.2](2.14,-0.495)
\psline[linecolor=darkblue]{->}(1.16,-1.995)(3.4,1.425)
\psbezier(3.48,1.485)(3.66,1.805)(4.04,1.325)(4.44,1.665)

\psline[dotsep=0.06cm]{<->}(2.42,-0.015)(1.94,-0.255)
\psline[dotsep=0.06cm]{<->}(2.94,0.245)(2.52,0.0050)

%\psbezier(3.58,0.625)(3.72,0.125)(4.32,0.505)(4.46,0.125)

\psbezier(3.0325,0.165)(3.2725,-0.195)(3.3325,0.185)(3.6125,-0.095)

%\psbezier(0.5125,1.325)(0.6325,0.865)(0.8925,1.2154762)(0.9725,0.865)
\psbezier(0.86,1.525)(0.9,1.125)(1.26,1.305)(1.28,0.945)

\psline{<->}(0.8325,0.725)(3.5725,0.725)

\rput(0.76671875,0.175){\small\color{gdarkgray}Licht}
\rput(3.73,-1.285){\small\color{gdarkgray}ruhender Beobachter}
\rput(4.1,1.855){\small\color{gdarkgray}bewegter Beobachter}
\rput(4.6,-0.205){\small\color{gdarkgray}Ereignis bew.}

\rput(1.9309375,0.095){\small\color{gdarkgray}$v_2$}
\rput(2.9764063,0.435){\small\color{gdarkgray}$v_1$}

\rput(1.9685937,1.415){\small\color{gdarkgray}$t$}
\rput(4.445469,-0.645){\small\color{gdarkgray}$x$}

\rput(1.2,1.795){\small\color{gdarkgray}Ereignis ruh.}

\end{pspicture} 

\caption{Gleichzeitige Ereignisse in der KM (links) und der SRT (rechts).}
\label{figure:5:AbsGleichzeitigkeit}
\end{figure}

Gibt man die absolute Gleichzeitigkeit auf und fordert dagegen, dass die
Lichtgeschwindigkeit eine universelle konstante für alle Bezugssysteme ist,
nehmen ruhender und bewegter Beobachter unterschiedliche
Ereignisse als ``gleichzeitig'' wahr. Ein bewegter
Beobachter hat eine ``verschobene Idee'' davon, was Gleichzeitigkeit
bedeutet.\bsphere
\end{bsp}

\addtocounter{subsection}{-1}
\subsection{Mathematische Formulierung}
Einstein baut seine ``Spezielle Relativitätstheorie'' auf zwei Axiomen auf.
\begin{enumerate}[label=(\roman{*})]
  \item\label{prop:5:SRT:1} \textit{Spezielles Relativitätsprinzip.}\\
Alle Naturgesetze haben in allen Inertialsystemen die selbe Form.
  \item\label{prop:5:SRT:2} \textit{Konstanz der Lichtgeschwindigkeit.}\\
$c$ ist eine universelle Konstante für alle Inertialsysteme.
\end{enumerate}

Zur mathematischen Formulierung dieser Theorie bietet es sich an, \emph{4-er
Ortsvektoren} einzuführen,
\begin{align*}
\rvec{x}^\mu =
\begin{pmatrix}
ct\\\vec{x}              
\end{pmatrix} = 
\begin{pmatrix}
x^0 \\ x^1 \\ x^2 \\ x^3
\end{pmatrix},\quad \text{d.h. } x_0 = ct,
\end{align*}
die ein Ereignis, welches in einem System $K$ stattfindet, durch Angabe von
Zeit und Ort des Geschehens charakterisieren. Indem man die Zeit $t$ mit
der Lichtgeschwindigkeit multipliziert, haben alle Komponenten des 4-er Vektors
die Dimension einer Länge. Die 4-er Vektoren sind Elemente des
\emph{Minkowski-Raums}, ein vierdimensionaler Vektorraum, der nicht mehr mit
der euklidischen Metrik versehen ist. Wir werden uns später damit ausführlicher
befassen.

\begin{bemn}[Bemerkung zur Notation.]
Für einen 4-er Vektor verwenden wir das Symbol $\rvec{x}^\mu$ bzw.
$\rvec{x}$. Die $\mu$-te Komponente des 4-er Vektors bezeichnen wir mit
$x^\mu$, wobei $\mu=0,\ldots,3$.

Die \emph{Einstein'sche Summennotation} erlaubt eine kompakte Schreibweise für
Summen,
\begin{align*}
&x_\mu x^\mu\equiv\sum_\mu x_\mu x^\mu,\\
&A_{\mu\nu} x^\mu x^\nu\equiv\sum_{\mu,\nu} A_{\mu\nu} x^\mu x^\nu. 
\end{align*}
Wann immer der gleiche Index oben ${}^\mu$ und unten ${}_\mu$ in einem Ausdruck
auftritt, wird darüber summiert.\maphere
\end{bemn}
Es lassen sich nun Transformationen definieren, die ein Inertialsystem
bezüglich dieser Darstellung in ein anderes überführen. Homogenität und
Isotropie des Raumes verlangen, dass diese Transformationen affine Abbildungen
sind
\begin{align*}
\rvec{x}' = \Lambda \rvec{x} + \rvec{a},
\end{align*}
wobei $\Lambda$ eine $4\times 4$-Matrix und $\rvec{a}$ einen konstanten 4-er
Vektor beschreibt.

Zur Beschreibung dieser Transformationen genügt es, sich zunächst auf einen
Boost eines Inertialsystems $K'$, dessen Ursprung sich mit konstanter
Geschwindigkeit entlang der $x$-Achse relativ zu $K$ bewegt, zu beschränken
($\rvec{a}=0$). Die Transformation hat daher die Form,
\begin{align*}
&\begin{rcases}
{x^0}' = a(v)x^0 + b(v)x^1\\
{x^1}' = d(v)x^1 + e(v)x^0\\
{x^2}' = x^2\\
{x^3}' = x^3
\end{rcases}
\Rightarrow\Lambda =
\begin{pmatrix}
a & b & 0 & 0\\
e & d & 0 & 0\\
0 & 0 & 1 & 0\\
0 & 0 & 0 & 1
\end{pmatrix}.
\end{align*}
Homogenität und Isotropie des Raumes erfordern, dass die Koeffizienten
lediglich von $v$ abhängen können. Kehren wir die Richtung der $x$-Achse in $K$
und $K'$ sowie $v$ um,
\begin{align*}
&{x^0}' = a(-v)x^0 - b(-v)x^1,\\
&-{x^1}' = -d(-v)x^1 + e(-v)x^0,\\
&{x^2}' = x^2,\\
&{x^3}' = x^3.
\end{align*}
 erzwingt die Isotropie, dass sich die
Transformationsformel nicht ändert, d.h.
\begin{align*}
&a(-v) = a(v),\qquad b(v) = -b(-v),\\
&d(v) = d(-v),\qquad e(-v) = -e(v).
\end{align*}
Betrachten wir nun den Nullvektor im $K'$ zum Zeitpunkt $t$, ergibt sich
\begin{align*}
{x^1}' = 0 = d(v)vt + e(v)ct \Rightarrow \frac{e(v)}{d(v)} = -\frac{v}{c} \equiv
-\beta.
\end{align*}
Wir sehen, dass die Koeffizienten der Matrix $\Lambda$ bereits durch die Art
der Bewegung eingeschränkt werden.

Da es unabhängig ist, ob sich $K'$ mit der Geschwindigkeit $v$ zu $K$ oder $K$
sich mit der Geschwindigkeit $-v$ zu $K'$ bewegt, hat die Rücktransformation
die Form,
\begin{align*}
&x^0 = a(-v){x^0}' + b(-v){x^1}' = a(v){x^0}' - b(v){x^1}'\\
&x^1 = d(-v){x^1}' + e(-v){x^0}' = d(v){x^1}' - e(v){x^0}'\\
&x^2 = {x^2}'\\
&x^3 = {x^3}'\\
\Rightarrow\;&\Lambda^{-1} =
\begin{pmatrix}
a & -b & 0 & 0\\
-e & d & 0 & 0\\
0 & 0 & 1 & 0\\
0 & 0 & 0 & 1
\end{pmatrix}.
\end{align*}
$\Lambda$ und $\Lambda^{-1}$ sind invers zueinander, d.h.
$\Lambda^{-1} \Lambda = \Id$,
woraus sich folgendes Gleichungssystem ergibt,
\begin{align*}
&a^2 -be = 1, && -ab + bd = 0,\\
&-ae + de = 0, && -be +d^2 = 1.
\end{align*}
Wählen wir $\kappa\in\R$ als freien Parameter, so besitzt das System eine
eindeutige Lösung. Setzen wir
\begin{align*}
b = -a\beta \kappa^2,
\end{align*}
ergibt sich für die Transformationen die Form
\begin{align*}
&{x^0}' = \gamma x^0 - \gamma\beta\kappa^2x^1\\
&{x^1}' = -\gamma\beta\kappa^2 x^0 + \gamma x^1\\
&{x^2}' = x^2\\
&{x^3}' = x^3
\end{align*}
wobei $\beta = \dfrac{v}{c}$, $\gamma=\dfrac{1}{\sqrt{1-\kappa^2\beta^2}}$.
\begin{bemn}
Für $\kappa = 0$ ist $\gamma=1$ und wir erhalten die
Galilei-Transformation für einen boost.

Prinzip \ref{prop:5:SRT:1} kann also auch auf die Gallilei-Transformationen
führen. Diese sind jedoch nicht mit \ref{prop:5:SRT:2} verträglich. Wählt man
$\kappa = 1$, so sind \ref{prop:5:SRT:1} und \ref{prop:5:SRT:2} erfüllt. Die
Größe $\frac{c}{\kappa}$ spielt die Rolle einer maximalen
Signalgeschwindigkeit. Für $\kappa=0$ werden alle Wechselwirkungen instantan,
für $\kappa=1$ ist die Ausbreitungsgeschwindigkeit die
Lichtgeschwindigkeit.\maphere
\end{bemn}

Wir erhalten für $\kappa=1$ die \emph{Lorentz-Transformationen}, welche die
Lichtgeschwindigkeit bei Transformation zwischen Inertialsystemen invariant
lassen. Die Lorenz-Transformation für einen Boost in $x$-Richtung ist gegeben
durch,
\begin{align*}
\begin{rcases}
{x^0}' = \gamma x^0 - \gamma\beta x^1\\
{x^1}' = -\gamma\beta x^0 + \gamma x^1\\
{x^2}' = x^2\\
{x^3}' = x^3
\end{rcases}
\Rightarrow \Lambda = 
\begin{pmatrix}
\gamma & -\gamma\beta & 0 & 0\\
-\gamma\beta & \gamma & 0 & 0\\
0 & 0 & 1 & 0\\
0 & 0 & 0 & 1
\end{pmatrix}.
\end{align*}
\begin{figure}[!htbp]
\begin{pspicture}(-0.1,-1.151875)(3.0834374,1.151875)

\psline[linecolor=darkblue]{->}(0.3,-0.85)(1.0078125,-0.44703126)

\psline{->}(0.3078125,-0.971875)(0.3078125,0.968125)
\psline{->}(0.1478125,-0.851875)(3.0078125,-0.851875)
\psline{->}(1.0278125,-0.451875)(1.4878125,0.928125)
\psline{->}(0.9478125,-0.391875)(2.4878125,-0.151875)

\rput(0.11203125,0.998125){\color{gdarkgray}$ct$}
\rput(2.9532812,-1.001875){\color{gdarkgray}$x$}
\rput(1.2365625,0.978125){\color{gdarkgray}$ct'$}
\rput(2.5703125,-0.321875){\color{gdarkgray}$x'$}

\rput(1.5,-0.65703124){\color{gdarkgray}$(ct,vt)$}
\end{pspicture} 
\caption{Lorentztransformation im Minkowski-Raum}
\end{figure}

\begin{bemn}
Die Lorentz-Transformation lässt die Größe
\begin{align*}
(x^0)^2 - \vec{x}^2 = ct^2-(x^1)^2-(x^2)^2-(x^3)^2 = g_{\mu\nu}x^\nu x^\mu
\end{align*} 
unverändert. $g_{\mu\nu}$ bezeichnet hier den metrischen Tensor,
\begin{align*}
(g_{\mu\nu}) =
\begin{pmatrix}
1 & 0& 0 & 0\\
0 & -1 & 0 & 0\\
0 & 0 & -1 & 0\\
0 & 0 & 0 & -1
\end{pmatrix}.
\end{align*}
\begin{proof}
Wir betrachten nur den Spezialfall des $x$-Richtung Boost, die übrigen folgen
analog.
\begin{align*}
(x'^0)^2 - \vec{x}^{'2} &=
(\gamma x^0 - \gamma\beta x^2)^2 - (-\gamma\beta x^0 + \gamma x^1) - (x^2)^2 +
(x^3)^2 \\
&= \gamma^2(x^0)^2 + \gamma^2\beta^2(x^1)^2 - \gamma^2\beta^2(x^0)^2 -
\gamma^2(x^1)^2 - (x^2)^2 - (x^3)^2\\
&= (x^0)^2\underbrace{\gamma^2(1-\beta^2)}_{=1} +
(x^1)^2\underbrace{\gamma^2(\beta^2)-1}_{=-1} - (x^2)^2 - (x^3)^2\\
&= (x^0)^2 - (x^1)^2 - (x^2)^2 - (x^3)^2\qedhere\maphere
\end{align*}
\end{proof}
\end{bemn}

\subsubsection{Die Lichtgeschwindigkeit als universelle Konstante}

Wir wollen nun zeigen, dass die Lichtgeschwindigkeit $c$ unabhängig vom
Inertialsystem ist.
\begin{proof}
Bewege sich ein Teilchen in einem Ineratialsystem $K$ konstant mit
Geschwindigkeit $c$ in Richtung $\vec{n}$ ($\vec{n}$ Einheitsvektor). Die
Bewegungsgleichung ist dann gegeben durch,
\begin{align*}
&\vec{x}(t) = ct\vec{n},\qquad \abs{\dvec{x}} = c.
\end{align*} 
Transformieren wir nun in Ineratialsystem $K'$, das sich relativ zu $K$ mit
Geschwindigkeit $v$ bewegt, so erhalten wir die Gleichungen
\begin{align*}
&ct' = \gamma (x^0) -\gamma\beta (x^1) = \gamma ct - \gamma\beta ctn^1 = \gamma
ct(1-\beta n^1),\\
&(x^1)' = -\gamma\beta (x^0) + \gamma (x^1) = -\gamma\beta ct + \gamma ct n^1 =
\gamma ct(n^1-\beta) = ct'\frac{n^1-\beta}{1-\beta n^1},\\
&(x^2)' = (x^2) = ctn^2 = \frac{1}{\gamma}\frac{ct'}{1-\beta n^1}n^2,\\
&(x^3)' = (x^3) = ctn^3 = \frac{1}{\gamma}\frac{ct'}{1-\beta n^1}n^3.
\end{align*}
Um die Geschwindigkeit des Teilchens in $K'$ zu erhalten, müssen wir die
Bewegungsgleichung in $K'$ aufstellen,
\begin{align*}
\vec{x}'(t') = ct'
\begin{pmatrix}
\frac{n^1-\beta}{1-\beta n^1}\\
\frac{1}{\gamma}\frac{n^2}{1-\beta n^1}\\
\frac{1}{\gamma}\frac{n^3}{1-\beta n^1}
\end{pmatrix}
= ct'\vec{n}'
\end{align*}
Nun ist $\abs{\dvec{x}'} = c\abs{\vec{n}'}$. Wir müssen also zeigen $\abs{n}' =
1$. Dazu betrachten wir den Spezialfall
\begin{align*}
\vec{n} = 
\begin{pmatrix}
1 \\ 0\\ 0
\end{pmatrix}
\Rightarrow
\vec{n}' =
\begin{pmatrix}
\frac{1-\beta}{1-\beta}\\
\frac{1}{\gamma}0\\
\frac{1}{\gamma}0
\end{pmatrix}
= \begin{pmatrix}
  1 \\ 0\\ 0
  \end{pmatrix}.\qedhere
\end{align*}
\end{proof}

% Gilt
% \begin{align*}
% &(x^0)' = \gamma x^0 - \gamma\beta x^1 = \const,\\
% \Leftrightarrow & ct - \beta x^1 = \const,
% \end{align*}
% so treten die Ereignisse gleichzeitig ein. Ist
% \begin{align*}
% &(x^1)' = -\gamma\beta x^0 + \gamma x^1 = \const,\\
% \Leftrightarrow & ct = \const + \beta x^1,
% \end{align*}
% so finden die Ereignisse am gleichen Ort statt. 

\subsection{Einfache Folgerungen aus der SRT}

Die Lorenztransformationen zeigen, dass es nicht möglich ist, sich schneller
als das Licht zu bewegen.

\begin{defnn}
Sei $\rvec{x}^\nu$ ein $4$-er Vektor. Wir bezeichnen Ergeignisse abhängig vom
Skalarpodukt als
\begin{align*}
&g_{\mu\nu}x^\mu x^\nu > 0 \qquad: \text{\emph{zeitartig}},\\
&g_{\mu\nu}x^\mu x^\nu = 0 \qquad: \text{\emph{lichtartig}},\\
&g_{\mu\nu}x^\mu x^\nu < 0 \qquad: \text{\emph{raumartig}}.\fishhere
\end{align*}
\end{defnn}

\begin{figure}[!htbp]
  \centering
\begin{pspicture}(0,-2.2939062)(4.8934374,2.2939062)
\psline[linecolor=yellow](0.0,-2.1560917)(3.9798,1.8239063)
\psline[linecolor=yellow](3.9798,-2.1560917)(0.0,1.8239063)
\psline{->}(0.0,-0.15609375)(4.0,-0.15609375)
\psline{->}(1.98,-2.2160938)(1.98,1.9239062)

\psbezier{->}(3.9,-1.2760937)(3.46,-1.6960938)(3.68,-0.99609375)(3.24,-1.2560937)

\psdots[dotstyle=x](3.26,-0.57609373)
\psdots[dotstyle=x](3.12,-2.0360937)
\psdots[linecolor=darkblue,dotsize=0.2](1.98,-0.15609375)

%\psbezier[linecolor=darkblue]{<-}(2.74,1.9439063)(3.2,1.4039062)(3.0,0.86390626)(2.46,0.80390626)(1.92,0.74390626)(2.2457285,0.2946278)(1.98,-0.17609376)(1.7142714,-0.6468153)(1.94,-1.1760937)(1.48,-1.2560937)(1.02,-1.3360938)(1.12,-1.6760937)(1.74,-2.0160937)

\psbezier[linecolor=darkblue]{<-}(2.745889,1.92)(2.9258888,1.22)(2.705889,1.08)(2.46,0.8058889)(2.214111,0.5317778)(2.2457285,0.29661044)(1.98,-0.17411113)(1.7142715,-0.6448327)(1.9058889,-0.82)(1.4858888,-1.28)(1.0658889,-1.74)(0.4858889,-1.84)(0.44588888,-2.22)

\rput(3.665625,0.25390625){\small\color{gdarkgray}raumartig}

\rput(0.923125,1.7476562){\small\color{gdarkgray}zeitartig}

\rput(4.231406,-1.0660938){\small\color{gdarkgray}lichtartig}

\rput(1.6065625,-0.00609375){\small\color{gdarkgray}A}

\rput(3.4871874,-0.66609377){\small\color{gdarkgray}B}

\rput(3.3271875,-2.1260939){\small\color{gdarkgray}C}

\rput(3.2885938,2.1139061){\small\color{gdarkgray}Teilchen mit $v<c$}
\end{pspicture}

\caption{Lichtkegel mit ruhendem Punkt $A$, raumartigen Punkt $B$ und
zeitartigem Punkt $C$.}
\end{figure}

Als Beobachter im Punkt $A$ zur Zeit $t=0$ können wir nur zeitartige
Ereignisse, die sich im unteren Lichtkegel befinden, wahrnehmen bzw. von ihnen
beeinflusst werden. Zeitartige Ereignisse, die sich im oberen Lichtkegel
befinden, können beeinflusst werden. Sind zwei Ereignisse raumartig getrennt,
können sie sich gegenseitig weder wahrnehmen noch beeinflussen.
% Lichtartige
%Ereignisse können von Beobachtern, die sich mit Lichtgeschwindigkeit bewegen,
%wahrgenommen bzw. beeinflusst werden.

Gilt in einem Bezugssystem für zwei zeitartige Ereignisse $A$ und $C$, dass $C$
vor $A$ stattfindet, d.h. $x_A^0 > x_C^0$, so gilt dies in jedem Bezugssystem.
Um dies einzusehen betrachte $\rvec{x}^\mu = \rvec{x}_A^\mu - \rvec{x}_C^\mu$,
% =\begin{pmatrix} x_A^0-x_C^0\\\vec{x}_A -\vec{x}_C\end{pmatrix}$, 
wobei $\rvec{x}_\mu\rvec{x}^\mu  = (x_A^0-x_C^0)^2-(\vec{x}_A-\vec{x}_C)^2 > 0$.
Angenommen es gibt ein Bezugssystem mit $x_A^0=x_C^0$, dann ist
$\rvec{x}_\mu\rvec{x}^\mu < 0$, das Skalarprodukt ist jedoch invariant unter
Lorentz-Transformationen, d.h. es kann kein solches Bezugssystem geben.
Es gibt also eine absolute Vergangenheit und Zukunft für zeitartige Ereignisse.
Für raumartige Ereignisse ist die zeitliche Folge jedoch relativ und nicht
absolut. Raumartig getrennte Ereignisse können also nicht kausal verknüpft sein.

Die Lichtgeschwindigkeit $c$ ist daher die maximale Geschwindigkeit für
Informationen und Teilchen.

\subsubsection{Relativität der Gleichzeitigkeit}

Für einen ruhenden Beobachter finden zwei Ereignisse $A$ und $B$ gleichzeitig
statt, wenn
\begin{align*}
x_A^0 = x_B^0.
\end{align*}
Für einen gleichförmig mit der Geschwindigkeit $v$ bewegten Beobachter finden
sie gleichzeitig statt, wenn
\begin{align*}
{x_A^0}' = {x_B^{0}}' = c\tau.
\end{align*}
Für den ruhenden Beobachter entspricht dies den Ereignissen
\begin{align*}
&c\tau = \gamma x^0 - \beta\gamma x^1\\
&x^0 = \frac{c\tau + \beta\gamma x^1}{\gamma} = \frac{c\tau}{\gamma} +
\beta x^1,
\end{align*}
die auf einer Geraden mit Steigung $\beta$ liegen.

\begin{figure}[!htbp]
  \centering
\begin{pspicture}(0.8,-1.011875)(4.5,2.131875)
\psarc(1.39,-0.641875){1.39}{69.17911}{88.60282}
\rput{-90.0}(2.111875,0.708125){\psarc(1.41,-0.701875){1.39}{91.39718}{110.82089}}
\psline[linecolor=darkblue]{->}(1.42,-0.671875)(2.24,1.668125)
\psline[linecolor=darkblue]{->}(1.42,-0.671875)(3.76,0.148125)
\psline{->}(1.42,-0.831875)(1.42,1.668125)
\psline{->}(1.25,-0.661875)(3.75,-0.661875)

\rput(1.593125,0.538125){\color{gdarkgray}$\alpha$}
\rput(2.553125,-0.441875){\color{gdarkgray}$\alpha$}
\rput(1.22125,1.798125){\color{gdarkgray}$x^0$}
\rput(3.925625,-0.681875){\color{gdarkgray}$x^1$}
\rput(4.0925,0.278125){\color{gdarkgray}${x^1}'$}
\rput(2.4725,1.958125){\color{gdarkgray}${x^0}'$}
\rput(2.011875,1.578125){\color{gdarkgray}$K'$}
\rput(1.2685938,-0.861875){\color{gdarkgray}$K$}

\rput(2.6265626,0.458125){\color{gdarkgray}$A$}
\rput(3.5471876,0.858125){\color{gdarkgray}$B$}
\rput(2.9471874,1.058125){\color{gdarkgray}$C$}

\psline[linecolor=yellow](1.42,0.048125)(3.96,1.328125)
\psdots[dotstyle=x](2.64,0.668125)
\psdots[dotstyle=x](3.36,1.028125)
\psdots[dotstyle=x](3.0,0.848125)

\end{pspicture} 

\caption{Ereignisse $A$ und $B$ mit Detektor $C$.}
\end{figure}

Die allgemeine Definition von Gleichzeitigkeit lautet daher,
\begin{defnn}
Zwei Ereignisse $A$ und $B$ heißen \emph{gleichzeitig}, wenn von $A$ und $B$
zur gleichen Zeit ausgesendtes Licht bei einem in der Mitte ruhenden Detektor
zur gleichen Zeit eintrifft.\fishhere
\end{defnn}

\subsubsection{Längenkontraktion}

Betrachte einen Stab mit der Länge $L_0$, der im Bezugsystem $K'$ ruht. Das
Bezugsystem $K'$ bewege sich relativ zu $K$ mit der Geschwindigkeit $v$.

\begin{figure}[!htbp]
  \centering
\begin{pspicture}(-0.1,-2.1)(5.625,2.061875)
\psline{->}(0.3553125,-1.841875)(0.3553125,0.658125)
\psline{->}(0.1853125,-1.671875)(2.6853125,-1.671875)
\psline{->}(2.8353126,-0.741875)(2.8353126,1.758125)
\psline{->}(2.6653125,-0.571875)(5.1653123,-0.571875)

\rput(0.1565625,0.788125){\color{gdarkgray}$x^0$}
\rput(2.8609376,-1.691875){\color{gdarkgray}$x^1$}
\rput(0.20390625,-1.871875){\color{gdarkgray}$K$}

\rput(2.6678126,1.888125){\color{gdarkgray}${x^0}'$}
\rput(5.3878126,-0.591875){\color{gdarkgray}${x^1}'$}
\rput(2.6271875,-0.771875){\color{gdarkgray}$K'$}

\psline(0.8753125,-1.681875)(1.3953125,-0.581875)
\psline(1.7753125,-1.681875)(2.2953124,-0.581875)

\psframe[linecolor=darkblue,fillstyle=solid,fillcolor=darkblue](4.9,-0.521875)(3.1553125,-0.601875)
\psframe[linecolor=darkblue,fillstyle=solid,fillcolor=darkblue](2.1153126,-0.901875)(1.2353125,-0.981875)
\rput(0.8609375,-1.891875){\color{gdarkgray}$x_L^1$}
\rput(1.7809376,-1.911875){\color{gdarkgray}$x_R^1$}
\rput(3.1809375,-0.891875){\color{gdarkgray}${x_L^1}'$}
\rput(4.85,-0.891875){\color{gdarkgray}${x_R^1}'$}

\psline{->}(3.2153125,1.298125)(4.0153127,1.298125)

\rput(3.5620313,1.488125){\color{gdarkgray}$v$}
\end{pspicture} 

\caption{Längenkontraktion eines Stabes.}
\end{figure}

Um die Länge $L=\left(x_R^1- x_L^1\right)$ in $K$ zu messen, müssen linkes und
rechtes Ende zur gleichen Zeit lokalisiert werden ($x_R^0-x_L^0=0$),
\begin{align*}
L_0 &= {x_R^1}'-{x_L^1}' = 
\gamma \left(x_R^1-\beta x_R^0 \right)
- \gamma \left(x_L^1-\beta x_L^0 \right)
= \gamma\left(x_R^1- x_L^1\right)=\gamma L\\
\Rightarrow L &=\sqrt{1-\frac{v^2}{c^2}}L_0 < L_0.
\end{align*}
Die Länge des Stabes ist also in jedem relativ zum Ruhesystem bewegten System
\textit{kleiner}. Diesen Effekt nennt man \emph{Lorentzkontraktion}.

\subsubsection{Zeitdilatation}

Betrachte eine Uhr, die im Koordinatenursprung von $K$ ruht. Ein Zeitintervall
$\tau=t_A-t_B$, das wir im Ruhesystem $K$ ablesen, heißt \emph{Eigenzeit}. Lesen
wir nun zusätzlich die Zeit in einem Bezugssystem $K'$ ab, das sich relativ zu $K$
mit der Geschwindigkeit $v$ bewegt.

\begin{figure}[H]
  \centering
\begin{pspicture}(-0.1,-1.6)(3.18375,1.5017188)
\psline{->}(0.4878125,-1.3017187)(0.4878125,1.1982813)
\psline{->}(0.3178125,-1.1317188)(2.8178124,-1.1317188)

\psline[linecolor=yellow]{->}(0.4878125,-1.1417187)(2.0078125,0.93828124)

\psline(1.6078125,0.37828124)(1.6078125,-1.1217188)
\psline(1.6278125,0.39828125)(0.4878125,0.39828125)

\psdots[linecolor=darkblue](0.4878125,-1.1417187)
\psdots[linecolor=darkblue](1.6078125,0.37828124)

\rput(0.20203125,0.40828124){\color{gdarkgray}$c\Delta t$}
\rput(0.2890625,1.3282813){\color{gdarkgray}$x^0$}

\rput(2.9934375,-1.1517187){\color{gdarkgray}$x^1$}
\rput(1.625625,-1.3517188){\color{gdarkgray}$x_B^1$}

\rput(2.3,0.96828127){\color{gdarkgray}${x^0}'$}
\rput(0.334375,-1.3317188){\color{gdarkgray}$A$}
\rput(1.855,0.22828124){\color{gdarkgray}$B$}

\end{pspicture} 

\caption{Zeitdilatation einer bewegten Uhr.}
\end{figure}

Mit den bekannten Transformationsformeln ergibt sich in $K'$,
\begin{align*}
&{x_A^0}'-{x_B^0}' = \gamma \left({x_A^0} + \beta {x_A^1}\right) -
\gamma\left({x_B^0} - \beta {x_B^1}\right) =
\gamma\left({x_B^0}-{x_A^0}\right)\\
\Rightarrow &\tau' =  \gamma \tau > \tau.
\end{align*}
Von $K'$ aus betrachtet vergeht,
bis auf der Uhr das Zeitintervall $\tau$ verstrichen ist, in $K'$ tatsächlich \textit{mehr Zeit}.
Eine bewegte Uhr geht also um den konstanten Faktor
\begin{align*}
\frac{1}{\sqrt{1 - \frac{v^2}{c^2}}}
\end{align*}
``langsamer'' als in ihrem Ruhesystem.

\begin{bemn}
Das Differential der Eigenzeit ist
\begin{align*}
\dtau = \frac{1}{\gamma}\dt = \sqrt{1-\frac{v^2}{c^2}}\dt.
\end{align*}
Messen wir im Laborsystem die Zeitdifferenz $t_2-t_1$, erhalten wir für 
eine allgemeine Bewegung die Zeitdifferenz der bewegten Uhr in ihrem
Ruhesystem
\begin{align*}
\tau_2 - \tau_1 = \int\limits_{t_1}^{t_2} \dt\sqrt{1-\frac{v^2}{c^2}}.\maphere
\end{align*}
\end{bemn}

\begin{bsp}
\textit{Die Lichtuhr}. Zwischen zwei in $K'$ ruhenenden Spiegeln mit Abstand $l$
wird ein Lichtstrahl hin- und hergesandt. Immer wenn der Lichtstrahl auf einem
der Spiegel registriert wird, wird die Lichtuhr um eine Zeiteinheit weitergestellt.

\begin{figure}[!htbp]
  \centering
\begin{pspicture}(0,-0.643)(4.279568,0.622)
\psline(0.0,-0.198)(1.7805682,-0.198)
\psline(0.0,0.603)(1.7805682,0.603)

\psline(2.48,-0.198)(4.260568,-0.198)
\psline(2.48,0.603)(4.260568,0.603)

\psline[linecolor=yellow]{<->}(0.9,-0.078)(0.9,0.522)
\psline[linecolor=yellow]{<-}(3.4,-0.138)(2.88,0.542)
\psline[linecolor=yellow]{->}(3.36,-0.11)(3.88,0.562)

\psline(3.38,-0.138)(3.38,-0.318)
\psline(2.88,-0.138)(2.88,-0.318)

\rput(0.1009375,0.232){\color{gdarkgray}$l$}
\rput(2.5809374,0.232){\color{gdarkgray}$l$}
\rput(3.10875,-0.488){\color{gdarkgray}$v\Delta t$}
\end{pspicture} 

\caption{Lichtuhr in ruhendem (links) und bewegtem (link) Zustand.}
\end{figure}

Im Ruhesystem $K'$ benötigt der Lichtstrahl, um von einem Spiegel zum anderen zu
gelangen, die Zeit $\Delta\tau' = l/c$. Nun bewege sich $K'$ relativ zu $K$ mit
der Geschwindigkeit $v$ entlang der Spiegelachse. Aus der Sicht von $K$ hat das
Licht zwischen zwei Spiegeln eine größere Distanz zurückzulegen, die
Lichtgeschwindigkeit ist aber in allen Inertialsystem gleich, d.h. in $K$
vergeht zwischen zwei Spiegeln die Zeit
\begin{align*}
&\Delta\tau =  \frac{\sqrt{l^2 + v^2\Delta\tau^2}}{c}\\
&\Delta\tau^2 = \frac{l^2 + v^2\Delta\tau^2}{c^2} = {\Delta\tau'}^2 +
\frac{v^2}{c^2}\Delta\tau^2\\
&\Delta\tau = \frac{1}{\sqrt{1-\frac{v^2}{c^2}}}\Delta\tau' > \Delta\tau'.
\end{align*}
Liest man in $K$ ab, geht die in $K'$ ruhende Uhr langsamer, als eine in $K$
ruhende Uhr, da mehr Zeit vergeht, bis die Lichtuhr um ein Zeitintervall
weitergestellt wird.

Andersherum würde ein in $K'$ ruhender Beobachter aufgrund der selben
Überlegung sehen, dass eine zu ihm bewegte Lichtuhr in $K$ langsamer geht, als
seine in $K'$ ruhende. Dies ist die Symmetrie der Zeitdilatation, jeder misst, dass die
Uhr des anderen langsamer geht.\bsphere
\end{bsp}

\begin{bemn}
Betrachten wir die Trajektorie eines bewegten Teilchens, so muss diese so
geformt sein, dass sie sich in jedem Punkt innerhalb des von dort ausgehenden
Lichtkegels befindet.
\begin{figure}[!htbp]
  \centering
\begin{pspicture}(0,-1.2941111)(4.034111,1.3)
\psline[linecolor=yellow](2.005889,-1.08)(3.9998,0.9058889)
\psline[linecolor=yellow](1.9858888,-1.08)(0.0,0.9058889)
\psline{->}(0.02,-1.0741111)(4.02,-1.0741111)
\psline{->}(1.9858888,-1.28)(2.0,1.0058888)

\psline(2.5647683,0.2)(2.3458889,0.52)
\psline(2.5816052,0.2)(2.785889,0.52)
\psline(1.1647682,0.44)(0.9458889,0.76)
\psline(1.1816051,0.44)(1.3858889,0.76)

\psbezier[linecolor=darkblue]{->}(1.9858888,-1.08)(2.3858888,-0.48)(2.185889,-0.36)(2.485889,0.04)(2.785889,0.44)(2.485889,0.92)(2.8858888,1.28)
\psbezier[linecolor=darkblue](1.5058889,-0.28)(1.3658888,0.28)(1.2858889,-0.14)(1.1658889,0.44)(1.0458889,1.02)(1.4858888,0.2)(1.745889,0.82)

\psellipse(2.5647683,0.53103274)(0.22600006,0.06896515)
\psellipse(1.1647683,0.7710327)(0.22399993,0.06896515)

\psdots(2.5698888,0.2)
\psdots(1.1738889,0.44)
\end{pspicture} 
\caption{Trajektore innerhalb und außerhalb der zulässigen Geschwindigkeit.}
\end{figure}
\end{bemn}

\begin{bsp}
\textit{Der Myon-Zerfall.}
In ca. 10km Höhe werden aufgrund der Wechselwirkung von Protonen aus dem
Weltraum mit unserer Atmosphäre Myonen gebildet. Sie haben eine sehr hohe
Geschwindigkeit $v\sim 0.998c\Rightarrow\gamma\approx 16$. Die Teilchen sind
nicht stabil und haben eine Lebensdauer von wenigen $\mathrm{\mu s}$. Mögliche
Myon-Zerfälle sind,
\begin{align*}
&\mu^+ \longrightarrow e^+ + \nu_e  + \overline{\nu}_\mu,\\
&\mu^- \longrightarrow e^- + \nu_\mu  + \overline{\nu}_e.
\end{align*}
Im Ruhesystem der Myonen kann der Zerfall beschrieben werden durch,
\begin{align*}
N(t) = N_0 e^{-\frac{t}{\tau}},\qquad \tau = 2.2\mathrm{\mu s}\quad
\text{mittlere Lebensdauer}.
\end{align*}

Rechnet man klassisch, benötigen die Myonen bis zum Erreichen der Erde die Zeit,
\begin{align*}
t_{\text{Flug}} = \frac{10\mathrm{km}}{0.998c} \approx 30 \mathrm{\mu s}.
\end{align*}
Es sind also kaum Myonen auf der Erde zu erwarten, man kann jedoch eine sehr
große Anzahl von eintreffenden Myonen messen.

Aufgrund der hohen Geschwindigkeit der Myonen kommen relativistische Effekte
zu tragen. Wir müssen die Eigenzeit der Myonen betrachten,
\begin{align*}
\tau_{\text{Flug}} = \frac{1}{\gamma}t_{\text{Flug}} \approx
\frac{1}{16}30\mathrm{\mu s} = 1.9\mathrm{\mu s}.
\end{align*}
Dies ist einer von vielen experimentellen Befunden, der die Korrektheit der SRT
bestätigt.\bsphere
\end{bsp}

\subsubsection{Zwillingsparadoxon}

Betrachtet man die Zwillingsbrüder $A$ und $B$. $A$ befinde sich in Ruhe,
während $B$ eine längere Reise mit $v\approx c$ macht. Trifft $B$ wieder auf der
Erde ein, ist er viel jünger als $A$.
\begin{figure}[H]
  \centering
\begin{pspicture}(0,-1.3717188)(4.2418265,1.3717188)
\psline[linecolor=yellow](2.005889,-1.0317187)(3.9998,0.95417017)
\psline[linecolor=yellow](1.9858888,-1.0317187)(0.0,0.95417017)
\psline{->}(0.02,-1.0258299)(4.02,-1.0258299)
\psline{->}(1.9858888,-1.2317188)(1.9858888,1.0541701)
\psbezier[linecolor=darkblue](1.9858888,-1.0117188)(2.305889,-0.31171876)(2.305889,-0.07171875)(1.9858888,0.50828123)
\psdots(1.9858888,0.48828125)
\psdots(1.9858888,-1.0117188)

\rput(2.2530763,0.49828124){\color{gdarkgray}$B$}
\rput(1.8124514,-1.2217188){\color{gdarkgray}$A$}
\rput(1.8671389,1.1982813){\color{gdarkgray}$x^0$}
\rput(4.0515137,-1.2017188){\color{gdarkgray}$x^1$}
\end{pspicture} 
\caption{Reise von $B$ im Minkowski-Raum.}
\end{figure}
Man kann dies mit der Zeitdilatation erklären,
\begin{align*}
\tau_2 - \tau_1 = \int\limits_{t_1}^{t_2}
\underbrace{\sqrt{1-\frac{v^2}{c^2}}}_{<1}\dt < t_2-t_1.
\end{align*}
Die für $B$ vergangene Zeit ist kleiner als die für $A$ vergangene.
\begin{bsp}
Betrachten wir den exotischen Fall der instantanen Beschleunigung.

\begin{figure}[!htbp]
  \centering
\begin{pspicture}(0,-1.3717188)(4.2418265,1.3717188)
\psline[linecolor=yellow](2.005889,-1.0317187)(3.9998,0.95417017)
\psline[linecolor=yellow](1.9858888,-1.0317187)(0.0,0.95417017)
\psline{->}(0.02,-1.0258299)(4.02,-1.0258299)
\psline{->}(1.9858888,-1.2317188)(1.9858888,1.0541701)
\psline[linecolor=darkblue](1.9858888,-1.0117188)(2.4058888,-0.15171875)(1.9858888,0.48828125)
\psdots(1.9858888,0.48828125)
\psdots(1.9858888,-1.0117188)

\rput(2.2530763,0.49828124){\color{gdarkgray}$B$}
\rput(1.8124514,-1.2217188){\color{gdarkgray}$A$}
\rput(1.8671389,1.1982813){\color{gdarkgray}$x^0$}
\rput(4.0515137,-1.2017188){\color{gdarkgray}$x^1$}
\end{pspicture}
\caption{Reise mit instantaner Beschleunigung.}
\end{figure}

Sei $v=0.8c$ und $t_2-t_1=10\text{ Jahre}$, so ergibt sich für die
Zeitdilatation,
\begin{align*}
\tau_2 -\tau_1 = 6\text{ Jahre}.
\end{align*}
Dies scheint paradox, da die Situation auf den ersten Blick Symmetrisch
erscheint, d.h. aus der Sicht von $B$ müsste ebenfalls weniger Zeit vergangen sein, als aus der
Sicht von $A$. Die Situation kann jedoch nicht symmetrisch sein, denn während
$A$ sich stets im selben Inertialsystem befindet, wechselt $B$ das
Inertialsystem. Es handelt sich hier also um kein Paradoxon.

Aus der Sicht von $B$ ist die in der Raumzeit zurückzulegende Strecke
``Lorentz-kontrahiert'', d.h. er benötigt weniger Zeit als $A$.\bsphere
\end{bsp}
\begin{bsp}
Um Auszuschließen, dass die Beschleunigungsphase von $B$ den entscheidenden
Unterschied macht, sollen nun $A$ und $B$ eine Reise machen und dabei identisch
beschleunigen, $B$ soll nur länger mit der hohen Geschwindigkeit fliegen.
$A$ komme nach $10$ Jahren auf die Erde zurück, $B$ nach $30$.

Für $A$ ist zwischen Abflug und $B$s Ankunft die Zeit
\begin{align*}
(\text{Beschleunigungsbeitrag}) + \frac{1}{\gamma}\left(10 \text{
Jahre}\right) + 20\text{ Jahre}
\end{align*}
vergangen. Für $B$ ist zwischen Abflug und Ankunft die Zeit
\begin{align*}
(\text{Beschleunigungs Beitrag}) + \frac{1}{\gamma}\left(30 \text{
Jahre}\right)
\end{align*}
vergangen. Für die Differenz ergibt sich,
\begin{align*}
\Delta T_A - \Delta T_B &=  \frac{1}{\gamma}\left(10 \text{
Jahre}\right) + 20\text{ Jahre} - \frac{1}{\gamma}\left(30 \text{
Jahre}\right) \\ 
&= \left(1-\frac{1}{\gamma}\right)\underbrace{\left(20\text{
Jahre}\right)}_{\text{Reisezeit diff.}}.
\end{align*}
Bei $v=0.8c$ beträgt die Differenz $\left(1-\frac{3}{5}\right)20\text{ Jahre} =
8\text{ Jahre}$.\bsphere
\end{bsp}

\subsection{Lorentztransformation}

In der Speziellen Relativitätstheorie werden die Galilei Transformationen, die
in der Klassischen Mechanik ein Inertialsystem in ein anderes überführen,
durch Poincaretransformationen ersetzt. Diese Transformationen bilden ebenfalls
eine Gruppe, die \emph{Poincare-Gruppe}, die die Galilei Gruppe ersetzt. Eine
Poincaretransformation hat die Form
\begin{align*}
\rvec{x}' = \Lambda\rvec{x}+\rvec{a},
\end{align*}
d.h. sie setzt sich aus einer Lorentztransformation $\rvec{\Lambda}$ und einer
Translation $\rvec{a}$ zusammen. Dabei ist die Lorentztransformation eine
lineare Abbildung
\begin{align*}
{x^\mu}' = \Lambda_\nu^\mu x^\nu,
\end{align*}
die dadurch charakterisiert ist, dass sie die Minkovski Metrik
\begin{align*}
g=g_{\alpha\beta} = g^\mu\nu = \begin{pmatrix}
                               1 & \\
                               & -1 \\
                               && -1 \\
                               &&& -1
                               \end{pmatrix}
\end{align*}
invariant lässt, d.h. $\lin{\Lambda \rvec{x},\Lambda \rvec{x}}_g =
\lin{\rvec{x},\rvec{x}}_g$.

Somit ist auch der Betrag des $4$-er Vektors $\rvec{x}^\nu$ invariant,
\begin{align*}
g_{\mu\nu}x^\mu x^\nu = (x^0)^2 - (\vec{x})^2 = \const.
\end{align*}
\begin{proof}[Beweis der Invarianz.]
Mit $y^\mu = \Lambda_\nu^\mu x^\nu$ erhalten wir,
\begin{align*}
g_{\mu\nu}y^\mu y^\nu = g_{\mu\nu}\Lambda_{\alpha}^\mu x^\alpha
\Lambda_\beta^\nu x^\beta = \left(\Lambda_\alpha^\mu
g_{\mu\nu}\Lambda_\beta^\nu\right)x^\alpha x^\beta = g_{\alpha\beta}x^\alpha
x^\beta.\qedhere
\end{align*}
\end{proof}

Für jeden 4-er Vektor $\rvec{x}^\mu$ ist daher die Größe
\begin{align*}
g_{\mu\nu}x^\mu x^\nu = {x^0}^2 - \vec{x}^2 = \const 
\end{align*}
eine feste Größe in allen Inertialsystemen. Insbesondere wird der Lichtkegel mit
\begin{align*}
g_{\mu\nu}x^\mu x^\nu = 0
\end{align*}
auf sich selbst abgebildet.

Eine spezielle Lorentztransformation haben wir bereits kennen gelernt, den
Boost in $x$-Richtung,
\begin{align*}
\Lambda_\mu^\nu = \begin{pmatrix}
                  \gamma & -\gamma\beta\\
                  -\gamma\beta & \gamma\\
                  && 1\\
                  &&& 1
                  \end{pmatrix}
= \begin{pmatrix}
\cosh\psi & -\sinh\psi\\
-\sinh\psi & \cosh\psi\\
&& 1\\
&&& 1
\end{pmatrix}
\end{align*}
mit der \emph{Rapidität} $\psi$ und den Relationen
\begin{align*}
\tanh \psi = \frac{v}{c},\quad \gamma =
\frac{1}{\sqrt{1-\frac{v^2}{c^2}}},\quad \beta = \frac{v}{c}.
\end{align*}
Rotationen haben die Form,
\begin{align*}
\Lambda_\mu^\nu = \begin{pmatrix}
                  1 & 0 & 0 & 0\\
                  0 & \\
                  0 & & \vec{R}\\
                  0 
                  \end{pmatrix}
\end{align*}
mit einer $3\times 3$-Rotationsmatrix $\vec{R}$.

Analog zur klassischen Mechanik gibt es auch die zwei diskreten
Transformationen Raumspiegelung und Zeitumkehr
\begin{align*}
P =
\begin{pmatrix}
1 &\\
& -1\\
&& -1\\
&&& -1
\end{pmatrix},\qquad 
T=\begin{pmatrix}
-1 &\\
& 1\\
&& 1\\
&&& 1
\end{pmatrix}.
\end{align*}
\begin{bemn}
Man kann zeigen, dass sich jede Lorentztransformation als Produkt dieser
einfachen Transformationen schreiben lässt.\maphere
\end{bemn}

\subsubsection{Addition von Geschwindigkeiten}

In der Speziellen Relativitätstheorie ist die Geschwindigkeitsaddition nicht
mehr linear wie in der Newtonschen Mechanik.

\begin{figure}[!htbp]
  \centering

\begin{pspicture}(-0.1,-1.491875)(3.5175,1.491875)
\psline{->}(0.3478125,-1.331875)(0.3478125,0.968125)
\psline{->}(0.2478125,-1.191875)(2.4878125,-1.191875)
\psline{->}(1.3878125,-0.471875)(1.3878125,1.128125)
\psline{->}(1.2678125,-0.371875)(3.1078124,-0.371875)
\psline{->}(2.3278124,0.288125)(2.3278124,1.248125)
\psline{->}(2.2478125,0.408125)(3.2878125,0.408125)
\psline[linecolor=darkblue]{->}(0.3478125,-1.191875)(1.3478125,-0.431875)
\psline[linecolor=darkblue]{->}(1.3878125,-0.371875)(2.3078125,0.348125)

\rput(1.71875,0.198125){\color{gdarkgray}$v_2$}
\rput(0.7242187,-0.621875){\color{gdarkgray}$v_1$}
\rput(2.4332812,-1.341875){\color{gdarkgray}$x$}
\rput(0.11203125,1.018125){\color{gdarkgray}$ct$}
\rput(3.0903125,-0.541875){\color{gdarkgray}$x'$}
\rput(1.1765625,1.198125){\color{gdarkgray}$ct'$}
\rput(3.3303125,0.258125){\color{gdarkgray}$x''$}
\rput(2.1965625,1.318125){\color{gdarkgray}$ct''$}
\end{pspicture} 
\caption{Relativbewegung von drei Koordinatensystemen.}
\end{figure}


 Um die genaue Form herzuleiten, betrachten wir die
Intertialsysteme $K$, $K'$ und $K''$ wobei sich $K'$ mit $v_1$ relativ zu $K$
und $K''$ mit $v_2$ relativ zu $K'$ bewegt.
Die Transformation von $K$ nach $K''$ erhalten wir, indem wir die einzelnen
Transformationen hintereinanderausführen,
\begin{align*}
\Lambda(v_2)\Lambda(v_1) &= 
\begin{pmatrix}
\cosh\chi_2 & -\sinh\chi_2\\
-\sinh\chi_2 & \cosh\chi_2\\
&& 1\\
&&& 1
\end{pmatrix}\\
&\quad \cdot
\begin{pmatrix}
\cosh\chi_1 & -\sinh\chi_1\\
-\sinh\chi_1 & \cosh\chi_1\\
&& 1\\
&&& 1
\end{pmatrix}\\
% &=
% \begin{pmatrix}
% \tiny\cosh\chi_2 \cosh\chi_1 + \sinh\chi_1\sinh\chi_2 & -\sinh\chi_1\cosh\chi_2
% - \sinh\chi_2\cosh\chi_1\\
% \cosh\chi_2 \cosh\chi_1 + \sinh\chi_1\sinh\chi_2 &-\sinh\chi_1\cosh\chi_2 -
% \sinh\chi_2\cosh\chi_1\\
% && 1\\ 
% &&& 1
% \end{pmatrix}\\
&=
\begin{pmatrix}
\cosh(\chi_1+\chi_2) & -\sinh(\chi_1+\chi_2)\\
-\sinh(\chi_1+\chi_2) & \cosh(\chi_1+\chi_2)\\
&& 1\\
&&& 1
\end{pmatrix}
\end{align*}
Es werden also nicht die Geschwindigkeiten sondern die Rapiditäten addiert,
\begin{align*}
\chi_3 = \chi_1+\chi_2\Rightarrow v_3 = c\tanh \chi_3 =
\frac{v_1+v_2}{1+\frac{v_1v_2}{c^2}}.
\end{align*}

\begin{bemn}
$v_3$ ist stets kleiner als die Lichtgeschwindigkeit $c$.\maphere
\end{bemn}

\subsection{Energie und Impuls}

Das Differential
\begin{align*}
\ds^2 = c^2\dt^2 - \diffd\vec{x}^2 =
c^2\left(1-\frac{\vec{v}^2}{c^2}\right)\dt^2 \equiv c^2\dtau^2
\end{align*}
ist lorentzinvariant und somit ist die Eigenzeit
\begin{align*}
\dtau = \frac{\ds}{c} = \sqrt{1-\frac{v^2}{c^2}}\dt
\end{align*}
eine lorentzinvariante skalare Größe.

Somit erhalten wir aus dem $4$-er Vektor $\rvec{x} = \begin{pmatrix}ct \\
\vec{x}\end{pmatrix}$ einen neuen Vektor,
\begin{align*}
&\rvec{u} = \frac{\diffd}{\dtau}\rvec{x} = \gamma \frac{\diffd}{\dt}\rvec{x} =
\gamma\begin{pmatrix}c \\ \vec{v}\end{pmatrix} = \begin{pmatrix} \gamma c \\
\gamma\vec{v}\end{pmatrix},\\
&\rvec{x} \mapsto \Lambda\rvec{x},\\
&\rvec{u} \mapsto \Lambda \rvec{u} = \Lambda \frac{\diffd}{\dtau}\rvec{x} =
\Lambda \frac{\diffd}{\dtau}\Lambda^{-1}\Lambda \rvec{x} =
\frac{\diffd}{\dtau}\Lambda\rvec{x}.
\end{align*}
\begin{bemn}[Bemerkungen.]
\begin{enumerate}[label=\arabic{*}.)]
  \item $\rvec{u}g\rvec{u} = \gamma^2c^2 - \gamma^2\vec{v}^2 =
c^2\gamma^2\left(1-\frac{v^2}{c^2}\right) = c^2$, d.h. 
 die Minkowski Metrik ist eine Konstante für alle Bewegungen.
  \item  Die Geschwindigkeit $\vec{v}$ ist kein Anteil eines 4-er
  Vektors.\maphere
\end{enumerate}
\end{bemn}

Wir definieren den $4$-er Impuls
\begin{align*}
&\rvec{p} = m\rvec{u} = m\frac{\diffd}{\dtau}\rvec{x},\\
&\rvec{p}^\mu = \begin{pmatrix}mc\gamma \\ m\vec{v}\gamma\end{pmatrix},\quad
\rvec{p}_\mu = g_{\mu\nu}\rvec{p}^\nu \begin{pmatrix}mc\gamma \\
-m\vec{v}\gamma\end{pmatrix}
\end{align*}
In der modernen Physik sieht man davon ab, die Masse geschwindigkeitsabhängig
zu definieren, sondern fasst sie als skalare lorentzinvariante Größe auf.
Es ist oft viel geschickter so vorzugehen, auch wenn in Deutschen Lehrbüchern -
historisch bedingt - oftmals von einer geschwindigkeitsabhängigen Masse
ausgegangen wird.

Mit dem so definierten Impuls erhalten wir,
\begin{align*}
p^\mu p_\mu =g_{\mu\nu} p^\mu p^\nu = m^2c^2.
\end{align*}
Für ein abgeschlossenes System ist daher $\rvec{p}^\mu$ eine
Erhaltungsgröße. Die Zeitkomponente von $\rvec{p}$ übernimmt die Rolle einer Energie,
\begin{align*}
\rvec{p}  = \begin{pmatrix}\frac{E}{c}\\ \vec{p}\end{pmatrix}.
\end{align*}
Für kleine Geschwindigkeiten $\frac{v}{c} << c$ erhalten wir so 
\begin{align*}
E &= mc^2\gamma = mc^2\left(\frac{1}{\sqrt{1-\frac{v^2}{c^2}}}\right)
= mc^2 \left( 1+ \frac{v^2}{2c^2} + \frac{3}{8}\frac{v^4}{c^4} + \ldots\right)\\
&\approx mc^2 + \frac{1}{2}mv^2. 
\end{align*}
\begin{bemn}[Bemerkungen.]
\begin{enumerate}[label=\arabic{*}.)]
  \item Der 4-er Impuls eines Elementarteilchens ist eine kovariante Größe
  $\rvec{p}^\mu$ und seine Masse der Skalar $p^\mu p_\mu = c^2m^2$.
  \item Für Photonen gilt $m=0$ und daher $\frac{E}{c} = \abs{\vec{p}}$, was
  auf die Dispersion führt $\frac{\omega}{c} = \abs{\vec{k}}$.\maphere
\end{enumerate}
\end{bemn}
\begin{bsp}
\textit{Elektron-Positron-Kollision}.
\begin{align*}
e^- + e^+ \longrightarrow 2\gamma.
\end{align*}
Es gilt die relativistische Energie und Impulserhaltung,
\begin{align*}
\rvec{p}_A + \rvec{p}_B = \rvec{p}_C + \rvec{p}_D.
\end{align*} 
Wir gehen davon aus, dass sich $e^-$ und $e^+$ in Ruhe befinden.
\begin{align*}
\Rightarrow 2mc^2 = E_c+ E_D
\end{align*}
Nach dem Stoß muss der Gesamtimpuls Null sein,
\begin{align*}
\Rightarrow &\rvec{p}_C = -\rvec{p}_D\\
\Rightarrow &E_C = E_D = mc^2.
\end{align*}
Wir sehen somit, dass die relativistische Energie-Impulserhaltung die Erzeugung
und die Annihilation von Teilchen ermöglicht.\bsphere
\end{bsp}
\begin{bsp}
Die Sonne strahlt Energie in Form von Photonen ab. Durch diesen Energieverlust
verliert sie auch stetig an Masse.\bsphere
\end{bsp}

\subsubsection{Weitere 4-er Vektoren}

Nachdem wir $4$-er Vektoren für Ort, Geschwindigkeit und Impuls eingeführt
haben, wollen wir prüfen, welche Größen sich außerdem so darstellen lassen.
Betrachten wir zunächst die Beschleunigung,
\begin{align*}
\rvec{a} &= \frac{\diffd}{\dtau}\rvec{u} = \frac{\diffd^2}{\dtau^2}\rvec{x} =
\gamma\frac{\diffd}{\dt}\gamma\begin{pmatrix}c\\ \vec{v}\end{pmatrix}
= \gamma^2 \begin{pmatrix}0\\ \vec{a}\end{pmatrix} +
\gamma\left(\frac{-2\vec{v}\vec{a}}{c^2}\right)\gamma^3\left(-\frac{1}{2}\right)
\begin{pmatrix}c \\ \vec{v}\end{pmatrix}\\
&= \gamma^2 \begin{pmatrix}0\\\vec{a}\end{pmatrix} + \gamma^4
\frac{\vec{v}\vec{a}}{c^2}\begin{pmatrix}c\\\vec{v}\end{pmatrix}.
\end{align*}
\begin{bemn}
Beschleunigung und Impuls stehen senkrecht aufeinander
\begin{align*}
a^\mu p_\mu =
\frac{1}{2}\frac{\diffd}{\dtau}p^\mu p_\mu = 0.\maphere
\end{align*}
\end{bemn}
Der Vorteil der Darstellung durch $4$er-Vektoren ist, dass eine solche Größe
``trivial'' unter Lorentztransformationen transformiert. D.h. sie transformiert
so wie $\rvec{x}$ oder $\rvec{p}$,
\begin{align*}
{A^\mu}' = \Lambda_\nu^\mu A^\nu.
\end{align*}
Wir wollen nun weitere Größen betrachten, die sich durch $4$er-Vektoren
darstellen lassen:
\begin{itemize}[label=\labelitem]
\item Wellenvektor des Lichts,
\begin{align*}
&\rvec{k}^\mu = 
\begin{pmatrix}\frac{\omega}{c}\\\vec{k}\end{pmatrix},\quad\rvec{k}_\mu = 
\begin{pmatrix}\frac{\omega}{c}\\-\vec{k}\end{pmatrix}.
\end{align*}
Somit ist auch die Phase
\begin{align*}
\Rightarrow & e^{ix^\mu k_\mu} = e^{i\left(\omega t -
\vec{k}\vec{x}\right)}
\end{align*}
eine skalare Größe. $\omega t - \vec{k}\vec{x}$ ist hierbei lorentzinvariant.
\item Elektromagnetisches Potential mit dem Skalarpotential $\phi$ und dem
Vektorpotential $\vec{A}$,
\begin{align*}
\rvec{A}^\mu = \begin{pmatrix}\phi\\\vec{A}\end{pmatrix}.
\end{align*}
\item Ableitungsoperatoren
\begin{align*}
&\rvec{\partial_\mu} = %TODO: Bold \partial?
\begin{pmatrix}
\frac{\partial_t}{c}\\
\vec{\nabla}
\end{pmatrix},\\
&\rvec{\partial^\mu} = 
\begin{pmatrix}
\frac{\partial_t}{c}\\
-\vec{\nabla}
\end{pmatrix}.
\end{align*}
\item D'Alembertoperator (Wellengleichung $\square E = 0$)
\begin{align*}
\rvec{\square} = \partial_\mu\partial^\mu = g_{\mu\nu}\partial^\mu\partial^\nu
= \frac{\partial_t^2}{c^2}-\Delta. %TODO: Bold \square?
\end{align*}
\item Stromdichte mit Ladungsstromdichte $\rho$ und Teilchenstromdichte
$\vec{j}$,
\begin{align*}
\rvec{j}^\mu =
\gamma\begin{pmatrix} \rho\\ \vec{j}\end{pmatrix}.
\end{align*}
\item Für das $E$- und $B$-Feld erhalten wir den elektromagnetischen
Feldtensor, einen Tensor 2. Stufe,
\begin{align*}
\rvec{F}^{\mu\nu} 
= \rvec{\partial^\mu} \rvec{A}^\nu - \rvec{\partial^\nu} \rvec{A}^\mu.
\end{align*}
In karthesischen Koordinaten mit Minkowski-Metrik ist die Matrixdarstellung des
Tensors gegeben durch
\begin{align*}
\rvec{F}^{\mu\nu}  = 
\begin{pmatrix}
0 & -E_x & -E_y & -E_z\\
E_x & 0 & -B_z & B_y\\
E_y & B_z & 0 & -B_x\\
E_z & -B_y & B_x & 0
\end{pmatrix}.
\end{align*}
Wenden wir eine Lorentztransformation an, so ergibt sich
\begin{align*}
(F^{\mu\nu})' &= (\partial^\mu)'(A^\nu)' - (\partial^\nu)'(A^\mu)'
= \Lambda_\alpha^\mu \Lambda_\beta^\mu \left[\partial^\alpha A^\beta -
\partial^\beta A^\alpha\right] \\ &= \Lambda_\alpha^\mu\Lambda_\beta^\nu
F^{\alpha\beta}.
\end{align*}
Die Maxwellgleichungen haben nun die Form,
\begin{align*}
&\rvec{\partial_\mu} \rvec{F}^{\mu\nu} = -\frac{4\pi}{c}\rvec{j}^\nu,\\
&\rvec{\partial^\alpha} \rvec{F}^{\mu\nu} + \rvec{\partial^\mu}
\rvec{F}^{\nu\alpha} + \rvec{\partial^\nu} \rvec{F}^{\alpha\mu} = 0,
\end{align*}
sie sind lorentzinvariant.
\end{itemize}

\subsection{Relativistischer Dopplereffekt}

Betrachte ein Inertialsystem $K'$, das sich mit der Geschwindigkeit $v$ in
$z$-Richtung relativ zu $K$ bewegt.

\begin{figure}[H]
  \centering
\begin{pspicture}(0,-1.501875)(3.92,1.541875)
\psline[linecolor=darkblue]{->}(0.44,-1.061875)(2.12,-0.261875)
\psline{->}(0.44,-1.201875)(0.44,1.098125)
\psline{->}(0.34,-1.061875)(2.58,-1.061875)
\psline{->}(2.18,-0.301875)(2.18,1.298125)
\psline{->}(2.06,-0.201875)(3.9,-0.201875)
\pscircle[linecolor=yellow](0.43,-1.071875){0.43}
\pscircle[linecolor=yellow](0.43,-1.071875){0.27}
\pscircle[linecolor=yellow](0.43,-1.071875){0.14}

\rput(1.5764062,-0.271875){\color{gdarkgray}$v$}
%\rput(2.5254688,-1.211875){\color{gdarkgray}$v_1$}
\rput(0.20859376,1.148125){\color{gdarkgray}$K$}
\rput(1.971875,1.368125){\color{gdarkgray}$K'$}

\end{pspicture} 
\caption{Transformation der Lichtausbreitung.}
\end{figure}

Der Wellenvektor $\rvec{k}^\mu = \begin{pmatrix}\omega/c\\\vec{k}\end{pmatrix}$
transformiert trivial unter Lorentztransformationen, wir erhalten daher für ${\rvec{k}^\mu}'$,
\begin{align*}
%&{k^\mu}' = \Lambda_\nu^\mu k^\nu,\\
&k_x' = k_x,\quad
k_y' = k_y,\quad
k_z' = \gamma\left(k_z - \beta\frac{\omega}{c}\right),\\
&\omega' = \gamma\left(\omega - vk_z\right)
%= \gamma\omega\left(1-\frac{v}{c}\cos\th\right),
\end{align*}
Stellen wir $\vec{k}$ und $\vec{k}'$ in Kugelkoordinaten dar mit $\omega/c =
\abs{\vec{k}}$ und $\omega'/c = \abs{\vec{k}'}$, so folgt
\begin{align*}
\omega' = \gamma\omega\left(1-v/c\cos\th\right)
\end{align*}
und dies ist gerade der relativistische Dopplereffekt.

Für eine Ausbreitung entlang des boost, d.h. $\th = 0$ ergibt sich,
\begin{align*}
&\omega' = \omega\sqrt{\frac{1-\beta}{1+\beta}},\qquad\lambda' =
\lambda\sqrt{\frac{1+\beta}{1-\beta}}.
\end{align*}
% 
% \begin{bsp}
% Lichtquelle im bewegten Bezugssystem.


% Wenden wir einen Lorentzboost in $z$-Richtung an,
% \begin{align*}
% &{k^\mu}' = \Lambda_\nu^\mu k^\nu,\\
% &k_x' = k_x,\quad
% k_y' = k_y,\quad
% k_z' = \gamma\left(k_z - \beta\frac{\omega}{c}\right),\\
% &\omega' = \gamma\left(\omega - vk_z\right)
% = \gamma\omega\left(1-\frac{v}{c}\cos\th\right),
% \end{align*}
% so erhalten wir gerade den relativistischen Dopplereffekt.\bsphere
% \end{bsp}

\subsection{Relativistische Kraftgesetze}

Für $v\to 0$ gilt das Newtonsche Gesetz
\begin{align*}
\vec{F} = m\frac{\diffd^2\vec{r}}{\dt^2}.
\end{align*}
Wir können dies auch als $4$er-Vektor schreiben,
\begin{align*}
&m\frac{\diffd^2 }{\dtau^2} \rvec{x}^\alpha= \rvec{f}^\alpha \Leftrightarrow
\frac{\diffd}{\dtau} \rvec{p}^\alpha= \rvec{f}^\alpha,
\end{align*}
wobei die Kraft im Ruhesystem zwangsläufig die Form haben muss,
\begin{align*}
&\rvec{f}_0 = \begin{pmatrix}0\\\vec{F}\end{pmatrix},
\end{align*}
mit der nicht relativistischen Kraft $\vec{F}$ für $\vec{v} = 0$.

Die Idee ist nun, in das Ruhesystem des Teilchens ($t=\tau$) zu transformieren,
dort das Newtonsche Gesetz zu verwenden und anschließend zurückzutransformieren.
\begin{align*}
\frac{\diffd^2}{\dtau^2}\begin{pmatrix}ct\\\vec{x}\end{pmatrix}
= \frac{\diffd^2}{\dt^2}\begin{pmatrix}ct\\\vec{r}\end{pmatrix}
= \begin{pmatrix}0\\\frac{\diffd^2}{\dt^2}\vec{r}\end{pmatrix}
\end{align*}
Im Laborsystem wirkt somit die $4$er-Kraft,
\begin{align*}
&f^\alpha = \Lambda_\beta^\alpha(-\vec{v})f_0^\beta,\qquad
\rvec{f} = \rvec{\Lambda}\rvec{f_0},
\end{align*}
wobei hier die Rücktransformation ins Laborsystem $\Lambda_\beta^\alpha$ von
$-\vec{v}$ abhängt.
\begin{align*}
\Rightarrow
\rvec{f} = \begin{pmatrix}f^0\\\vec{f}\end{pmatrix}
= 
\begin{pmatrix}
\gamma\frac{\vec{v}}{c}\vec{F}\\
\vec{F} + (\gamma-1)\frac{\vec{v}\vec{F}}{v^2}\vec{v}
\end{pmatrix}.
\end{align*}
$f_0$ beschreibt hier die Änderung der Energie.
\begin{bsp}
Wir können nun die relativistische Kraftgleichung der Elektrodynamik
angeben,
\begin{align*}
&\frac{\diffd}{\dtau}\rvec{p}^\alpha =
\frac{e}{c}\rvec{F}^{\alpha\beta}\rvec{u}_\beta,\\
&\frac{\diffd}{\dtau}\rvec{p}_\alpha =
\frac{e}{c}\rvec{F}_{\alpha\beta}\rvec{u}^\beta.
\end{align*}
In Komponenten geschrieben,
\begin{align*}
&\frac{\diffd}{\dt}(\gamma m\vec{v}) = e\left(\vec{E} + \frac{1}{c}
\vec{v}\times\vec{B}\right), && \text{Lorentzkraft},\\
&\frac{\diffd}{\dt}(\gamma mc^2) = e\vec{E}\vec{v}, &&\text{geleistete
Abreit}.
\end{align*}
\begin{proof}[Beweis der Darstellung.]
Im Ruhesystem gilt die Elektrostatik
\begin{align*}
\vec{F} = e\vec{E},\qquad \vec{E}\text{ elektrisches Feld}.
\end{align*}
Mit dem elektromagnetischen Feldtensor im Ruhesystem
\begin{align*}
&\rvec{F}_0^{\alpha\beta} = 
\begin{pmatrix}
0 & -E_x & -E_y & -E_z\\
E_x & 0 & -B_z & B_y\\
E_y & B_z & 0 & -B_x\\
E_z & -B_y & B_x & 0
\end{pmatrix},
\end{align*}
folgt das relativistische Kraftgesetz,
\begin{align*}
\rvec{f}_0^\alpha =
\begin{pmatrix}
0\\\vec{F}
\end{pmatrix}
= \frac{e}{c}F_0^{\alpha\beta} u_\beta,
\end{align*}
wobei im Ruhesystem,
\begin{align*}
&\rvec{u}_\beta = \rvec{u}_\beta^0 = g_{\beta\nu}\rvec{u}_0^\nu =
g_{\beta\nu}\frac{\diffd}{\dtau}\rvec{x}_0^\nu =
\begin{pmatrix}c\\\vec{0}\end{pmatrix}.
\end{align*}
Rücktransformation ins Laborsystem ergibt
\begin{align*}
f^\alpha =
\Lambda_\beta^\alpha f_0^\beta = \frac{e}{c}\Lambda_\beta^\alpha F_0^{\beta\nu}
g_{\nu\mu} u_0^\mu = 
\frac{e}{c}\underbrace{\Lambda_\beta^\alpha
F_0^{\beta\nu}\Lambda_\nu^\gamma}_{F^{\alpha\gamma}} g_{\gamma\delta}
\underbrace{\Lambda_{\mu}^\delta u_0^\mu}_{u^\delta}
= \frac{e}{c}F^{\alpha\gamma}u_\gamma,
\end{align*}
wobei $F^{\alpha\gamma}$ den Feldtensor im Laborsystem und $u^\delta$ die
Geschwindigkeit im Laborsystem beschreiben,
\begin{align*}
\rvec{u} = 
\begin{pmatrix}
c\gamma\\ -\gamma\vec{v}
\end{pmatrix},\qquad \rvec{f}
=
\frac{e}{c}
\begin{pmatrix}
c\gamma \vec{E}\vec{v}\\
c\gamma\vec{E} + \gamma \vec{v}\times\vec{B}
\end{pmatrix}
= \gamma\frac{\diffd}{\dt}m u^\alpha
= \gamma\frac{\diffd}{\dt}
\begin{pmatrix}
\frac{E}{c}\\ m\gamma\vec{v}
\end{pmatrix}.\qedhere\bsphere
\end{align*}
\end{proof}
\end{bsp}

\subsection{Variationsprinzip}
Um den Lagrangeformalismus auch im relativistischen Fall anwenden zu können,
ist es geschickt, die Trajektorien nicht mit $t$ sondern einem beliebigen
Parameter $\lambda$ zu parametrisieren. Man erhält so die relativistische
Wirkung
\begin{align*}
S_{\text{rel}} = \int \dlambda\ L,
\end{align*}
und den Lagrange
\begin{align*}
L\left(\rvec{x}^\mu(\lambda), \frac{\diffd \rvec{x}^\mu}{\dlambda} \right)
= \underbrace{-mc\sqrt{g_{\mu\nu}\frac{\diffd
x^\mu}{\dlambda}\frac{\diffd x^\nu}{\dlambda}}}_{\text{neue rel.
Energie}} - \underbrace{\frac{e}{c}A_\mu (x^\nu(\lambda))\frac{\diffd x^\mu}{\dlambda}}_{\text{Wechsel zwischen
Feldern}}.
\end{align*}
Die Wirkung ist lorentzinvariant und unabhängig von der Parametrisierung
$\lambda$.

Die Variationsrechnung liefert nun
\begin{align*}
\frac{\diffd}{\dlambda}\left(\frac{\partial L}{\partial \frac{\diffd
x^\mu}{\dlambda}}\right) - \frac{\partial L}{\partial x^\mu} = 0,
\end{align*}
wobei
\begin{align*}
&\frac{\partial L}{\partial \frac{\diffd
x^\mu}{\dlambda}} = 
- 
\frac{1}{2}\underbrace{\frac{mc}{\sqrt{g_{\mu\nu}\ldots}}}_{c}2\frac{\diffd}{\dlambda}x_\mu
- \frac{e}{c}A_\mu = - m\frac{\diffd x_\mu}{\dlambda} - \frac{e}{c} A_\mu\\
&\frac{\partial L}{\partial x^\mu} = 
-\frac{e}{c}\partial_\mu A_\nu \frac{\diffd}{\dlambda}x^\nu\\
&\frac{\diffd}{\dlambda}\left(\frac{\partial L}{\partial \frac{\diffd
x^\mu}{\dlambda}}\right) =
- m\frac{\diffd}{\dlambda} \frac{\diffd}{\dlambda} x_\mu -
\frac{c}{c}\partial_\nu A_\mu \frac{\diffd}{\dlambda}x^\nu.
\end{align*}
Einsetzen in die Euler-Lagrange-Gleichungen ergibt,
\begin{align*}
m\frac{\diffd^2}{\dlambda^2}x_\mu = \frac{e}{c}\frac{\diffd}{\dlambda}x^\nu
\underbrace{\left[\partial_\mu A_\nu - \partial_\nu A_\mu\right]}_{F_{\mu\nu}}.
\end{align*}
Setzte $\lambda=\tau$ der Eigenzeit, so folgt
\begin{align*}
\frac{\diffd}{\dtau} \rvec{p}_\mu = \frac{e}{c} \rvec{F}_{\mu\nu}\rvec{u}^\nu.
\end{align*}
\begin{bemn}
Für $\lambda=t$ und kleine Geschwindigkeiten $v<<c$ erhalten wir die klassische
Lagrangefunktion und Wirkung
\begin{align*}
&S_{\text{rel}} \longrightarrow \int\dt L\\
&L\longrightarrow - \underbrace{mc^2}_{\text{irrelevante Konstante}} +
\underbrace{\frac{m}{2}v^2}_{T} - \underbrace{e\rho}_{V} +
\underbrace{\frac{e}{c}\vec{v}\vec{A}}_{\text{Anteil für Lorentzkraft}}.\maphere
\end{align*}
\end{bemn}
\section{Systeme mit Zwangsbedingungen}

In praktischen Anwendungen treten in vielen Fällen Nebenbedingungen auf, die
den Konfigurationsraum des Problems einschränken. Die Idee ist nun, die
Lagrangefunktion so zu transformieren, dass sie sich unter diesen
Nebenbedingungen vereinfacht.

\begin{bsp}
Ein Pendel bestehe aus einer festen Stange der Länge $l$, an der eine Masse $m$
angebracht ist. Die feste Länge $l$ der Stange ergibt hier die
Zwangsbedingung.\bsphere
\end{bsp}
\begin{bsp}
Ein Zylinder, der sich in der schiefen Ebene befindet, erfährt die
Zwangsbedingungen dadurch, dass die Bewegung nur auf der Ebene stattfindet und
durch die Rollbedingung $\ds=R\dph$.\bsphere
\end{bsp}
\begin{figure}[!htbp]
  \centering
\begin{pspicture}(0,-0.88)(1.89,0.7)
\psline(0.856875,0.37)(0.176875,-0.53)
\psline[linestyle=dotted,dotsep=0.06cm](0.856875,0.37)(0.856875,-0.73)
\psdots(0.856875,0.35)
\psdots[linecolor=darkblue,dotstyle=square*](0.156875,-0.55)

\psarc(0.846875,0.32){0.51}{231.78897}{271.97495}
\psbezier(0.756875,0.01)(1.096875,0.37)(1.276875,-0.23)(1.536875,0.37)

\rput(0.13453124,-0.8){\color{gdarkgray}$m$}
\rput(0.4578125,0.14){\color{gdarkgray}$l$}
\rput(1.7214062,0.58){\color{gdarkgray}$\ph$}
\end{pspicture} \qquad
\begin{pspicture}(0.7,-0.5)(3.7,1.4)
\psline{->}(0.84,-0.4)(0.84,1.2471875)
\psline{->}(0.84,-0.4)(3.32,-0.4)
\psline(0.84,-0.4)(3.4,0.6271875)

\pscircle[linecolor=darkblue](2.75,0.8171875){0.40}
\psline[linestyle=dotted,dotsep=0.06cm]{->}(2.76,0.8271875)(2.76,-0.2728125)
\psline(2.76,0.8271875)(2.44,0.5871875)

\psdots(2.76,0.8271875)

\psarc(0.87,-0.3828125){0.87}{0.0}{22}

\psbezier(2.7,0.5071875)(3.1,1.1671875)(3.32,0.3671875)(3.54,0.9071875)

\rput(2.5185938,0.8771875){\color{gdarkgray}$R$}
\rput(3.0896876,0.0771875){\color{gdarkgray}$\vec{F}_g$}
\rput(1.513125,-0.25){\color{gdarkgray}$\alpha$}
\rput(3.5245314,1.1971875){\color{gdarkgray}$\ph$}
\end{pspicture} 

\caption{Pendel und Zylinder auf schiefer Ebene.}
\end{figure}

\subsubsection{Holonome Zwangsbedingungen}

Es gibt verschiedene Arten von Zwangsbedingungen. Die einfachste und häufigste
Art sind die \emph{holonomen Zwangsbedingungen}. Hier lässt sich die
Zwangsbedingung durch einen Satz Gleichungen, die die Koordinaten $q^i$
verknüpfen, darstellen.
\begin{align*}
f_j(q^1,\ldots,q^{3N},t) = 0,\qquad j = 1,\ldots,k.
\end{align*}
Bei $k$ Nebenbedingungen ist die Zahl der Freiheitsgrade auf $3N-k$ reduziert.
\begin{bsp}
Im Fall unseres Pendels ist
\begin{align*}
f_1(x,y) = x^2+y^2 - l^2 = 0.
\end{align*}
In Polarkoordinaten ist dadurch $r=l$ konstant.\bsphere
\end{bsp}
\begin{bsp}
Ein Teilchen gleitet reibungsfrei auf einer Oberfläche,
%TODO: Skizze
\begin{align*}
f(\vec{r}) = 0,
\end{align*}
wobei $f$ die Oberfläche beschreibt.\bsphere
\end{bsp}
\begin{bsp}
Zwei Teilchen, die durch einen starren Stab verbunden sind.
\begin{align*}
f_1(\vec{r}_1,\vec{r}_2) = \abs{\vec{r}_1-\vec{r_2}} - l = 0.
\end{align*}
\begin{figure}[!htbp]
  \centering
\begin{pspicture}(-0.1,-1)(2.52,0.9028125)
\psline{->}(0.02,-0.86)(0.02,0.7771875)
\psline{->}(0.02,-0.86)(2.5,-0.86)
\psdots(0.7,0.5171875)
\psdots(1.86,-0.1628125)
\psline[linecolor=darkblue](0.78,0.4771875)(1.78,-0.1228125)
\psline{->}(0.02,-0.86)(0.66,0.4371875)
\psline{->}(0.02,-0.86)(1.76,-0.2228125)

\rput(0.28,0.1471875){\color{gdarkgray}$\vec{r}_1$}
\rput(1.18,-0.5928125){\color{gdarkgray}$\vec{r}_2$}
\rput(0.65765625,0.8071875){\color{gdarkgray}$m_1$}
\rput(2.2,-0.2128125){\color{gdarkgray}$m_2$}
\rput(1.4009376,0.4471875){\color{gdarkgray}$l$}
\end{pspicture}

\caption{Starr verbundene Teilchen.}
\end{figure}

Eine Anwendung davon finden wir, wenn wir $O_2$-Moleküle bei tiefen
Temperaturen betrachten. Falls $k_BT << E_\text{\tiny Vibration}$, bleiben dem
Molekül nur noch die Translationsfreiheitsgrade.\bsphere
\end{bsp}

\begin{bemn}
Holonome Zwangsbedingungen definieren eine Hyperfläche im Konfigurationsraum
mit der Dimension $f=3N-k$.

Im Fall des Pendels wäre die Hyperfläche ein Kreis mit Radius $l$, die
verallgemeinerte Koordinate auf der Hyperfläche der Winkel $\ph$.\maphere
\end{bemn}

\subsubsection{Nichtholome Zwangsbedingungen}

Holonome Zwangsbedingungen sind sehr gutartig, da sie Probleme durch die
Reduktion von Freiheitsgraden vereinfachen. Es gibt jedoch auch
\emph{nichtholonome Zwangsbedinungen}, das sind Zwangsbedingungen, die sich
nicht auf die Form,
\begin{align*}
f_i(\vec{r},t) = 0,
\end{align*}
bringen lassen. In vielen Fällen verkomplizieren sie das Problem zusätzlich. Es
kann passieren, dass ein analytisch lösbares System durch nichtholonome
Zwangsbedingungen nicht mehr analytisch lösbar wird.

\begin{bsp}
Das Teilchen in der Box hat die Zwangsbedingungen,
\begin{align*}
&0\le x\le a,\\
&0\le y\le b.
\end{align*}
Die Nebenbedingungen liegen hier in Form einer Ungleichung vor; sie reduzieren
die Anzahl an Freiheitsgraden nicht.\bsphere

\begin{figure}[!htbp]
  \centering
\begin{pspicture}(0,-1.0117188)(2.7809374,0.99171877)
\psline{->}(0.28,-0.69)(0.28,0.9717187)
\psline{->}(0.26,-0.69)(2.7609375,-0.69)
\psline{->}(0.28,-0.69)(1.5609375,0.15171875)
\psdots[linecolor=darkblue](1.6209375,0.19171876)
\psline[linestyle=dotted,dotsep=0.06cm](0.28,0.63)(2.28,0.63)
\psline[linestyle=dotted,dotsep=0.06cm](2.28,-0.69)(2.28,0.63)
\psdots[dotstyle=square*](2.28,0.63)

\rput(2.2740624,-0.85828125){\color{gdarkgray}$a$}
\rput(0.08515625,0.6617187){\color{gdarkgray}$b$}
\rput(0.8409375,-0.03828125){\color{gdarkgray}$\vec{r}$}
\end{pspicture} 

\caption{Teilchen in der Box.}
\end{figure}
\end{bsp}

\begin{bsp}
Rollen eines Rades in der Ebene, mit dem Berührpunkt $(x,y)$, dem Winkel $\th$
zur Orientierung der Achse zur $x$-Richtung und dem Rollwinkel $\ph$.

\begin{figure}[!htbp]
  \centering
\begin{pspicture}(0,-1.02)(2.14,1.02)
\psline{->}(0.0,-0.18)(0.76,0.6)
\psline{->}(0.0,-0.18)(1.6,-1.0)
\psbezier[linecolor=darkblue]{->}(0.68,0.2)(1.44,0.48)(0.92,-0.92)(2.12,-0.72)
\psline{->}(0.0,-0.18)(0.98,0.18)
\psbezier(1.1883721,0.6150961)(1.26,0.23019223)(0.9595349,0.0)(0.88976747,0.40851015)(0.82,0.8170203)(1.1167442,1.0)(1.1883721,0.6150961)
\psline(1.02,0.5)(1.2,0.4)

\rput(1.7585938,0.53){\color{gdarkgray}$R$}
\psbezier[linewidth=0.02cm](1.1,0.5)(1.48,0.78)(1.26,0.18)(1.6,0.54)
\end{pspicture} 
\caption{Rollen eines Raeds in der Ebene.}
\end{figure}

Die Rollbedingung ist hier $\abs{\vec{v}} = R\dot{\ph}$, was auf
\begin{align*}
&\dot{x} = R\dot{\ph}\cos\th,\\
&\dot{y} = R\dot{\ph}\sin\th,
\end{align*}
führt. Man kann diese Nebenbedingungen nicht auf holonome Form bringen, da
$\th$ und $\ph$ vom Weg abhängen.\bsphere
\end{bsp}

Abgesehen von holonom und nicht-holonom gibt es noch weitere Möglichkeiten,
Zwangsbedingungen zu charakterisieren. \emph{Skleronome} (``starre'')
Zwangsbedingungen,
\begin{align*}
f_j(q^1,\ldots,q^{3N}) = 0,
\end{align*}
hängen nicht explizit von der Zeit ab. \emph{Rheonome} (``fließende'')
Zwangsbedingungen,
\begin{align*}
f_j(q^1,\ldots,q^{3N},t) = 0,
\end{align*}
hängen hingegen explizit von der Zeit ab.

\begin{bsp}
Eine Schiefe Ebene mit zeitabhängigem Neigungswinkel.\bsphere
\end{bsp}

\subsection{Dynamik eines Systems mit holonomen Zwangsbedingungen}

Im Folgenden wollen wir das Vorgehen zum Lösen von Problemen mit
Zwangsbedingungen erarbeiten. Auf Beweise soll dabei zunächst verzichtet
werden; die Korrektheit der Methode werden wir zum Schluss elegant mit dem
d'Almbertschen Prinzip zeigen.

\begin{enumerate}[label=\arabic{*}.)]
  \item Bestimme den Konfigurationsraum des Systems und führe verallgemeinerte
  Koordinaten ein, welche die Zwangsbedingungen erfüllen,
\begin{align*}
q^i,\qquad i = 1,\ldots,f.
\end{align*}
Die natürlichen Koordinaten lassen sich dann mit Hilfe der verallgemeinerten
ausdrücken,
\begin{align*}
\vec{r}(q^1,\ldots,q^f).
\end{align*}
\item Drücke die kinetische Energie in diesen verallgemeinerten Koordinaten aus,
\begin{align*}
T = \sum_i \frac{m_i}{2}\dvec{r}_i^2 \longrightarrow T =
\frac{1}{2}\sum\limits_{i\neq j}^f A_{ij}(q^k)\dot{q}^i\dot{q}^j.
\end{align*}
Verfahre analog mit dem Potential,
\begin{align*}
V(\vec{r}_1,\ldots,\vec{r}_N) \longrightarrow V(q^1,\ldots,q^f).
\end{align*}
\item Wir erhalten so die Lagrangefunktion in den verallgemeinerten Koordinaten,
\begin{align*}
L(q^i,\dot{q}^i,t) = T-V.
\end{align*}
Die Bewegungsgleichungen folgen anschließend aus den Euler-Lagrange-Gleichungen,
\begin{align*}
\frac{\diffd}{\dt}\frac{\partial L}{\partial \dot{q}^i} - \frac{\partial
L}{\partial q^i} = 0,\qquad i=1,\ldots,f.
\end{align*}
\end{enumerate}

\begin{bsp}
Wir wollen nun das Problem unseres Pendels unter Zwangsbedingungen lösen.
%TODO: Skizze
Sei $m$ die Masse und $\vec{r} = \begin{pmatrix}x\\y\end{pmatrix}$ die
Koordiante, sowie
\begin{align*}
&x^2+y^2 = l^2\\
&z = 0
\end{align*}
die Zwangsbedingung.
\begin{enumerate}[label=\arabic{*}.)]
  \item Durch die Einführung von Polarkoordinaten wird die Zwangsbedingung zu
  $r=l$. Wir erhalten damit
\begin{align*}
&x = l\sin\ph,\\
&y = -l\cos\ph,\\
&z = 0
\end{align*} 
wobei $\ph$ die verallgemeinerte Koordinate des Problems ist.
\item Um die kinetische Energie zu bestimmen, benötigen wir die Ableitungen,
\begin{align*}
&\dot{x} = l\dot{\ph}\cos\ph,\\
&\dot{y} = l\dot{\ph}\sin\ph.
\end{align*}
Die kinetische Energie hat somit die Form,
\begin{align*}
T = \frac{1}{2}m\left(\dot{x}^2+\dot{y}^2\right) = \frac{1}{2}m l^2\dot{\ph}^2.
\end{align*}
Das Potential ist gegeben durch,
\begin{align*}
V = mgy = -mgl\cos\ph.
\end{align*}
\item Die Lagrangefunktion hat damit in verallgemeinerten Koordinaten die
Gestalt,
\begin{align*}
L(\ph,\dot{\ph}) = \frac{1}{2}ml^2\dot{\ph}^2+mgl\cos\ph.
\end{align*}
Die Bewegungsgleichungen erhalten wir somit durch
\begin{align*}
&\ddot{\ph} = -\frac{g}{l}\sin\ph.
\end{align*}
Für $\dot{\ph}=0$ und $\ph=0,\pi,2\pi$ erhalten wir stationäre Lösungen.
Für $\dot{\ph}$ klein ergibt sich in $\ph=0,2\pi,4\pi,\ldots$ eine harmonische
Schwingung, während das Gleichgewicht in $\pi,3\pi,\ldots$ instabil ist.\bsphere
%TODO: Phasendiagramm.
\end{enumerate}
\end{bsp}

\begin{bemn}
In einem System mit holonomen Zwangsbedingungen müssen zusätzliche Kräfte
wirken, die die Teilchen auf die Hyperfläche zwängen. Die Newtonschen Gleichungen haben die
Form,
\begin{align*}
m_i \ddvec{r}_i = -\nabla_i V + \vec{Z}_i,
\end{align*}
wobei $\vec{Z}_i$ die zusätzliche Kraft beschreibt.\maphere
\end{bemn}

\begin{defnn}
Die Größen,
\begin{align*}
\vec{Z}_i = m_i\ddvec{r}_i - \vec{F}_i = m_i\ddvec{r}_i + \nabla_i V,\qquad i =
1,\ldots,N
\end{align*}
heißen \emph{Zwangskräfte}. Diese Kräfte sind apriori unbekannt, können aber
aus der Lösung des Problems berechnet werden.\fishhere
\end{defnn}

\begin{bsp}
%TODO: Pendel Ruhelage
Durchquert das Pendel die Ruhelage, so ist
\begin{align*}
\ddvec{r} = 0 \Rightarrow \vec{Z} = -\vec{F}_g.\bsphere
\end{align*}
\end{bsp}

\begin{defnn}
Eine \emph{virtuelle Verrückung} $\delta \vec{r}_i$ oder $\delta q^i$ ist eine
infinitesimale Änderung der Lagekoordinaten, welche mit den Zwangsbedingungen
verträglich ist.\fishhere
\end{defnn}
%TODO: Bild Virtuelle Verrückung

\subsubsection{Prinzip von d'Alembert}
\begin{propn}[Prinzip von d'Alembert]
In einem System mit Zwangsbedingungen erfüllt die Bewegungsgleichung,
\begin{align*}
\sum\limits_{i=1}^N \left(m_i\ddvec{r}_i - \vec{F}_i \right)\delta\vec{r}_i = 0,
\end{align*}
für alle virtuellen Verrückungen $\delta\vec{r}_i$.\fishhere
\end{propn}

Diese Aussage ist äquivalent dazu, dass Zwangskräfte unter einer virtuellen
Verrückung keine Arbeit leisten,
\begin{align*}
\delta A = \vec{Z}_i \cdot \delta\vec{r}_i = \left(m_i\ddvec{r}_i -
\vec{F}_i\right) \delta\vec{r}_i = 0.
\end{align*}
In einigen Lehrbüchern wird das Prinzip von d'Alembert als Axiom festgelegt,
für alle Standardprobleme der klassischen Mechanik leisten die Zwangskräfte
jedoch keine Arbeit.
%\begin{bsp}
%TODO: Zwangskraft = Normalkraft
%\end{bsp}

\begin{bemn}
Im statischen Fall reduziert sich das Prinzip von d'Almebert auf eine Bedingung
für die Gleichgewichtslage,
\begin{align*}
\sum_i \vec{F}_i\delta\vec{r}_i = 0.\maphere
\end{align*}
\end{bemn}

\begin{bsp}
Betrachten wir einen Flaschenzug.
%TODO: Flaschenzug
Es gilt $\delta h_1 = -2 \delta h_2$ und
\begin{align*}
&\vec{F}_1 \delta\vec{h}_1 + \vec{F}_2\delta\vec{h}_2 = 0\\
\Rightarrow & m_1 g \delta h_1 - m_2 g \frac{\delta h_1}{2} = 0 \Leftrightarrow
 \delta h_1 g \left(m_1 - \frac{m_2}{2}\right) = 0.
 \end{align*}
Für ein Gleichgewicht muss gelten
\begin{align*}
&\frac{m_2}{m_1} = 2.\bsphere
\end{align*}
\end{bsp}

Wir wollen nun zeigen, dass aus dem Prinzip von d'Alembert das Hamiltonsche
Variationsprinzip folgt und somit unser Lösungsschema für holonome
Zwangsbedingungen äquivalent zum Prinzip von d'Alembert ist.

\begin{proof}[Beweis der Äquivalenz.]
Wir führen dazu geeignete Koordinaten ein, die die holonomen Zwangsbedingungen
erfüllen,
\begin{align*}
&\vec{r}_i(q^1,\ldots,q^f,t).
\end{align*}
Eine virtuelle Verrückung hat somit die Form,
\begin{align*}
&\delta\vec{r}_i = \sum\limits_{j=1}^f \frac{\partial \vec{r}_i}{\partial
q^j}\delta q^j.
\end{align*}
Betrachten wir nun das Prinzip von d'Alembert
\begin{align*}
0 = \sum_i \left(m_i \ddvec{r}_i - \vec{F}_i\right)\delta\vec{r}_i,
\end{align*}
so erhalten wir für die Summanden:
\begin{itemize}
  \item 
\begin{align*}
\sum_i \vec{F}_i \delta \vec{r}_i = \sum_{i,j} \vec{F}_i \frac{\partial
\vec{r}_i}{\partial q^j}\delta q^j
= \sum_j \underbrace{\left(\sum_i \vec{F}_i \frac{\partial\vec{r}_i}{\partial
q^j}\right)}_{:=\vec{Q}_j}\delta q^j.
\end{align*}
Wobei wir $\vec{Q}_j$ als verallgemeinerte Kraft bezeichnen. Es gilt
\begin{align*}
\vec{Q}_j = \sum_i \vec{F}_i \frac{\partial\vec{r}_i}{\partial q^j}
= - \sum_i \nabla_i V \frac{\partial\vec{r}_i}{\partial q^j}
= -\frac{\diffd}{\diffd q^j}V(q^1,\ldots,q^f).
\end{align*}
\item
\begin{align*}
\sum_i m_i\ddvec{r}_i \delta\vec{r}_i  &= \sum_i\sum_j m_i\ddvec{r}_i 
\frac{\partial \vec{r}_i}{\partial q^j}\delta q^j = 
\sum_j \left(\sum_i m_i\ddvec{r}_i 
\frac{\partial \vec{r}_i}{\partial q^j}\right)\delta q^j\\
&= \sum_j \delta q^j \left[\sum_i \frac{\diffd}{\dt}\left( m_i\dvec{r}_i 
\frac{\partial \vec{r}_i}{\partial q^j}\right) -
m_i\dvec{r}_i\frac{\diffd}{\dt}\frac{\partial \vec{r}_i}{\partial q^j}\right]
\end{align*}
Nun ist
\begin{align*}
\frac{\partial
\dvec{r}_i}{\partial\dot{q}^j} = \frac{\partial}{\partial \dot{q}^j} \left[
\sum\limits_k \frac{\partial \vec{r}_i}{\partial q^k}\dot{q}^k +
\frac{\partial}{\partial t}\vec{r}_i \right] = \frac{\partial
\vec{r}_i}{\partial q^j}
\end{align*}
und daher gilt,
\begin{align*}
\sum_i m_i\ddvec{r}_i \delta\vec{r}_i = \sum_j \delta q^j \left[\sum_i
\frac{\diffd}{\dt}\underbrace{\left( m_i\dvec{r}_i \frac{\partial
\dvec{r}_i}{\partial \dot{q}^j}\right)}_{=\frac{\partial T}{\partial \dot{q}^j}}
- \underbrace{m_i\dvec{r}_i \frac{\partial \dvec{r}_i}{\partial
q^j}}_{\frac{\partial T}{\partial q^j}}\right]
\end{align*}
\end{itemize}
Zusammengefasst gilt also,
\begin{align*}
0 = \sum_j \left[\frac{\diffd}{\dt}\frac{\partial T}{\partial
\dot{q}^j}-\frac{\partial T}{\partial q^j} + \frac{\partial V}{\partial
q^j} \right]\delta q^j,
\end{align*}
und da das Potential nicht geschwindigkeitsabhängig ist, erhalten wir
\begin{align*}
0 = \sum_j \left[\frac{\diffd}{\dt}\frac{\partial L}{\partial
\dot{q}^j}-\frac{\partial L}{\partial q^j}\right]\delta q^j,
\end{align*}
für alle Variationen $\delta q^j$. Da die Variationen unabhängig sind, ist
somit der Ausdruck in der Klammer Null und wir erhalten die
Euler-Lagrange-Gleichungen.\qedhere
\end{proof}

Das Prinzip vn d'Alembert ist also äquivalent zum Hamiltonischen Prinzip
angewandt auf die Lagrange Funktion mit verallgemeinerten Koordianten.

\begin{bemn}
Ein freies Teilchen, das sich unter Zwangsbedingungen auf einer Fläche bewegt,
bewegt sich auf einer Geodäte, d.h. entlang der kürzesten Verbindungen zwischen
zwei Punkten.\maphere
\end{bemn}
\begin{bsp}
Ein käftefreies Teilchen das sich durch Zwangsbedingungen auf einer
Kugeloberfläche bewegt, bewegt sich entlang eines Großkreises.
\begin{proof}
Für ein allgemeines kräftefreies Teilchen ist der Lagrange
\begin{align*}
L = T = \frac{1}{2}m\dvec{r}^2.
\end{align*}
Die Euler-Lagrange-Gleichungen haben die Form,
\begin{align*}
\frac{\diffd}{\dt}\frac{\partial T}{\partial \dot{q}^i} - \frac{\partial
T}{\partial q^i}  = 0.
\end{align*}
Für eine Geodäte ist der Ausdruck,
\begin{align*}
\int\dt \sqrt{\dvec{r}^2} = \int\dt \sqrt{\frac{2}{m}T}
\end{align*}
extremal. Ableiten dieses Ausdrucks ergibt,
\begin{align*}
\frac{1}{2}\frac{\diffd}{\dt}\frac{\partial T}{\partial
\dot{q}^i}\frac{1}{\sqrt{T}} - \frac{1}{2}\frac{\partial T}{\partial
q^i}\frac{1}{\sqrt{T}} = 0.
\end{align*}
Die Energie ist erhalten, also ist $T=\const$ und wir erhalten die
Euler-Lagrange-Gleichung,
\begin{align*}
\frac{\diffd}{\dt}\frac{\partial T}{\partial \dot{q}^i} - \frac{\partial
T}{\partial q^i}  = 0.\qedhere\bsphere
\end{align*}
\end{proof}
\end{bsp}

\subsection{Mathematische Struktur}

Wir wollen noch kurz auf die mathematische Struktur der Probleme mit holonomen
Zwangsbedingungen eingehen.

Die Zwangsbedingungen
\begin{align*}
f_i(\vec{r}_1,\ldots,\vec{r}_N) = 0,\qquad i = 1,\ldots,k,
\end{align*}
definieren eine \emph{Mannigfaltigkeit} $M\subseteq\R^{3N}$, durch $M =
\bigcap_i f_i^{-1}(0)$ der Dimension $3N-k$.

Die verallgeminerten Koordinaten $q^i$, $i=1,\ldots,f$ stellen \emph{Karten} zu
dieser Mannigfaltigkeit $M$ dar.
%TODO: Bild Karte

An jedem Punkt $p$ der Mannigfaltigkeit haben wir eine Tangentialfläche $TM_p$.
Diese bildet im Gegensatz zur Mannigfaltigkeit einen Vektorraum. Die
Tangentialvektoren sind alle möglichen $\dot{q}^i$ in $p$, d.h. $TM_p$ ist
die Menge der Geschwindigkeitsvektoren der Kurven in $M$ durch $p$.

Für eine Mannigfaltigkeit, die sich in den $\R^n$ einbetten lässt, kann ein
Skalarprodukt definiert werden. Eine solche Mannigfaltigkeit heißt
\emph{Riemannsche Mannigfaltigkeit}. Durch holonome Zwangsbedingungen werden
also immer Riemannsche Mannigfaltigkeiten definiert. Das Skalarprodukt auf $M$ ist die
\emph{kinetische Energie} gegeben durch,
\begin{align*}
T: TM_p\times TM_p \to \R,\quad \dot{q}^i,\dot{q}^j \mapsto
\lin{\dot{q}^i,\dot{q}^j} = \sum\limits_{i,j} a_{ij}(q^k)\dot{q}^i\dot{q}^j,
\end{align*}
wobei $a_{ij}(q^k)$ einen metrischen Tensor bezeichnet.
Die \emph{potentielle Energie} ist eine rellwerte Funktion auf $M$,
\begin{align*}
V: M\to \R,\quad q^i \mapsto V(q^i).
\end{align*}
Die Vereinigung aller Tangentialräume an $M$ heißt \emph{Tangentialbündel},
\begin{align*}
TM = \bigcup\limits_{p\in M} TM_p.
\end{align*}
Die Lagrangefunktion ist somit eine skalare Funktion auf dem Tangentialbündel,
\begin{align*}
L : TM\to\R,\quad \dot{q}^i,q^i  \mapsto L(\dot{q}^i,q^i).
\end{align*} 
Die verallgemeinerten Impulse
\begin{align*}
p_j = \frac{\partial L}{\partial \dot{q}^j}
\end{align*}
sind Elemente aus dem \emph{Dualraum} von $TM_p$, d.h. eine Abbildung von
$TM_p$ in die reellen Zahlen,
\begin{align*}
p_i : TM_p \to \R,\quad q^i\mapsto \sum_i p_i q^i,\qquad p_i\in TM_p^*.
\end{align*}
Analog ist auch $\frac{\partial L}{\partial q^i}$ ein Element aus dem Dualraum.

Eine virtuelle Verrückung $\delta q^i$ ist ein Vektor im Tangentialraum, d.h.
ein Geschwindigkeitsvektor einer Kurve $q^i(t)$ am Punkt $p$.

Betrachten wir nun eine Variablentransformation,
\begin{align*}
q^i \mapsto q^i(\xi^1, \ldots, \xi^f),
\end{align*}
so transformieren die $\dot{q}^i$ \emph{kontravariant}
\begin{align*}
\dot{q}^i = \sum_j \frac{\partial q^i}{\partial \xi^j}\dot{\xi}^j,
\end{align*}
und die verallgemeinerten Impulse \emph{kovariant}
\begin{align*}
p_i = \frac{\partial L}{\partial \dot{q}^i} = \sum_j \frac{\partial
\eta_j}{\partial \dot{q}^i}\frac{\partial L}{\partial \eta^j}.
\end{align*}
Beim Prinzip von d'Alembert muss beachtet werden, dass 
Vektoren und Impulse unterschiedlich transformieren.
Die Euler-Lagrange-Gleichung ist jedoch vollständig kontravariant. Beim
Übergang von Geschwindigkeiten zu Impulsen müssen wir den metrischen Tensor
berücksichtigen,
\begin{align*}
p_i = \sum\limits_j a_ij(q^i)\dot{q}^j.
\end{align*}

\begin{bemn}
Ein kräftefreies Teilchen, das sich auf einer Hyperfläche bewegt, folgt einer
Geodäte. Die Distanz wird dabei von der durch das Skalarprodukt (kinetische
Energie) induzierten Metrik gemessen.\maphere
\end{bemn}

\begin{bsp}
Die Bewegung auf einer rotationssymmetrischen Fläche in Zylinderkoordinaten
($r,\ph,z$) wird durch die holonome Zwangsbedingung $z=f(r)$ beschrieben. Die
kinetische Energie ist
\begin{align*}
T = \frac{m}{2}\left[\dot{r}^2 + r^2\dot{\ph}^2 + (f'(r)\dot{r})^2\right] =
L(r,\ph,\dot{r},\dot{\ph}).
\end{align*}
Die Drehimpulserhaltung liefert $p_\ph = r^2\dot{\ph} = \const =
r\abs{v}\sin\alpha$. Die Energieerhaltung $T=\frac{1}{2}\abs{v}^2 = \const
\Rightarrow \abs{v}  = \const$.
\begin{align*}
\Rightarrow &r \sin\alpha = \const\\
\Rightarrow &r = \frac{p_\rho}{\abs{v} \sin\alpha} > \const. 
\end{align*}
Die Bewegung ist auf ein Band eingeschränkt.\bsphere
%TODO: Bild des Bandes
\end{bsp}
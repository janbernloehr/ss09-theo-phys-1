\section{Starre Körper}

\subsection{Bewegung im beschleunigten Bezugssystem und Scheinkräfte}

Ein Punktteilchen wird in einem Inertialsystem vollständig durch den Lagrange
\begin{align*}
L(\vec{q},\dvec{q}) = T-V
\end{align*}
beschrieben und folgt der Newtonschen Gleichung,
\begin{align*}
m\ddvec{q} = -\nabla V = \vec{F}.
\end{align*}

In einem beschleunigten Bezugssystem treten zusätzliche Terme auf, die wir als
\emph{Scheinkräfte} bezeichnen.

Im Inertialsystem $K$ habe das Teilchen die Koordinate $\vec{q}$, im bewegten
System $K'$ die Koordinate $\vec{Q}$. Der Übergang von $K$ in $K'$ findet durch
Translationen und Rotationen statt,
\begin{align*}
\vec{q} = B\vec{Q} + \vec{r}(t),
\end{align*}
wobei $B\in\SO(3)$ eine Rotationsmatrix und $\vec{r}(t)$ den Translationsvektor
beschreibt.

\begin{figure}[htbp]
\centering
\begin{pspicture}(0,-1.215)(3.125,1.25)
\psline{->}(0.3,-1.21)(0.3,1.09)
\psline{->}(0.1,-1.01)(3.12,-1.01)
\psline{->}(1.98,0.01)(2.46,0.93)
\psline{->}(1.94,0.19)(2.94,-0.33)
\psline{->}(0.3,-1.01)(2.0,0.11)
\psdots(1.76,0.87)
\psline[linecolor=yellow]{->}(2.04,0.15)(1.82,0.77)
\psline[linecolor=darkblue]{->}(0.3,-1.01)(1.7,0.75)

\rput(1.52,-0.4){\color{gdarkgray}$r$}
\rput(0.99,0.295){\color{gdarkgray}$q$}
\rput(2.08,0.655){\color{gdarkgray}$Q$}
\rput(0.14,1.075){\color{gdarkgray}$K$}
\rput(2.62,0.895){\color{gdarkgray}$K'$}
\end{pspicture} 
\caption{Intertialsystem $K$ und bewegtes System $K'$}
\end{figure}

Die Geschwindigkeit $\vec{v} = \frac{\diffd}{\dt}\vec{q}$ lässt
sich somit schreiben als
\begin{align*}
\dvec{q} = \dot{B}\vec{Q} + B\dvec{Q} + \dvec{r}(t).
\end{align*}
Wir betrachten zunächst zwei Spezialfälle:
\begin{enumerate}[label=\arabic{*})]
  \item \textit{Translationsbewegung}.
\begin{align*}
&B = \Id,\\
\Rightarrow & \dvec{q} = \dvec{Q} + \dvec{r}(t).
\end{align*}
Der Lagrange hat daher im bewegten System die Form,
\begin{align*}
L(\vec{Q},\dvec{Q}) &= \frac{m}{2}\left(\dvec{Q}+\dvec{r}\right)^2 - V =
\frac{m}{2}\dvec{Q}^2 + m\dvec{Q}\dvec{r} + \frac{m}{2}\dvec{r}^2-V\\
&= \frac{m}{2}\dvec{Q}^2 - m\vec{Q}\ddvec{r} - V +
\underbrace{\frac{m}{2}\dvec{r}^2 +
\frac{\diffd}{\dt}\left(m\vec{Q}\dvec{r}\right)}_{(*)}.
\end{align*}
(*) ist für die Bewegung irrelevant, da sie den Lagrange nur um eine Konstante
und eine totale Ableitung ändert.
\begin{align*}
L(\vec{Q},\dvec{Q}) = \frac{m}{2}\dvec{Q}^2 -
\left(V+m\vec{Q}\vec{a}_s(t)\right),
\end{align*}
wobei $\vec{a}_s(t)$ die Beschleunigung des bewegten Systems beschreibt.
Mittels Euler-Lagrange-Gleichungen erhalten wir so,
\begin{align*}
m\ddvec{Q} = -\frac{\partial V}{\partial \vec{Q}} -
\underbrace{m\vec{a}_s(t)}_{\text{Scheinkraft durch beschl. Bewegung des
Systems}}
\end{align*}
Ein Beobachter im bewegten Bezugssystem nimmt die Scheinkraft als Kraft wahr,
die zusätzlich auf den Kröper wirkt.
\item \textit{Rotationen}.
\begin{align*}
&\vec{r} = 0,\\
\Rightarrow & \dvec{q} = \dot{B}\vec{Q} + B\dvec{Q}.
\end{align*}
Für beliebige Rotationen lässt sich $\dot{B}\vec{Q}$ schreiben als
\begin{align*}
\dot{B}\vec{Q} = \dot{B}B^{-1}\vec{q} = \vec{\omega}\times\vec{q},\tag{**}
\end{align*}
wobei $\vec{\omega}$ die Winkelgeschwindigkeit darstellt.
\begin{bemn}
$\vec{\omega}$ ist die Winkelgeschwindigkeit im Ruhesystem,
$\vec{\Omega}=B^{-1}\vec{\omega}$ ist die Winkelgeschwindigkeit im
beschleunigten System,
\begin{align*}
B^{-1}\dot{B}\vec{Q} = B^{-1}(\vec{\omega}\times \vec{q}) =
(B^{-1}\vec{\omega}\times B^{-1}\vec{q}) = \vec{\Omega}\times\vec{Q}.\maphere
\end{align*}
\end{bemn}
\begin{proof}[Beweis der Darstellung (**).]
Es gilt $\dot{B}\vec{Q} = \dot{B}B^{-1}\vec{q}$. $B$ ist eine Rotation, d.h.
\begin{align*}
B B^\top = B^\top B = \Id \Leftrightarrow B^\top = B^{-1}.
\end{align*}
Die Ableitung der Identität verschwindet, d.h.
\begin{align*}
&\frac{\diffd}{\dt}(B B^{-1}) = \dot{B}B^\top  + B\dot{B}^\top = 0,\\
\Leftrightarrow & \dot{B}B^\top  + \left(\dot{B}B^{-1}\right)^\top = 0,
\end{align*}
und daher ist $\dot{B} B^\top$ eine schiefe (antisymmetrische) Matrix. Sie hat
die Form,
\begin{align*}
&\dot{B}B^\top = 
\begin{pmatrix}
0 & -\omega_3 & \omega_2\\
\omega_3 & 0 & -\omega_1\\
-\omega_2 & \omega_1 & 0
\end{pmatrix},\qquad \omega_1,\omega_2,\omega_3\in\R.\\
&\dot{B}B^{-1}\vec{q} = 
\begin{pmatrix}
\omega_2 q_3 - \omega_3 q_2\\
\omega_3 q_1 - \omega_1 q_3\\
\omega_1 q_2 - \omega_2 q_1
\end{pmatrix}
= \vec{\omega}\times \vec{q},\qquad \vec{\omega} =
\begin{pmatrix}\omega_1\\\omega_2\\\omega_3\end{pmatrix}.\qedhere
\end{align*}
\end{proof}
Für die Geschwindigkeit erhalten wir also,
\begin{align*}
\dvec{q} = \vec{\omega}\times\vec{q} + B\dvec{Q}.
\end{align*}
Damit können wir die kinetische Energie angeben,
\begin{align*}
T &= \frac{1}{2}m\left(\vec{\omega}\times \vec{q} + B\dvec{Q} \right)^2
= \frac{m}{2} \left(\left(B\vec{\Omega}\times B\vec{Q}\right) + B\dvec{Q}
\right)^2\\
&= \frac{m}{2} \left(B\left(\vec{\Omega}\times\vec{Q} +
\dvec{Q} \right)\right)^2 = 
\frac{m}{2} \lin{B\left(\vec{\Omega}\times\vec{Q} +
\dvec{Q} \right),B\left(\vec{\Omega}\times\vec{Q} +
\dvec{Q} \right)}
\\
&= \frac{m}{2} \left(\vec{\Omega}\times\vec{Q} + \dvec{Q}\right)^2
= \frac{m}{2} \dvec{Q}^2 + m\dvec{Q}\left(\vec{\Omega}\times\vec{Q}\right) +
\frac{m}{2}\left(\vec{\Omega}\times\vec{Q}\right)^2.
\end{align*}
Einsetzen in die Lagrangefunktion ergibt,
\begin{align*}
L(\vec{Q},\dvec{Q}) = \frac{m}{2}\dvec{Q}^2 + m\dvec{Q}\left(\vec{\Omega}\times\vec{Q}\right) +
\frac{m}{2}\left(\vec{\Omega}\times\vec{Q}\right)^2 - V,
\end{align*}
die Euler-Lagrange-Gleichungen liefern,
\begin{align*}
m\ddvec{Q} = -\frac{\partial V}{\partial \vec{Q}} +
\underbrace{m\vec{\Omega}\times \left(
\vec{Q}\times\vec{\Omega}\right)}_{\text{Zentrifugalkraft}} +
\underbrace{2m\dvec{Q}\times\vec{\Omega}}_{\text{Corioliskraft}} +
m\vec{Q}\times\dvec{\Omega}.
\end{align*}
\end{enumerate}

\begin{bsp}
Die durch die Erddrehung ``wirkende'' Zentrifugalkraft ändert das
Gravitationsfeld der Erde.
\begin{figure}[htbp]
\centering
\begin{pspicture}(0,-1.51)(6.76,1.51)
\pscircle[linecolor=darkblue](1.5,-0.01){0.9}
\psline{->}(1.5,1.49)(1.5,0.97)
\psline{->}(3.0,-0.01)(2.46,-0.01)
\psline{->}(1.5,-1.49)(1.5,-0.97)
\psline{->}(0.0,-0.01)(0.52,-0.01)
\psline{->}(2.62,1.11)(2.2,0.69)
\psline{->}(2.62,-1.13)(2.2,-0.71)
\psline{->}(0.4,-1.11)(0.82,-0.69)
\psline{->}(0.38,1.09)(0.8,0.67)

\pscircle[linecolor=darkblue](5.4,-0.01){0.9}
\psline{->}(5.4,1.49)(5.4,0.97)
\psline{->}(6.74,-0.01)(6.36,-0.01)
\psline{->}(5.4,-1.49)(5.4,-0.97)
\psline{->}(4.04,-0.01)(4.42,-0.01)
\psline{->}(6.4,0.99)(6.1,0.69)
\psline{->}(6.4,-0.99)(6.1,-0.71)
\psline{->}(4.38,-0.97)(4.72,-0.69)
\psline{->}(4.38,0.99)(4.7,0.67)
\end{pspicture} 
\caption{Wirkung der Zentrifugalkraft auf die Erdanziehung (Übertrieben
dargestllt)}
\end{figure}

Der Einfluss ist jedoch sehr klein. Die Verzerrung, die entsteht, da die Erde
keine perfekte Kugel ist und eine inhomogene Dichteverteilung hat, ist viel
größer.\bsphere
\end{bsp}
\begin{bsp}
Das Foucault'sche Pendel ist ein Nachweis für die Corioliskraft.\bsphere
\end{bsp}

\subsection{Starre Körper}
Ein starrer Körper ist eine Idealisierung eines festen Körpers, wobei alle
Abstände zwischen den Teilchen fixiert sind,
\begin{align*}
\abs{\vec{q}_i - \vec{q}_j} = q_{ij}  = \const. 
\end{align*}
Dies ist ein Spezialfall einer holonomen Zwangsbedingung.

\begin{bemn}
Die Bewegung des starren Körpers ist ausgezeichnet durch die Lage des Körpers
(Rotation um einen Punkt) und die Bewegung dieses Punktes. Die
Konfigurationsmannigfaltigkeit des Körpers ist
\begin{align*}
\R^3\times \SO(3),
\end{align*}
eine 6-dimensionale Hyperfläche.

Wir führen daher zwei Koordinatensysteme ein. Das Laborsystem $(K,\vec{q})$ und
das körperfeste System $(K',\vec{Q})$ im Ursprung $\vec{r}(t)$.\maphere
\begin{figure}
\centering
\begin{pspicture}(0,-1.215)(3.125,1.25)
\psline{->}(0.3,-1.21)(0.3,1.09)
\psline{->}(0.1,-1.01)(3.12,-1.01)
\psline{->}(1.98,0.01)(2.46,0.93)
\psline{->}(1.94,0.19)(2.94,-0.33)
\psline{->}(0.3,-1.01)(2.0,0.11)
\psdots(1.76,0.87)
\psline[linecolor=yellow]{->}(2.04,0.15)(1.82,0.77)
\psline[linecolor=darkblue]{->}(0.3,-1.01)(1.7,0.75)

\rput(1.52,-0.4){\color{gdarkgray}$r$}
\rput(0.99,0.295){\color{gdarkgray}$q$}
\rput(2.08,0.655){\color{gdarkgray}$Q$}
\rput(0.14,1.075){\color{gdarkgray}$K$}
\rput(2.62,0.895){\color{gdarkgray}$K'$}
\end{pspicture} 
\caption{Laborsystem und körperfestes System.}
\end{figure}
\end{bemn} 

\subsubsection{Erhaltungsgrößen}

Für einen freien starren Körper bewegt sich der Schwerpunkt gleichförmig. Die
Bewegung des freien Körpers um den Schwerpunkt hat den Drehimpuls und die
Energie als Erhaltungsgröße.

\begin{tabular}[h]{l|l|ll}
 & Laborsystem & Körperfestsystem\\\hline
 Position des Teilchens $i$ & $\vec{q}_i$ & $\vec{Q}_i$ & $\vec{q}_i =
 B\vec{Q}_i +\vec{r}$\\ Winkelgeschwindigkeit & $\vec{\omega}$ & $\vec{\Omega}$ & $\vec{\omega} =
 B\vec{\Omega}$\\
 Drehimpuls & $\vec{m}$ & $\vec{M}$ & $\vec{m} = B\vec{M}$
\end{tabular}

Die Geschwindigkeit ist gegeben durch,
\begin{align*}
\vec{v}_i = \vec{\omega}\times \left(\vec{q}_i - \vec{r}\right) + \dvec{r}.
\end{align*}
\begin{proof}[Beweis der Geschwindigkeitsdarstellung.]
$\vec{v}_i = \dot{B}\vec{Q}_i + \dvec{r} + B\dvec{Q}_i$. Im starren Körper sind
jedoch alle Teilchen fixiert, d.h. $B\dvec{Q}_i = 0$.\qedhere
\end{proof}
Einsetzen in die kinetische Energie ergibt,
\begin{align*}
T = \sum_i \frac{m_i}{2}\vec{v}_i^2 = 
\underbrace{\sum_i \frac{m_i}{2}\dvec{r}^2}_{\frac{1}{2}M\dvec{r}^2} +
\sum_i m_i \dvec{r} \left(\vec{\omega}\times \left(\vec{q}_i
-\vec{r} \right) \right) + \frac{1}{2}\sum_i m_i
\left[\vec{\omega}\times \left(\vec{q}_i - \vec{r}\right) \right]^2.
\end{align*}
Betrachten wir die $2$-Fälle,
\begin{enumerate}[label=(\roman{*})]
  \item Bewegung im Schwerpunkt, dann ist $\sum_i m_i(\vec{q}_i -\vec{r}) = 0$,
  \item Bewegung mit $\vec{r}$ fixiert, dann ist $\dvec{r} = 0$.
\end{enumerate}
Somit lässt sich die kinetische Energie schreiben als,
\begin{align*}
T = \underbrace{\frac{1}{2}M\dvec{r}^2}_{\text{Schwerpunktsenergie}}+
\underbrace{\frac{1}{2}\sum_i m_i \left(\vec{\Omega}\times \vec{Q}_i
\right)^2}_{\text{Rotationsenergie}}.
\end{align*}
Die Rotationsenergie lässt sich schreiben als,
\begin{align*}
T_R &= \frac{1}{2}\sum_i m_i \left[\vec{\Omega}^2\vec{Q}_i^2 -
\left(\vec{\Omega}\vec{Q}_i\right)^2 \right] = \frac{1}{2}
\sum\limits_{j,k=1}^3 \Omega_j\Omega_k \underbrace{\sum_i m_i\left[ \vec{Q}_i^2
\delta^{jk} - Q_i^k Q_i^j \right]}_{=I^{jk}}\\
&= \frac{1}{2}\sum\limits_{j,k=1}^3 \Omega_i \Omega_k I^{jk}
= \frac{1}{2}\vec{\Omega} I \vec{\Omega}.
\end{align*}
Die Elemente des Trägheitstensors $I$ haben die Form,
\begin{align*}
I^{jk} = \sum\limits_{i}m_i \left[\delta^{jk} Q_i^2 - Q_i^jQ_i^k \right]
= \int\dr^3 \rho(\vec{r}) \left[\vec{r}^2 - r^jr^k\right],
\end{align*}
wobei $\rho(\vec{r})$ die Massendichte des starren Körper bezeichnet,
\begin{align*}
\rho(\vec{r}) = \sum_ i m_i \delta(\vec{r}-\vec{Q}_i).
\end{align*}
\begin{bemn}
Die totale Masse ist gegeben durch,
\begin{align*}
M  = \int\dr^3 \rho(\vec{r}),
\end{align*}
der Schwerpunkt durch,
\begin{align*}
\vec{r}_s = \frac{1}{M}\int\dr^3 \rho(\vec{r})\vec{r}.\maphere
\end{align*}
\end{bemn}

Der Trägheitstensor ist symmetrisch, $I^{jk} = I^{kj}$, und positiv. Er kann
durch geeignete Wahl des Körperfesten Systems auf Diagonalgestalt gebracht
werden,
\begin{align*}
I = \begin{pmatrix}
    I_1 & 0 & 0\\
    0 & I_2 & 0\\
    0 & 0 & I_3
    \end{pmatrix}.
\end{align*}
$I_1,I_2,I_3$ heißen \emph{Hauptträgheitsmomente}, die zugehörigen
Basisvektoren $\vec{e}_k$ \emph{Hauptträgheitsachsen}. Die Rotationsenergie
nimmt dann folgende Form an
\begin{align*}
T_R = \frac{1}{2}\sum\limits_{i=1}^3 I_i \Omega_i^2.
\end{align*}

\begin{pspicture}(0,-1.56)(3.92,1.56)
\psframe(1.98,-0.22)(0.2,-1.0)
\psline{->}(1.74,0.28)(1.74,1.26)
\psline{->}(2.56,-0.06)(3.48,-0.06)
\psline{->}(1.04,-0.56)(0.38,-1.2)

\psline(0.2,-0.22)(1.5,0.74)(3.12,0.74)
\psline(1.98,-0.22)(3.12,0.74)(3.12,0.08)(1.98,-1.0)

\rput(1.98,1.365){\color{gdarkgray}$\vec{e}_3$}
\rput(3.71,-0.035){\color{gdarkgray}$\vec{e}_2$}
\rput(0.29,-1.395){\color{gdarkgray}$\vec{e}_1$}

\rput(1.08,-1.135){\color{gdarkgray}$a$}
\rput(0.09,-0.555){\color{gdarkgray}$b$}
\rput(2.8,-0.395){\color{gdarkgray}$c$}
\end{pspicture} 

\begin{bemn}[Bemerkungen.]
\begin{enumerate}[label=\arabic{*}.)]
  \item 
$I_1+I_2\ge I_3$, für alle Permutationen.
\begin{proof}
Verwende dazu die Definition des Trägheitsmoments,
\begin{align*}
I_1 + I_2 &= \int\dr^3 \rho(\vec{r})\left[2\vec{r}^2 - r_1^2 -r_2^2\right]
= \int\dr^3 \rho(\vec{r})\left[r_1^2 + r_2^2 + \underbrace{2r_3^2}_{\ge
0}\right]\\
& \ge \int\dr^3 \rho(\vec{r})\left[r_1^2 + r_2^2 \right]
= I_3.\qedhere\maphere
\end{align*}
\end{proof}
\item $I_1 = I_2 = I_3$, Kugelkreisel,\\
$I_1=I_2\neq I_3$, symmetrischer Kreisel,\\
$I_1\neq I_2\neq I_3$, unsymmetrischer Kreisel.
\end{enumerate}
\end{bemn}

\begin{bsp}
Homogener Quader,
\begin{align*}
&I_1 = \frac{1}{12}m(b^2+c^2),\\
&I_2 = \frac{1}{12}m(a^2+c^2),\\
&I_3 = \frac{1}{12}m(a^2+b^2).\bsphere
\end{align*}
\end{bsp}
\begin{bsp}
Homogene Kugel
\begin{align*}
I=\frac{2}{5}mR^2.\bsphere
\end{align*}
\end{bsp}
\begin{bsp}
Homogener Zylinder (Vollzylinder der Länge $l$)
\begin{align*}
&I_1 = \frac{1}{2}mR^2,\\
&I_2=I_3 = \frac{1}{4}mr^2 + \frac{1}{12}ml^2.\bsphere
\end{align*}
\end{bsp}

\begin{propn}[Satz von Steiner]
Es sei $I^{jk}$ der Trägheitstensor relativ zum Schwerpunkt. So gilt für den
Trägheitstensor $\tilde{I}^{jk}$ bezügl eines um $\vec{a}$ veschobenen Punkts
\begin{align*}
\tilde{I}^{jk} = I^{jk} + M\left[\delta^{jk} a^2 - a^j a^k\right].\fishhere
\end{align*}
\end{propn}
\begin{proof}
Der Beweis ist eine leichte Übung.\qedhere
\end{proof}

\subsubsection{Drehimpuls}

Der Drehimpuls bezüglich eines körperfesten Punktes $O$ ist gegeben durch,
\begin{align*}
\vec{m} = \sum_i \vec{q}_i \times \vec{v}_i m_i.
\end{align*}
$O$ ist fixiert im Laborsystem. Setzen wir $\vec{v}_i =
\vec{\omega}\times\vec{q}_i$, so erhalten wir
\begin{align*}
\vec{m} = \sum_i m_i \vec{q}_i \times \left( \vec{\omega}\times\vec{q}_i\right).
\end{align*}
Im bewegten System ist
\begin{align*}
\vec{M} = B^{-1}\vec{m} = \sum_i m_i \vec{Q}_i \times \left(\vec{\Omega}\times
\vec{Q}_i\right) = \sum_i m_i \left[\vec{\Omega}\left(\vec{Q}_i \right)^2 -
\vec{Q}_i \left(\vec{\Omega}\vec{Q}_i\right)\right] = I\vec{\Omega},\tag{*}
\end{align*}
denn
\begin{align*}
M^j = \sum_i \sum_k \left[\delta^{jk} \left(\vec{Q}_i\right)^2 - Q_i^j Q_i^k
\right]m_i \Omega_k  = \sum_k I^{jk}\Omega_k.
\end{align*}
\begin{bemn}[Bemerkungen.]
\begin{enumerate}[label=\arabic{*}.)]
  \item $T_R = \frac{1}{2}\vec{M}\vec{Q}$.
  \item Der Drehimpuls $\vec{M}$ ist im Allgemeinen nicht parallel zur
Drehachse $\vec{\Omega}$.
\item Für ein freies System ist der Drehimpuls im Laborsystem erhalten
$\vec{m}=\const.$ Der Drehimpuls im rotierenden System ist aber nicht
zwangsläufig erhalten $\vec{M}\neq\const$.\maphere
\end{enumerate}
\end{bemn}

Falls ein Körper nur rotiert, so ist es optimal den Ursprung des bewegten
Koordinatensystems in den Schwerpunkt zu setzen.
% Für einen bewegten Körper mit Ursprung des rotierenden Systems im Schwerpunkt
% gilt für den totalen Drehimpuls,
\begin{align*}
&\vec{q}_i  = B\vec{Q}_i +\vec{r}_s,\\
&\vec{v}_i = B\left(\vec{\Omega}\times \vec{Q}_i\right)+\dvec{r}_s,\\
&\vec{J} = \sum_i m_i(\vec{Q}_i + \vec{R}_s) \times
\left[\vec{\Omega}\times \vec{Q}_i + \vec{v}_s\right].
\end{align*}
Da wir uns im Schwerpunkt befinden, verschwinden die Kreuzterme und es gilt,
\begin{align*}
\vec{J} = \underbrace{M_\tot \vec{R}_s \times \vec{v}_s}_{\text{Bahndrehimpuls}}
+
\underbrace{\vec{I}\vec{Q}}_{\vec{M}}.
\end{align*}
$\vec{M}$ ist der innere Drehimpuls (Spin). Der totale Drehimpuls im
Laborsystem ist nun
\begin{align*}
\vec{j} = B\vec{J} = \underbrace{M_\tot \vec{r}_s \times
\vec{v}_s}_{\text{Bahndrehimpuls}} +
\underbrace{\vec{m}}_{\text{Spin}}.
\end{align*}
$J$ bzw. $j$ lässt sich daher aufteilen in die äußere Bewegung (z.b. Kreisbahn)
und die innere Bewegung (Eigendrehimpuls).

Unter Einfluss einer äußeren Kraft ändert sich der Drehimpuls eines Körpers mit
seinem Drehmoment (fixiere Punkt $O$ des Körpers),
\begin{align*}
\frac{\diffd}{\dt}\vec{m} &= \frac{\diffd}{\dt}\sum_i
m_i\vec{q}_i\times\dvec{q}_i
= \sum_i m_i \vec{q}_i\times\ddvec{q}_i = \sum_i \vec{q}_i\times \vec{F_i} =
\vec{n}.
\end{align*}
$\vec{n}$ ist das realtive Drehmoment zum Körperfesten Punkt $O$.

Für das mitrotierende System bedeutet dies,
\begin{align*}
\frac{\diffd}{\dt}\vec{M} = \frac{\diffd}{\dt}B^{-1}\vec{m}
= B^{-1}\vec{n} + \dot{B}^{-1}B\vec{M} = \vec{N} + \vec{M}\times\vec{\Omega}.
\end{align*}

Dies lässt sich in der \emph{Euler-Gleichung} zusammenfassen
\begin{align*}
\frac{\diffd}{\dt}\vec{M} = \vec{N} + \vec{M}\times\vec{\Omega},
\end{align*}
d.h. auch in Abwesenheit äußerer Kräfte $\vec{N}=0$ ist der Drehimpuls im
rotierenden System nicht erhalten.

Die Lösung des freien Kreisels lässt sich einfach geometrisch interpretieren.
Wir haben zwei Erhaltungsgrößen
\begin{align*}
&E = \frac{M_1^2}{I_1} + \frac{M_2^2}{I_2} + \frac{M_3^2}{I_3} = \const.\\
&M^2 = M_1^2+M_2^2 + M_3^2 = \const.
\end{align*}
Die Schnittmenge beschreibt Kurven auf einem Ellipsoiden.
\begin{itemize}[label=\labelitem]
  \item 6 stationäre Lösungen entlang der Hauptachsen.
  \item 4 stabile Lösungen entlang $\vec{e}_1$ und $\vec{e}_3$ ($I_1<I_2<I_3$).
  \item 2 instabile Lösungen entlang der mittleren Hauptachse.
\end{itemize}

\subsection{Lagrange's top}

Wir betrachten nun den symmetrischen Kreisel im Schwerepotential.

In diesem System sind zwei Größen erhalten:
\begin{itemize}[label=\labelitem]
  \item Die Energie $E = T+V$.
  \item Der Drehimpuls $m_z$ entlang der $z$-Achse.
\end{itemize}

Für einen symmetrischen Kreisel $(I_1=I_2\neq I_3)$ ist das Problem analytisch
lösbar. Wähle dazu geeignete Koordinaten $\ph,\psi,\th$.
%TODO: Skizze Kreisel

Das Laborsystem sei $K$ mit $\vec{e}_x$, $\vec{e}_y$, $\vec{e}_z$, das
körperfeste System sei $K'$ mit $\vec{e}_1$, $\vec{e}_2$, $\vec{e}_3$ entlang
der Trägheitsachsen des Kreisels. Es gilt
\begin{align*}
\vec{e}_N = \vec{e}_z\times \vec{e}_3.
\end{align*}
Um $K$ in $K'$ zu transformieren verwenden wir die 3 Transformationen,
\begin{align*}
&\ph: \text{Rotation um } \vec{e}_z: \vec{e}_x\mapsto \vec{e}_N\\
&\th: \text{Rotation um } \vec{e}_N: \vec{e}_z\mapsto \vec{e}_3\\
&\psi: \text{Rotation um } \vec{e}_3: \vec{e}_N\mapsto \vec{e}_1
\end{align*}
Weiter gilt,
\begin{align*}
&\vec{e}_N = \cos\psi \vec{e}_1 - \sin\psi \vec{e}_2\\
&\vec{e}_z = \cos\ph \vec{e}_3 + \cos\psi \sin\th \vec{e}_2 + \sin\psi \sin\th
\vec{e}_1
\end{align*}
Wir haben jetzt ein geeignetes Koordinatensystem. Der nächste Schritt ist die
Lagrangefunktion in diesen Koordinaten aufzustellen.

Potentielle Energie,
\begin{align*}
U = m g l \cos\th,
\end{align*}
wobei $l$ die Position des Schwerpunkts bezeichnet.

Um die Kinetische Energie zu bestimmen, verwenden wir den Hilfsvektor
\begin{align*}
\vec{\Omega} &= \dot{\th}\vec{e}_N + \dot{\psi}\vec{e}_3 + \dot{\ph}\vec{e}_z =
\Omega_1 \vec{e}_1 + \Omega_2\vec{e}_2 + \Omega_3\vec{e}_3\\
&= \vec{e}_1\left(\dot{\th}\cos\psi + \dot{\ph}\sin\psi\sin\th \right)
+ \vec{e}_2 \left( -\dot{\th}\sin\psi + \dot{\ph}\cos\psi\sin\th \right)
+ \vec{e}_3 \left(\dot{\psi} + \dot{\ph}\cos\th\right)
\end{align*}

Für $K'$ folgt somit:
\begin{align*}
&\Omega_1 = \dot{\th}\cos\psi + \dot{\ph}\sin\psi\sin\th,\\
&\Omega_2 = -\dot{\th}\sin\psi + \dot{\ph}\cos\psi\sin\th,\\
&\Omega_3 = \dot{\psi} + \dot{\ph}\cos\th.
\end{align*}

Einsetzen in die kinetische Energie ergibt,
\begin{align*}
T = \frac{1}{2}\left( I_1\Omega_1^2 + I_2 \Omega_2^2 + I_3\Omega_3^2 \right)
= \frac{I_1}{2}\left( \dot{\th}^2 + \dot{\ph}^2\sin^2\th \right) +
\frac{I_3}{2}\left( \dot{\psi} + \dot{\ph}\cos\th\right)^2.
\end{align*}

Die Lagrangefunktion ist nun gegeben durch,
\begin{align*}
L &= L(\ph,\th,\psi,\dot{\ph},\dot{\th},\dot{\psi}) = T-U \\ &=
\frac{I_1}{2}\left( \dot{\th}^2 + \dot{\ph}^2\sin^2\th \right) +
\frac{I_3}{2}\left( \dot{\psi} + \dot{\ph}\cos\th\right)^2
-mgl\cos\th
\end{align*}
Der Aufwand der durch die Einführung von Eulerkoordinaten entstand zahlt sich
nun dadurch aus, dass in der Lagrangefunktion zwei zyklische Koordinaten
auftreten, nämlich $\ph$ und $\psi$.

Die Erhaltungsgrößen erhalten wir durch Differenzieren,
\begin{align*}
&\frac{\partial L}{\partial\dot{\ph}} = \dot{\ph}\left(I_1 \sin^2\th + I_3
\cos^2\th  \right) + \dot{\psi}I_3 \cos\th = M_z = \const,\\
&\frac{\partial L}{\partial\dot{\psi}} = \dot{\ph}I_3\cos\th + \dot{\psi}I_3 =
M_3 = \const\\
\Rightarrow&\dot{\ph} = \frac{M_z - M_3\cos\th}{I_1 \sin^2 \th}\\
&\dot{\psi} = \frac{M_3}{I_3} - \cos\th \frac{M_z - M_3\cos\th}{I_1\sin^2\th}
\end{align*}
Dies können wir nun verwenden um die $\dot{\ph}$ und $\dot{\psi}$ Abhängigkeit
des Lagranges zu eliminieren.

Verbleibende Gleichung für die Inklination $\th$ folgt aus der
Energieerhaltung
\begin{align*}
\underbrace{E - \frac{M_3^2}{2I_3}}_{E'} = \frac{I_1}{2}\dot{\th}^2 +
\underbrace{\frac{\left(M_z - M_3\cos\th\right)^2}{2I\sin^2\th} +
mgl\cos\th}_{U_{\text{eff}}(\th)}.
\end{align*}
Somit haben wir nur noch ein 1-dimensionales Problem. Ein solches ist formal
exakt lösbar.

\subsubsection{Qualitative Diskussion}

Wir sind jetzt nicht daran interessiert das Problem exakt zu lösen, da dies
sehr aufwändig ist und nicht viel zum Verständnis beiträgt.  Jedoch wollen wir
das System qualitativ betrachten.

Einführen einer neuen Kooridnaten
\begin{align*}
&u = \cos\th, \qquad -1\le u\le 1,
\end{align*}
und den Abkürzungen
\begin{align*}
&a \equiv  \frac{M_2}{I_1},\quad b  \equiv \frac{M_3}{I_1},
\quad\alpha \equiv \frac{2E'}{I_1},\quad
\beta \equiv \frac{2mgl}{I_1} \ge 0.
\end{align*}
Wir erhalten somit für
\begin{align*}
\dot{u}^2 = f(u) = \left(\alpha-\beta u \right)(1-u^2)-(a-bu)^2
\end{align*}
ein Polynom 3. Grades in $u$.
\begin{align*}
\dot{\ph} = \frac{a-bu}{1-u^2}.
\end{align*}

\begin{figure}[htbp]
  \centering
\begin{pspicture}(-1,-1)(4,3)
 \psaxes[labels=none,ticks=none]{->}%
 	(0,0)(-0.5,-0.5)(3.5,2.5)%
 	[\color{gdarkgray}$u$,-90]%
 	[\color{gdarkgray}$f(u)$,0]
 	
 \psxTick(1){\color{gdarkgray}u_1}
 \psxTick(2){\color{gdarkgray}u_2}
 
\psplot[linewidth=1.2pt,%
	     linecolor=darkblue,%
	     algebraic=true]%
	     {0.5}{3.5}{(x-1)*(x-2)*(x-3)}

%\rput(2,-0.4){\color{gdarkgray}$f(u)$}
\end{pspicture} 
  \caption{Phasendiagramm für $u$.}
\end{figure}

Die Bewegung findet zwischen $u_1$ und $u_2$ statt. D.h. die Inklination $\th$
oszilliert zwischen den Winkeln $\th_1$ und $\th_2$ periodisch. Diese Bewegung
wird \emph{Nutation} genannt.

Der azimutale Winkel $\ph$ folgt aus
\begin{align*}
\dot{\ph} = \frac{a-bu}{1-u^2}.
\end{align*}
Folgende 3 Fälle sind möglich

\begin{figure}[htbp]
\centering
\begin{pspicture}(0,-1.7)(2.24,1.3)
\pscircle(0.9,-0.36){0.9}
\psellipse[linestyle=dashed](0.9,0.34)(0.44,0.2)
\psline{->}(0.9,0.42)(0.9,1.02)
\psbezier(0.18,0.18)(0.18,-0.46)(1.62,-0.46)(1.62,0.16)
\psbezier(0.04,-0.12)(0.04,-0.98)(1.76,-0.98)(1.76,-0.12)

\psbezier[linecolor=darkblue]{->}(0.1,0.04)(0.14,-0.14)(0.09427359,-0.3594825)(0.16,-0.46)(0.22572641,-0.5605175)(0.32,-0.14)(0.44,-0.2)(0.56,-0.26)(0.54,-0.72)(0.68,-0.74)(0.82,-0.76)(0.94,-0.3)(1.04,-0.3)(1.14,-0.3)(1.18,-0.42)(1.26,-0.54)

\rput(0.9,1.205){\color{gdarkgray}$z$}
\rput(0.57,0.705){\color{gdarkgray}$\th'$}
\rput(1.77,0.365){\color{gdarkgray}$\th_1$}
\rput(2.03,-0.255){\color{gdarkgray}$\th_2$}

\rput(0.9,-1.5){\color{gdarkgray}$(I)$}
\end{pspicture}
\begin{pspicture}(0,-1.7)(2.24,1.3)
\pscircle(0.9,-0.36){0.9}
\psellipse[linestyle=dashed](0.9,0.34)(0.44,0.2)
\psline{->}(0.9,0.42)(0.9,1.02)
\psbezier(0.18,0.18)(0.18,-0.46)(1.62,-0.46)(1.62,0.16)
\psbezier(0.04,-0.12)(0.04,-0.98)(1.76,-0.98)(1.76,-0.12)

\psbezier[linecolor=darkblue]{->}(0.1,0.04)(0.04,-0.14)(0.06,-0.38)(0.16,-0.46)(0.26,-0.54)(0.64,-0.28)(0.44,-0.2)(0.24,-0.12)(0.42,-0.72)(0.68,-0.74)(0.94,-0.76)(1.28,-0.26)(1.04,-0.3)(0.8,-0.34)(1.28,-0.98)(1.38,-0.48)

\rput(0.9,1.205){\color{gdarkgray}$z$}
\rput(0.57,0.705){\color{gdarkgray}$\th'$}
\rput(1.77,0.365){\color{gdarkgray}$\th_1$}
\rput(2.03,-0.255){\color{gdarkgray}$\th_2$}

\rput(0.9,-1.5){\color{gdarkgray}$(II)$}
\end{pspicture} 
\begin{pspicture}(0,-1.7)(2.24,1.3)
\pscircle(0.9,-0.36){0.9}
\psellipse[linestyle=dashed](0.9,0.34)(0.44,0.2)
\psline{->}(0.9,0.42)(0.9,1.02)
\psbezier(0.18,0.18)(0.18,-0.46)(1.62,-0.46)(1.62,0.16)
\psbezier(0.04,-0.12)(0.04,-0.98)(1.76,-0.98)(1.76,-0.12)

\psbezier[linecolor=darkblue]{->}(0.1,0.04)(0.04,-0.14)(0.06,-0.38)(0.16,-0.46)(0.26,-0.54)(0.44,-0.22)(0.44,-0.22)(0.44,-0.22)(0.46,-0.76)(0.68,-0.74)(0.9,-0.72)(0.98,-0.32)(0.98,-0.32)(0.98,-0.32)(1.22,-1.1)(1.48,-0.38)

\rput(0.9,1.205){\color{gdarkgray}$z$}
\rput(0.57,0.705){\color{gdarkgray}$\th'$}
\rput(1.77,0.365){\color{gdarkgray}$\th_1$}
\rput(2.03,-0.255){\color{gdarkgray}$\th_2$}

\rput(0.9,-1.5){\color{gdarkgray}$(III)$}
\end{pspicture}
\caption{Bewegung des Kreisels.}
\end{figure}

Falls $a=bu$ außerhalb von $(u_1,u_2)$ liegt, steigt
$\ph$ monoton an (Fall I).

Falls $u=\frac{a}{b}$ im Intervall $(u_1,u_2)$, so hat $\dot{\ph}$ einen  
Vorzeichenwechsel, $\ph$ bewegt sich vor und zurück (Fall II).

Fall III folgt, wenn wir den Kreisel mit $\dot{\ph} = 0$ fallen lassen.

Die azimutale Bewegung wird \emph{Präzession} genannt.

\subsubsection{Vorestellung der $SO(3)$}

Die $SU(2)$, die \emph{spezielle unitäre Gruppe}, das sind die unitären
$2\times 2$-Matritzen mit Determinante $1$, ist eine Kugel im 4-dimensionalen
Raum mit Radius 1. Identifizieren wir jeweils zwei  Punkte der $SU(2)$, so
erhalten wir die $SO(3)$.

Kehren wir nun zum Kreis zurück.
Durch die Einführung der Eulerwinkel wird jede Rotation im $\R^3$ durch drei
Rotationen um die jeweiligen Achsen darstellbar.
\begin{align*}
T = \Omega_1^2 I_1 + \Omega_2^2 I_2 + \Omega_3^2 I_3,\\
\vec{\Omega} = \dot{\th}\vec{e}_N + \dot{\psi}\vec{e}_3 + \dot{\ph}\vec{e}_2.
\end{align*}
Für $I_1 =I_2$ ist das Problem exakt lösbar,
\begin{align*}
&T = \frac{1}{2}I_1 \left(\dot{\th}^2 + \dot{\ph}^2\sin^2\th\right) +
\frac{I_3}{2}\left(\dot{\psi} + \dot{\ph}\cos\th \right)^2,\\
&V = mgl\cos\th.
\end{align*}
$\ph$ und $\psi$ sind zyklische Koordinaten. Wir haben daher 2 Erhaltungsgrößen.

Unter Einfluss von äußeren Kräften ändert sich der Drehimpuls eines Körpers mit
seinem Drehmoment.
\begin{align*}
\frac{\diffd }{\dt}\vec{m}
&= \frac{\diffd}{\dt} \sum_i m_i \vec{q}_i \times \dvec{q}_i
= \frac{\diffd}{\dt}
\sum_i m_i \dvec{q}_i \times \dvec{q}_i + \vec{q}_i \times \ddvec{q}_i\\
&= \sum_i \vec{q}_i \times m\ddvec{q}_i = \sum_i \vec{q}_i \times \vec{F}_i
\equiv \vec{n},
\end{align*}
wobei wir $\vec{n}$ als \emph{Drehmoment} bezeichnen.

\begin{bsp}
Kreiselkräfte beim Fahrradfahren. Um eine Rechtskurve zu fahren, drückt man den
Lenker rechts nach vorne, wodurch das Rad nach rechts ausweicht.\bsphere
\end{bsp}

Für das mitrotierende System ergibt sich daher,
\begin{align*}
\frac{\diffd}{\dt} \vec{M} &= \frac{\diffd}{\dt}\left(B^{-1}\vec{m} \right)
= \left(\frac{\diffd}{\dt}B^{-1}\right)\vec{m} +
B^{-1}\frac{\diffd}{\dt}\vec{m} =\left(\frac{\diffd}{\dt}B^{-1}\right)B\vec{M}
+ B^{-1}\vec{n} \\ &= \vec{M}\times \vec{\Omega} + \vec{N},
\end{align*}
d.h. auch ohne äußere Kräfte ist der Drehimpuls des rotierenden Systems im
Allgeinen nicht erhalten. Eine Ausnahme bilden Rotationen um die
Hauptträgheitsachsen.

Die Lösung des freien Kreisels lässt sich einfach geometrisch interpretieren.
Wir haben zwei Erhaltungsgrößen,
\begin{align*}
E = \frac{1}{2}\vec{\Omega}\cdot \underbrace{I\vec{\Omega}}_{\vec{M}} =
\frac{1}{2}\vec{M}I^{-1}\vec{M} = \frac{M_1^2}{I_1} + \vec{M_2^2}{I_2} +
\vec{M_3^2}{I_3} = \const,
\end{align*}
dies beschreibt uns einen Ellipsoid,
\begin{align*}
\vec{M}^2 = \left(B^{-1}\vec{m}\right)\left(B^{-1}\vec{m}\right) =
\underbrace{B^{-1}B^{-1}}_{\Id}\left(\vec{m}\cdot\vec{m}\right) = \vec{m}^2
= M_1^2 + M_2^2 + M_3^2 = \const,
\end{align*}
dies beschreibt eine Kugel. Die Schnittmenge dieses Ellipsoids mit der Kugel
schränkt die Bewegung von $\vec{M}$ auf Linien ein.
%TODO: Flußbild
Rotationen um $\vec{e}_1$ und $\vec{e}_3$ sind stabile Lösungen. Insgesamt gibt
es 6 stationäre Lösungen. Davon sind 4 stabil und 2 instabil.
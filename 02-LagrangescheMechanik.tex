\section{Lagrangsche Mechanik}
Im Folgenden wollen wir die Newtonschen Gleichungen mit Hilfe eines
Variationsprinzips, dem so genannten ``Hamiltonschen Prinzip der kleinsten
Wirkung'' formulieren, mit dem wir die Newtonschen Gesetze elegant und
mathematisch exakt in beliebige Koordiantensysteme transformieren können.

\subsection{Euler-Lagrange Gleichungen}
Betrachte ein System von $N$ Teilchen mit den Koordinaten $q^i(t)$,
$i=1,\ldots,3N$ in einem konservativen Kraftfeld.
\begin{align*}
F^i(q^1,\ldots,q^{3N},t) = -\frac{\partial}{\partial q^i}V(q^1,\ldots,q^{3N},t).
\end{align*}

\begin{bemn}[Bemerkung zur Notation.]
Die Beziehung zwischen unseren bisherigen Ortsvektoren $\vec{r}_j$ und den
Koordinaten $q^i$ ist folgende:
\begin{align*}
&\vec{r}_j = \begin{pmatrix}
q^{3j-2}\\q^{3j-1}\\q^{3j}
\end{pmatrix},
\vec{r}_1 = \begin{pmatrix}
q^{1}\\q^{2}\\q^{3}
\end{pmatrix},
\vec{r}_2 = \begin{pmatrix}
q^{4}\\q^{5}\\q^{6}
\end{pmatrix},
\ldots\\
&\vec{F}_j =
\begin{pmatrix}
F^{3j-2}\\F^{3j-1}\\F^{3j}
\end{pmatrix},
\vec{F}_1 = \begin{pmatrix}
F^{1}\\F^{2}\\F^{3}
\end{pmatrix},
\vec{F}_2 = \begin{pmatrix}
F^{4}\\F^{5}\\F^{6}
\end{pmatrix},
\ldots\maphere
\end{align*}
\end{bemn}

Die Newtonschen Gleichungen haben somit die Form,
\begin{align*}
m_i\ddot{q}^i  = -\frac{\partial}{\partial q^i} V(q^j,t) = F^i(q^j,t)
\end{align*}
mit der  Kurzschreibweise $V(q^1,\ldots,q^{3N},t) = V(q^i,t)$.

Wir können den linken Teil auch als partielle Ableitung der kinetischen Energie
$T$ schreiben,
\begin{align*}
m_i \ddot{q}^i = \frac{\diffd}{\dt}\left(m_i \dot{q}^i\right)
= \frac{\diffd}{\dt}\left(\frac{1}{2}m_i
\frac{\partial}{\partial \dot{q}^i}\left(\dot{q}^i\right)^2\right)
= \frac{\diffd}{\dt}\frac{\partial}{\partial \dot{q}^i}T,
\end{align*}
mit $T = \sum_i \frac{m_i}{2}\left(q^i\right)^2$.

Wir definieren die \emph{Lagrange Funktion} durch,
\begin{align*}
L(q^i,\dot{q}^i,t) &= \sum_i \frac{m}{2} \left(\dot{q}^i\right)^2 - V(q^i,t),
\end{align*}
oder kurz $L=T-V$.

Die Bewegungsgleichungen folgen somit aus den \emph{Euler-Lagrange
Gleichungen},
\begin{align*}
&\frac{\diffd}{\dt}\frac{\partial L}{\partial \dot{q}^i} - \frac{\partial L
}{\partial q^i} = 0 \Leftrightarrow\; \frac{\diffd}{\dt}\frac{\partial T}{\partial\dot{q}^i} +
\frac{\partial V}{\partial q^i} = 0.
\end{align*}

\begin{bemn}[Bemerkung zur Notation.]
\begin{align*}
\frac{\partial L}{\partial \vec{r}_1} :=
\begin{pmatrix}
\frac{\partial L}{\partial q_1}\\ \frac{\partial L}{\partial q_2}\\
\frac{\partial L}{\partial q_3}
\end{pmatrix},\qquad 
\frac{\partial L}{\partial \dvec{r}_1} :=
\begin{pmatrix}
\frac{\partial L}{\partial \dot{q}_1}\\ \frac{\partial L}{\partial \dot{q}_2}\\
\frac{\partial L}{\partial \dot{q}_3}
\end{pmatrix}.\maphere
\end{align*}
\end{bemn}

\begin{bsp}
%\begin{enumerate}[label=\arabic{*}.)]
%  \item
 Ein Teilchen befinde sich in einem Gravitationsfeld auf Höhe $x$. Die
  kinetische Energie ist gegeben durch $T= \frac{m}{2}\dot{x}^2$, die
  potentielle Energie ist $V=mgx$, wodurch sich die Kraft $F_x =
  -\frac{\partial V}{\partial x} = -mg$ ergibt.
\begin{figure}[!htbp]
  \centering
\begin{pspicture}(-0.1,-1.25)(2.5334375,1.2304688)

\psframe[fillstyle=solid,fillcolor=glightgray,linestyle=none](2.2,-0.95)(0.1,-1.15)

\psline{->}(0.05,-0.95)(2.36,-0.95)
\psline{->}(0.25,-1.15)(0.25,1.05)
\psline{->}(1,0.2)(1,-0.45)
\psdots[linecolor=darkblue](1,0.2)

\rput(1.2275,-0.08046875){\color{gdarkgray}$\vec{F}$}

\rput(2.4032812,-1.0804688){\color{gdarkgray}$y$}
\rput(0.085625,1.1395313){\color{gdarkgray}$x$}
\end{pspicture} 
  \caption{Teilchen im Gravitationspotential.}
\end{figure} 

  Die Lagrange Funktion hat nun die Form,
\begin{align*}
&L(x,\dot{x}) = T-V = \frac{m}{2}\dot{x}^2 - mgx,\\
&\frac{\diffd}{\dt}\frac{\partial L}{\partial \dot{x}} - \frac{\partial
L}{\partial x} = m\ddot{x}+ mg = 0\\
\Rightarrow\; & \ddot{x} = -g.\bsphere
\end{align*}
\end{bsp}
\begin{bsp}
%\item
Zwei Teilchen mit dem Wechselwirkungspotential
$V\left(\abs{\vec{r}_1-\vec{r}_2}\right)$.

\begin{figure}[!htbp]
  \centering
\begin{pspicture}(-0.5,-1.5)(3,1.2)

\psline{->}(-0.2,-1.2)(2.8,-1.2)
\psline{->}(0,-1.4)(0,1)

\psline{->}(2.52,-0.71171874)(2.02,-0.21171875)
\psline{->}(1.02,0.7682812)(1.52,0.26828125)
\psline{->}(0.0,-1.2)(0.92,0.6482813)
\psline{->}(0.0,-1.2)(2.4,-0.75171876)

\psdots[linecolor=darkblue](2.52,-0.71171874)
\psdots[linecolor=darkblue](1.0,0.78828126)

\rput(1.8,0.5){\color{gdarkgray}$\vec{F}_{21}$}
\rput(2.4,-0.1){\color{gdarkgray}$\vec{F}_{12}$}

\rput(0.3,-0.08171875){\color{gdarkgray}$\vec{r}_2$}
\rput(1.4,-0.7){\color{gdarkgray}$\vec{r}_1$}
\end{pspicture} 
  \caption{Zwei Teilchen mit Wechselwirkungspotential.}
\end{figure}

Kinetische Energie,
\begin{align*}
T = \frac{m_1}{2}\dvec{r}_1^2 + \frac{m_2}{2}\dvec{r}_2^2\entspr \sum_i
\frac{m_i}{2} (\dot{q}^i)^2.
\end{align*}
Potentielle Energie,
\begin{align*}
V\left(\abs{\vec{r}_1-\vec{r}_2}\right).
\end{align*}
Die Lagrange-Funktion hat somit die Form,
\begin{align*}
&L(\vec{r}_1,\vec{r}_2,\dvec{r}_1,\dvec{r}_2) =  \sum_i \frac{m_i}{2} (\dot{q}^i)^2 -
V\left(\abs{\vec{r}_1-\vec{r}_2}\right),\\
&\frac{\partial L}{\partial \vec{r}_1} =
-\frac{\vec{r}_1-\vec{r}_2}{\abs{\vec{r}_1-\vec{r}_2}}
V'\left(\abs{\vec{r}_1-\vec{r}_2}\right) = \vec{F}_{21}\\
&\frac{\partial L}{\partial \vec{r}_2} =
-\frac{\vec{r}_2-\vec{r}_1}{\abs{\vec{r}_2-\vec{r}_1}}
V'\left(\abs{\vec{r}_1-\vec{r}_2}\right) = \vec{F}_{12} = -\vec{F}_{21}\\
&\frac{\partial L}{\partial \dvec{r}_1}
= m_1\dvec{r}_1,\qquad \frac{\partial L}{\partial \dvec{r}_2}
= m_2\dvec{r}_2\\
\Rightarrow\; & \frac{\diffd}{\dt}\frac{\partial L}{\partial \dvec{r}_1}
- \frac{\partial L}{\partial \vec{r}_1}
= m_1\ddvec{r}_1 + \frac{\vec{r}_1-\vec{r}_2}{\abs{\vec{r}_1-\vec{r}_2}}
V'\left(\abs{\vec{r}_1-\vec{r}_2}\right)
= m_1\ddvec{r}_1 - \vec{F}_{21}.\\
& \frac{\diffd}{\dt}\frac{\partial L}{\partial \dvec{r}_2}
- \frac{\partial L}{\partial \vec{r}_2}
= m_2\ddvec{r}_2 + \frac{\vec{r}_2-\vec{r}_1}{\abs{\vec{r}_2-\vec{r}_1}}
V'\left(\abs{\vec{r}_2-\vec{r}_1}\right)
= m_2\ddvec{r}_2 - \vec{F}_{12}.\bsphere
\end{align*}
%\end{enumerate}
\end{bsp}

\subsection{Variationsrechnung}
Im Folgenden stellen wir den mathematischen Rahmen, der zum Verständnis des
Hamilton'schen Variationsprinzips wichtig ist.

\begin{figure}[!htbp]
  \centering
\begin{pspicture}(-0.5,-1.5)(3,1.2)

\psline{->}(-0.2,-1.2)(2.8,-1.2)
\psline{->}(0,-1.4)(0,1)


\psplot[linewidth=1.2pt,%
	     linecolor=darkblue,%
	     algebraic=true]%
	     {0.2}{2.8}{(x-0.8)^3-2*(x-0.8)^2}

\rput(2,-0.4){\color{gdarkgray}$f(x)$}
\end{pspicture} 
  \caption{Funktion $f(x)$ mit Extremstellen.}
\end{figure}
Um festzustellen, ob $x_0$ ein Extremum\footnote{Gemeint sind hier kritische
Punkte von $f$.} von $f:\R\to\R,\; x\mapsto f(x)$ ist,
betrachten wir dazu $x=x_0+\alpha$, wobei $\alpha$ eine \emph{Variation} von $x_0$ darstellt. Eine
hinreichende Bedingung für ein Extremum ist,
\begin{align*}
\frac{\partial}{\partial\alpha} f(x_0+\alpha)\big|_{\alpha=0} = f'(x_0) = 0.
\end{align*}

Betrachte nun eine Kurve $\gamma$ im $\R^n$,
\begin{align*}
\gamma: [t_1,t_2]\to\R^n,\; t\mapsto \vec{r}(t) =
\begin{pmatrix}
q^1(t)\\\vdots\\ q^n(t)
\end{pmatrix},
\end{align*}
mit festen Anfangs- und Endpunkten $\vec{r}_1$ und $\vec{r}_2$. Das Integral,
\begin{align*}
I(\gamma) := \int_{t_1}^{t_2} \dt L(q^i,\dot{q}^i,t)
\end{align*}
beschreibt ein Funktional, das jeder Raumkurve $\gamma$ eine Zahl zuordnet.
Wir sind an den Bedingungen interessiert, unter denen die Kurve $\gamma$ ein
Extremum von $I(\gamma)$ ist. Dazu betrachten wir eine kleine Variation von
$\gamma$,
\begin{align*}
\gamma'(\alpha): t\mapsto &\vec{r}(t) + \alpha\vec{\eta}(t)\equiv \vec{r}'(t),\\
&q^i(t) + \alpha\eta^i(t) \equiv \tilde{q}^i(t). 
\end{align*}
\begin{figure}[!htbp]
  \centering
\begin{pspicture}(0,-1.2411667)(3.014111,1.2470555)
\psline{->}(0.0,-1.0270555)(3.0,-1.0270555)
\psline{->}(0.2,-1.2270555)(0.2,1.1729444)
\psbezier[linestyle=dotted,dotsep=0.06cm,linecolor=darkblue](0.7546875,0.78705555)(1.4946876,1.2270555)(1.7546875,1.0670556)(1.6946875,0.64705557)(1.6346875,0.22705555)(1.5546875,0.06705555)(1.7146875,-0.23294444)
\psbezier{->}(0.4746875,0.48705557)(0.8546875,1.0870556)(1.3546875,0.90705556)(1.5146875,0.32705554)(1.6746875,-0.25294444)(1.8946875,-0.8729445)(2.7146876,-0.45294446)

\rput(2.2,0.89705557){\color{gdarkgray}$\vec{\eta}(t)$}

\rput(2.5960937,-0.8){\color{gdarkgray}$\vec{\gamma}(t)$}
\end{pspicture} 
  \caption{Variation von $\gamma$.}
\end{figure}

Fixierte Anfangs- und Endpunkte verlangen, dass
$\vec{\eta}(t_1)=\vec{\eta}(t_2)=0$. Dies führt uns auf eine neue Zahl,
\begin{align*}
I(\gamma'(\alpha)) = \int_{t_1}^{t_2} \dt L(\tilde{q}^i, \dot{\tilde{q}}^i, t).
\end{align*}

\begin{defnn}
Die Kurve $\gamma$ heißt \emph{extremal} zu dem Funktional $I(\gamma)$, falls
für alle Variationen $\vec{\eta}(t)$ gilt,
\begin{align*}
\frac{\partial}{\partial\alpha}I(\gamma'(\alpha))\bigg|_{\alpha=0} = 0.\fishhere
\end{align*}
\end{defnn}

Wir wollen nun untersuchen, für welche $\gamma$, die Wirkung maximal wird,
\begin{align*}
\frac{\partial}{\partial\alpha} I(\gamma'(\alpha))
&= \int_{t_1}^{t_2}\dt \frac{\partial}{\partial\alpha}
L(q^i + \alpha \eta^i, \dot{q}^i + \alpha\dot{\eta}^i,t)
\bigg|_{\alpha=0}\\
&= \int_{t_1}^{t_2}\dt \sum_i
\bigg[
\frac{\partial L}{\partial q^i} L(q^i + \alpha \eta^i, \dot{q}^i +
\alpha\dot{\eta}^i,t)\eta^i \\
 &\qquad + \frac{\partial}{\partial \dot{q}^i}L(q^i +
\alpha \eta^i, \dot{q}^i + \alpha\dot{\eta}^i,t)\dot{\eta}^i
\bigg]_{\alpha=0}\\
&= \int_{t_1}^{t_2}\dt \sum_i
\left[
\frac{\partial }{\partial q^i} L(q^i, \dot{q}^i, t)\eta^i +
\frac{\partial}{\partial \dot{q}^i}L(q^i, \dot{q}^i, t)\dot{\eta}^i
\right]\\
&\overset{\text{\tiny part.int.}}{=}
\int_{t_1}^{t_2}\dt \sum_i \left[
\frac{\partial}{\partial q^i} - \frac{\diffd}{\dt}\frac{\partial}{\partial
\dot{q}^i}\right]L\eta^i + \underbrace{\sum_i \frac{\partial}{\partial
\dot{q}^i}L\eta^i\bigg|_{t_1}^{t_2}}_{=0\;\text{\tiny nach Vor.}}\\
&= \int\limits_{t_1}^{t_2}
\sum_i \left[
\frac{\partial}{\partial q^i}L(q^i, \dot{q}^i, t) -
\frac{\diffd}{\dt}\frac{\partial}{\partial \dot{q}^i} 
L(q^i, \dot{q}^i, t)
\right]\eta^i \overset{!}{=} 0
\end{align*}
Der Ausdruck verschwindet genau dann für jede Variation $\eta$, wenn der
Integrand bereits verschwindet.
\begin{proof}
Angenommen das Integral verschwindet für einen Integrand, der an einem Punkt
$x$ $\neq 0$ ist, dann folgt Aufgrund der Stetigkeit, dass der Integrand in
einer Umgebung von $x$ nicht verschwindet. Wählt man nun $\eta$ so, dass es nur
in dieser Umgebung ungleich Null ist, ist das Integral nicht
null.\dipper\qedhere
\end{proof}

Wir haben somit folgenden Satz bewiesen.

\begin{propn}
Eine notwendige und hinreichende Bedingung, damit die Kurve $q^i(t)$ ein
extremal zum Funktional $I(\gamma)$ ist, ist das Erfüllen der
Euler-Lagrange-Gleichungen
\begin{align*}
\frac{\diffd}{\dt}\frac{\partial}{\partial \dot{q}^i}L(q^i,\dot{q}^i,t) -
\frac{\partial}{\partial q^i}L(q^i,\dot{q}^i,t) = 0.\fishhere
\end{align*}
\end{propn}

\begin{bemn}
Die Funktionalableitung hat die Form,
\begin{align*}
\frac{\delta I}{\delta q^i(t)} = \frac{\partial L}{\partial q^i} -
\frac{\diffd}{\dt}\frac{\partial L}{\partial \dot{q}^i}.\maphere
\end{align*}
\end{bemn}

\subsection{Hamilton's Prinzip der kleinsten Wirkung}

\begin{propn}[Hamilton's Prinzip der kleinsten Wirkung]
Die Bewegung eines Systems von der Zeit $t_1$ nach $t_2$ ist so, dass die
Wirkung
\begin{align*}
S:= \int_{t_1}^{t_2} \dt L
\end{align*}
mit dem Lagrange $L=T-V$ extremal wird.\fishhere
\end{propn}
\begin{proof}
Die Variationsrechnung liefert, dass ein Extremum der Wirkung die
Euler-Lagrange-Gleichungen erfüllt und diese sind äquivalent zu den Newtonschen
Bewegungsgleichungen.\qedhere
\end{proof}

\begin{bemn}
Die Dimension der Wirkung ist,
\begin{align*}
[\text{Wirkung}] = [\text{Energie}]\cdot[\text{Zeit}] =
[\text{Impuls}]\cdot[\text{Länge}].
\end{align*}
Somit entspricht die Einheit der Wirkung der von $h$ bzw $\hbar$.\maphere
\end{bemn}

Durch die Variationsrechnung haben wir die lokalen Bewegungsgleichungen durch
ein globales Konzept der extremalen Wirkung ersetzt. Dadurch ergeben sich sehr
markante Vorteile.
\begin{itemize}[label=\labelitem]
  \item Der Lagrange besteht aus den experimentell zugänglichen Größen $T$
  kinetische Energie und $V$ potentielle Energie.
  \item Das Variationsprinzip ist invariant unter Koordinatentransformation.
  Man erhält sehr einfach die Bewegungsgleichungen in krumlinigen Koordinaten. 
  \item Die Größen $q^i$ können beliebige verallgemeinerte Koordinaten sein 
 und beliebige Enheiten annehmen.
  \item Symmetrien und Erhaltungsgrößen sind im Lagrange Formalismus sehr
  elegant zur formulieren.
\end{itemize}

\begin{bsp}
Betrachte ein Teilchen in der Ebene im Zentralpotential $V(r)$.
\begin{figure}[!htbp]
  \centering
\begin{pspicture}(0,-1.2)(3.45,1.6)
\psline{->}(0.2453125,-0.87457985)(3.2453125,-0.87457985)
\psline{->}(0.4453125,-1.0745798)(0.4453125,1.3254201)
\psline{->}(0.44,-0.8804687)(1.5,0.19953126)
\psline[linecolor=yellow]{->}(1.56,0.25953126)(2.3,0.35953125)
\psarc(0.6,-0.9){0.6}{2.29061}{57.380756}
\psdots[linecolor=darkblue](1.56,0.25953126)

\rput(0.9,-0.65){\color{gdarkgray}$\ph$}
\rput(1.06,0.10953125){\color{gdarkgray}$\vec{r}$}
\rput(2.2867188,0.58953124){\color{gdarkgray}$\vec{v}$}
\rput(3.3054688,-1.0304687){\color{gdarkgray}$x$}
\rput(0.2878125,1.4495312){\color{gdarkgray}$y$}
\end{pspicture}
  \caption{Teilchen in der Ebene.}
\end{figure}

In karthesischen Koordinaten ist
\begin{align*}
&T = \frac{m}{2}(\dot{x}^2+\dot{y}^2)\\
&V=V(\sqrt{x^2+y^2})
\end{align*}
und die verallgeminerten Koordinaten haben die Form $q^1 = x, q^2 = y$.


In Polarkoordinaten ist $q^1 = r\cos \ph, q^2 = r\sin \ph$.
Die kinetische Energie hat hier die Form,
\begin{align*}
T&=\frac{m}{2}\left(\left(\dot{r}\cos\ph - r\sin\ph\right)^2
+ \left(\dot{r}\sin\ph + r\cos\ph\right)^2\right)\\
&= \frac{m}{2}\left(
\dot{r}^2 \cos^2\ph + r^2\sin^2\ph \dot{\ph} + \dot{r}^2\sin^2\ph
+ r^2\cos^2\ph\dot{\ph}^2 \right)\\
&= \frac{m}{2}\left(\dot{r}^2 + r^2\dot{\ph}^2\right).
\end{align*}
Der Lagrange ist also,
\begin{align*}
L(r,\ph,\dot{r},\dot{\ph}) =\frac{m}{2}\left(\dot{r}^2+r^2\dot{\ph}^2\right) -
V(r).
\end{align*}
Die Euler-Lagrange-Gleichung hat nun die Form,
\begin{align*}
&\frac{\diffd}{\dt}\frac{\partial L}{\partial \dot{r}} - \frac{\partial
L}{\partial r}
= m\ddot{r} - mr\dot{\ph}^2 + V'(r) = 0.\\
&\frac{\diffd}{\dt}\frac{\partial L}{\partial \dot{\ph}} - \frac{\partial
L}{\partial \ph}
= \frac{\diffd}{\dt}mr^2\dot{\ph}  = 0 \Rightarrow mr^2\dot{\ph} = \const.
\end{align*}
Wir erhalten so sofort die Drehimpulserhaltung,
\begin{align*}
L_z = m(\dot{x}y-x\dot{y}) = mr^2\dot{\ph} = \const.
\end{align*}
\end{bsp}

\begin{bemn}[Bemerkungen zu den Größen.]
\begin{itemize}[label=\labelitem]
  \item $q^i$: Verallgemeinerte Koordinaten.
  \item $p_i = \frac{\partial L}{\partial \dot{q}^i}$: Verallgemeinerte
  Impulse.\footnote{Auch generalisierter, kanonischer, kanonisch
  konjugierter oder konjugierter Impuls.}
  \item $\frac{\partial L}{\partial q^i}$: Verallgemeinerte Kräfte.
\end{itemize}
Die verallgemeinerten Größen können unter Koordinatentransformation ihre
Dimension ändern, die $q^i$ beispielsweise müssen also in beliebigen
krummlinigen Koordinaten nicht mehr die Einheit einer Länge haben.\maphere
\end{bemn}

\begin{defnn}
Eine Koordinate $q^i$ heißt \emph{zyklisch}, wenn $L$ nicht von $q^i$ abhängt,
\begin{align*}
\frac{\partial L}{\partial q^i} = 0.\fishhere
\end{align*}
\end{defnn}

\begin{propn}
Für zyklische Koordinaten ist der dazugehörige kanonische Impuls
 eine Erhaltungsgröße.
\begin{align*}
\frac{\partial L}{\partial q^i} = 0 \Rightarrow \frac{\diffd p_i}{\dt} =
0 \Rightarrow p_i=\frac{\partial L}{\partial \dot{q}^i} = \const.\fishhere
\end{align*}
\end{propn}

\begin{bsp}
Betrachte erneut ein Teilchen in der Ebene im Zentralpotential. Der Lagrange
hat in Polarkoordinaten die Form
\begin{align*}
L(r,\ph,\dot{r},\dot{\ph}) = \frac{m}{2}\left(\dot{r}^2 + r^2\dot{\ph}^2
\right)-V(r).
\end{align*}
Offensichtlich ist $\ph$ zyklisch, wir erhalten so die Drehimpulserhaltung
\begin{align*}
L = p_\ph = \frac{\partial L}{\partial\dot{\ph}} = mr^2\dot{\ph} =
\const.\bsphere
\end{align*}
\end{bsp}

\begin{bemn}
Unterscheiden sich zwei Lagrange Funktionen $L$ und $L'$ nur durch eine totale
Ableitung,
\begin{align*}
L' = L + \frac{\diffd}{\dt}F,
\end{align*}
so führt dies auf dieselben Bewegungsgleichungen.\maphere
\end{bemn}
\begin{proof}
Dies ergibt sich direkt, wenn wir die Wirkung betrachten,
\begin{align*}
S'(\gamma) = 
\int\limits_{t_1}^{t_2} \dt L' =
\int\limits_{t_1}^{t_2} \dt \left(L + \frac{\diffd}{\dt} F\right) =
\left(\int\limits_{t_1}^{t_2} \dt L\right) + F\big|_{t_1}^{t_2}. 
\end{align*}
Nun hängt $F$ nicht von $\gamma$ ab und daher ist ein zu $L$ extremales
$\gamma$ auch extremal zu $L'$.\qedhere
\end{proof}

\begin{bsp}
Potentiale lassen sich nicht eindeutig definieren, denn zwei Potentiale, die
sich durch eine Konstante unterscheiden
\begin{align*}
V(q^i),\qquad V'(q^i) = V(q^i) + V_0,
\end{align*}
führen zu zwei Lagrange Gleichungen, die sich lediglich durch eine totale
Ableitung unterscheiden,
\begin{align*}
L' - L = V_0 = \frac{\diffd}{\dt}(t\cdot V_0).\bsphere
\end{align*}
\end{bsp}

\subsection{Lorentz-Kraft}
Aus Experimenten wissen wir, dass ein geladenes Teilchen im elektrischen Feld
$\vec{E}$ und magnetischem Feld $\vec{B}$ die Lorentz-Kraft erfährt,
\begin{align*}
\vec{F} = q\cdot\vec{E} + \frac{q}{c}\vec{v}\times\vec{B}.
\end{align*}
Nun lassen sich diese Felder über Potentiale definieren,
\begin{align*}
&\vec{E} = -\nabla \phi - \frac{1}{c}\frac{\partial}{\partial
t}\vec{A},\\& \vec{B} = \rot\vec{A},
\end{align*} 
mit einem Skalarpotential $\phi$ und einem Vektorpotential $\vec{A}$. $\vec{A}$
und $\phi$ sind nicht eindeutig und insbesondere nicht experimentell zugänglich;
lediglich das vom Potential erzeugte Feld lässt sich messen.

Lagrange Funktion und Euler-Lagrange-Gleichung haben die Form,
\begin{align*}
&L(\vec{r},\dvec{r},t) = \frac{m}{2}\dvec{r}^2 - q\left[\phi(\vec{r},t) -
\frac{\dvec{r}}{c}\vec{A}(\vec{r},t) \right],\\
&m\ddvec{r} = \vec{F}_L = q\vec{E} + \frac{q}{c}\vec{v}\times\vec{B}.
\end{align*}
Der kanonische Impuls hat nun die Form,
\begin{align*}
\vec{p} = \frac{\partial L}{\partial \dvec{r}} = m\dvec{r} +
\frac{q}{c}\vec{A}(\vec{r},t).
\end{align*}
\begin{bemn}[Übungsaufgabe:]
Zeige, dass die Euler-Lagrange-Gleichungen die Lorentzkraft ergeben.
\end{bemn}

Da die Potentiale $\phi$ und $\vec{A}$ nicht eindeutig sind, lassen sich
Transformationen definieren unter denen $\vec{E}$- und $\vec{B}$-Feld
invariant sind:
\begin{align*}
&\vec{A}\mapsto \vec{A}' = \vec{A} + \nabla \lambda(\vec{r},t),\\
&\phi\mapsto \phi' = \phi - \frac{1}{c}\partial_t \lambda(\vec{r},t).
\end{align*}
Diese Transformationen werden \emph{Eichtransformationen} genannt. Wendet man
eine Eichtransformation an, unterscheidet sich der Lagrange wieder nur um eine
totale Ableitung,
\begin{align*}
L' &= \frac{m}{2}\dvec{r}^2 - q\left[\phi - \frac{1}{c}\partial_t \lambda -
\frac{\dvec{r}}{c}\left(\vec{A} + \nabla\lambda\right)\right]
= L + \frac{q}{c}\left[\partial_t \lambda + \dvec{r}\nabla\lambda \right] \\ 
&=L+ \frac{q}{c}\frac{\diffd}{\dt}\lambda.
\end{align*}

\subsection{Noether Theorem}

Wir kennen bereits eine ganz spezielle Symmetrie, die zyklischen Koordinaten,
und haben gesehen, dass diese uns stets eine Erhaltungsgröße liefert. Wir
wollen nun einen allgemeinen Zusammenhangen zwischen Symmetrien und
Erhaltungssätzen aufstellen.

Betrachte eine kontinuierliche Schar von Koordinatentransformationen,
\begin{align*}
h_s^i:\; &q^i \mapsto \tilde{q}^i = h_s^i(q^j),\\
&\dot{q}^i \mapsto \dot{\tilde{q}}^i = \sum_j \frac{\partial h_s^i}{\partial
q^j}\dot{q}^j\\
h_0^i:\; &q^i \mapsto q^i.
\end{align*}

\begin{bsp}
Translationen haben so die Form,
\begin{align*}
&q^i \mapsto \tilde{q}^i + s\cdot a^i,\\
&\dot{q}^i \mapsto \dot{\tilde{q}}^i  = \dot{q}^i.\bsphere
\end{align*}
\end{bsp}
\begin{bsp}
Eine Rotation um die $z$-Achse ist gegeben durch,
\begin{align*}
&\vec{r} \mapsto \vec{r}' = R(s)\vec{r} =
\begin{pmatrix}
\cos s & \sin s & 0\\
-\sin s & \cos s & 0\\
0 & 0 & 1
\end{pmatrix}
\begin{pmatrix}
q^1 \\ q^2 \\ q^3 
\end{pmatrix},\\
&\dvec{r}\mapsto \dvec{r}' = R(s)\dvec{r}.
\end{align*}
Für $s=0$ ist $R=\Id$.\bsphere
\end{bsp}

Die Schar von Koordinatentransformationen ist eine \emph{Symmetrie des Systems},
wenn sie den Lagrange nicht verändert, d.h. wenn
\begin{align*}
&L(\tilde{q}^i,\dot{\tilde{q}}^i,t) = L(q^i,\dot{q}^i,t), \forall s,\\
\Leftrightarrow\; & \frac{\diffd}{\ds}L(\tilde{q}^i,\dot{\tilde{q}}^i,t) = 0.
\end{align*}

\begin{propn}[Noether Theorem]
Für eine Symmetrie $h_s^i$ gibt es eine Erhaltungsgröße der Form,
\begin{align*}
I(q^i,\dot{q}^i) = \sum_i \frac{\partial L}{\partial \dot{q}^i}
\frac{\diffd h_s^i(q^i)}{\ds} \bigg|_{s=0}.\fishhere
\end{align*}
\end{propn}

\begin{bsp}
$L(r,\dot{r},\ph,\dot{\ph})$ und $\ph$ ist zyklisch.
\begin{align*}
\left.
\begin{aligned}
&h_s^i: \ph\mapsto \ph+s,\\
&\frac{\diffd h_s}{\diffd s} = 1
\end{aligned}
\right\}\Rightarrow \frac{\partial L}{\partial \dot{\ph}} =
I(q^i,\dot{q}^i).\bsphere
\end{align*}
\end{bsp}

\begin{proof}[Beweis des Noether Theorems.]
Sei $q^i(t)$ eine Lösung der Bewegungsgleichung, dann ist $\tilde{q}^i(t) =
h_s^i(q^i(t))$ ebenfalls eine Lösung der Bewegungsgleichung.
\begin{align*}
0 &= \frac{\diffd}{\ds}L(\tilde{q}^i,\dot{\tilde{q}}^i,t)
= \sum_i \left[\frac{\partial L}{\partial q^i}\frac{\diffd}{\ds} \tilde{q}^i +
\frac{\partial L}{\partial \dot{q}^i}\frac{\diffd}{\ds}\dot{\tilde{q}}^i
\right]\\ &= \sum_i \left[\left(\frac{\diffd}{\dt}\frac{\partial L}{\partial
\dot{q}^i}\right)\frac{\diffd}{\ds}\tilde{q}^i + \frac{\partial
L}{\partial \dot{q}^i}\frac{\diffd}{\dt}\left(\frac{\diffd}{\ds}\tilde{q}^i
\right) \right]\\
&= \sum_i \frac{\diffd}{\dt} \left[\frac{\partial L}{\partial
\dot{q}^i}\frac{\diffd}{\ds}\tilde{q}^i \right].
\end{align*}
Also ist $\sum_i\dfrac{\partial L}{\partial
\dot{q}^i}\dfrac{\diffd}{\ds}\tilde{q}^i$ eine Erhaltungsgröße für alle $s$, also auch für $s=0$ und damit gilt,
\begin{align*}
\left.\frac{\diffd}{\dt} \sum_i\left[\frac{\partial L}{\partial
\dot{q}^i}\frac{\diffd}{\ds}\tilde{q}^i \right]\right|_{s=0} = 0.\qedhere
\end{align*}
\end{proof}

\begin{bsp}
%\begin{enumerate}[label=\arabic{*}.)]
\textit{Translationssymmetrie in einem System von $N$ Teilchen.}
\begin{align*}
&\vec{r}_i \mapsto \vec{r}_i + s\vec{a},\\
&\dvec{r}_i \mapsto \dvec{r}_i',
\end{align*}
wobei $\vec{a}$ ein beliebiger aber von $i$ unabhängiger Translationsvektor
ist, d.h. das ganze Teilchensystem wird um $s\vec{a}$ verschoben. 
\begin{align*}
&\frac{\diffd}{\ds} h_s^i = \frac{\diffd}{\ds} \left(\vec{r}_i + s\vec{a}\right)
= \vec{a}.\\
&\frac{\partial L}{\partial \dvec{r}_i} = \vec{p}_i. 
\end{align*}
Wir erhalten somit die Erhaltungsgröße,
\begin{align*}
I(\vec{r}_i,\dvec{r}_i) = \sum_i \left(\vec{p}_i\cdot \vec{a}\right) =
\left(\sum_i \vec{p}_i\right)\cdot\vec{a} = \vec{p}_\tot\cdot\vec{a}.
\end{align*}
Liegt eine Translationssymmetrie entlang allen Koordinatenachsen vor, so	 gilt
\begin{align*}
\vec{p}_\tot = \sum_i \vec{p}_i = \const.
\end{align*}
Im Schwerefeld mit Gravitation in $z$-Richtung und Translationssymmetrie in $x$-
und $y$-Richtung wären nur $\vec{p}_x$ und $\vec{p}_y$ erhalten.\bsphere
\end{bsp}
\begin{bsp}
\textit{Rotationssymmetrie.} Für kleine Winkel können wir die Rotation um
die Achse $\vec{n}$ schreiben als,
\begin{align*}
\vec{r}_i \mapsto \vec{r}_i + s\vec{n}\times\vec{r}_i,\\
\dvec{r}_i \mapsto \dvec{r}_i + s\vec{n}\times\dvec{r}_i.
\end{align*}
\begin{align*}
&\frac{\diffd}{\ds} h_s^i = \frac{\diffd}{\ds}\left(\vec{r}_i +
s\vec{n}\times\vec{r}_i\right) = \vec{n}\times\vec{r}_i,\\
& I(\vec{r}_i,\dvec{r}_i) = \sum_i \vec{p}_i \cdot \vec{n}\times \vec{r}_i =
\sum_i \vec{n}\cdot \vec{r}_i\times\vec{p}_i = \vec{n}\sum_i \vec{L}_i =
\vec{n}\cdot \vec{L}_\tot.
\end{align*}
Bei Rotationssymmetrie entlang aller Koordinatenachsen ist der
Gesamtdrehimpuls $\vec{L}_\tot$ erhalten.
\end{bsp}
\begin{bemn}[Übungsaufgabe:]
Welcher Teil des Drehimpulses ist nicht erhalten, wenn die
Rotationssymmetrie nur um die $z$-Achse gegeben ist?
\end{bemn}
\begin{bsp}
\textit{Galilei Boost.}
\begin{align*}
&\vec{r}_i \mapsto \vec{r}_i + s\vec{v}t\\
&\dvec{r}_i \mapsto \dvec{r}_i + s\vec{v}
\end{align*}
Die Lagrangefunktion ist nicht invariant unter dieser Operation, sondern ändert
sich um eine totale Ableitung,
\begin{align*}
L\mapsto L'=L + \sum_i\left[\frac{m_i}{2}s^2\vec{v}^2 + m_i\dvec{r} \vec{v}s
\right]= L + \frac{\diffd}{\dt}
\left(\underbrace{\sum_i \frac{m_i}{2}s^2\vec{v}^2t + m_i
s\vec{r}\vec{v}}_{F(q^i,\dot{q}^i)}\right).
\end{align*}
Jedoch gilt $\dfrac{\diffd}{\ds}L = 0$ und damit auch
\begin{align*}
0 = \dfrac{\diffd}{\ds}\left[L' - \frac{\diffd}{\dt}F\right]
= \dfrac{\diffd}{\dt}\left[\sum_i\frac{\partial L}{\partial
\dot{q}^i}\frac{\diffd h_s^i}{\ds} - \frac{\diffd}{\ds}F\right].
\end{align*}
Die Erhaltunsgröße hat somit die Form
\begin{align*}
I(q^i,\dot{q}^i) = \sum_i \frac{\partial L}{\partial q^i}\frac{\diffd
h_s^i(q^i)}{\ds} - \frac{\diffd}{\ds} F(q^i,\dot{q}^i)\bigg|_{s=0}.
\end{align*}
Damit folgt der Schwerpunktssatz
\begin{align*}
\sum_i \vec{p}_i\vec{v}t - \sum_i m_i\vec{r}_i\vec{v} =
\underbrace{\left(\sum_i \vec{p}_i t - \sum_i
m\vec{r}_i\right)}_{\text{Erhaltungsgröße}}\vec{v} = \const.\bsphere
\end{align*}
%\end{enumerate}
\end{bsp}

\subsection{Energieerhaltung}
Unser System ist homogen in der Zeit, d.h. kein Zeitpunkt ist ausgezeichnet.
Betrachten wir also eine Translation der Zeit
\begin{align*}
&t\mapsto t' = t+a,
\end{align*}
so bleibt die Lagrangefunktion unverändert und ist daher nicht explizit von
der Zeit abhängig,
\begin{align*}
&L = L' \quad \frac{\partial L}{\partial t} = 0.
\end{align*}
Sie ist natürlich implizit von der Zeit abhängig, da $q^i$ und $\dot{q}^i$ in
$L(q^i(t),\dot{q}^i(t))$ von der Zeit abhängen. Die totale Ableitung
$\frac{\diffd}{\dt} L$ muss daher auch nicht verschwinden, sondern es gilt
\begin{align*}
\frac{\diffd}{\dt} L(q^{i},\dot{q}^i) &= \sum_i \left[\frac{\partial L}{\partial
q^{i}}\frac{\diffd}{\dt}q^{i} + \frac{\partial L}{\partial
\dot{q}^{i}}\frac{\diffd}{\dt}\dot{q}^{i} \right] + \frac{\partial L}{\partial
t}\\
&= \sum_i \left[\left(\frac{\diffd}{\dt}\frac{\partial
L}{\partial \dot{q}^i}\right)\frac{\diffd}{\dt}q^{i} + \frac{\partial
L}{\partial \dot{q}^{i}}\frac{\diffd}{\dt}\dot{q}^{i} \right]\\
&= \frac{\diffd}{\dt}\sum_i \frac{\partial L}{\partial\dot{q}^i}\dot{q}^i,\\
\Rightarrow\; & \frac{\diffd}{\dt}\underbrace{\left[\sum_i \frac{\partial
L}{\partial \dot{q}^i} \dot{q}^i - L\right]}_{:=H}=0.
\end{align*}
Die Energie,
\begin{align*}
H = \sum_i \vec{p}_i \dot{q}^i - L(q^i,\dot{q}^i),\qquad \vec{p}_i :=
\frac{\partial L}{\partial \dot{q}^i}
\end{align*}
ist eine Erhaltungsgröße, die aus der Translationsinvarianz der Zeit folgt.

\begin{bemn}
Für geschwindigkeitsunabhängige Kräfte im Potential $V$ gilt,
\begin{align*}
&\frac{\partial L}{\partial \dot{q}^i} = \frac{\partial T}{\partial
\dot{q}^i},\\
&H = \sum_i \frac{\partial L}{\partial \dot{q}^i}\dot{q}^i - (T-V)  =
T+V.\maphere
\end{align*}
\end{bemn}

Wir haben somit für jede der 10 Symmetrien der Galilei-Gruppe eine
Erhaltungsgröße gefunden.

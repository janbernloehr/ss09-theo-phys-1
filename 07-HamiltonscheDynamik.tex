\section{Hamilton'sche Dynamik}

Ein Teilchen im Potential ist beschrieben durch,
\begin{align*}
L(q,\dot{q}) = \frac{m}{2}\dot{q}^2 - V(q).
\end{align*}
Hierbei sind $q$ und $\dot{q}$ nicht unabhängig von einander. Die
Euler-Lagrange-Gleichungen, die eine Beziehung zwischen ihnen herstellen, sind
Differentialgleichungen 2. Ordnung.

Wir wollen nun zu neuen Koordinaten $q$, $p$ übergehen, so dass wir eine
Differentialgleichung 1. Ordnung erhalten.

Der Hamilton'sche Impuls $p=\frac{\partial L}{\partial \dot{q}}$ erfüllt die
Relationen
\begin{align*}
&\dot{q} = \frac{p}{m},\\
&\dot{p} = m\ddot{q} = -\frac{\partial V}{\partial q}.
\end{align*}
Wir wollen diese Relationen durch eine neue Funktion $H$ beschreiben,
\begin{align*}
&\dot{q} = \frac{p}{m} = \frac{\partial H}{\partial p},\\
&\dot{p} = -\frac{\partial V}{\partial q} = -\frac{\partial H}{\partial q}.
\end{align*}
Diese Funktion heißt \emph{Hamiltonfunktion} und ist gegeben durch,
\begin{align*}
H(q,p) = \frac{p^2}{2m} + V(q).
\end{align*}
Die Gleichungen
\begin{align*}
&\dot{q} = \frac{p}{m} = \frac{\partial H}{\partial p},\\
&\dot{p} = -\frac{\partial V}{\partial q} = -\frac{\partial H}{\partial q},
\end{align*}
heißen \emph{Hamilton'sche Bewegungsgleichungen}, $(q,p)$ sind harmonische
Variablen.

\subsection{Legendre Transformation}

Wir betrachten die Funktion $f(x)$ mit Variable $x$ und definieren uns eine
neue Funktion $g(y)$, die \emph{Legendre Transformierte} von $f(x)$, mit der
neuen Variablen $y$ mittels,
\begin{align*}
&y := \frac{\diffd f}{\dx},\\
&g(y) := [xy-f(x)](y).
\end{align*}
Um die ``alte" Variable $x$ durch einen von $y$ abhängigen Ausdruck ersetzen zu
können, muss die Gleichung $y = \frac{\diffd f}{\dx}$ nach $x$ aufgelöst
werden $x(y)$. Dies ist nur dann eindeutig, falls $\frac{\diffd^2
f}{\dx^2}\neq 0$ für alle $x$.

\begin{bsp}
Sei $f(x) = x^2$, dann ist $y=\frac{\diffd f}{\dx} = 2x\Rightarrow
x=\frac{1}{2}y$. Die Legendre-Transformierte $g(y)=xy-f(x)$ ist gegeben durch,
\begin{align*}
g(y) = \frac{y^2}{2} - x^2 = \frac{y^2}{2} - \frac{y^2}{4} =
\frac{y^2}{4}.\bsphere
\end{align*}
\end{bsp}

\begin{bemn}[Geometrische Deutung.]
Die Legendre-Transformation beschreibt eine Funktion $f$ in eindeutiger
Weise, indem sie jeder Steigung $y=\frac{\df}{\dx}$ den $y$-Achsen-Abschnitt der dort
anliegenden Tangente zuordnet. Die Funktion $f$ wird somit vollständig durch ihre
Einhüllenden charakterisiert.\maphere
%TODO: Skizze
\end{bemn}

\begin{propn}[Theorem]
Die Legendre-Transformation von $g(y)$ ergibt wieder die Funktion $f(x)$, falls
$g(y)$ die Legendre-Transformation von $f(x)$ ist.\fishhere
\end{propn}
\begin{proof}
Betrachte dazu,
\begin{align*}
z &= \frac{\diffd g}{\dy} = \frac{\diffd}{\dy}\left[ xy-f(x) \right](y)
= x \frac{\diffd }{\dy}y + y \frac{\diffd}{\dy}x  - \frac{\diffd}{\dy}f(x)
\\ &= x + y\frac{\diffd}{\dy}x - \frac{\diffd f(x)}{\dx}\frac{\dx}{\dy}
= x + \frac{\diffd f}{\dx}\frac{\dx}{\dy} - \frac{\diffd
f}{\dx}\frac{\dx}{\dy}.
\end{align*}
Die Legendre-Transformation von $g(y)$ ist
\begin{align*}
h(z) = zy - g(y) = zy-\left[xy-f(x)\right](y) = 
xy - \left[xy-f(x)\right] = f(x).\qedhere
\end{align*}
\end{proof}

\subsection{Hamilton-Funktion und Hamiltonsche Bewegungsgleichung}

Wir erhalten die Hamiltonfunktion indem wir eine Legendre-Transformation auf
die Lagrangefunktion anwenden,
\begin{align*}
L(q^\alpha,\dot{q}^\alpha,t).
\end{align*}
Dabei sei $q^\alpha$ fest und unsere neue Variable der kanonischer Impuls,
\begin{align*}
p^\alpha \equiv \frac{\partial L}{\partial \dot{q}^\alpha}.
\end{align*}
Die Legendre-Transformation hat dann die Form,
\begin{align*}
H(q^\alpha, p_\alpha,t) = \left[\sum_\alpha p_\alpha \dot{q}^\alpha -
L\right]_{q^\alpha,p_\alpha,t}.
\end{align*}
Um die Bewegungsgleichungen zu erhalten, betrachte das Differential,
\begin{align*}
\diffd H &= \sum_\alpha\left[ \frac{\partial H}{\partial p_\alpha}\ddp_\alpha +
\frac{\partial H}{\partial q^\alpha} \ddq^\alpha\right] + \frac{\partial
H}{\partial t}\dt \overset{!}{=} \diffd\left[\sum_\alpha p_\alpha\dot{q}^\alpha - L 
\right]\\
&= \sum_\alpha \left[\dot{q}^\alpha\ddp_\alpha +
p_\alpha\ddotq^\alpha  - \frac{\partial L}{\partial q^\alpha}\ddq^\alpha -
\frac{\partial L}{\partial \dot{q}_\alpha} \diffd\dot{q}^\alpha  \right] +
\frac{\partial L}{\partial t}\dt.
\end{align*}
Somit folgt für die Bewegungsgleichungen,
\begin{align*}
&\frac{\partial H}{\partial p_\alpha} = \dot{q}^\alpha,\\
&\frac{\partial H}{\partial q^\alpha} = -\dot{p}_\alpha.
\end{align*}
Die Hamiltonfunktion und die Hamilton'schen Bewegungsgleichungen sind
äquivalent zur Lagrangefunktion und den Euler-Lagrange-Gleichungen.
\begin{bemn}[Alternative Herleitung.]
Obwohl das totale Differential $\diffd H$ sehr elegant auf die Hamiltonschen
Bewegungsgleichungen führt, wollen wir noch eine alternative Herleitung
betrachten.
\begin{align*}
\frac{\partial H}{\partial p_\alpha} &= \frac{\partial}{\partial p_\alpha}\left[
\sum_\beta \dot{q}^\beta p_\beta -L \right] = \dot{q}^\alpha + \sum_\beta
p_\beta \frac{\partial\dot{q}^\beta}{\partial p_\alpha} - \frac{\partial
L}{\partial p_\alpha}
\\ &= \dot{q}^\alpha + \sum_\beta \frac{\partial \dot{q}^\beta}{\partial
p_\alpha}\left[p_\beta - \frac{\partial L}{\partial \dot{q}^\beta}\right] =
\dot{q}^\alpha.
\end{align*}
Die Rechnung für $\frac{\partial H}{\partial q^\alpha}$ funktioniert
analog.\maphere
\end{bemn}

\begin{bsp}
%\begin{enumerate}[label=\arabic{*}.)]
%  \item
 Sei $L(q,\dot{q}) = \frac{m}{2}\dot{q}^2 - V(q)$. Wir wollen die
  Hamiltonfunktion nun mit Hilfe der Legendretransformation berechnen,
\begin{align*}
p &= \frac{\partial L}{\partial \dot{q}} = m\dot{q} \Rightarrow \dot{q} =
\frac{p}{m}.\\
H(q,p) &= \dot{q}p - L = p\left(\frac{p}{m}\right) -
\frac{m}{2}\left(\frac{p}{m}\right)^2 + V(q) = \frac{p^2}{2m} + V(q).
\end{align*}
Die Hamiltonschen Bewegungsgleichungen lauten also,
\begin{align*}
&\dot{q} = \frac{\partial H}{\partial q} = \frac{p}{m},\\
&\dot{p} = -\frac{\partial H}{\partial q} = - V'(q) =
\frac{\diffd}{\dt}(m\dot{q}) = m\ddot{q}.\bsphere
\end{align*}
\end{bsp}
\begin{bsp}
%\item
 Wir betrachten die Lagrangefunktion in Polarkoordinaten,
\begin{align*}
L(r,\ph,\dot{r},\dot{\ph}) = \frac{m}{2}\left(\dot{r}^2 + r^2 \dot{\ph}^2
\right) - V(r).
\end{align*}
Die Impulse haben nun die Form,
\begin{align*}
&p_r = \frac{\partial L}{\partial \dot{r}} = m\dot{r} \Rightarrow \dot{r} =
\frac{p_r}{m},\\
&p_\ph = \frac{\partial L}{\partial \dot{\ph}} = mr^2\dot{\ph} \Rightarrow
\dot{\ph} = \frac{p_\ph}{mr^2}.
\end{align*}
Die Hamiltonfunktion ist daher
\begin{align*}
H(r,\ph,p_r,p_\ph) &= p_\ph \dot{\ph} + p_r \dot{r} - L(r,\ph,\dot{r},\dot{\ph})
\\ &= \frac{p_\ph^2}{mr^2} + \frac{p_r^2}{m} -
\frac{m}{2}\left(\frac{p_r}{m}\right)^2 -
\frac{m}{2}r^2\left(\frac{p_\ph}{mr^2}\right)^2 + V(r)
\\ &= \frac{1}{2}\frac{p_\ph^2}{mr^2} + \frac{1}{2}\frac{p_r^2}{m} + V(r).
\end{align*}
$\ph$ ist zyklisch, d.h. $p_\ph$ ist erhalten. Im Lagrange müssten wir noch
$\dot{\ph}$ durch $p_\ph$ ausdrücken, in der Hamiltonfunktion ist dies nicht
mehr notwendig.\bsphere
\end{bsp}
\begin{bsp}
%\item
Teilchen im äußeren EM-Feld.
\begin{align*}
&L(\vec{x},\dvec{x}) = \frac{m}{2}\dvec{x}^2 - e\left(\phi(\vec{x}) -
\frac{\dvec{x}}{c}\vec{A} \right)\\
&\vec{p}_x = m\dvec{x} + \frac{e}{c}\vec{A}
\end{align*}
Die Hamiltonfunktion hat nun die Form,
\begin{align*}
H(\vec{x},\vec{p}_x) &= \vec{p}\dvec{x} - L =
\frac{\vec{p}}{m}\left(\vec{p}-\frac{e}{c}\vec{A}(x) \right) -
\frac{1}{2m}\left(\vec{p}-\frac{e}{c}\vec{A}(x) \right)^2 \\ &\quad+ e\phi(x) 
- \frac{e}{cm}\left(\vec{p}-\frac{e}{c}\vec{A}(x) \right)\vec{A}(x)\\
&= \left(\vec{p}-\frac{e}{c}\vec{A}(x)
\right)\left[\frac{\vec{p}}{m}-\frac{1}{2m}\left(\vec{p}-\frac{e}{c}\vec{A}(x)
\right) - \frac{e}{cm}\vec{A}(x)\right] \\ &\quad+ e\phi(x)\\
&= \left(\vec{p}-\frac{e}{c}\vec{A}(x) \right)\left[\frac{\vec{p}}{2m} -
\frac{e}{2mc}\vec{A}(x) \right] + e\phi(x)\\
&= \frac{1}{2m}\left(\vec{p}-\frac{e}{c}\vec{A}(x) \right)^2 +
e\phi(x).\bsphere
\end{align*}
%\end{enumerate}
\end{bsp}

\begin{bemn}
Falls das Problem mit verallgemeinerten Koordinaten zeitunabhängig ist, so gilt,
\begin{align*}
L = T-V \Rightarrow H=T+V
\end{align*} 
und der Wert von $H$ ist die Energie, denn
\begin{align*}
H= \underbrace{\sum_\alpha \dot{p}^\alpha p_\alpha}_{2T} - \underbrace{L}_{T-V}
= T+V.
\end{align*}
Die allgemeine Form der kinetischen Energie ist,
\begin{align*}
T = \sum_{\alpha,\beta} g_{\alpha\beta} \dot{q}^\alpha\dot{q}^\beta.\maphere
\end{align*}
\end{bemn}

\subsection{Poisson-Klammern}
Wir betrachten eine Messgröße $F(q,p,t)$ und berechnen ihre Zeitableitung.
\begin{bsp}
$L_z = xp_y - yp_x$, Drehimpuls in $z$-Richtung.\bsphere
\end{bsp}
\begin{align*}
\frac{\diffd F}{\dt} &= \frac{\partial F}{\partial t} + \sum_\alpha \left[
\frac{\partial F}{\partial q^\alpha}\frac{\diffd q^\alpha}{\dt} +
\frac{\partial F}{\partial p_\alpha} \frac{\diffd p_\alpha}{\dt}
\right] =
\frac{\partial F}{\partial t} + \sum_\alpha \left[
\frac{\partial F}{\partial q^\alpha}\frac{\partial H}{\partial p_\alpha} -
\frac{\partial F}{\partial p_\alpha} \frac{\partial H}{\partial q^\alpha}
\right]\\
&\equiv \frac{\partial F}{\partial t} + \setd{F,H}.
\end{align*}
Dabei haben wir die \emph{Poissonklammer} eingeführt, die definiert ist als
\begin{align*}
\setd{A,B} := \sum_\alpha \left[
\frac{\partial A}{\partial q^\alpha}\frac{\partial B}{\partial p_\alpha} -
\frac{\partial A}{\partial p_\alpha} \frac{\partial B}{\partial q^\alpha}
\right].
\end{align*}
\begin{bemn}[Bemerkungen.]
\begin{enumerate}[label=\arabic{*}.)]
  \item 
Falls $F$ nicht explizit von der Zeit abhängig ist, d.h. $\frac{\partial
F}{\partial t} = 0$, ist $F(q,p)$ genau dann eine Erhaltungsgröße, wenn
$\setd{F,H} = 0$.
\item
Falls $H$ nicht explizit von der Zeit abhängig ist, ist $H$ erhalten,
\begin{align*}
\setd{H,H} = \sum_\alpha \left[
\frac{\partial H}{\partial q^\alpha}\frac{\partial H}{\partial p_\alpha} -
\frac{\partial H}{\partial p_\alpha} \frac{\partial H}{\partial q^\alpha}
\right].\maphere
\end{align*}
\end{enumerate}
\end{bemn}

\begin{bemn}[Spezialfälle.]
Sei $F(q^\alpha,p^\alpha)=q^\alpha$ oder $p_\alpha$, so ist $\frac{\partial
F}{\partial t} = 0$ und,
\begin{align*}
\dot{q}^\alpha &= \setd{q^\alpha,H} = \sum\limits_\beta \left[ \frac{\partial
q^\alpha}{\partial q^\beta}\frac{\partial H}{\partial p^\beta} - \frac{\partial
q^\alpha}{\partial p_\beta}\frac{\partial H}{\partial q^\beta} \right]
= \frac{\partial H}{\partial p^\alpha},\\
\dot{p}_\alpha &= \setd{p_\alpha, H} = \sum_\beta \left[ 
\frac{\partial p_\alpha}{\partial q^\beta}\frac{\partial H}{\partial p_\beta} -
\frac{\partial p_\alpha}{\partial p_\beta}\frac{\partial H}{\partial q^\beta}
\right]
= -\frac{\partial H}{\partial q^\alpha}.\maphere
\end{align*}
\end{bemn}

\begin{propn}[Eigenschaften]
Die Poisson-Klammer ist ein bilinearer Differentialoperator mit den
Eigenschaften,
\begin{enumerate}[label=(\roman{*})]
  \item $\setd{A,B+C}=\setd{A,B}+\setd{A,C}$,\qquad (\textit{Linearität}),
  \item $\setd{A,B} = -\setd{B,A} \Rightarrow \setd{A,A} = 0$,\qquad
  (\textit{Antisymmetrie}),
  \item $\setd{A,c} = 0$ für $c\in\R$,
  \item $\setd{A,\setd{B,C}} + \setd{B,\setd{C,A}} + \setd{C,\setd{A,B}} =
  0$,\qquad (\textit{Jacobi Identität}).\fishhere
\end{enumerate}
\end{propn}

\begin{bemn}
$\setd{q^\alpha,p_\beta} = \delta_{\alpha\beta}$.
\begin{proof}
$\sum_\gamma \underbrace{\frac{\partial q^\alpha}{\partial
q^\gamma}}_{\delta_{\alpha\gamma}}
\underbrace{\frac{\partial p_\beta}{p_\gamma}}_{\delta_{\beta\gamma}} -
\underbrace{\frac{\partial q^\alpha}{\partial
p_\gamma}}_{0}\underbrace{\frac{\partial p^\beta}{\partial
q^\gamma}}_{0}$.\qedhere\maphere
\end{proof}
\end{bemn}

\subsubsection{Symplektische Notation}

Die Hamiltonschen Bewegungsgleichungen lassen sich in einer Gleichung in
folgender Form zusammenfassen,
\begin{align*}
\begin{pmatrix}
\dot{q}\\\dot{p}
\end{pmatrix}
=
\underbrace{\begin{pmatrix}
0 & 1\\
-1 & 0
\end{pmatrix}}_{\ep_{ij}}
\begin{pmatrix}
\frac{\partial H}{\partial q}\\
\frac{\partial H}{\partial p}
\end{pmatrix},
\end{align*}
wobei $\ep_{ij}$ den \emph{antisymmetrischen Tensor} bezeichnet.

Wir definieren nun einen neuen Vektor,
\begin{align*}
&\sym{x} = 
\begin{pmatrix}
q^1\\ p^1 \\ \vdots \\ q^f\\p^f
\end{pmatrix}
=
\begin{pmatrix}
x^1\\\vdots\\ x^{2f}
\end{pmatrix},\\
&\frac{\partial x^i}{\partial t}
= 
\sum_j \ep_{ij} \frac{\partial H}{\partial x_j},
\end{align*}
mit dem total antisymmetrischen Tensor
\begin{align*}
\ep =
\begin{pmatrix}
\begin{matrix}
0 & 1\\
-1 & 0
\end{matrix}\\
& \ddots\\
&& \begin{matrix}
0 & 1\\
-1 & 0
\end{matrix}
\end{pmatrix}.
\end{align*}
Zur Abkürzung führen wir folgende Notation ein,
\begin{align*}
&\frac{\diffd \sym{x}}{\dt} = \ep\frac{\partial H}{\partial \sym{x}} =
\ep\nabla H(\sym{x})\\
& \setd{A,B} = \sum_{i,j} \frac{\partial A}{\partial x^i} \ep_{ij}
\frac{\partial B}{\partial x^j} = \frac{\partial A}{\partial
\sym{x}}\ep\frac{\partial B}{\partial \sym{x}}.
\end{align*}

\subsection{Extremalprinzip}

Analog zum Lagrange existiert ein Extremalprinzip zu dem die Hamiltonschen
Bewegungsgleichungen äquivalent sind.

Für die Lagrangefunktion impliziert eine Variation von $q(t)$ auch eine
Variation von $\dot{q}(t)$. Die Euler-Lagrange-Gleichungen sind das Extremal
der Wirkung
\begin{align*}
S=\int\limits_{t_1}^{t_2}\dt L(q(t),\dot{q}(t),t).
\end{align*}

Für die Hamiltonfunktion sind nun $q(t)$ und $p(t)$ voneinander unabhängige
Größen. Übertragen wir die Wirkung für die Lagrangefunktion auf den Hamilton,
ergibt sich,
\begin{align*}
S = \int\limits_{t_1}^{t_2} \dt \underbrace{\sum_\alpha p_\alpha \dot{q}^\alpha
- H(q^\alpha,p_\alpha,t)}_{L'},
\end{align*}
wobei $L'$ hier für den Zahlenwert der Lagrangefunktion zum Zeitpunkt $t$
steht.

Anwendung der Euler-Lagrange-Gleichungen auf die unabhängigen Größen
$q^\alpha$ und $p^\beta$ ergibt,
\begin{align*}
&\frac{\diffd}{\dt}\frac{\partial L'}{\partial \dot{q}^\alpha} - \frac{\partial
L}{\partial q^\alpha} = 0,\\
&\frac{\diffd}{\dt}\frac{\partial L'}{\partial \dot{p}^\alpha} - \frac{\partial
L}{\partial p^\alpha} = 0.
\end{align*}
Einsetzen des Ausdrucks für $L'$ führt wieder auf die Hamiltonschen
Bewegungsgleichungen,
\begin{align*}
&\frac{\partial L'}{\partial \dot{q}^\alpha} = p_\alpha\\
\Rightarrow &\frac{\diffd}{\dt}p_\alpha = \dot{p}_\alpha = \frac{\partial
L'}{\partial q^\alpha} = -\frac{\partial H}{\partial q^\alpha}\\
&\frac{\partial L'}{\partial \dot{p}_\alpha} = 0\\
\Rightarrow &0 = \frac{\partial L'}{\partial p_\alpha} = \dot{q}^\alpha -
\frac{\partial H}{\partial p_\alpha} \Rightarrow \dot{q}^\alpha =
\frac{\partial H}{\partial p_\alpha}.
\end{align*}
\begin{bemn}
Die Wirkung entlang einer physikalischen Bahn mit $H=E$ ist gegeben durch,
\begin{align*}
S = \int\limits_{t_1}^{t_2} \dt \left[\sum_\alpha p_\alpha\dot{q}^\alpha -H
\right] = 
\sum_\alpha \int_\gamma p_\alpha \ddq^\alpha - E(t_2-t_1).\maphere
\end{align*}
\end{bemn}
\begin{figure}[htbp]
\centering
\begin{pspicture}(0,-1.41)(3.3,1.41)
\psline{->}(0.22,-1.19)(0.24,1.27)
\psline{->}(0.04,-1.01)(3.2,-1.01)
\psbezier[linecolor=darkblue]{**-**}(0.74,-0.49)(1.0719048,0.45040816)(2.1,-0.59)(2.38,0.47)

\rput(3.17,-1.185){\color{gdarkgray}$q^\alpha$}
\rput(0.05,1.255){\color{gdarkgray}$t$}
\rput(0.68,-0.645){\color{gdarkgray}$t_1$}
\rput(2.56,0.555){\color{gdarkgray}$t_2$}
\end{pspicture}

\caption{Wirkung entlang einer Trajektorie}
\end{figure}

\subsection{Kanonische Transformationen}
 
Wir wollen nun untersuchen, welche Transformationen die Hamiltonfunktion
invariant lasssen. Für die Newtonschen Gleichungen waren das lediglich die
Galilei-Transformationen. Der Lagrangeformalismus hingegen ist invariant unter
beliebigen Transformationen, d.h.
 \begin{align*}
 &Q^\alpha = f(q^\beta,t)\\
 &\dot{Q}^\alpha = \frac{\diffd}{\dt}f(q^\beta,t)
 \end{align*}
für jede Transformation $f$ im Konfigurationsraum. Beispielsweise ließen sich
durch das Einführen von Kugelkoordinaten zahlreiche Probleme leicht auflösen bzw. reduzieren.

 Im Hamiltonformalismus sind $q^\alpha$ und $p^\alpha$ von einander unabhängig,
 d.h. Transformationen können $p^\alpha$ ändern während sie $q^\alpha$ invariant
 lassen und umgekehrt.
 %Im Lagrangeformalismus waren durch die feste
 %Verknüpfung von $q$ und $\dot{q}$ beliebige Transformationen zugelassen. 

Die Klasse von Transformationen im Phasenraum ($(q,p)$-Raum), die die
Hamiltonschen Bewegungsgleichungen
invariant lassen heißen \emph{kanonische Transformationen}.

Wir wollen uns zunächst auf die wichtigste Untergruppe dieser Transformationen
\begin{align*}
(q^\alpha,p^\alpha)\mapsto (Q^\alpha,P^\alpha)
\end{align*}
beschränken, die die folgenden Eigenschaften erfüllen:
\begin{enumerate}[label=(\roman{*})]
  \item zeitunabhängig,
  \item kontinuierlich in einem Parameter $s$.
\end{enumerate}

Wir betrachten eine Messgröße $F$, die nicht explizit von der Zeit abhängt,
entlang einer Bewegungskurve. Aufgrund der Zeitunabhängigkeit der
Transformation ist $\frac{\diffd F}{\dt}$ invariant,
\begin{align*}
\frac{\diffd F}{\dt} = \setd{F,H}_{q,p} = \setd{F,H}_{Q,P},
\end{align*}
d.h. die Poissonklammer in neuen und alten Koordinaten ist identisch.
Da dies für alle Hamiltonfunktionen gelten muss, folgt allgemein
\begin{align*}
&\setd{A,B}_{q,p} = \setd{A,B}_{Q,P},\\
&\sum_\alpha \frac{\partial A}{\partial q^\alpha}\frac{\partial B}{\partial
p_\alpha} - \frac{\partial A}{\partial p_\alpha}\frac{\partial B}{\partial
q^\alpha}
=
\sum_\alpha \frac{\partial A}{\partial Q^\alpha}\frac{\partial B}{\partial
P_\alpha} - \frac{\partial A}{\partial P_\alpha}\frac{\partial B}{\partial
Q^\alpha},
\end{align*}
bzw. in symplektischer Schreibweise,
\begin{align*}
\sum_{i,j} \frac{\partial A}{\partial \sym{x}_i}\ep_{ij}\frac{\partial
B}{\partial \sym{x}_j}
=
\sum_{i,j} \frac{\partial A}{\partial \sym{y}_i}\ep_{ij}\frac{\partial
B}{\partial \sym{y}_j},
\end{align*}
wobei
\begin{align*}
\sym{x} =
\begin{pmatrix}
q^1\\p^1\\\vdots \\ q^f \\ p^f
\end{pmatrix},\qquad
\sym{y} =
\begin{pmatrix}
Q^1\\P^1\\\vdots \\ Q^f \\ P^f
\end{pmatrix}.
\end{align*}
\begin{bemn}[Spezialfall.]
$A=P_\alpha$, $B=Q^\beta$ $\Rightarrow$ $\setd{P_\alpha,Q^\beta}_{q,p} =
\setd{P_\alpha, Q^\beta}_{Q,P} = \delta_{\alpha\beta}$.

\begin{bsp}
Sei $Q^\beta = \lambda q^\beta$, so folgt $P_\alpha =
\frac{1}{\lambda}p_\alpha$, denn $\setd{Q^\alpha,P_\alpha} =
\delta_{\alpha\beta}$.\bsphere
\end{bsp}

Seien $\sym{y}$ die neuen und $\sym{x}$ die alten Koordinaten in symplektischer
Notation, so ergibt sich,
\begin{align*}
\setd{y_i,y_j}_x = \ep_{ij} = \sum_{l,m} \frac{\partial y_i}{\partial
x_k}\ep_{lm} \frac{\partial y_j}{\partial x_m} = \ep_{ij}
\end{align*}
Bezeichne $M_{il}:=\frac{\partial y_i}{\partial x_l}$ die Funktionalmatrix der
kanonischen Transformation, so gilt
\begin{align*}
M\ep M^\top = \ep.\maphere
\end{align*}
\end{bemn}

\begin{defnn}
Die \emph{Symplektische Gruppe}
\begin{align*}
Sp(f,\R) := \setdef{M \in \R^{2f\times 2f}}{M\ep M^\top =\ep}.
\end{align*}
ist die Gruppe der $2f\times 2f$-Matritzen mit $M\ep M^\top = \ep$.\fishhere
\end{defnn}

Entwickeln wir $\sym{y}$ in $s$, so ergibt sich,
\begin{align*}
y_i = x_i + s\cdot v_i + o(s^2),
\end{align*}
mit $\sym{v}_i(x) = \frac{\partial y_i}{\partial s}\bigg|_{s=0}$.
Differentiation der Entwicklung nach $x_k$ ergibt,
\begin{align*}
M_{ik}  = \frac{\partial y_i}{\partial x_k} = \delta_{ik} + s
\underbrace{\frac{\partial v_i}{\partial x_k}}_{=:m_{ik}},
\end{align*}
d.h. wir können die Funktionalmatrix $M$ in $s$ entwickeln,
\begin{align*}
M = \Id + s\ m + o(s^2).
\end{align*}
Einsetzen in die Bedingung für eine kanonische Transformation ergibt,
\begin{align*}
\ep &= M\ep M^\top = \left(\Id + s\ m + o(s^2)\right)\ep\left(\Id + s\ m +
o(s^2) \right) \\ &= \ep + s\left[m\ep + \ep m^\top \right] + o(s^2)\\
\Rightarrow & m\ep + \ep m^\top = 0
\end{align*}
Multiplikation von rechts und links mit $\ep$ unter verwendung von $\ep^2 =\Id$
und $\ep^\top = -\ep$ ergibt,
\begin{align*}
\ep m \ep^2 + \ep^2 m^\top\ep = 0
\Leftrightarrow
\ep m -m^\top e^\top  = 0
\Leftrightarrow
\ep m - (\ep m)^\top = 0.\tag{*}
\end{align*}
Setze nun
\begin{align*}
\sum_j e_{ij} m_{jk} = \sum_j e_{ij}\frac{\partial v_j}{\partial x_k}
= \frac{\partial}{\partial x_k}\underbrace{\left(\sum_j \ep_{ij}v_j
\right)}_{=:g_i},
\end{align*}
so nimmt (*) die Form an,
\begin{align*}
\frac{\partial g_i}{\partial x_k} - \frac{\partial g_k}{\partial x_i} = 0.
\end{align*}
In drei Dimensionen wäre dies äquivalent damit, dass $\rot g = 0$. Hier haben
wir die höherdimensionale Verallgemeinerung. Analog zum dreidimensionalen Fall
existiert daher ein Skalarfeld $G$ so, dass
\begin{align*}
g_i = -\nabla_i G = - \frac{\partial G}{\partial x_i}(\sym{x}). 
\end{align*}
$G$ heißt \emph{Erzeugende} Funktion für die kanonische Transformation (bzw. den
kanonischen Fluss).

\subsubsection{Zusammenfassung}

\begin{enumerate}[label=\arabic{*}.)]
  \item $v_i = \frac{\diffd y_i}{\ds}\bigg|_{s=0} = -\sum_j \ep_{ij}g_j =
  \sum_j \ep \frac{\partial G}{\partial y_j}$.
  \item Zu jeder kanonischen Transformation $g$ existiert eine erzeugende
  Funktion $G$.
  \item Jede Funktion $G$ erzeugt eine kanonische Transformation.

Insbesondere erzeugt die Hamiltonfunktion eine kanonische Transformation. Diese
ist die Transformation im Koordinatenraum, die zu gegebenen Anfangsbedingungen
die Hamiltonschen Bewegungsgleichungen liefert.
\end{enumerate}

\begin{bsp}
\textit{Erzeugende für eine Drehung um die $z$-Achse}. Die Drehung ist
charakterisiert durch,
\begin{align*}
\vec{y} = \vec{x} + s\vec{v}\times\vec{x}.
\end{align*}
Dabei ist,
\begin{align*}
&\frac{\diffd y_1}{\ds} = -x_2\\
&\frac{\diffd y_2}{\ds} = x_1\\
&\frac{\diffd y_3}{\ds} = 0.
\end{align*}
D.h. $G=L_3 = x_1p_2 - x_2p_1$, die $z$-Komponente des Drehimpulses, ist
Erzeugende für die Rotation.

Ananlog sieht man, dass der Impuls die Erzeugende für die Translation im Raum
ist.\bsphere
\end{bsp}

\subsubsection{Symmetrien und Erhaltungssätze}

Eine kanonische Transformation $g$ ist eine \emph{Symmetrie}, wenn sie die
Hamiltonfunktion nicht ändert,
\begin{align*}
0 = \frac{\diffd H}{\ds}(\sym{y}) = \sum_i \frac{\partial H}{\partial
y_i}\frac{\partial y_i}{\partial s} =
\sum_{i,j} \frac{\partial H}{\partial y_i}\ep_{ij}\frac{\partial
G}{\partial y_j} = \setd{H,G}.
\end{align*}
D.h. $g$ ist genau dann eine Symmetrie, wenn die Poissonklammer von $H$ und
ihrer Erzeugenden $G$ verschwindet.

Die Erzeugende einer Symmetrie ist daher stets eine Erhaltungsgröße, denn
\begin{align*}
\frac{\diffd G}{\dt} = \setd{G,H} = 0.
\end{align*}

\subsection{Endliche kanonische Transformationen}

Die Transformation lässt die Bewegungsgleichungen
\begin{align*}
&\dot{q}^\alpha = \frac{\partial H}{\partial p_\alpha},\quad
\dot{p}_\alpha = -\frac{\partial H}{\partial q^\alpha}
\end{align*}
invariant. In neuen Koordinaten,
\begin{align*}
&\dot{Q}^\alpha = \frac{\partial H'}{\partial P_\alpha},\quad
\dot{P}_\alpha = -\frac{\partial H'}{\partial Q^\alpha}.
\end{align*}
Falls $H$ zeitabhängig ist, erhalten wir eine neue Hamiltonfunktion $H'$.

Die Hamiltonschen Bewegungsgleichungen sind äquivalent zum Hamiltonschen
Prinzip der kleinsten Wirkung,
\begin{align*}
&\delta\left(\sum_\alpha p_\alpha\dot{q}^\alpha - H\right) = 0,\\
&\delta\left(\sum_\alpha P_\alpha\dot{Q}^\alpha - H'\right) = 0,
\end{align*}
d.h. $\sum_\alpha p_\alpha\dot{q}^\alpha - H$ und $\sum_\alpha
P_\alpha\dot{Q}^\alpha - H'$ dürfen sich lediglich um eine totale
Zeitableitung unterscheiden. Dies ergibt eine neue Funktion,
\begin{align*}
&\frac{\diffd F}{\dt} = \sum_\alpha p_\alpha \dot{q}^\alpha -
P_\alpha\dot{Q}^\alpha - (H-H'),\\
&\diffd F = \sum_\alpha p_\alpha \ddq^\alpha -
P_\alpha\dQ^\alpha - (H-H')\dt.
\end{align*}

Wir nehmen an, dass $F(q,Q,t)$ die Form,
\begin{align*}
\diffd F = \sum_\alpha p_\alpha \ddq^\alpha -
P_\alpha\dQ^\alpha - \frac{\partial F}{\partial t}\dt
\end{align*}
hat. Dadurch ergeben sich die Bedingungen,
\begin{align*}
\frac{\partial F}{\partial q^\alpha} = p_\alpha,\quad
-\frac{\partial F}{\partial Q^\alpha} = P_\alpha,\quad
\frac{\partial F}{\partial t} = H-H'.
\end{align*}
Unter diesen Bedingungen erzeugt $F$ eine kanonische Transformation.

\begin{bsp}
Betrachte den Harmonischen Oszillator mit,
\begin{align*}
H=\frac{p^2}{2m} + k\frac{q^2}{2},\quad \omega^2=\frac{k}{m}.
\end{align*}
Durch kühne Überlegung erhält man die Erzeugende,
\begin{align*}
F = \frac{m\omega q^2}{2}\cot(Q),
\end{align*}
dann ist
\begin{align*}
&p = \frac{\partial F}{\partial q} = m\omega q \cot Q,\\
&P = -\frac{\partial F}{\partial Q} = -\frac{m\omega q^2}{2}\frac{1}{\sin^2 Q}.
\end{align*}
Auflösen der Gleichung ergibt,
\begin{align*}
&q = \sqrt{\frac{2P}{m\omega}}\sin Q\\
&p = \sqrt{2Pm\omega}\cos Q.
\end{align*}
Die neue Hamiltonfunktion hat die Form,
\begin{align*}
H' &= H + \underbrace{\frac{\partial F}{\partial t}}_{=0}
= \frac{2 Pm\omega \cos^2 Q}{2m} + \omega^2 m\frac{2P}{\omega}\frac{\sin^2
Q}{2}\\ &= P\omega \cos^2 Q + P\omega \sin^2 Q = P\omega.
\end{align*}
Nach der kanonischen Transformation ist $Q$ eine zyklische Koordinate,
\begin{align*}
&Q = \omega t + \ph,\\
&E = \omega P \Rightarrow P = \frac{E}{\omega}.\bsphere
\end{align*}
\end{bsp}

\subsection{Hamilton-Jacobi Differentialgleichungen}

Wir wollen nun untersuchen, welche kanonische Transformation $F_2$ zur Folge
hat, dass
\begin{align*}
&H'(Q^\alpha, P_\alpha) = \const = 0,\\
&\dot{Q}^\alpha = \frac{\partial H'}{\partial p_\alpha} = 0\\
&\dot{P}_\alpha = -\frac{\partial H'}{\partial q^\alpha} = 0.
\end{align*}
Wir erhalten so eine Differentialgleichung, deren Lösung die gesuchte kanonische
Transformation ist.

Die Erzeugende dieser Transformation ist die Wirkung,
\begin{align*}
S = \int\limits_{t_1}^{t_2}\dt L.
\end{align*}

Betrachte nun zu festem Anfangspunkt $q^\alpha$, $S(q^\alpha,t)$ als Funktion
des Endpunkts für \textit{physikalische Bahnen}. Für die Ortsableitung ergibt
sich,
\begin{align*}
\frac{\partial S}{\partial q^\alpha} &= \frac{\partial}{\partial
q^\alpha}\int\limits_{t_1}^{t} \dt L = 
\int\limits_{t_1}^t \ds \frac{\partial L}{\partial q^\alpha}
+ \sum_\beta \frac{\partial L}{\partial \dot{q}^\beta}\frac{\partial
\dot{q}^\beta}{\partial q^\alpha}
\\ &= \underbrace{\int\limits_{t_1}^t \ds \left[\frac{\partial L}{\partial
q^\alpha} - \frac{\diffd}{\dt}\frac{\partial L}{\partial q^\alpha}\right]}_{=0}
+ \frac{\partial L}{\partial \dot{q}^\alpha}\bigg|_{t_1}^t = p_\alpha.
\end{align*}
Betrachte nun $q^\alpha$ als fixiert und differenziere nach $t$,
\begin{align*}
&\frac{\diffd S}{\dt} =\frac{\diffd }{\dt}\int\limits_{t_1}^t \dt L = L
= \frac{\partial S}{\partial t} + \sum_\alpha \frac{\partial S}{\partial
q^\alpha}\frac{\partial q^\alpha}{\partial \dot{q}^\alpha}\\
\Rightarrow & \frac{\partial S}{\partial t} = \frac{\diffd S}{\dt} -
\sum_\alpha p_\alpha\dot{q}^\alpha =- H.
\end{align*}

Die Wirkungsfunktion $S(q^\alpha,t)$ erfüllt die folgenden
Differentialgleichungen,
\begin{align*}
&\frac{\partial S}{\partial t} = -H, \quad
\frac{\partial S}{\partial q^\alpha} = p_\alpha\\
&\diffd S = \sum_\alpha p_\alpha \ddq^\alpha - H\dt
\end{align*}
Damit folgt die \emph{Hamilton-Jacobi Differentialgleichung},
\begin{align*}
\frac{\partial S}{\partial t} = -H(q^\alpha,p_\alpha,t) =
-H(q^\alpha,\frac{\partial S}{\partial q^\alpha},t).
\end{align*}
$S(q^\alpha,t)$ ist somit durch eine partielle Differentialgleichung 1. Ordnung
in $f+1$ Variablen bestimmt. Für partielle Differentialgleichungen existieren
weitreichende Lösungsverfahren und -sätze, mit denen man sich in der
Quantenelektrodynamik ausführlich beschäftigt.

Die Lösung ist durch $f+1$ Integrationskonstanten bestimmt. Eine
Integrationkonstante ist trivial,
\begin{align*}
S(q^\alpha,t,a^\alpha),\quad \alpha = 1,\ldots,f.
\end{align*}
Die Lösung dieser Differentialgleichung können wir als Erzeugende einer
kanonische Transformation mit $p_\alpha = a^\alpha$ auffassen,
\begin{align*}
&F_2(q^\alpha,p_\alpha) = S(q^\alpha, t, a^\alpha).\\
&p_\alpha = \frac{\partial S}{\partial q^\alpha}\\
&Q^\alpha = \frac{\partial S}{\partial p_\alpha}\\
&H' = H+\frac{\partial S}{\partial t} = 0.
\end{align*}

\begin{bsp}
Wir betrachten erneut den harmonischen Oszillator,
\begin{align*}
H = \frac{p^2}{2m} + m\omega^2q^2,
\end{align*}
und lösen die Hamilton-Jacobi Differentialgleichungen.

Separation führt auf,
\begin{align*}
S(q^\alpha,t) = S_0(q^\alpha) + S_1(t).
\end{align*}
Für den zeitabhängigen Teil ist
\begin{align*}
&\frac{\partial S_1}{\partial t} =
-\frac{1}{2m}\left(\frac{\partial S_0(q)}{\partial q} \right)^2 - m\omega^2
q^2\\
\Rightarrow &
\frac{\partial S_1}{\partial t} = -a^1 
\Rightarrow  
\frac{1}{2m}\left(\frac{\partial S_0(q)}{\partial q} \right)^2 + m\omega^2 q^2
= a^1
\end{align*}
Für den Ortsanteil erhalten wir so,
\begin{align*}
\frac{\partial S_0}{\partial q} = \sqrt{2m}\sqrt{a^1 -m\omega^2 q^2}
\end{align*}
dies kann man integrieren,
\begin{align*}
S_0(q) = \int \ddq \sqrt{2m}\sqrt{a^1 - m\omega^2 q^2}. 
\end{align*}
Die Lösung hat daher die Form,
\begin{align*}
S(q,t) = \int \ddq \sqrt{2m}\sqrt{a^1 - m\omega^2 q^2} + a^1 t.
\end{align*}
Man kann hiervon eine analytische Lösung berechnen. Betrachten wir die neuen
Koordinaten,
\begin{align*}
&P_\alpha = a^1,\quad Q^\alpha = \frac{\partial S}{\partial a^1}
= \int\ddq \sqrt{\frac{m}{2}} \frac{1}{\sqrt{a^1 - m\omega^2 q^2}} - t\\
\Rightarrow &
Q^\alpha + t =
\frac{1}{\omega}\arcsin\left(q\sqrt{\frac{m\omega^2}{2a^1}}\right).
\end{align*}
In den neuen Koordinaten $(Q^\alpha,P_\alpha)$ verschwindet die
Hamiltonfunktion $H'=0$, d.h. $P=\const$, $Q=\const$ und daher
\begin{align*}
&q = \sqrt{\frac{2P}{m\omega^2}}\sin\left(\omega(t+Q) \right)\\
&p = \ldots
\end{align*}
D.h. in den neuen Koordinaten ist der Impuls $P$ die Amplitude und $Q$ die
Phase des Oszillators, beide Konstanten.\bsphere
\end{bsp}
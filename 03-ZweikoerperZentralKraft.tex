\section{Zweikörper Zentralkraft Problem}
%TODO: Bild
Als Anwendung des Lagrange Formalismus untersuchen wir zwei Körper mit Massen
$m_1$ und $m_2$ in einem rotationssymmetrischen Wechselwirkungspotential $V$.

Das Zweikörperproblem ist exakt lösbar und von besonderem Interesse, da es die
Bewegung der Erde um die Sonne mit hinreichender Genauigkeit beschreibt.
Später werden wir so auch die \emph{Keplerschen Gesetze} herleiten, zunächst
wollen wir uns jedoch allgemein mit dem Problem befassen.

Zur Lösung des Zweikörperproblems werden wir die entwickelten Erhaltungsgrößen
verwenden, durch die die exakte Lösung sehr einfach zu bestimmen ist. Die
Lagrangefunktion hat hier die Form,
\begin{align*}
L(\vec{r}_1,\vec{r}_2,\dvec{r}_1,\dvec{r}_2) = \frac{m_1}{2}\dvec{r}_1^2 +
\frac{m_2}{2}\dvec{r}_2^2 - V(\abs{\vec{r}_1-\vec{r}_2}).
\end{align*}
Da unsere Vektoren 3-dimensional sind, erhalten wir einen 6-dimensionalen
Konfigurationsraum, d.h. ein System von 6 gekoppelte Differentialgleichungen.
Mit Hilfe der Symmetrien können wir jedoch die Freiheitsgrade so reduzieren,
dass das Problem exakt lösbar wird.

\subsection{Reduktion auf 1-Teilchen Problem}
Dank der Impulserhaltung und dem Schwerpunktsatz wissen wir, dass der
Schwerpunkt einer trivialen Bewegung folgt. Dies können wir durch das
Einführen von Schwerpunkts- $\vec{R}$ und Relativkoordinaten $\vec{r}$
ausnutzen,
\begin{align*}
&\vec{R} = \frac{m_1\vec{r}_1+m_2\vec{r}_2}{m_1+m_2},\\
&\vec{r} = \vec{r}_2-\vec{r}_1.
\end{align*}
Die bisherigen Koordinaten haben dann die Form,
\begin{align*}
&\vec{r}_1 = \vec{R} - \frac{m_2}{m_1+m_2}\vec{r},\\
&\dvec{r}_1 = \dvec{R} - \frac{m_2}{m_1+m_2}\dvec{r},\\
&\vec{r}_2 = \vec{R} + \frac{m_1}{m_1+m_2}\vec{r},\\
&\dvec{r}_2 = \dvec{R} + \frac{m_1}{m_1+m_2}\dvec{r}.
\end{align*}
Dadurch erhalten wir für die kinetische Energie den Ausdruck,
\begin{align*}
T &= \frac{m_1}{2}\left(\dvec{R} - \frac{m_2}{m_1+m_2}\dvec{r}\right)^2
+ \frac{m_2}{2}\left(\dvec{R} + \frac{m_1}{m_1+m_2}\dvec{r}\right)^2\\
&= \frac{1}{2}\left[(m_1+m_2)\dvec{R}^2 + \frac{m_1m_2^2}{(m_1+m_2)^2}\dvec{r}^2
+ \frac{m_2m_1^2}{(m_1+m_2)^2}\dvec{r}^2\right]\\
&= \frac{1}{2}\left[(m_1+m_2)\dvec{R}^2 +
\frac{m_1m_2}{m_1+m_2}\dvec{r}^2\right]\\
&= \frac{1}{2}\left[(m_1+m_2)\dvec{R}^2 +
\mu\dvec{r}^2\right],
\end{align*}
wobei wir $\mu=\dfrac{m_1m_2}{m_1+m_2}$ als \emph{reduzierte Masse} einführen.
Die Lagrangefunktion vereinfacht sich somit zu,
\begin{align*}
L(\vec{R},\vec{r},\dvec{R},\dvec{r}) = \frac{1}{2}\left[(m_1+m_2)\dvec{R}^2 +
\mu\dvec{r}^2\right] - V(\abs{\vec{r}}).
\end{align*}
Offensichtlich ist $\vec{R}$ eine zyklische Koordinate und es gilt,
\begin{align*}
\vec{p} = \frac{\partial L}{\partial \vec{R}} = (m_1+m_2)\dvec{R} =\const =
(m_1+m_2)\vec{v}.
\end{align*}
Daraus ergibt sich sofort der Schwerpunktssatz,
\begin{align*}
\vec{R}(t) = \vec{R}_0 + \vec{v}t.
\end{align*}
Es genügt daher nur noch die Relativkoordinate zu betrachten. Wir erhalten ein
effektives Einteilchenproblem mit Koordinate $\vec{r}$ und effektiver Masse
$\mu$. Das Problem hat sich um 3 Freiheitsgrade reduziert.

Im Folgenden betrachten wir somit den Lagrange 
\begin{align*}
L(\vec{r},\dvec{r}) = \frac{1}{2}\mu\dvec{r}^2 - V(\abs{\vec{r}}).
\end{align*}
\begin{bemn}[Bemerkungen.]
Betrachten wir den Grenzafll $m_2>>m_1$, dann gilt
\begin{align*}
&\mu = \frac{m_1m_2}{m_1+m_2}\approx \frac{m_1m_2}{m_2} = m_1,\\
&\vec{R} = \frac{m_1\vec{r}_1+m_2\vec{r}_2}{m_1+m_2} \approx 
\frac{m_1}{m_2}\vec{r}_1 + \vec{r}_2\approx \vec{r}_2.
\end{align*}
Dies entspricht dem System Erde-Sonne, bei dem der Schwerpunkt in der Sonne
liegt und die Erde um die Sonne kreist.

Ist hingegen $m_1=m_2=m$, so gilt
\begin{align*}
&\mu = \frac{m_1m_2}{m_1+m_2} = \frac{m^2}{2m} = \frac{1}{2}m,\\
&\vec{R} = \frac{m_1\vec{r}_1+m_2\vec{r}_2}{m_1+m_2} =
\frac{1}{2}\left(\vec{r}_1+\vec{r}_2\right).
\end{align*}
Der Schwerpunkt liegt im Mittelpunkt der Verbindungsstrecke der Massen und die
Massen kreisen darum.\maphere
\end{bemn}

\subsection{Integration der Bewegungsgleichung}
Wir wollen die Bewegungsgleichung nun exakt lösen. Dazu müssen wir das 
Potential $V$, das als rotationssymmetrisch vorausgesetzt ist, etwas genauer
studieren. Betrachten eine Rotation im Raum,
\begin{align*}
&\vec{r}\mapsto \vec{M}\vec{r},\qquad \vec{M}\in\mathrm{SO}(3),\\
&\abs{\vec{r}} = \abs{\vec{M}\vec{r}} =
\sqrt{\lin{\vec{M}\vec{r},\vec{M}\vec{r}}} =
\sqrt{\lin{\vec{M}\vec{M}\vec{r},\vec{r}}}
= \sqrt{\lin{\vec{r},\vec{r}}} = \abs{r}.
\end{align*}
Die Länge des Vektors $\vec{r}$ ist also invariant unter Rotation. Aufgrund der
Rotationssymmetrie des Potentials ist der Drehimpuls
\begin{align*}
\vec{L} = \vec{r}\times\vec{p},
\end{align*}
eine Erhaltungsgröße mit $\abs{\vec{L}}\equiv l = \const$. Es gilt daher
\begin{align*}
\lin{\vec{r},\vec{L}} = \lin{\vec{r},\vec{r}\times\vec{p}} =
\lin{\vec{p},\vec{r}\times\vec{r}} = 0.
\end{align*}
Aufgrund der Konstanz von $\vec{L}$ steht somit $\vec{r}$ stets senkrecht auf
$\vec{L}$ und daher verläuft die Kurve $t\mapsto \vec{r}(t)$ in einer Ebene.
\begin{bemn}
Falls $l=0$ bricht die obige Argumentation zusammen. In diesem Fall ist die
Bewegung jedoch geradlinig und findet daher ebenfalls in einer Ebene
statt.\maphere
\end{bemn}

\begin{figure}[!htbp]
  \centering
\begin{pspicture}(0,-1.07)(2.4,1.07)
\psline(0.0,-0.37)(1.6,1.05)
\psline(0.0,-0.37)(0.8,-1.05)
\psline(0.8,-1.05)(2.38,0.37)
\psline(1.6,1.05)(2.38,0.37)

%\psline{->}(1.06,0.01)(1.7,0.21)
\psarc(1.08,0.07){0.2}{5}{153.43495}
\psdots[dotsize=0.02](1.1,0.13)
\psline{->}(1.06,0.01)(0.34,0.69)

\psbezier[linecolor=darkblue]{->}(1.0658889,0.02411111)(1.4258889,0.2641111)(1.7058889,-0.2758889)(1.9658889,0.4441111)
\psarc[linewidth=0.02]{->}(0.4558889,0.5741111){0.17}{201.0375}{127.874985}

\psdots[dotsize=0.1](1.06,0.01)

\rput(0.2025,0.9){\color{gdarkgray}$\vec{L}$}
\rput(1.72,0.44){\color{gdarkgray}$\vec{r}$}
\end{pspicture}
  \caption{Wahl der Bewegungsebene.}
\end{figure}

Nun kann diese Ebene beliebig im Raum liegen, wir können aber ein
Koordinatensystem so wählen, dass die $z$-Achse entlang der Drehimpulsachse
verläuft. Die Bewegung findet dann in der $x,y$-Ebene statt. Es
bietet sich nun die Wahl von Polarkoordianten an, wodurch der Lagrange die Form
annimmt,
\begin{align*}
L(r,\ph,\dot{r},\dot{\ph}) = \frac{1}{2}\mu\left(\dot{r}^2
+r^2\dot{\ph}^2\right) - V(r).
\end{align*}
Die Koordinate $\ph$ ist zyklisch und die Erhaltungsgröße
\begin{align*}
p_\ph = \frac{\partial L}{\partial \dot{\ph}} = \mu r^2\dot{\ph} \equiv l.
\end{align*}
Die Drehimpulserhaltung eliminiert somit 2 Freiheitsgrade. Die Richtung des
Drehimpulses schränkt die Bewegung auf eine Ebene ein ($r,\ph$), während der
Betrag des Drehimpulses $\ph$ eliminiert. Es bleibt ein eindimensionales
Problem zu lösen.
\begin{bemn}
Das 2. Keplersche Gesetz ist äquivalent zur Drehimpulserhaltung,
%TODO: Skizze Keppler,
\begin{align*}
&\dA = \frac{1}{2}r r\dph = \frac{1}{2}r^2\dph\\
&\frac{\dA}{\dt} = \frac{1}{2}r^2\dot{\ph} =\frac{l}{2\mu} = \const.\maphere
\end{align*}
\end{bemn}
Die Bewegungsgleichung für die Koordinate $r$ reduziert sich zu
\begin{align*}
&\frac{\diffd}{\dt}\frac{\partial L}{\partial \dot{r}} = \frac{\partial
L}{\partial r}\\
\Leftrightarrow & \mu\ddot{r} = \mu r \dot{\ph}^2 - V'(r).
\end{align*}
Mit $\dot{\ph} = \frac{l}{\mu r^2}$ können wir die $\ph$ Abhängigkeit
eliminieren und erhalten somit,
\begin{align*}
\mu\ddot{r} = \mu r \frac{l^2}{\mu^2r^4} - V'(r) = \frac{l^2}{\mu r^3} - V'(r)
= - \frac{\diffd}{\dr}\left[\frac{1}{2\mu}\frac{l^2}{r^2} +
V(r)\right] = -U'(r),
\end{align*}
wobei wir $U(r) = \frac{1}{2\mu}\frac{l^2}{r^2} + V(r)$ als \emph{effektives
Potential} bezeichnen.

Die Gleichung verhält sich nun, wie ein eindimensionales Teilchen im Potential
$U(r)$ mit der Energie,
\begin{align*}
E = \frac{\mu}{2}\left(\dot{r}^2 + \dot{\ph}^2r^2 \right) + V(r)
= \frac{\mu}{2}\dot{r}^2 + \underbrace{\frac{1}{2\mu}\frac{l^2}{r^2}  +
V(r)}_{=U(r)}.
\end{align*}

\begin{figure}[!htbp]
  \centering
\begin{pspicture}(0,-2.4)(4,2.7)
\psline{->}(0.37,-1.6)(0.37,1.8)
\psline{->}(0.17645577,0.6)(3.8,0.6)
\psbezier[linecolor=yellow](1.0564557,-1.590038)(1.0564557,-1.1100379)(1.8564558,0.48996207)(3.6764557,0.5099621)
\psbezier[linecolor=darkblue](0.79645574,1.4699621)(0.79645574,0.66996205)(0.83645576,-0.5500379)(1.2764558,-0.5500379)(1.7164558,-0.5500379)(2.0564559,0.46996206)(3.6964557,0.5099621)
\psdots(1.2764558,-0.47003794)


\rput{90}(0.2,1.7){\small\color{gdarkgray}Streuzustände}
\rput(0.2,-0.9){\rput{-90}{\small\color{gdarkgray}Gebundene Zustände}}

\rput(1.6,1.2){\color{gdarkgray}$U(r)=\frac{1}{r^2}$}
\rput[l](1.4,-1.4){\color{gdarkgray} $V(r)=\frac{1}{r^\alpha},\;\alpha <2$}

\end{pspicture}

  \caption{Teilchenpotential und effektives Potential mit Gleichgewichtslage.}
\end{figure}

Eine formale Lösung erhalten wir aus der Energieerhaltung,
\begin{align*}
\dot{r}^2 = \frac{2}{\mu}\left(E - U(r) \right),
\end{align*}
wobei wir lediglich den auslaufenden Teil ($\dot{r}>0$) betrachten.
\begin{align*}
&\dot{r} = \sqrt{\frac{2}{\mu}}\sqrt{E-U(r)},\\
\Rightarrow & \int\dt \frac{\dot{r}}{\sqrt{\frac{2}{\mu}(E-U(r))}} = \int\dr
\sqrt{\frac{\mu}{2}}\frac{1}{\sqrt{E-U(r)}} = \int dt = t-t_0.
\end{align*}
Die formale Lösung ist somit,
\begin{align*}
&t-t_0 = \underbrace{\int\limits_{r_0}^r \dr
\sqrt{\frac{\mu}{2}}\frac{1}{\sqrt{E-U(r)}}}_{:=F(r)}\\
\Rightarrow & r(t) = F^{-1}(t-t_0).
\end{align*}
Das Integral lässt sich jedoch nur in wenigen Fällen analytisch lösen. Wir sind
meist aber nicht an der Bewegungsgleichung selbst, sondern nur an der Form
der Bahnkurve interessiert und erhalten für die Bahntrajektorien
$\ph(r), r(\ph)$,
\begin{align*}
\frac{\dph}{\dr} = \frac{\dph}{\dt}\frac{\dt}{\dr} = \frac{l}{\mu
r^2}\frac{1}{\dot{r}} = \frac{l}{\mu
r^2}\sqrt{\frac{\mu}{2}}\frac{1}{\sqrt{E-U(r)}}
\end{align*}
\begin{align*}
\ph-\ph_0 =  \int\dr \frac{l}{\mu
r^2}\sqrt{\frac{\mu}{2}}\frac{1}{\sqrt{E-U(r)}} = \frac{l}{\sqrt{2\mu}}\int\dr
\frac{1}{r^2}\frac{1}{\sqrt{E-U(r)}}.
\end{align*}

Qualitativ lässt sich durch die Analyse des effektiven Potentials bereits viel
über die Bahn aussagen.
\begin{itemize}[label=\labelitem]
  \item $V(r)$ dominiert über $\frac{1}{r^2}$ für $r\to\infty$,
  \item $\frac{1}{r^2}$ dominiert über $V(r)$ für $r\to 0$.
\end{itemize}

 Für $E = E_\text{min}$ erhalten wir eine Kreisbahn, während sich für $E<0$ i.A.
 Rosettenbahnen mit Preiheldrehung ergeben.

\begin{figure}[H]
  \centering
\begin{pspicture}(-0.2,-3)(5.2,3.4)

\psline{->}(1.24,0.44)(1.26,3.12)
\psline{->}(1.06,2.42)(4.86,2.42)
\psbezier[linewidth=0.04](1.68,3.28)(1.68,2.48)(1.72,1.26)(2.16,1.26)(2.6,1.26)(2.94,2.28)(4.58,2.32)

\psline[linestyle=dashed,dash=0.06cm 0.04cm,linewidth=0.02cm]%
	   (1.06,1.96)(4.86,1.96)
\psline[linestyle=dashed,dash=0.06cm 0.04cm,linewidth=0.02cm]%
	   (1.06,1.26)(4.86,1.26)

\psline[linestyle=dotted,dotsep=0.06cm](2.18,2.42)(2.16,-0.32)
\psline[linestyle=dotted,dotsep=0.06cm](2.76,2.42)(2.74,-1.51)
\psline[linestyle=dotted,dotsep=0.06cm](1.78,2.42)(1.78,-1.52)

\pscircle[linecolor=yellow](1.27,-1.51){0.49}
\pscircle[linecolor=yellow](1.27,-1.51){1.42}
\psbezier[linecolor=darkblue](2.74,-1.16)(2.74,-0.18)(1.9,0.02)(1.12,-0.78)(0.34,-1.58)(0.88,-3.28)(1.6,-2.02)(2.32,-0.76)(0.68,0.78)(0.0,-0.7)

\psdots(1.27,-1.51)

\rput(2.5,3.03){\color{gdarkgray}$U(r)=\frac{1}{r^2}$}
\rput(4.88,2.27){\color{gdarkgray}$\vec{r}$}
\rput[l](0.4,1.96){\color{gdarkgray}$E$}
\rput[l](0.4,1.26){\color{gdarkgray}$E_\text{min}$}
\rput(3.0,2.55){\color{gdarkgray}\tiny$r_\text{max}$}
\rput(2.4,2.55){\color{gdarkgray}\tiny$r_0$}
\rput(2.0,2.55){\color{gdarkgray}\tiny$r_\text{min}$}
\end{pspicture} 

  \caption{Potential und Trajektorie.}
\end{figure}
Für die Oszillation erhält man im Allgemeinen keine rationalen Winkel $\ph =
\frac{2\pi n}{m}$, d.h. im Allgemeinen sind die Trajektorien nicht geschlossen.
Man kann analytisch beweisen, dass es genau zwei Potentiale gibt für die die
Trajektorien geschlossen sind, nämlich $V(r) = r^2$ und $V(r) = \frac{1}{r}$.


\subsection{Gravitationspotential}

Das Potential hat die Form
\begin{align*}
V(r) = -\frac{k}{r},
\end{align*}
wobei $k= G m_1 m_2$ im Fall der Gravitation oder $k = q_1 q_2$ im Fall der
elektrostatischen Wechselwirkung. Das effektive Potential hat die Form
\begin{align*}
U(r) = \frac{l^2}{2\mu r^2} - \frac{k}{r},
\end{align*}
und für die Bahntrajektorie gilt
\begin{align*}
\ph - \ph_0 = \frac{l}{\sqrt{2\mu}}\int\dr \frac{1}{r^2}\frac{1}{\sqrt{E-U(r)}}
= \frac{l}{\sqrt{2\mu}}\int\frac{\dr}{r^2}\frac{1}{\sqrt{E-\frac{l^2}{2\mu r^2}
+ \frac{k}{r}}}.
\end{align*}
Substituieren wir $x = \frac{1}{r}$, dann gilt für die Differentiale $\dx =
-\frac{1}{r^2}\dr$,
\begin{align*}
\ph - \ph_0 = -\frac{l}{\sqrt{2\mu}}\int\dx \frac{1}{\sqrt{E-\frac{l^2}{2\mu}x^2 +
kx}}
= -\int\dx \frac{1}{\sqrt{\frac{2E\mu}{l^2}-x^2 +
\frac{2\mu k}{l^2}x}}
\end{align*}
Ergänzen wir quadratisch mit $-\left(x-\frac{\mu k}{l^2}\right)^2 = -x^2 +
\frac{2\mu k}{l^2}x - \frac{\mu^2 k^2}{l^4}$, erhalten wir
\begin{align*}
\ph - \ph_0 &= -\int\dx \frac{1}{\sqrt{\underbrace{\frac{2E\mu}{l^2} +
\frac{\mu^2 k^2}{l^4}}_{=a^2} +\left(x-\underbrace{\frac{\mu
k}{l^2}}_{=b}\right)^2}}\\ 
&= - \int\dx \frac{1}{\sqrt{a^2-(x-b)^2}}= - \int\dx \frac{1}{a}\frac{1}{\sqrt{1-\left(\frac{x}{a}-\frac{b}{a}\right)^2}}
\\ &= - \int\dy \frac{1}{\sqrt{1-y^2}}= \arccos(y) = \arccos\left(\frac{x}{a}-\frac{b}{a} \right)
\\ &= \arccos\left(\frac{\frac{x}{b} - 1}{\frac{a}{b}}\right)
= \arccos\left(\frac{\frac{x}{\frac{\mu k}{l^2}} -
1}{\frac{\sqrt{\frac{2E\mu}{l^2} + \frac{\mu^2 k^2}{l^4}}}{\frac{\mu
k}{l^2}}}\right)\\ 
&= \arccos\left(\frac{\frac{x}{\frac{\mu k}{l^2}} - 1}{\sqrt{1 +
\frac{2El^2}{\mu k^2}}}\right)
= \arccos\left(\frac{\frac{l^2x}{\mu k} - 1}{\sqrt{1+\frac{2El^2}{\mu
k^2}}}\right) \\ &=
\arccos\left(\frac{\frac{l^2}{\mu k}\frac{1}{r} -1}{\sqrt{1+\frac{2El^2}{\mu
k^2}}} \right).
\end{align*}
Die Bahntrajektorie ist somit gegeben durch,
\begin{align*}
&\frac{p}{r} = 1 + \ep\cos(\ph-\ph_0),\\
&p = \frac{l^2}{\mu k},\qquad \ep = \sqrt{1+\frac{2 E l^2}{\mu k^2}}, 
\end{align*}
was einem Kegelschnitt entspricht.

\begin{figure}[!htbp]
  \centering
  
\begin{pspicture}(0,-0.8)(3.8621874,1.42)
\psline{->}(0.18,-0.6)(0.2,1.28)
\psline{->}(0.0,0.58)(3.8,0.58)

\psbezier[linewidth=0.04](0.58,1.4)(0.58,0.58)(0.66,-0.58)(1.1,-0.58)(1.54,-0.58)(1.88,0.44)(3.52,0.48)

\psline[linestyle=dashed,dash=0.06cm 0.04cm,linewidth=0.02cm](0.0,0.24)(3.8,0.24)
\psline[linestyle=dotted,dotsep=0.06cm](0.66,0.24)(0.66,0.66)
\psline[linestyle=dotted,dotsep=0.06cm](2.34,0.22)(2.34,0.66)

\rput(3.76,0.81){\color{gdarkgray}$r$}
\rput(0.9,0.99){\color{gdarkgray}$\frac{p}{1+\ep}$}
\rput(2.3,0.99){\color{gdarkgray}$\frac{p}{1-\ep}$}
\end{pspicture} 

  \caption{Trajektorie des Teilchens im effektiven Potential.}
\end{figure}
Der Radialteil der Trajektorie oszilliert zwischen $\frac{p}{1+\ep}$ und
$\frac{p}{1-\ep}$.
Wir wollen nun untersuchen wie eine Änderung von $E$ auf $\ep$ wirkt.

\begin{tabular}[h]{lll}
$E > 0 \Rightarrow \ep > 1$ & Hyperbel\\
$E = 0 \Rightarrow \ep = 1$ & Parabel\\
$E < 0 \Rightarrow \ep < 1$ & Ellipse\\
$E=E_\text{\tiny min} \Rightarrow \ep = 0$ & Kreisbahn 
\end{tabular} 

\begin{figure}[!htbp]
  \centering
  
\begin{pspicture}(-0.2,-1.3)(4.2,1.2)

\psellipse[linewidth=0.04,dimen=outer](2.12,0)(1.58,0.8)

\psdots[linecolor=yellow](1.22125,0)
\psdots[linecolor=darkblue](0.56125,0)
\psdots[linecolor=darkblue](3.68125,0)

\psline(0.64125,0)(1.12125,0)
\psline(1.32125,0)(3.60125,0)

\psbezier(3.04125,0.055625)(3.2679167,0.695625)(3.56125,-0.024375)(3.78125,0.555625)
\psbezier(0.10125,-0.624375)(0.32791665,0.015625)(0.62125,-0.704375)(0.84125,-0.124375)

\rput(3.9257812,0.885625){\color{gdarkgray}$\frac{p}{1-\ep}$}
\rput(0.08578125,-1){\color{gdarkgray}$\frac{p}{1+\ep}$}
\end{pspicture}

\caption{Ellipsenbahn der Erde.}
\end{figure}

\subsubsection{Repitition: Kegelschnitte}
%TODO: Skizze Kegelschnitt
Der Kegel kann durch
\begin{align*}
z = \sqrt{x^2+y^2} = r
\end{align*}
parametrisiert werden, die Ebene durch
\begin{align*}
z = z_0 - \lambda x.
\end{align*}

Die Schnittkurve der beiden Flächen ist gegeben durch,
\begin{align*}
&z_0-\lambda x = \sqrt{x^2+y^2}=r\\ 
\Leftrightarrow &z_0 -\lambda r\cos\ph = r\\
\Leftrightarrow &\frac{z_0}{r}  = 1-\lambda\cos\ph.
\end{align*}
und dies ist genau die Form der Trajektorie, die wir bereits berechnet haben.
%TODO: Bildchen Kegelschnitte

Man kann die Schnittkurve auch als \emph{quadratische Form} auffassen, wodurch
wir die Ellipsengleichung erhalten,
\begin{align*}
r^2 = (x^2+y^2) = (p-\ep x)^2 \Rightarrow \frac{\left(x+x_0\right)^2}{a^2} +
\frac{y^2}{b^2} = 1.
\end{align*}
Die \emph{große Halbachse} ist hierbei durch
\begin{align*}
a = \frac{p}{1-\ep^2},
\end{align*}
die \emph{kleine Halbachse} durch
\begin{align*}
b = \frac{p}{\sqrt{1-\ep^2}},
\end{align*}
und der \emph{Brennpunkt} durch
\begin{align*}
x_0 = \frac{\ep p}{1-\ep^2},
\end{align*}
gegeben.

\subsubsection{Keplerschen Gesetze}

Wir sind nun in der Lage mithilfe unserer Lösung des Zweikörper
Zentralkraftproblems die drei Keplerschen Gesetze zu beweisen.
\begin{propn}[1. Keplersches Gesetz]
Die Planeten bewegen sich auf Ellipsen mit der Sonne im Brennpunkt.\fishhere
\end{propn}
Dieses Gesetz gilt, wenn die Masse der Planeten klein im Vergleich zur Masse
der Sonne ist. Kepler hatte in diesem Fall Glück, da unser Sonnensystem gerade
so beschaffen ist.
\begin{proof}
Ist die Masse des Planeten klein im Vergleich zur Sonne, dann liegt der
Brennpunkt näherungsweise in der Sonne.\qedhere
\end{proof}

\begin{propn}[2. Keplersche Gesetz]
Eine von der Sonne zu einem Planeten gezogene Strecke überstreicht in gleichen
Zeiträumen gleiche Flächen.\fishhere
\end{propn}
\begin{proof}
Wir haben bereits gezeigt, dass dieses Gesetz äquivalent zur Drehimpulserhaltung
ist.\qedhere
\end{proof}

\begin{propn}[3. Keplersche Gesetz] Betrachtet man die Bahn eines Planetens
mit großer Halbachse $a$ und der Umlaufzeit $T$, dann ist 
$\dfrac{T^2}{a^3}$ unabhängig vom Planeten.\fishhere
\end{propn}
Auch dieses Gesetzt gilt nur für Planeten mit einer im Vergleich zur Sonne
kleinen Masse.
\begin{proof}
Wir verwenden $\frac{\dA}{\dt} = \frac{l}{2\mu}$. Die Fläche einer Ellipse ist
\begin{align*}
&A = \frac{\dA}{\dt}T = \pi a b = T\frac{l}{2\mu}\\
\Rightarrow & T = \frac{2\mu}{l}\pi a \frac{p}{\sqrt{1-\ep^2}}
= \frac{2\mu}{l}\pi a\frac{\frac{l^2}{\mu k}}{\sqrt{1-1+\frac{2El^2}{\mu k^2}}}
= 2\pi a\sqrt{\frac{\mu}{2E}}.
\end{align*}
Mit $a = \dfrac{p}{1-\ep^2} = \dfrac{l^2}{\mu k}\dfrac{1}{\frac{2El^2}{\mu k^2}}
= \dfrac{k}{2E}$ erhalten wir $\dfrac{1}{2E} = \dfrac{a}{k}$, wobei
\begin{align*}
k = G M_s m_e,\qquad \mu = \frac{M_s m_e}{M_s+m_e}\approx m_e.
\end{align*}
Einsetzen ergibt nun,
\begin{align*}
&T = 2\pi a^{\frac{3}{2}}\frac{1}{\sqrt{G M_S}},\\
\Leftrightarrow & T^2 = \frac{4\pi^2}{G M_s}a^3,
\end{align*}
und der Vorfaktor ist für das Sonnensystem konstant.\qedhere
\end{proof}

Interessant ist, dass für $E\le 0$ die Bahnkurven im Zentralpotential stets
geschlossen sind. Verantwortlich dafür ist eine weitere
Erhaltungsgröße, der \emph{Laplace-Runge-Lenz Vektor}.

\subsubsection{Laplace-Runge-Lenz Vektor}

Das Coulomb Potential $V(\abs{\vec{r}}) = -\frac{k}{\abs{\vec{r}}}$ zeichnet
sich durch eine weitere Erhaltungsgröße aus, den Laplace-Runge-Lenz Vektor
\begin{align*}
\vec{A} = \vec{p}\times\vec{L} - \mu k\frac{\vec{r}}{\abs{\vec{r}}}.
\end{align*}
\begin{proof}
Die Erhaltung des Vektors sieht man wie folgt ein,
\begin{align*}
\frac{\diffd}{\dt} \vec{A} &= \dot{\vec{p}}\times\vec{L} - \mu k\left( 
\frac{\dvec{r}}{\abs{\vec{r}}} -
\frac{\vec{r}}{\abs{\vec{r}}^2}\frac{\diffd}{\dt}\abs{\vec{r}} \right)\\
&=
-\frac{\mu k}{\abs{\vec{r}}^3}\vec{r}\times (\vec{r}\times\dvec{r}) - \frac{\mu
k}{\abs{\vec{r}}^3}\left(\abs{\vec{r}}^2\dvec{r} -
\vec{r}(\vec{r}\dvec{r})\right) = 0.\qedhere
\end{align*}
\end{proof}

Dieser Vektor steht senkrecht zum Drehimpuls $L$
\begin{align*}
\vec{A}\cdot \vec{L} = 0.
\end{align*}
Zudem zeigt er in Richtung des Perihelion. Konsequenz seiner Erhaltung ist,
dass das Perihelion sich nicht dreht.

\subsection{Streuexperimente}

Streuexperimente sind die Schlüsselversuche der Kern- und
Elementarteilchenphysik. Hierbei werden leichte Teilchen auf schwere ruhende
Teilchen geschossen und dadurch abgelenkt. Die Ablenkung kann Aufschluss über
die Struktur der Materie geben.
\begin{figure}[!htbp]
  \centering
  
\begin{pspicture}(-0.2,-1.4)(5.8,1.5)

\psline(0.0,-0.6416429)(3.92,-0.6416429)
\psdots[linecolor=yellow](4.04,-0.6416429)

\psline(4.14,-0.6416429)(5.36,-0.6416429)
\psline(4.1,-0.54164296)(5.1,0.45835707)

\psarc(4.2,-0.6416429){0.5}{0.0}{60.945396}

\psline(5.38,0.35835707)(4.66,1.0783571)
\psbezier[linecolor=darkblue]{->}(0.2,-0.061642937)(1.28,-0.061642937)(2.34,-0.061642937)(2.8,-0.061642937)(3.26,-0.061642937)(3.68,-0.08164294)(4.02,0.21835706)(4.36,0.51835704)(4.54,0.69835705)(4.66,0.8383571)
\psline{<->}(0.86,-0.5816429)(0.86,-0.10164294)

\psbezier(4.58,-0.56164294)(4.86,-0.88164294)(4.16,-0.6416429)(4.4,-1.0216429)
\psbezier(0.6,-0.50164294)(0.56,-1.0216429)(1.02,-0.50164294)(1.04,-1.0616429)
\psdots(0.08,-0.061642937)

\rput(0.58421874,-0.31164294){\color{gdarkgray}$b$}
\rput(0.51453125,0.34835705){\color{gdarkgray}$E=\frac{mv^2}{2}$}
\rput(4.4085937,-0.47164294){\color{gdarkgray}$\th$}

\rput{-45.56502}(0.95505434,3.9956415){\color{gdarkgray}\rput(5.2279687,0.8483571){Detektor}}
\rput(1.0198437,-1.211643){\color{gdarkgray}Stoßparameter}
\rput(4.465,-1.211643){\color{gdarkgray}Streuwinkel}
\end{pspicture} 

\caption{Streuexperiment.}
\end{figure}

Rutherford untersuchte 1909, wie $\alpha$-Teilchen an einer Goldfolie gestreut
werden und entdeckte dabei, dass der Kerndurchmesser um ca. 5 Größenordnungen
kleiner sein muss als der Atomdurchmesser. Viele sehen dies als die
Geburtsstunde der Kernphysik.

Im Folgenden untersuchen wir die Streuung von 2 Teilchen mit Zentralkraft. Wir
arbeiten wieder im Schwerpunktsystem mit der Relativkoordinate $\vec{r}$ als
Variable. Dabei vernachlässigen wir, dass das schwere Teilchen durch die
Streuung kinetische Energie erhält, für sehr schwere Teilchen ist dies eine
gute Näherung.

\begin{bemn}
Wir betrachten hier nur elastische Stöße, d.h. die Energie ist erhalten.\maphere 
\end{bemn}

\begin{defnn}
Der \emph{Stoßparameter} $b$ ist der Abstand zwischen Einfallsachse und
Streuzentrum,  $\th$ der \emph{Streuwinkel}.\fishhere
\end{defnn}

Betrachte einen einfallenden Teilchenstrahl mit Fluss
$F\entspr\frac{\text{Anzahl Teilchen}}{cm^2\ s}$. Messe die Zahl
der gestreuten Teilchen $\dN$ im Raumwinkel $\dOmega=
\sin\th\dth\dph$. Wir können keine diskreten Winkel messen, da der Detektor
eine endliche Ausdehnung und Auflösung hat. Es wird daher stets ein
Winkelbereich gemessen, wobei wir daran interessiert sind, diesen möglichst
klein zu halten, um genaue Ergebnisse zu erhalten.

Dies führt zur Definition des \emph{differentiellen Wirkungsquerschnitts},
\begin{align*}
\frac{\dsigma}{\dOmega} = \frac{1}{F}\frac{\dN}{\dOmega}.
\end{align*}
Der totale Streuquerschnitt ist dann gegeben durch,
\begin{align*}
&\sigma = \int \dOmega \left(\frac{\dsigma}{\dOmega}\right)
:= \int\limits_{0}^{2\pi}\dph\int\limits_{0}^\pi \dth \sin\th
\frac{\dsigma}{\dOmega},\\
&\left[\sigma\right] = \text{ Fläche},\quad F\sigma = \text{ Anzahl der
gestreuten Teilchen}
\end{align*}
\begin{bsp}
Im Fall einer Kugel, die von Teilchen umströmt wird, ist $\sigma = \pi
R^2$.\bsphere
\begin{figure}[!htbp]
  \centering
\begin{pspicture}(0,-0.6)(4.6,0.6)
\pscircle(2.21,-0.02){0.39}
\psline[linewidth=0.02]{->}(0.0,0.19)(1.82,0.19)
\psline[linewidth=0.02]{->}(0.0,-0.01)(1.76,-0.01)
\psline[linewidth=0.02]{->}(0.0,-0.21)(1.82,-0.21)
\psline[linewidth=0.02]{->}(0.0,-0.39)(1.96,-0.39)
\psline[linewidth=0.02]{->}(0.0,0.37)(1.96,0.37)
\psline[linewidth=0.02]{->}(0.0,-0.57)(3.04,-0.55)
\psline[linewidth=0.02]{->}(0.0,0.57)(3.04,0.57)
\psarc[linestyle=none,fillstyle=solid,fillcolor=glightgray](2.21,-0.02){0.37}{87.70939}{275.90613}
\psbezier(2.28,-0.05)(2.82,0.47)(3.16,-0.49)(3.68,0.17)

\rput(3.755,0.4){\color{gdarkgray}$\sigma=\pi R^2$}
\end{pspicture} 

\caption{Umströmte Kugel.}
\end{figure}
\end{bsp}

Für rotationssymmetrische Potentiale hängt $\frac{\dsigma}{\dOmega}$ nur vom
Winkel $\th$ und der Energie der einfallenden Teilchen ab. Für jedes Tupel aus
kinetischer Energie $E_\text{\tiny kin}$ und Stoßparameter $b$ können wir die
Trajektorie berechnen und den Wert $\th$ bestimmen. Wir haben damit das inverse
Problem $\th(E,b)$ gelöst, welches uns auf $b(E,\th)$ führt.

Aus der Energieerhaltung folgt, dass der Betrag der Geschwindigkeit für
einfallendes und ausfallendes Teilchen gleich ist,
\begin{align*}
v = \sqrt{\frac{2E}{m}}.
\end{align*}

Der Drehimpuls $l=mbv$ ist ebenfalls erhalten, das ausfallende Teilchen hat
somit auch den Stoßparameter $b$. Haben wir nun eine Lösung $\th(E,b)$, so
können wir eine kleine Winkeländerung $\dth$ durch eine kleine Änderung des
Stoßparameters $\db$ ausdrücken
\begin{align*}
\dth = \abs{\frac{\dth}{\db}}\db.
\end{align*}

\begin{figure}[!htbp]
  \centering
\begin{pspicture}(0,-0.5)(5.38,1.08)
\psline(0.0,-0.42)(3.92,-0.42)
\psdots(4.04,-0.42)
\psline(4.14,-0.42)(5.36,-0.42)
\psline(4.1,-0.32)(5.1,0.68)
\psarc(4.04,-0.42){0.64}{0.0}{178.31532}
\psbezier[linecolor=darkblue]{->}(0.2,0.16)(1.28,0.16)(2.34,0.16)(2.8,0.16)(3.26,0.16)(3.68,0.14)(4.02,0.44)(4.36,0.74)(4.54,0.92)(4.66,1.06)
\psline{<->}(0.86,-0.36)(0.86,0.12)
\psdots(0.08,0.16)
\psline[linestyle=dotted,dotsep=0.06cm](3.94,-0.3)(3.44,0.38)

\rput(4.4085937,-0.25){\color{gdarkgray}$\th$}
\rput(0.58421874,-0.09){\color{gdarkgray}$b$}

\rput(0.30671874,0.57){\color{gdarkgray}$\vec{v}$}

\rput(4.0285935,-0.01){\color{gdarkgray}$\ph$}

\rput(3.6485937,-0.25){\color{gdarkgray}$\ph$}
\end{pspicture} 

\caption{Streuexperiment.}
\end{figure}

Für die gestreuten Teilchen gilt $\dN = b \db\dph F$, also erhalten wir
\begin{align*}
\frac{\dsigma}{\dOmega} &= \frac{1}{F}\frac{\dN}{\dOmega}
= \frac{1}{F}\frac{b\db\dph F}{\sin\th\dth\dph} =
\frac{b\db}{\sin\th\abs{\frac{\dth}{\db}}\db}\\
&= \frac{1}{\sin\th}b\abs{\frac{\diffd}{\dth}b(\th,E)}.\tag{*}
\end{align*}

\subsubsection{Rutherfordstreuung}

Wir betrachten das Potential $\dfrac{1}{r}$ im repulsiven Fall
\begin{align*}
V(r) = -\frac{k}{r},\qquad k = -e e',
\end{align*}
wobei $e,e'$ die Ladung von Streuteilchen und Target sind.

Verwenden wir nun unsere Lösung
\begin{align*}
\frac{1}{r} = \frac{mee'}{l^2}\left(\ep\cos\ph - 1\right),
\end{align*}
so erhalten wir für $r\to\infty$,
\begin{align*}
&0 = \ep\cos\ph - 1
\Leftrightarrow \cos \ph = \frac{1}{\ep}
\overset{\pi = 2\ph+\th}{\Rightarrow} \sin\left(\frac{\th}{2}\right) =
\frac{1}{\ep}\\
&\cot\left(\frac{\th}{2}\right) =
\frac{\cos\left(\frac{\th}{2}\right)}{\sin\left(\frac{\th}{2}\right)}
= \frac{\sqrt{1-\frac{1}{\ep^2}}}{\frac{1}{\ep}} = \sqrt{\ep^2-1}
\end{align*}
Mit $\ep = \sqrt{1+\left(\dfrac{2Eb}{ee'}\right)^2}$ ergibt sich somit,
\begin{align*}
b(\th,E) = \cot\left(\frac{\th}{2}\right) \frac{ee'}{2\abs{E}}.
\end{align*}
Setzen wir dies in (*) ein, ergibt sich
\begin{align*}
\frac{\dsigma}{\dOmega} &= 
\frac{\cot\left(\frac{\th}{2}\right)}{\sin\th}\frac{ee'}{2\abs{E}} \abs{\frac{\diffd}{\dth}
\left(\cot\left(\frac{\th}{2}\right) \frac{ee'}{2\abs{E}}\right)}\\
&=
\abs{\frac{ee'}{2\abs{E}}\cot\left(\frac{\th}{2}\right)\frac{ee'}{2\abs{E}}
\frac{1}{2}\frac{1}{\sin^2\left(\frac{\th}{2}\right)}\frac{1}{\sin\th}}\\
&=
\frac{1}{4}\left(\frac{ee'}{2E}\right)^2\frac{1}{\abs{\sin\left(\frac{\th}{2}\right)}^4}.
\end{align*}
Dies ist die berühmte Rutherfordsche-Streuformel mit der
$\dfrac{1}{\sin\left(\frac{\th}{2}\right)^4}$-Abhängigkeit.

\section{Grundprinzipien}

Wir beginnen mit den wichtigsten experimentellen Fakten, auf denen die
Klassische Mechanik aufbaut.

\subsection{Raum und Zeit}

Unser \emph{Raum} ist ein dreidimensionaler euklidischer Raum und die Zeit ist
eindimensional. Ein \emph{Teilchen} beschreibt eine Kurve
\begin{align*}
t\mapsto\vec{r}(t),
\end{align*}
in diesem $(3+1)$-dimensionalen Raum.
\begin{figure}[!htbp]
  \centering
\begin{pspicture}(-1,-1)(4.5,2.5)
 \psaxes[labels=none,ticks=none]{->}%
 	(0,0)(-0.5,-0.5)(4,2)%
 	[\color{gdarkgray}$\vec{r}\in\R^3$,-90]%
 	[\color{gdarkgray}$t$,0]
 \psyTick(0.5){\color{gdarkgray}t_0}
 \psyTick(1.5){\color{gdarkgray}t_1}
 
 \psline[linewidth=0.5pt](0,0.5)(4,0.5)
 \psline[linewidth=0.5pt](0,1.5)(4,1.5)
 
 \psbezier[linecolor=darkblue,arrows=->,linewidth=1.2pt]
 (2.42,-0.6170833)(0.78,0.8229167)(4.18,0.42291668)(3.38,1.7829167)
 
 \end{pspicture}
  \caption{Teilchen Trajektorie.}
\end{figure}

Der Raum ist homogen und isotrop, d.h. kein Punkt und keine Richtung sind
ausgezeichnet. Ebenso ist die Zeit homogen, das heißt kein Zeitpunkt ist
ausgezeichnet.

Die \emph{Geschwindigkeit} ist die Ableitung der Raumkurve nach der Zeit,
\begin{align*}
\vec{v}(t) =\frac{\diffd}{\dt} \vec{r}(t) = \dot{\vec{r}}(t),
\end{align*}
die \emph{Beschleunigung} ist die zweite Ableitung der Raumkurve,
\begin{align*}
\vec{a}(t) = \frac{\diffd^2}{\dt^2}\vec{r}(t) = \ddot{\vec{r}}(t).
\end{align*}

\subsubsection{Trägheitsgesetz (1. Newtonsche Gesetz)}
Es gibt \emph{Inertialsysteme}, in denen alle Naturgesetze zu allen
Zeiten gleich sind. Ein System, das sich gleichförmig zu einem Inertialsystem
bewegt, ist selbst ein Inertialsystem.
\begin{figure}[!htbp]
  \centering
\begin{pspicture}(-0.5,-0.5)(5.5,4.5)

\psline{->}(-0.2,0)(3.5,0)
\psline{->}(0,-0.2)(0,2)

\psline{->}(1.3,2)(5,2)
\psline{->}(1.5,1.8)(1.5,4)

\psline[linestyle=dashed](0,0)(2.8,2.4)
\psline[linestyle=dashed](1.5,2)(2.8,2.4)

\psdot[linecolor=darkblue](2.8,2.4)

\rput(-0.2,2){\color{gdarkgray}$y$}
\rput(3.8,-0.2){\color{gdarkgray}$x$}

\rput(1.3,4){\color{gdarkgray}$y'$}
\rput(5.3,1.8){\color{gdarkgray}$x'$}
 
 \rput(2,2.4){\color{gdarkgray}$\vec{r}'$}
 \rput(1,1.1){\color{gdarkgray}$\vec{r}$}
 

\end{pspicture} 
  \caption{Zwei Inertialsysteme.}
\end{figure}

Die Beschleunigung, die ein Teilchen in einem Inertialsystem erfährt, ist somit
in allen Inertialsystemen gleich. Insbesondere erfährt ein kräftefreies Teilchen
keine Beschleunigung und bewegt sich gleichförmig,
\begin{align*}
\vec{r}(t) = \vec{r}_0 + \vec{v}t \Rightarrow \ddvec{r} = 0.
\end{align*}

Eine gute Approximation eines Inertialsystems erhalten wir, indem wir uns an
den Fixsternen orientieren. Doch selbst hier gibt es noch Abweichungen, da sich
auch unserer Galaxie dreht. Für die meisten Experimente genügt es jedoch, die
Erdoberfläche als Ineratialsystem zu wählen. Für große Geschwindigkeiten und
Strecken müssen jedoch Effekte wie die Corioliskraft, die durch die Erdrotation
entstehen, berücksichtigt werden, was man am Foucaultsches Pendel oder an der
Rotation von Wirbelstürmen beobachten kann.

\subsubsection{Bewegungsgesetz (2. Newtonsche Gesetz)}
Die \emph{Masse} $m$ eines Teilchens ist eine skalare Größe und daher vom
Inertialsystem unabhängig. Sie ist eine Eigenschaft der Teilchen.

Der \emph{Impuls} eines Teilchens ist gegeben durch,
\begin{align*}
\vec{p}(t) = m\vec{v}(t) = m\dot{\vec{r}}(t).
\end{align*}
Somit folgt die Bewegungsgleichung im Inertialsystem,
\begin{align*}
\frac{\diffd \vec{p}}{\dt} = m\ddvec{r} = \vec{F},
\end{align*}
wobei $\vec{F}$ die \emph{Kraft}, die auf das Teilchen wirkt, bezeichnet.
Für $N$-Teilchen erhalten wir,
\begin{align*}
m_i \ddvec{r}_i = \vec{F}_i(\vec{r}_1,\ldots, \vec{r}_N,
\vec{v}_1,\ldots,\vec{v}_N,t).
\end{align*}
Dies ist eine Differentialgleichung 2. Ordnung. Die Trajektorien $x_i(t)$ sind
eindeutig durch Ort und Geschwindigkeit zur Anfangszeit $t_0$ bestimmt. Die
klassische Mechanik beschreibt also ein vollkommen deterministisches Weltbild.

Bisher lässt sich die Masse experimentell nicht absolut sondern nur in Relation
zur Masse eines anderen Körpers bestimmen. Dies erfolgt durch den Vergleich der
Beschleunigungen für eine feste Kraft,
\begin{align*}
&\ddot{r}_a = \frac{F}{m_a},\quad\ddot{r}_b = \frac{F}{m_b}\\
\Rightarrow & \frac{\ddot{r}_a}{\ddot{r}_b} = \frac{m_b}{m_a}.
\end{align*}

%Es ist hier nicht auszuschließen, dass die Kraft, die auf ein Teilchen wirkt,
%auch von den Geschwindigkeiten der $N$-Teilchen abhängt.

\subsubsection{Actio = Reactio (3. Newtonsche Gesetz)}
Die Kraft $\vec{F}_{21}$, die das zweite Teilchen auf das erste ausübt, ist
entgegengesetzt zu der Kraft $\vec{F}_{12}$, die das erste Teilchen auf das
zweite ausübt. $\vec{F}_{21}=-\vec{F}_{12}$.
\begin{figure}[!htbp]
\centering
\begin{pspicture}(-0.1,-1.15)(2.5,1.2)

\psline[linecolor=darkblue]{->}(2.14,0.74828124)(1.34,0.20828125)
\psline[linecolor=yellow]{<-}(0.88,-0.11171875)(0.08,-0.65171874)

\psdots[dotsize=0.12](0.06,-0.6717188)
\psdots[dotsize=0.12](2.14,0.74828124)

\rput(0.006875,-1.0017188){\color{gdarkgray}$1$}
\rput(2.3785937,0.97828126){\color{gdarkgray}$2$}
\rput(0.7496875,-0.48171875){\color{gdarkgray}$\vec{F}_{21}$}
\rput(1.4296875,0.6382812){\color{gdarkgray}$\vec{F}_{12}$}
\end{pspicture}

\caption{Actio und Reactio.}
\end{figure}

Für zwei wechselwirkende Teilchen gilt daher die Beziehung,
\begin{align*}
\ddvec{r}_1 &= \frac{\vec{F}_{21}}{m_1} = - \frac{\vec{F}_{12}}{m_1} = -
\frac{m_2}{m_1}\ddvec{r}_2,\\
\frac{\ddvec{r}_1}{\ddvec{r}_2} &= -\frac{m_2}{m_1}.
\end{align*}

\subsubsection{Kraftgesetze}
In der Natur kommen vier fundamentale Wechselwirkungen vor, nämlich Gravitation,
Elektromagnetische Wechselwirkung, Starke und Schwache Wechselwirkung. In de
Klassischen Mechanik spielen lediglich Gravitation und Elektromagnetismus 
eine Rolle, die durch folgende Gesetze beschreiben werden.

Das \emph{Coulombsche Gesetz} hat die Form,
\begin{align*}
\vec{F}_{12} = \frac{\vec{r}_1-\vec{r}_2}{\abs{\vec{r}_1-\vec{r}_2}}
\frac{e_1 e_2}{\abs{\vec{r}_1-\vec{r}_2}^2},\qquad \text{im cgs-System},
\end{align*}
wobei die Ladung $e_i$ im Gegensatz zur Masse $m_i$ unabhängig vom Träger ist.

Man kann die schwere Masse als Ladung bezüglich der Gravitation
betrachten. Die entsprechende Beziehung stellt das \emph{Gravitationsgesetz}
her
\begin{align*}
\vec{F}_{12} = -G \frac{\vec{r}_1-\vec{r}_2}{\abs{\vec{r}_1-\vec{r}_2}}
\frac{\hat{m}_1 \hat{m}_2}{\abs{\vec{r}_1-\vec{r}_2}^2}.
\end{align*}
Es stellt sich nun die Fragem ob die schwere Masse $\hat{m}_i$ des
Gravitationsgesetzes gleich der trägen Massen $m_i$ des Trägheitsgesetzes ist.
Loránd Eötvös\footnote{Loránd Eötvös (* 27. Juli 1848 in Buda; † 8. April 1919
in Budapest) war ein ungarischer Physiker. International bekannt war er als
(Baron) Roland von Eötvös} erzielte 1909 das experimentelle Resultat,
\begin{align*}
\frac{\hat{m}}{m} = \text{universelle Konstante}.
\end{align*}
$G$ ist nun so bestimmt, dass die schwere Masse identisch zur trägen Masse ist.
Diese Gleichheit ist auch eine der Grundannahmen in der allgemeinen
Relativitätstheorie.
\begin{figure}[!htbp]
  \centering
\begin{pspicture}(-0.1,-1.3)(3.2,1.2)

\psline{->}(2.52,-0.71171874)(2.02,-0.21171875)
\psline{->}(1.02,0.7682812)(1.52,0.26828125)
\psline{->}(0.0,-1.2317188)(0.92,0.6482813)
\psline{->}(0.0,-1.2317188)(2.4,-0.75171876)

\psdots[linecolor=darkblue](2.52,-0.71171874)
\psdots[linecolor=darkblue](1.0,0.78828126)

\psdots(0,-1.2317188)

\rput(2.686875,-1){\color{gdarkgray}$m_1,e_1$}
\rput(0.75859374,1.0982813){\color{gdarkgray}$m_2,e_2$}

\rput(0.3,-0.08171875){\color{gdarkgray}$\vec{r}_2$}
\rput(1.46,-1.1217188){\color{gdarkgray}$\vec{r}_1$}
\end{pspicture}
  \caption{2-Teilchen Kraft.}
\end{figure}



\subsubsection{Galilei-Transformation}
Eine \emph{Galilei-Transformation} beschreibt den Übergang von einem
Inertialsystem in ein anderes.
\begin{align*}
&t' = t + a,\\
&\vec{r}' = \mathrm{R}\vec{r} + \vec{v}t + \vec{b},
\end{align*}
dabei sind $a,\vec{b}$ und $\vec{v}$ konstant und $R$ eine Drehmatrix.
\begin{figure}[!htbp]
  \centering
\begin{pspicture}(-0.5,-0.5)(5.5,4.5)

\psline{->}(-0.2,0)(3.5,0)
\psline{->}(0,-0.2)(0,2)

\psline{->}(1.3,2)(5,2)
\psline{->}(1.5,1.8)(1.5,4)

\psline[linecolor=darkblue]{->}(0,0)(1.4,1.9)

\rput(-0.2,2){\color{gdarkgray}$t$}
\rput(3.8,-0.2){\color{gdarkgray}$\vec{r}$}

\rput(1.3,4){\color{gdarkgray}$t'$}
\rput(5.3,1.8){\color{gdarkgray}$\vec{r}'$}
 
 \rput[tl](1.2,1.6){\color{gdarkgray}$(a,\vec{b})$}
\end{pspicture} 
  \caption{Galilei Transformation.}
\end{figure}

Im $(3+1)$ dimensionalen Raum hat die Transformation die Form
\begin{align*}
\begin{pmatrix}
t'\\
\vec{r}'
\end{pmatrix}
= \begin{pmatrix}
  1 & 0 \\
  \vec{v} & \mathrm{R} 
  \end{pmatrix}
\begin{pmatrix}
t\\
\vec{r}
\end{pmatrix}
+ \begin{pmatrix}
  a\\\vec{b}
  \end{pmatrix}.
\end{align*}
Die Translation hat somit insgesamt 10 Parameter,
\begin{itemize}[label=\labelitem]
  \item Translation in Raum und Zeit $\vec{b}, a$ ($4$-Parameter)
  \item Boost $\vec{v}$ ($3$-Parameter)
  \item Rotationen im Raum $R\in\mathrm{SO}(3)$ ($3$-Parameter)
\end{itemize}
Unter $\mathrm{SO}(3)$ verstehen wir,
\begin{align*}
\mathrm{SO}(3) = \setdef{R\in M_{3\times 3}}{R^tR = RR^t = \Id\text{ und }\det
R = 1}.
\end{align*}
\begin{bsp}
Rotation um die $z$-Achse mit Winkel $\th$,
\begin{align*}
R = \begin{pmatrix}
\cos \th & \sin \th & 0\\
-\sin\th & \cos \th & 0\\
0 & 0 & 1
\end{pmatrix}
\end{align*}
Rotation um die $x$-Achse mit Winkel $\th$,
\begin{align*}
R = \begin{pmatrix}
    1& 0 & 0 \\
0 & \cos \th & \sin \th\\
0 & -\sin\th & \cos \th
\end{pmatrix}\bsphere
\end{align*}
\end{bsp}
Die Menge der Galilei-Transformationen bildet eine 10-parametrige Gruppe, die
\emph{Galilei-Gruppe} $\GG$.
\begin{bemn}
Die Galilei Transformationen $g\in\GG$ haben die Eigenschaften.
\begin{itemize}[label=\labelitem]
  \item $g_2\circ g_1\in G$,
  \item $(g_3\circ g_2)\circ g_1 = g_3 \circ (g_2\circ g_1)$ Assoziativität,
  \item $g\circ\Id = g = \Id \circ g$ Einselement,
  \item Zu $g\in\GG$ existiert ein Inverses $g^{-1}$ mit $g^{-1}\circ g = \Id =
  g \circ g^{-1}$,
  \item $\GG$ ist nicht kommutativ.\maphere
\end{itemize} 
\end{bemn}
 Neben den gewöhnlichen Transformationen gibt es
noch zwei diskrete Transformationen
\begin{itemize}[label=\labelitem]
  \item Raumspiegelung $\vec{r}' = -\vec{r}$
  \item Zeitumkehr $t' = -t$
\end{itemize}

Die 10 Parameter der Galilei-Transformation beschreiben 10 Symmetrien, aus
denen wir 10 Erhaltungssätze herleiten können.

\subsection{Symmetrien und Erhaltungssätze}

Die Gesetze der Klassischen Mechanik für ein abgeschlossenes
System\footnote{abgeschlossen bedeutet, dass keine Kräfte von Teilchen
ausgehen, die sich außerhalb des Systems befinden.} sind forminvariant unter
Galilei-Transformationen.

\begin{bemn}
Man sagt auch, dass die Galilei-Gruppe die Symmetrie-Gruppe der Klassischen
Mechanik ist (abgesehen von der Zeitumkehr).\maphere
\end{bemn}

\begin{propn}[Homogenität des Raums]
Kein Punkt des Raums ist ausgezeichnet, es sind daher nur Relativbewegungen von
Bedeutung und der Nullpunkt kann frei gewählt werden.\fishhere
\end{propn}
Betrachten wir die Kraft $\vec{F}_{12}(\vec{r}_1,\vec{r}_2,t)$, dann
bewirkt die Homogenität des Raums, dass die Kraft unter einer
Galilei-Transformation mit $\vec{r}_1' = \vec{r}_1+\vec{a}$ und $\vec{r}_2' =
\vec{r}_2+\vec{a}$ erhalten bleibt,
\begin{align*}
\vec{F}_{12}(\vec{r}_1',\vec{r}_2',t) =
\vec{F}_{12}(\vec{r}_1,\vec{r}_2,t).
\end{align*}

\begin{propn}[Homogenität der Zeit]
Es existiert kein ausgezeichneter Zeitpunkt, also sind alle Fundamentalkräfte
zeitunabhängig.\fishhere
\end{propn}

\begin{propn}[Isotropie des Raumes]
Der Raum hat keine ausgezeichnete Richtung, betrachtet man also zwei Teilchen,
so ist die einzig verfügbare Richtung $\vec{r}_1-\vec{r}_2$.\fishhere
\end{propn}

\begin{propn}[Invarianz unter boost]
Es gibt kein absolutes Ruhesystem, die Geschwindigkeit hat nur eine
relative Bedeutung $\vec{v}_1-\vec{v}_2$.\fishhere
\end{propn}

\subsubsection{Folgerungen aus der Galilei-Invarianz}
$2$-Teilchen Kräfte, die geschwindigkeitsunabhängig sind, können geschrieben
werden als
\begin{align*}
\vec{F}_{ij} =
\frac{\vec{r}_i-\vec{r}_j}{\abs{\vec{r}_i-\vec{r}_j}}f_{ij}\left(\abs{\vec{r}_i
- \vec{r}_j}\right),
\end{align*}
da nur der Abstand $\abs{\vec{r}_i - \vec{r}_j}$ und die Richtung
$\vec{r}_i-\vec{r}_j$ identifizierbar sind.
\begin{bemn}
\begin{enumerate}[label=\arabic{*}.)]
  \item 
Wir haben bereits gesehen, dass die Fundamentalkräfte Coulomb-Wechselwirkung
und Gravitationskraft von dieser Gestalt sind. Jetzt sehen wir, dass es auch
die einzig mögliche ist.
\item Das 3. Newtonsche Gesetzt folgt direkt aus der
Galilei-Invarianz.
\item Fundamentale Kräfte, die von der Geschwindigkeit der Teilchen abhängen
existieren in der Newtonschen Mechanik nicht.
\item In einem $3$-Teilchen System könnte neben der $2$-Teilchen Wechselwirkung auch
eine fundamentale $3$-Teilchen Wechselwirkung auftreten, also eine Kraft, die
genau dann auftrit, wenn $3$ Teilchen vorhanden sind, und nicht das Ergebnis
einer Überlagerung von $2$-Teilchen Wechselwirkungen ist. Das Experiment zeigt
jedoch, dass alle bis jetzt bekannten fundamentalen Kräfte reine $2$-Teilchen
Kräfte sind. 
\end{enumerate}
\end{bemn}

\begin{figure}[!htbp]
  \centering
\begin{pspicture}(-0.1,-1.2)(3,1.2)

\psline{->}(2.6453125,-0.77953124)(2.1453125,-0.27953124)
\psline{->}(1.1453125,0.7004688)(1.6453125,0.20046875)
\psline{->}(0.3253125,-0.77953124)(1.1253124,-0.77953124)
\psline{<-}(1.8453125,-0.77953124)(2.6453125,-0.77953124)
\psline{<-}(0.8253125,0.12046875)(1.1253124,0.7004688)
\psline{->}(0.3453125,-0.75953126)(0.6653125,-0.19953126)

\psdots[linecolor=darkblue](2.6453125,-0.77953124)
\psdots[linecolor=darkblue](1.1253124,0.72046876)
\psdots[linecolor=darkblue](0.3253125,-0.7995312)

\rput(2.8121874,-1.0295312){\color{gdarkgray}$1$}
\rput(1.0439062,1.0304687){\color{gdarkgray}$2$}
\rput(0.05296875,-1.0495312){\color{gdarkgray}$3$}

\end{pspicture} 
  \caption{3-Teilchen Kraft.}
\end{figure}

Für ein $N$-Teilchensystem gilt daher
\begin{align*}
\vec{F}_i(\vec{r}_1,\ldots,\vec{r}_N) = \sum_{j\neq i}
\vec{F}_{ji}(\vec{r}_i-\vec{r}_j).
\end{align*}

Es folgen nun direkt die 10 klassischen Erhaltungssätze. Diese gelten auch in
Theorien, die über die Klassische Mechanik hinausgehen.

\begin{prop}[Impulserhaltung]
Der \emph{totale Impuls} $\vec{p}_{\tot} = \sum_i \vec{p}_i$ mit $\vec{p}_i =
m_i\vec{v}_i$ ist erhalten.\fishhere
\end{prop}
\begin{proof}
Die Ableitung des totalen Impuls verschwindet, denn
\begin{align*}
\frac{\diffd}{\dt}  \sum_i \vec{p}_i &= \sum_i \frac{\diffd}{\dt} \vec{p}_i
= \sum_{i\neq j} \vec{F}_{ij} = \frac{1}{2} \left[
\sum_{i\neq j} \vec{F}_{ij} + \sum_{i\neq j} \vec{F}_{ij} \right] \\
&\overset{!}{=} \frac{1}{2} \left[
\sum_{i\neq j} \vec{F}_{ij} + \sum_{j\neq i}
\underbrace{\vec{F}_{ji}}_{-\vec{F}_{ij}} \right]
 = 0.\qedhere
\end{align*}
\end{proof}

\begin{prop}[Schwerpunktsatz]
Der \emph{Schwerpunkt}
\begin{align*}
\vec{r}_\tot = \frac{1}{M}\sum_i m_i \vec{r}_i,\quad M = \sum_i m_i,
\end{align*}
bewegt sich gleichförmig,
\begin{align*}
\vec{r}_\tot = \vec{r}_0 + \frac{\vec{p}_\tot t}{M}.
\end{align*}
$\vec{r}_0$ ist eine Erhaltungsgröße.\fishhere
\end{prop}

\begin{prop}[Drehimpulserhaltung]
Der \emph{totale Drehimpuls} $\vec{L}_\tot = \sum_i \vec{r}_i\times \vec{p}_i$
ist erhalten.\fishhere
\end{prop}
\begin{proof}
Der Beweis funktioniert analog zur Impulserhaltung,
\begin{align*}
\frac{\diffd}{\dt} \vec{L}_\tot &= \sum_i
\frac{\diffd}{\dt}\left(\vec{r}_i\times\vec{p}_i \right)
= \sum_i \left[\dvec{r}_i\times \vec{p}_i + \vec{r}_i \times\dvec{p}_i \right]
= \sum_i\vec{r}_i \times\dvec{p}_i = \sum_{i\neq j} \vec{r}_i\times \vec{F}_{ji}
\\ &= \frac{1}{2} \sum_{i\neq j} \left(\vec{r}_i-\vec{r}_j\right)\times
\frac{\vec{r}_i-\vec{r}_j}{\abs{\vec{r}_i-\vec{r}_j}}
f_{ji}(\abs{\vec{r}_i-\vec{r}_j}) = 0.\qedhere
\end{align*}
\end{proof}

\begin{prop}[Energieerhaltung]
Die \emph{Arbeit}, die ein Teilchen entlang der Kurve $\gamma$ im Kraftfeld
verrichtet, hat die Form,
\begin{align*}
A &= \int_\gamma \dvecs \vec{F}_i
= \int_{t_1}^{t_2} \dt \frac{\diffd}{\dt} \vec{r}_i(t) m\ddvec{r}_i(t)
= \int_{t_1}^{t_2} \dt \vec{v}_i \dvec{v}_i m_i
\\ &= \frac{1}{2} \int_{t_1}^{t_2} \dt \frac{\diffd}{\dt} \left(m_i
\vec{v}_i^2 \right) = T_2 - T_1,
\end{align*}
mit der kinetischen Energie $T = \dfrac{m_i}{2}\vec{v}_i^2 =
\dfrac{\vec{p}_i^2}{2m_i}$.

Ein Kraftfeld $\vec{F}_i$ heißt \emph{konservativ}, wenn das Integral
\begin{align*}
\int_\gamma \dvecs \vec{F}_i,
\end{align*}
unabhängig vom Weg ist.
Für konservative Kraftfelder existiert ein \emph{Potential}
$V(\vec{r})$, so dass
\begin{align*}
\vec{F} = -\nabla V.
\end{align*}
Somit gilt $A = T_1-T_0 = \int_\gamma \dvecs \vec{F} = V_0-V_1$.

Konservative Kraftfelder erfüllen somit die
Energieerhaltung,
\begin{align*}
H = T+V = \frac{\vec{p}^2}{2m} + V(\vec{r}) \equiv \text{const}.
\end{align*}
Die Summe der kinetischen und potentiellen Energie ist für alle
Punkte gleich.\fishhere
\end{prop}
\begin{figure}[!htbp]
  \centering
\begin{pspicture}(0,-1.1)(3,1.2)

\psline{->}(0.0,-0.84428024)(2.8,-0.84428024)
\psline{->}(0.2,-1.0442803)(0.2,1.1)
\psbezier{<-}(2.4258888,-0.25016913)(1.245889,0.009830868)(2.445889,1.1898309)(0.70588887,0.86983085)
\psbezier{->}(0.70588887,0.8898309)(0.6658889,0.24983087)(1.4258889,-0.47016913)(1.4258889,-0.09016913)(1.4258889,0.28983086)(0.9401748,0.23600471)(1.1858889,-0.47016913)(1.431603,-1.176343)(2.0837839,-1.1898309)(2.445889,-0.31016913)
\psdots[linecolor=darkblue](0.70588887,0.8898309)
\psdots[linecolor=darkblue](2.485889,-0.25016913)

\rput(2.087295,0.35308692){\color{gdarkgray}$\gamma$}
\rput(0.8808889,-0.04691308){\color{gdarkgray}$\sigma$}
\end{pspicture} 
  \caption{Äquivalente Wege in einem konservativen Kraftfeld.}
\end{figure}

\begin{bemn}[Bemerkungen.]
\begin{enumerate}[label=\arabic{*}.)]
\item 
Ein konservatives Kraftfeld erfüllt $\rot \vec{F} = 0$.
\item
Aus der Galilei-Invarianz haben wir gefolgert, dass die Kraft zwischen zwei
Teilchen die Form
\begin{align*}
\vec{F}_{ij} = \frac{\vec{r}_i -
\vec{r}_j}{\abs{\vec{r}_i-\vec{r}_j}}f_{ji}\left(\abs{\vec{r}_i -
\vec{r}_j}\right)
\end{align*}
hat und konservativ ist, d.h.
\begin{align*}
\vec{F}_{ij} &= -\nabla V\left(\abs{\vec{r}_i -
\vec{r}_j}\right),\quad V_{ij}(r) = \int\dr f_{ij}(r).
\end{align*}
\item Im Vielteilchensystem gilt daher,
\begin{align*}
H = \sum_i \frac{\vec{p}_i^2}{2m_i} + \frac{1}{2}\sum_{i\neq j}
V_{ij}\left(\abs{\vec{r}_i - \vec{r}_j}\right).
\end{align*}
\item Coulomb- und Gravitationspotential haben die Form,
\begin{align*}
V(\abs{\vec{r}_1-\vec{r}_2}) = \frac{e_1e_2}{\abs{\vec{r}_1-\vec{r}_2}}.\maphere
\end{align*}
\end{enumerate}
\end{bemn}

Die 10 Erhaltungssätze folgen direkt aus den 10 Symmetrie-Transformationen der
Galilei-Gruppe. Es ist ein fundamentales Konzept, Erhaltungsgrößen aus
Symmetrien zu gewinnen, das wir später mittels dem Noether Theorem elegant
formulieren können.

\subsubsection{Ausblick: Wann bricht die Klassische Mechanik zusammen?}
\begin{itemize}[label=\labelitem]
  \item Bei kleinen Distanzen wie dem Abstand
  zwischen Elektron und Atomkern verliert die Klassische Mechanik ihre
  Gültigkeit und man muss auf die Quantenmechanik ausweichen.
  \item Bei hohen Geschwindigkeiten $\frac{v}{c} \sim 1$ müssen wir auf die
  spezielle Relativitätstheorie.
  \item Bei Gravitation mit hohen Geschwindigkeiten oder starken
  Gravitationsfeldern stellt die allgemeine Relativitätstheorie den richtigen
  Rahmen.
\end{itemize}
Die Klassischen Mechanik ist deshalb aber nicht falsch, sondern folgt aus diesen
erweiterten Theorien immer als Spezialfall, wenn man zum Gültigkeitsbereich der
Klassischen Mechanik übergeht.
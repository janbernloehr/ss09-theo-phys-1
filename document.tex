% =============================================================================
% Titel:		Theoretische Physik I - Mitschrieb
% Erstellt:	SS 09
% Dozent:	Prof. H.P. Büchler
% Autor:	Jan-Cornelius Molnar
% =============================================================================
\documentclass[paper=a5,fleqn,DIV=calc]{scrartcl}

% =============================================================================
% 					Benötigte Pakete
% =============================================================================
\usepackage{janmcommon}
\usepackage{janmscript}

% =============================================================================
% 					Theorem Umgebungen
% =============================================================================
% Theorem Umgebungen *MIT* Numerierung
\theoremstyle{graymarginwithblueheader}
\theorembodyfont{\itshape}
\theoremseparator{}
\theoremsymbol{}

\newtheorem{prop}{Satz}[section]
\newtheorem{lem}[prop]{Lemma}
\newtheorem{defn}[prop]{Definition}
\newtheorem{cor}[prop]{Korollar}

\theoremstyle{yellowheader}
\theorembodyfont{\normalfont}
\theoremseparator{}
\theoremsymbol{}

\newtheorem{bsp}{Bsp}[section]

\theoremstyle{graymarginwithitblackheader}
\theorembodyfont{\normalfont}
\theoremseparator{}
\theoremsymbol{}

\newtheorem{bem}[prop]{Bemerkung.}

% Theorem Umgebungen *OHNE* Numerierung
\theoremstyle{graymarginwithblueheadern}
\theorembodyfont{\itshape}
\theoremseparator{}
\theoremsymbol{}

\newtheorem{propn}{Satz}
\newtheorem{lemn}{Lemma}
\newtheorem{corn}{Korollar}
\newtheorem{defnn}{Definition}

\theoremstyle{graymarginwithyellowheadern}
\theorembodyfont{\normalfont}
\theoremseparator{}
\theoremsymbol{}

\newtheorem{bspn}{Bsp}

\theoremstyle{graymarginwithitblackheadern}
\theorembodyfont{\normalfont}
\theoremseparator{}
\theoremsymbol{}

\newtheorem{bemn}{Bemerkung.}


% =============================================================================
% 					Überschriften-Style
% =============================================================================
\renewcommand\thesection{\arabic{section}}
\renewcommand\thesubsection{\arabic{section}.\arabic{subsection}}
\renewcommand\thesubsubsection{\small\ensuremath{\blacksquare}\normalsize}
\renewcommand\thebsp{\arabic{bsp}}

\titleformat{\section}%
  {\normalfont\bfseries\Huge\color{darkblue}}%
  {\thesection}%
  {0.6em}%
  {}%
  
\titleformat{\subsection}%
  {\normalfont\bfseries\Large\color{darkblue}}%
  {\thesubsection}%
  {0.6em}%
  {}%

\titleformat{\subsubsection}%
  {\normalfont\bfseries\color{darkblue}}%
  {\thesubsubsection}%
  {0.2em}%
  {}%
  
% =============================================================================
% 					Sonstiges
% =============================================================================

% 4-er Vektoren der SRT 
\newcommand{\rvec}[1]{\bm{\mathrm{#1}}}
% Symplektische Notation
\newcommand{\sym}[1]{\overline{#1}}

% =============================================================================
% 					Document-Body
% =============================================================================
\begin{document}

% Titel
\begin{center}
{\huge\bf Theoretische Physik I - Mitschrieb}

bei Prof. Dr. H.P. Büchler

Jan-Cornelius Molnar, Version: \today\ \thistime
\end{center}

% Inhaltsverzeichnis
\tableofcontents

% Inhalt
\newpage
\section{Grundprinzipien}

Wir beginnen mit den wichtigsten experimentellen Fakten, auf denen die
Klassische Mechanik aufbaut.

\subsection{Raum und Zeit}

Unser \emph{Raum} ist ein dreidimensionaler euklidischer Raum und die Zeit ist
eindimensional. Ein \emph{Teilchen} beschreibt eine Kurve
\begin{align*}
t\mapsto\vec{r}(t),
\end{align*}
in diesem $(3+1)$-dimensionalen Raum.
\begin{figure}[!htbp]
  \centering
\begin{pspicture}(-1,-1)(4.5,2.5)
 \psaxes[labels=none,ticks=none]{->}%
 	(0,0)(-0.5,-0.5)(4,2)%
 	[\color{gdarkgray}$\vec{r}\in\R^3$,-90]%
 	[\color{gdarkgray}$t$,0]
 \psyTick(0.5){\color{gdarkgray}t_0}
 \psyTick(1.5){\color{gdarkgray}t_1}
 
 \psline[linewidth=0.5pt](0,0.5)(4,0.5)
 \psline[linewidth=0.5pt](0,1.5)(4,1.5)
 
 \psbezier[linecolor=darkblue,arrows=->,linewidth=1.2pt]
 (2.42,-0.6170833)(0.78,0.8229167)(4.18,0.42291668)(3.38,1.7829167)
 
 \end{pspicture}
  \caption{Teilchen Trajektorie.}
\end{figure}

Der Raum ist homogen und isotrop, d.h. kein Punkt und keine Richtung sind
ausgezeichnet. Ebenso ist die Zeit homogen, das heißt kein Zeitpunkt ist
ausgezeichnet.

Die \emph{Geschwindigkeit} ist die Ableitung der Raumkurve nach der Zeit,
\begin{align*}
\vec{v}(t) =\frac{\diffd}{\dt} \vec{r}(t) = \dot{\vec{r}}(t),
\end{align*}
die \emph{Beschleunigung} ist die zweite Ableitung der Raumkurve,
\begin{align*}
\vec{a}(t) = \frac{\diffd^2}{\dt^2}\vec{r}(t) = \ddot{\vec{r}}(t).
\end{align*}

\subsubsection{Trägheitsgesetz (1. Newtonsche Gesetz)}
Es gibt \emph{Inertialsysteme}, in denen alle Naturgesetze zu allen
Zeiten gleich sind. Ein System, das sich gleichförmig zu einem Inertialsystem
bewegt, ist selbst ein Inertialsystem.
\begin{figure}[!htbp]
  \centering
\begin{pspicture}(-0.5,-0.5)(5.5,4.5)

\psline{->}(-0.2,0)(3.5,0)
\psline{->}(0,-0.2)(0,2)

\psline{->}(1.3,2)(5,2)
\psline{->}(1.5,1.8)(1.5,4)

\psline[linestyle=dashed](0,0)(2.8,2.4)
\psline[linestyle=dashed](1.5,2)(2.8,2.4)

\psdot[linecolor=darkblue](2.8,2.4)

\rput(-0.2,2){\color{gdarkgray}$y$}
\rput(3.8,-0.2){\color{gdarkgray}$x$}

\rput(1.3,4){\color{gdarkgray}$y'$}
\rput(5.3,1.8){\color{gdarkgray}$x'$}
 
 \rput(2,2.4){\color{gdarkgray}$\vec{r}'$}
 \rput(1,1.1){\color{gdarkgray}$\vec{r}$}
 

\end{pspicture} 
  \caption{Zwei Inertialsysteme.}
\end{figure}

Die Beschleunigung, die ein Teilchen in einem Inertialsystem erfährt, ist somit
in allen Inertialsystemen gleich. Insbesondere erfährt ein kräftefreies Teilchen
keine Beschleunigung und bewegt sich gleichförmig,
\begin{align*}
\vec{r}(t) = \vec{r}_0 + \vec{v}t \Rightarrow \ddvec{r} = 0.
\end{align*}

Eine gute Approximation eines Inertialsystems erhalten wir, indem wir uns an
den Fixsternen orientieren. Doch selbst hier gibt es noch Abweichungen, da sich
auch unserer Galaxie dreht. Für die meisten Experimente genügt es jedoch, die
Erdoberfläche als Ineratialsystem zu wählen. Für große Geschwindigkeiten und
Strecken müssen jedoch Effekte wie die Corioliskraft, die durch die Erdrotation
entstehen, berücksichtigt werden, was man am Foucaultsches Pendel oder an der
Rotation von Wirbelstürmen beobachten kann.

\subsubsection{Bewegungsgesetz (2. Newtonsche Gesetz)}
Die \emph{Masse} $m$ eines Teilchens ist eine skalare Größe und daher vom
Inertialsystem unabhängig. Sie ist eine Eigenschaft der Teilchen.

Der \emph{Impuls} eines Teilchens ist gegeben durch,
\begin{align*}
\vec{p}(t) = m\vec{v}(t) = m\dot{\vec{r}}(t).
\end{align*}
Somit folgt die Bewegungsgleichung im Inertialsystem,
\begin{align*}
\frac{\diffd \vec{p}}{\dt} = m\ddvec{r} = \vec{F},
\end{align*}
wobei $\vec{F}$ die \emph{Kraft}, die auf das Teilchen wirkt, bezeichnet.
Für $N$-Teilchen erhalten wir,
\begin{align*}
m_i \ddvec{r}_i = \vec{F}_i(\vec{r}_1,\ldots, \vec{r}_N,
\vec{v}_1,\ldots,\vec{v}_N,t).
\end{align*}
Dies ist eine Differentialgleichung 2. Ordnung. Die Trajektorien $x_i(t)$ sind
eindeutig durch Ort und Geschwindigkeit zur Anfangszeit $t_0$ bestimmt. Die
klassische Mechanik beschreibt also ein vollkommen deterministisches Weltbild.

Bisher lässt sich die Masse experimentell nicht absolut sondern nur in Relation
zur Masse eines anderen Körpers bestimmen. Dies erfolgt durch den Vergleich der
Beschleunigungen für eine feste Kraft,
\begin{align*}
&\ddot{r}_a = \frac{F}{m_a},\quad\ddot{r}_b = \frac{F}{m_b}\\
\Rightarrow & \frac{\ddot{r}_a}{\ddot{r}_b} = \frac{m_b}{m_a}.
\end{align*}

%Es ist hier nicht auszuschließen, dass die Kraft, die auf ein Teilchen wirkt,
%auch von den Geschwindigkeiten der $N$-Teilchen abhängt.

\subsubsection{Actio = Reactio (3. Newtonsche Gesetz)}
Die Kraft $\vec{F}_{21}$, die das zweite Teilchen auf das erste ausübt, ist
entgegengesetzt zu der Kraft $\vec{F}_{12}$, die das erste Teilchen auf das
zweite ausübt. $\vec{F}_{21}=-\vec{F}_{12}$.
\begin{figure}[!htbp]
\centering
\begin{pspicture}(-0.1,-1.15)(2.5,1.2)

\psline[linecolor=darkblue]{->}(2.14,0.74828124)(1.34,0.20828125)
\psline[linecolor=yellow]{<-}(0.88,-0.11171875)(0.08,-0.65171874)

\psdots[dotsize=0.12](0.06,-0.6717188)
\psdots[dotsize=0.12](2.14,0.74828124)

\rput(0.006875,-1.0017188){\color{gdarkgray}$1$}
\rput(2.3785937,0.97828126){\color{gdarkgray}$2$}
\rput(0.7496875,-0.48171875){\color{gdarkgray}$\vec{F}_{21}$}
\rput(1.4296875,0.6382812){\color{gdarkgray}$\vec{F}_{12}$}
\end{pspicture}

\caption{Actio und Reactio.}
\end{figure}

Für zwei wechselwirkende Teilchen gilt daher die Beziehung,
\begin{align*}
\ddvec{r}_1 &= \frac{\vec{F}_{21}}{m_1} = - \frac{\vec{F}_{12}}{m_1} = -
\frac{m_2}{m_1}\ddvec{r}_2,\\
\frac{\ddvec{r}_1}{\ddvec{r}_2} &= -\frac{m_2}{m_1}.
\end{align*}

\subsubsection{Kraftgesetze}
In der Natur kommen vier fundamentale Wechselwirkungen vor, nämlich Gravitation,
Elektromagnetische Wechselwirkung, Starke und Schwache Wechselwirkung. In de
Klassischen Mechanik spielen lediglich Gravitation und Elektromagnetismus 
eine Rolle, die durch folgende Gesetze beschreiben werden.

Das \emph{Coulombsche Gesetz} hat die Form,
\begin{align*}
\vec{F}_{12} = \frac{\vec{r}_1-\vec{r}_2}{\abs{\vec{r}_1-\vec{r}_2}}
\frac{e_1 e_2}{\abs{\vec{r}_1-\vec{r}_2}^2},\qquad \text{im cgs-System},
\end{align*}
wobei die Ladung $e_i$ im Gegensatz zur Masse $m_i$ unabhängig vom Träger ist.

Man kann die schwere Masse als Ladung bezüglich der Gravitation
betrachten. Die entsprechende Beziehung stellt das \emph{Gravitationsgesetz}
her
\begin{align*}
\vec{F}_{12} = -G \frac{\vec{r}_1-\vec{r}_2}{\abs{\vec{r}_1-\vec{r}_2}}
\frac{\hat{m}_1 \hat{m}_2}{\abs{\vec{r}_1-\vec{r}_2}^2}.
\end{align*}
Es stellt sich nun die Fragem ob die schwere Masse $\hat{m}_i$ des
Gravitationsgesetzes gleich der trägen Massen $m_i$ des Trägheitsgesetzes ist.
Loránd Eötvös\footnote{Loránd Eötvös (* 27. Juli 1848 in Buda; † 8. April 1919
in Budapest) war ein ungarischer Physiker. International bekannt war er als
(Baron) Roland von Eötvös} erzielte 1909 das experimentelle Resultat,
\begin{align*}
\frac{\hat{m}}{m} = \text{universelle Konstante}.
\end{align*}
$G$ ist nun so bestimmt, dass die schwere Masse identisch zur trägen Masse ist.
Diese Gleichheit ist auch eine der Grundannahmen in der allgemeinen
Relativitätstheorie.
\begin{figure}[!htbp]
  \centering
\begin{pspicture}(-0.1,-1.3)(3.2,1.2)

\psline{->}(2.52,-0.71171874)(2.02,-0.21171875)
\psline{->}(1.02,0.7682812)(1.52,0.26828125)
\psline{->}(0.0,-1.2317188)(0.92,0.6482813)
\psline{->}(0.0,-1.2317188)(2.4,-0.75171876)

\psdots[linecolor=darkblue](2.52,-0.71171874)
\psdots[linecolor=darkblue](1.0,0.78828126)

\psdots(0,-1.2317188)

\rput(2.686875,-1){\color{gdarkgray}$m_1,e_1$}
\rput(0.75859374,1.0982813){\color{gdarkgray}$m_2,e_2$}

\rput(0.3,-0.08171875){\color{gdarkgray}$\vec{r}_2$}
\rput(1.46,-1.1217188){\color{gdarkgray}$\vec{r}_1$}
\end{pspicture}
  \caption{2-Teilchen Kraft.}
\end{figure}



\subsubsection{Galilei-Transformation}
Eine \emph{Galilei-Transformation} beschreibt den Übergang von einem
Inertialsystem in ein anderes.
\begin{align*}
&t' = t + a,\\
&\vec{r}' = \mathrm{R}\vec{r} + \vec{v}t + \vec{b},
\end{align*}
dabei sind $a,\vec{b}$ und $\vec{v}$ konstant und $R$ eine Drehmatrix.
\begin{figure}[!htbp]
  \centering
\begin{pspicture}(-0.5,-0.5)(5.5,4.5)

\psline{->}(-0.2,0)(3.5,0)
\psline{->}(0,-0.2)(0,2)

\psline{->}(1.3,2)(5,2)
\psline{->}(1.5,1.8)(1.5,4)

\psline[linecolor=darkblue]{->}(0,0)(1.4,1.9)

\rput(-0.2,2){\color{gdarkgray}$t$}
\rput(3.8,-0.2){\color{gdarkgray}$\vec{r}$}

\rput(1.3,4){\color{gdarkgray}$t'$}
\rput(5.3,1.8){\color{gdarkgray}$\vec{r}'$}
 
 \rput[tl](1.2,1.6){\color{gdarkgray}$(a,\vec{b})$}
\end{pspicture} 
  \caption{Galilei Transformation.}
\end{figure}

Im $(3+1)$ dimensionalen Raum hat die Transformation die Form
\begin{align*}
\begin{pmatrix}
t'\\
\vec{r}'
\end{pmatrix}
= \begin{pmatrix}
  1 & 0 \\
  \vec{v} & \mathrm{R} 
  \end{pmatrix}
\begin{pmatrix}
t\\
\vec{r}
\end{pmatrix}
+ \begin{pmatrix}
  a\\\vec{b}
  \end{pmatrix}.
\end{align*}
Die Translation hat somit insgesamt 10 Parameter,
\begin{itemize}[label=\labelitem]
  \item Translation in Raum und Zeit $\vec{b}, a$ ($4$-Parameter)
  \item Boost $\vec{v}$ ($3$-Parameter)
  \item Rotationen im Raum $R\in\mathrm{SO}(3)$ ($3$-Parameter)
\end{itemize}
Unter $\mathrm{SO}(3)$ verstehen wir,
\begin{align*}
\mathrm{SO}(3) = \setdef{R\in M_{3\times 3}}{R^tR = RR^t = \Id\text{ und }\det
R = 1}.
\end{align*}
\begin{bsp}
Rotation um die $z$-Achse mit Winkel $\th$,
\begin{align*}
R = \begin{pmatrix}
\cos \th & \sin \th & 0\\
-\sin\th & \cos \th & 0\\
0 & 0 & 1
\end{pmatrix}
\end{align*}
Rotation um die $x$-Achse mit Winkel $\th$,
\begin{align*}
R = \begin{pmatrix}
    1& 0 & 0 \\
0 & \cos \th & \sin \th\\
0 & -\sin\th & \cos \th
\end{pmatrix}\bsphere
\end{align*}
\end{bsp}
Die Menge der Galilei-Transformationen bildet eine 10-parametrige Gruppe, die
\emph{Galilei-Gruppe} $\GG$.
\begin{bemn}
Die Galilei Transformationen $g\in\GG$ haben die Eigenschaften.
\begin{itemize}[label=\labelitem]
  \item $g_2\circ g_1\in G$,
  \item $(g_3\circ g_2)\circ g_1 = g_3 \circ (g_2\circ g_1)$ Assoziativität,
  \item $g\circ\Id = g = \Id \circ g$ Einselement,
  \item Zu $g\in\GG$ existiert ein Inverses $g^{-1}$ mit $g^{-1}\circ g = \Id =
  g \circ g^{-1}$,
  \item $\GG$ ist nicht kommutativ.\maphere
\end{itemize} 
\end{bemn}
 Neben den gewöhnlichen Transformationen gibt es
noch zwei diskrete Transformationen
\begin{itemize}[label=\labelitem]
  \item Raumspiegelung $\vec{r}' = -\vec{r}$
  \item Zeitumkehr $t' = -t$
\end{itemize}

Die 10 Parameter der Galilei-Transformation beschreiben 10 Symmetrien, aus
denen wir 10 Erhaltungssätze herleiten können.

\subsection{Symmetrien und Erhaltungssätze}

Die Gesetze der Klassischen Mechanik für ein abgeschlossenes
System\footnote{abgeschlossen bedeutet, dass keine Kräfte von Teilchen
ausgehen, die sich außerhalb des Systems befinden.} sind forminvariant unter
Galilei-Transformationen.

\begin{bemn}
Man sagt auch, dass die Galilei-Gruppe die Symmetrie-Gruppe der Klassischen
Mechanik ist (abgesehen von der Zeitumkehr).\maphere
\end{bemn}

\begin{propn}[Homogenität des Raums]
Kein Punkt des Raums ist ausgezeichnet, es sind daher nur Relativbewegungen von
Bedeutung und der Nullpunkt kann frei gewählt werden.\fishhere
\end{propn}
Betrachten wir die Kraft $\vec{F}_{12}(\vec{r}_1,\vec{r}_2,t)$, dann
bewirkt die Homogenität des Raums, dass die Kraft unter einer
Galilei-Transformation mit $\vec{r}_1' = \vec{r}_1+\vec{a}$ und $\vec{r}_2' =
\vec{r}_2+\vec{a}$ erhalten bleibt,
\begin{align*}
\vec{F}_{12}(\vec{r}_1',\vec{r}_2',t) =
\vec{F}_{12}(\vec{r}_1,\vec{r}_2,t).
\end{align*}

\begin{propn}[Homogenität der Zeit]
Es existiert kein ausgezeichneter Zeitpunkt, also sind alle Fundamentalkräfte
zeitunabhängig.\fishhere
\end{propn}

\begin{propn}[Isotropie des Raumes]
Der Raum hat keine ausgezeichnete Richtung, betrachtet man also zwei Teilchen,
so ist die einzig verfügbare Richtung $\vec{r}_1-\vec{r}_2$.\fishhere
\end{propn}

\begin{propn}[Invarianz unter boost]
Es gibt kein absolutes Ruhesystem, die Geschwindigkeit hat nur eine
relative Bedeutung $\vec{v}_1-\vec{v}_2$.\fishhere
\end{propn}

\subsubsection{Folgerungen aus der Galilei-Invarianz}
$2$-Teilchen Kräfte, die geschwindigkeitsunabhängig sind, können geschrieben
werden als
\begin{align*}
\vec{F}_{ij} =
\frac{\vec{r}_i-\vec{r}_j}{\abs{\vec{r}_i-\vec{r}_j}}f_{ij}\left(\abs{\vec{r}_i
- \vec{r}_j}\right),
\end{align*}
da nur der Abstand $\abs{\vec{r}_i - \vec{r}_j}$ und die Richtung
$\vec{r}_i-\vec{r}_j$ identifizierbar sind.
\begin{bemn}
\begin{enumerate}[label=\arabic{*}.)]
  \item 
Wir haben bereits gesehen, dass die Fundamentalkräfte Coulomb-Wechselwirkung
und Gravitationskraft von dieser Gestalt sind. Jetzt sehen wir, dass es auch
die einzig mögliche ist.
\item Das 3. Newtonsche Gesetzt folgt direkt aus der
Galilei-Invarianz.
\item Fundamentale Kräfte, die von der Geschwindigkeit der Teilchen abhängen
existieren in der Newtonschen Mechanik nicht.
\item In einem $3$-Teilchen System könnte neben der $2$-Teilchen Wechselwirkung auch
eine fundamentale $3$-Teilchen Wechselwirkung auftreten, also eine Kraft, die
genau dann auftrit, wenn $3$ Teilchen vorhanden sind, und nicht das Ergebnis
einer Überlagerung von $2$-Teilchen Wechselwirkungen ist. Das Experiment zeigt
jedoch, dass alle bis jetzt bekannten fundamentalen Kräfte reine $2$-Teilchen
Kräfte sind. 
\end{enumerate}
\end{bemn}

\begin{figure}[!htbp]
  \centering
\begin{pspicture}(-0.1,-1.2)(3,1.2)

\psline{->}(2.6453125,-0.77953124)(2.1453125,-0.27953124)
\psline{->}(1.1453125,0.7004688)(1.6453125,0.20046875)
\psline{->}(0.3253125,-0.77953124)(1.1253124,-0.77953124)
\psline{<-}(1.8453125,-0.77953124)(2.6453125,-0.77953124)
\psline{<-}(0.8253125,0.12046875)(1.1253124,0.7004688)
\psline{->}(0.3453125,-0.75953126)(0.6653125,-0.19953126)

\psdots[linecolor=darkblue](2.6453125,-0.77953124)
\psdots[linecolor=darkblue](1.1253124,0.72046876)
\psdots[linecolor=darkblue](0.3253125,-0.7995312)

\rput(2.8121874,-1.0295312){\color{gdarkgray}$1$}
\rput(1.0439062,1.0304687){\color{gdarkgray}$2$}
\rput(0.05296875,-1.0495312){\color{gdarkgray}$3$}

\end{pspicture} 
  \caption{3-Teilchen Kraft.}
\end{figure}

Für ein $N$-Teilchensystem gilt daher
\begin{align*}
\vec{F}_i(\vec{r}_1,\ldots,\vec{r}_N) = \sum_{j\neq i}
\vec{F}_{ji}(\vec{r}_i-\vec{r}_j).
\end{align*}

Es folgen nun direkt die 10 klassischen Erhaltungssätze. Diese gelten auch in
Theorien, die über die Klassische Mechanik hinausgehen.

\begin{prop}[Impulserhaltung]
Der \emph{totale Impuls} $\vec{p}_{\tot} = \sum_i \vec{p}_i$ mit $\vec{p}_i =
m_i\vec{v}_i$ ist erhalten.\fishhere
\end{prop}
\begin{proof}
Die Ableitung des totalen Impuls verschwindet, denn
\begin{align*}
\frac{\diffd}{\dt}  \sum_i \vec{p}_i &= \sum_i \frac{\diffd}{\dt} \vec{p}_i
= \sum_{i\neq j} \vec{F}_{ij} = \frac{1}{2} \left[
\sum_{i\neq j} \vec{F}_{ij} + \sum_{i\neq j} \vec{F}_{ij} \right] \\
&\overset{!}{=} \frac{1}{2} \left[
\sum_{i\neq j} \vec{F}_{ij} + \sum_{j\neq i}
\underbrace{\vec{F}_{ji}}_{-\vec{F}_{ij}} \right]
 = 0.\qedhere
\end{align*}
\end{proof}

\begin{prop}[Schwerpunktsatz]
Der \emph{Schwerpunkt}
\begin{align*}
\vec{r}_\tot = \frac{1}{M}\sum_i m_i \vec{r}_i,\quad M = \sum_i m_i,
\end{align*}
bewegt sich gleichförmig,
\begin{align*}
\vec{r}_\tot = \vec{r}_0 + \frac{\vec{p}_\tot t}{M}.
\end{align*}
$\vec{r}_0$ ist eine Erhaltungsgröße.\fishhere
\end{prop}

\begin{prop}[Drehimpulserhaltung]
Der \emph{totale Drehimpuls} $\vec{L}_\tot = \sum_i \vec{r}_i\times \vec{p}_i$
ist erhalten.\fishhere
\end{prop}
\begin{proof}
Der Beweis funktioniert analog zur Impulserhaltung,
\begin{align*}
\frac{\diffd}{\dt} \vec{L}_\tot &= \sum_i
\frac{\diffd}{\dt}\left(\vec{r}_i\times\vec{p}_i \right)
= \sum_i \left[\dvec{r}_i\times \vec{p}_i + \vec{r}_i \times\dvec{p}_i \right]
= \sum_i\vec{r}_i \times\dvec{p}_i = \sum_{i\neq j} \vec{r}_i\times \vec{F}_{ji}
\\ &= \frac{1}{2} \sum_{i\neq j} \left(\vec{r}_i-\vec{r}_j\right)\times
\frac{\vec{r}_i-\vec{r}_j}{\abs{\vec{r}_i-\vec{r}_j}}
f_{ji}(\abs{\vec{r}_i-\vec{r}_j}) = 0.\qedhere
\end{align*}
\end{proof}

\begin{prop}[Energieerhaltung]
Die \emph{Arbeit}, die ein Teilchen entlang der Kurve $\gamma$ im Kraftfeld
verrichtet, hat die Form,
\begin{align*}
A &= \int_\gamma \dvecs \vec{F}_i
= \int_{t_1}^{t_2} \dt \frac{\diffd}{\dt} \vec{r}_i(t) m\ddvec{r}_i(t)
= \int_{t_1}^{t_2} \dt \vec{v}_i \dvec{v}_i m_i
\\ &= \frac{1}{2} \int_{t_1}^{t_2} \dt \frac{\diffd}{\dt} \left(m_i
\vec{v}_i^2 \right) = T_2 - T_1,
\end{align*}
mit der kinetischen Energie $T = \dfrac{m_i}{2}\vec{v}_i^2 =
\dfrac{\vec{p}_i^2}{2m_i}$.

Ein Kraftfeld $\vec{F}_i$ heißt \emph{konservativ}, wenn das Integral
\begin{align*}
\int_\gamma \dvecs \vec{F}_i,
\end{align*}
unabhängig vom Weg ist.
Für konservative Kraftfelder existiert ein \emph{Potential}
$V(\vec{r})$, so dass
\begin{align*}
\vec{F} = -\nabla V.
\end{align*}
Somit gilt $A = T_1-T_0 = \int_\gamma \dvecs \vec{F} = V_0-V_1$.

Konservative Kraftfelder erfüllen somit die
Energieerhaltung,
\begin{align*}
H = T+V = \frac{\vec{p}^2}{2m} + V(\vec{r}) \equiv \text{const}.
\end{align*}
Die Summe der kinetischen und potentiellen Energie ist für alle
Punkte gleich.\fishhere
\end{prop}
\begin{figure}[!htbp]
  \centering
\begin{pspicture}(0,-1.1)(3,1.2)

\psline{->}(0.0,-0.84428024)(2.8,-0.84428024)
\psline{->}(0.2,-1.0442803)(0.2,1.1)
\psbezier{<-}(2.4258888,-0.25016913)(1.245889,0.009830868)(2.445889,1.1898309)(0.70588887,0.86983085)
\psbezier{->}(0.70588887,0.8898309)(0.6658889,0.24983087)(1.4258889,-0.47016913)(1.4258889,-0.09016913)(1.4258889,0.28983086)(0.9401748,0.23600471)(1.1858889,-0.47016913)(1.431603,-1.176343)(2.0837839,-1.1898309)(2.445889,-0.31016913)
\psdots[linecolor=darkblue](0.70588887,0.8898309)
\psdots[linecolor=darkblue](2.485889,-0.25016913)

\rput(2.087295,0.35308692){\color{gdarkgray}$\gamma$}
\rput(0.8808889,-0.04691308){\color{gdarkgray}$\sigma$}
\end{pspicture} 
  \caption{Äquivalente Wege in einem konservativen Kraftfeld.}
\end{figure}

\begin{bemn}[Bemerkungen.]
\begin{enumerate}[label=\arabic{*}.)]
\item 
Ein konservatives Kraftfeld erfüllt $\rot \vec{F} = 0$.
\item
Aus der Galilei-Invarianz haben wir gefolgert, dass die Kraft zwischen zwei
Teilchen die Form
\begin{align*}
\vec{F}_{ij} = \frac{\vec{r}_i -
\vec{r}_j}{\abs{\vec{r}_i-\vec{r}_j}}f_{ji}\left(\abs{\vec{r}_i -
\vec{r}_j}\right)
\end{align*}
hat und konservativ ist, d.h.
\begin{align*}
\vec{F}_{ij} &= -\nabla V\left(\abs{\vec{r}_i -
\vec{r}_j}\right),\quad V_{ij}(r) = \int\dr f_{ij}(r).
\end{align*}
\item Im Vielteilchensystem gilt daher,
\begin{align*}
H = \sum_i \frac{\vec{p}_i^2}{2m_i} + \frac{1}{2}\sum_{i\neq j}
V_{ij}\left(\abs{\vec{r}_i - \vec{r}_j}\right).
\end{align*}
\item Coulomb- und Gravitationspotential haben die Form,
\begin{align*}
V(\abs{\vec{r}_1-\vec{r}_2}) = \frac{e_1e_2}{\abs{\vec{r}_1-\vec{r}_2}}.\maphere
\end{align*}
\end{enumerate}
\end{bemn}

Die 10 Erhaltungssätze folgen direkt aus den 10 Symmetrie-Transformationen der
Galilei-Gruppe. Es ist ein fundamentales Konzept, Erhaltungsgrößen aus
Symmetrien zu gewinnen, das wir später mittels dem Noether Theorem elegant
formulieren können.

\subsubsection{Ausblick: Wann bricht die Klassische Mechanik zusammen?}
\begin{itemize}[label=\labelitem]
  \item Bei kleinen Distanzen wie dem Abstand
  zwischen Elektron und Atomkern verliert die Klassische Mechanik ihre
  Gültigkeit und man muss auf die Quantenmechanik ausweichen.
  \item Bei hohen Geschwindigkeiten $\frac{v}{c} \sim 1$ müssen wir auf die
  spezielle Relativitätstheorie.
  \item Bei Gravitation mit hohen Geschwindigkeiten oder starken
  Gravitationsfeldern stellt die allgemeine Relativitätstheorie den richtigen
  Rahmen.
\end{itemize}
Die Klassischen Mechanik ist deshalb aber nicht falsch, sondern folgt aus diesen
erweiterten Theorien immer als Spezialfall, wenn man zum Gültigkeitsbereich der
Klassischen Mechanik übergeht.

\newpage
\section{Lagrangsche Mechanik}
Im Folgenden wollen wir die Newtonschen Gleichungen mit Hilfe eines
Variationsprinzips, dem so genannten ``Hamiltonschen Prinzip der kleinsten
Wirkung'' formulieren, mit dem wir die Newtonschen Gesetze elegant und
mathematisch exakt in beliebige Koordiantensysteme transformieren können.

\subsection{Euler-Lagrange Gleichungen}
Betrachte ein System von $N$ Teilchen mit den Koordinaten $q^i(t)$,
$i=1,\ldots,3N$ in einem konservativen Kraftfeld.
\begin{align*}
F^i(q^1,\ldots,q^{3N},t) = -\frac{\partial}{\partial q^i}V(q^1,\ldots,q^{3N},t).
\end{align*}

\begin{bemn}[Bemerkung zur Notation.]
Die Beziehung zwischen unseren bisherigen Ortsvektoren $\vec{r}_j$ und den
Koordinaten $q^i$ ist folgende:
\begin{align*}
&\vec{r}_j = \begin{pmatrix}
q^{3j-2}\\q^{3j-1}\\q^{3j}
\end{pmatrix},
\vec{r}_1 = \begin{pmatrix}
q^{1}\\q^{2}\\q^{3}
\end{pmatrix},
\vec{r}_2 = \begin{pmatrix}
q^{4}\\q^{5}\\q^{6}
\end{pmatrix},
\ldots\\
&\vec{F}_j =
\begin{pmatrix}
F^{3j-2}\\F^{3j-1}\\F^{3j}
\end{pmatrix},
\vec{F}_1 = \begin{pmatrix}
F^{1}\\F^{2}\\F^{3}
\end{pmatrix},
\vec{F}_2 = \begin{pmatrix}
F^{4}\\F^{5}\\F^{6}
\end{pmatrix},
\ldots\maphere
\end{align*}
\end{bemn}

Die Newtonschen Gleichungen haben somit die Form,
\begin{align*}
m_i\ddot{q}^i  = -\frac{\partial}{\partial q^i} V(q^j,t) = F^i(q^j,t)
\end{align*}
mit der  Kurzschreibweise $V(q^1,\ldots,q^{3N},t) = V(q^i,t)$.

Wir können den linken Teil auch als partielle Ableitung der kinetischen Energie
$T$ schreiben,
\begin{align*}
m_i \ddot{q}^i = \frac{\diffd}{\dt}\left(m_i \dot{q}^i\right)
= \frac{\diffd}{\dt}\left(\frac{1}{2}m_i
\frac{\partial}{\partial \dot{q}^i}\left(\dot{q}^i\right)^2\right)
= \frac{\diffd}{\dt}\frac{\partial}{\partial \dot{q}^i}T,
\end{align*}
mit $T = \sum_i \frac{m_i}{2}\left(q^i\right)^2$.

Wir definieren die \emph{Lagrange Funktion} durch,
\begin{align*}
L(q^i,\dot{q}^i,t) &= \sum_i \frac{m}{2} \left(\dot{q}^i\right)^2 - V(q^i,t),
\end{align*}
oder kurz $L=T-V$.

Die Bewegungsgleichungen folgen somit aus den \emph{Euler-Lagrange
Gleichungen},
\begin{align*}
&\frac{\diffd}{\dt}\frac{\partial L}{\partial \dot{q}^i} - \frac{\partial L
}{\partial q^i} = 0 \Leftrightarrow\; \frac{\diffd}{\dt}\frac{\partial T}{\partial\dot{q}^i} +
\frac{\partial V}{\partial q^i} = 0.
\end{align*}

\begin{bemn}[Bemerkung zur Notation.]
\begin{align*}
\frac{\partial L}{\partial \vec{r}_1} :=
\begin{pmatrix}
\frac{\partial L}{\partial q_1}\\ \frac{\partial L}{\partial q_2}\\
\frac{\partial L}{\partial q_3}
\end{pmatrix},\qquad 
\frac{\partial L}{\partial \dvec{r}_1} :=
\begin{pmatrix}
\frac{\partial L}{\partial \dot{q}_1}\\ \frac{\partial L}{\partial \dot{q}_2}\\
\frac{\partial L}{\partial \dot{q}_3}
\end{pmatrix}.\maphere
\end{align*}
\end{bemn}

\begin{bsp}
%\begin{enumerate}[label=\arabic{*}.)]
%  \item
 Ein Teilchen befinde sich in einem Gravitationsfeld auf Höhe $x$. Die
  kinetische Energie ist gegeben durch $T= \frac{m}{2}\dot{x}^2$, die
  potentielle Energie ist $V=mgx$, wodurch sich die Kraft $F_x =
  -\frac{\partial V}{\partial x} = -mg$ ergibt.
\begin{figure}[!htbp]
  \centering
\begin{pspicture}(-0.1,-1.25)(2.5334375,1.2304688)

\psframe[fillstyle=solid,fillcolor=glightgray,linestyle=none](2.2,-0.95)(0.1,-1.15)

\psline{->}(0.05,-0.95)(2.36,-0.95)
\psline{->}(0.25,-1.15)(0.25,1.05)
\psline{->}(1,0.2)(1,-0.45)
\psdots[linecolor=darkblue](1,0.2)

\rput(1.2275,-0.08046875){\color{gdarkgray}$\vec{F}$}

\rput(2.4032812,-1.0804688){\color{gdarkgray}$y$}
\rput(0.085625,1.1395313){\color{gdarkgray}$x$}
\end{pspicture} 
  \caption{Teilchen im Gravitationspotential.}
\end{figure} 

  Die Lagrange Funktion hat nun die Form,
\begin{align*}
&L(x,\dot{x}) = T-V = \frac{m}{2}\dot{x}^2 - mgx,\\
&\frac{\diffd}{\dt}\frac{\partial L}{\partial \dot{x}} - \frac{\partial
L}{\partial x} = m\ddot{x}+ mg = 0\\
\Rightarrow\; & \ddot{x} = -g.\bsphere
\end{align*}
\end{bsp}
\begin{bsp}
%\item
Zwei Teilchen mit dem Wechselwirkungspotential
$V\left(\abs{\vec{r}_1-\vec{r}_2}\right)$.

\begin{figure}[!htbp]
  \centering
\begin{pspicture}(-0.5,-1.5)(3,1.2)

\psline{->}(-0.2,-1.2)(2.8,-1.2)
\psline{->}(0,-1.4)(0,1)

\psline{->}(2.52,-0.71171874)(2.02,-0.21171875)
\psline{->}(1.02,0.7682812)(1.52,0.26828125)
\psline{->}(0.0,-1.2)(0.92,0.6482813)
\psline{->}(0.0,-1.2)(2.4,-0.75171876)

\psdots[linecolor=darkblue](2.52,-0.71171874)
\psdots[linecolor=darkblue](1.0,0.78828126)

\rput(1.8,0.5){\color{gdarkgray}$\vec{F}_{21}$}
\rput(2.4,-0.1){\color{gdarkgray}$\vec{F}_{12}$}

\rput(0.3,-0.08171875){\color{gdarkgray}$\vec{r}_2$}
\rput(1.4,-0.7){\color{gdarkgray}$\vec{r}_1$}
\end{pspicture} 
  \caption{Zwei Teilchen mit Wechselwirkungspotential.}
\end{figure}

Kinetische Energie,
\begin{align*}
T = \frac{m_1}{2}\dvec{r}_1^2 + \frac{m_2}{2}\dvec{r}_2^2\entspr \sum_i
\frac{m_i}{2} (\dot{q}^i)^2.
\end{align*}
Potentielle Energie,
\begin{align*}
V\left(\abs{\vec{r}_1-\vec{r}_2}\right).
\end{align*}
Die Lagrange-Funktion hat somit die Form,
\begin{align*}
&L(\vec{r}_1,\vec{r}_2,\dvec{r}_1,\dvec{r}_2) =  \sum_i \frac{m_i}{2} (\dot{q}^i)^2 -
V\left(\abs{\vec{r}_1-\vec{r}_2}\right),\\
&\frac{\partial L}{\partial \vec{r}_1} =
-\frac{\vec{r}_1-\vec{r}_2}{\abs{\vec{r}_1-\vec{r}_2}}
V'\left(\abs{\vec{r}_1-\vec{r}_2}\right) = \vec{F}_{21}\\
&\frac{\partial L}{\partial \vec{r}_2} =
-\frac{\vec{r}_2-\vec{r}_1}{\abs{\vec{r}_2-\vec{r}_1}}
V'\left(\abs{\vec{r}_1-\vec{r}_2}\right) = \vec{F}_{12} = -\vec{F}_{21}\\
&\frac{\partial L}{\partial \dvec{r}_1}
= m_1\dvec{r}_1,\qquad \frac{\partial L}{\partial \dvec{r}_2}
= m_2\dvec{r}_2\\
\Rightarrow\; & \frac{\diffd}{\dt}\frac{\partial L}{\partial \dvec{r}_1}
- \frac{\partial L}{\partial \vec{r}_1}
= m_1\ddvec{r}_1 + \frac{\vec{r}_1-\vec{r}_2}{\abs{\vec{r}_1-\vec{r}_2}}
V'\left(\abs{\vec{r}_1-\vec{r}_2}\right)
= m_1\ddvec{r}_1 - \vec{F}_{21}.\\
& \frac{\diffd}{\dt}\frac{\partial L}{\partial \dvec{r}_2}
- \frac{\partial L}{\partial \vec{r}_2}
= m_2\ddvec{r}_2 + \frac{\vec{r}_2-\vec{r}_1}{\abs{\vec{r}_2-\vec{r}_1}}
V'\left(\abs{\vec{r}_2-\vec{r}_1}\right)
= m_2\ddvec{r}_2 - \vec{F}_{12}.\bsphere
\end{align*}
%\end{enumerate}
\end{bsp}

\subsection{Variationsrechnung}
Im Folgenden stellen wir den mathematischen Rahmen, der zum Verständnis des
Hamilton'schen Variationsprinzips wichtig ist.

\begin{figure}[!htbp]
  \centering
\begin{pspicture}(-0.5,-1.5)(3,1.2)

\psline{->}(-0.2,-1.2)(2.8,-1.2)
\psline{->}(0,-1.4)(0,1)


\psplot[linewidth=1.2pt,%
	     linecolor=darkblue,%
	     algebraic=true]%
	     {0.2}{2.8}{(x-0.8)^3-2*(x-0.8)^2}

\rput(2,-0.4){\color{gdarkgray}$f(x)$}
\end{pspicture} 
  \caption{Funktion $f(x)$ mit Extremstellen.}
\end{figure}
Um festzustellen, ob $x_0$ ein Extremum\footnote{Gemeint sind hier kritische
Punkte von $f$.} von $f:\R\to\R,\; x\mapsto f(x)$ ist,
betrachten wir dazu $x=x_0+\alpha$, wobei $\alpha$ eine \emph{Variation} von $x_0$ darstellt. Eine
hinreichende Bedingung für ein Extremum ist,
\begin{align*}
\frac{\partial}{\partial\alpha} f(x_0+\alpha)\big|_{\alpha=0} = f'(x_0) = 0.
\end{align*}

Betrachte nun eine Kurve $\gamma$ im $\R^n$,
\begin{align*}
\gamma: [t_1,t_2]\to\R^n,\; t\mapsto \vec{r}(t) =
\begin{pmatrix}
q^1(t)\\\vdots\\ q^n(t)
\end{pmatrix},
\end{align*}
mit festen Anfangs- und Endpunkten $\vec{r}_1$ und $\vec{r}_2$. Das Integral,
\begin{align*}
I(\gamma) := \int_{t_1}^{t_2} \dt L(q^i,\dot{q}^i,t)
\end{align*}
beschreibt ein Funktional, das jeder Raumkurve $\gamma$ eine Zahl zuordnet.
Wir sind an den Bedingungen interessiert, unter denen die Kurve $\gamma$ ein
Extremum von $I(\gamma)$ ist. Dazu betrachten wir eine kleine Variation von
$\gamma$,
\begin{align*}
\gamma'(\alpha): t\mapsto &\vec{r}(t) + \alpha\vec{\eta}(t)\equiv \vec{r}'(t),\\
&q^i(t) + \alpha\eta^i(t) \equiv \tilde{q}^i(t). 
\end{align*}
\begin{figure}[!htbp]
  \centering
\begin{pspicture}(0,-1.2411667)(3.014111,1.2470555)
\psline{->}(0.0,-1.0270555)(3.0,-1.0270555)
\psline{->}(0.2,-1.2270555)(0.2,1.1729444)
\psbezier[linestyle=dotted,dotsep=0.06cm,linecolor=darkblue](0.7546875,0.78705555)(1.4946876,1.2270555)(1.7546875,1.0670556)(1.6946875,0.64705557)(1.6346875,0.22705555)(1.5546875,0.06705555)(1.7146875,-0.23294444)
\psbezier{->}(0.4746875,0.48705557)(0.8546875,1.0870556)(1.3546875,0.90705556)(1.5146875,0.32705554)(1.6746875,-0.25294444)(1.8946875,-0.8729445)(2.7146876,-0.45294446)

\rput(2.2,0.89705557){\color{gdarkgray}$\vec{\eta}(t)$}

\rput(2.5960937,-0.8){\color{gdarkgray}$\vec{\gamma}(t)$}
\end{pspicture} 
  \caption{Variation von $\gamma$.}
\end{figure}

Fixierte Anfangs- und Endpunkte verlangen, dass
$\vec{\eta}(t_1)=\vec{\eta}(t_2)=0$. Dies führt uns auf eine neue Zahl,
\begin{align*}
I(\gamma'(\alpha)) = \int_{t_1}^{t_2} \dt L(\tilde{q}^i, \dot{\tilde{q}}^i, t).
\end{align*}

\begin{defnn}
Die Kurve $\gamma$ heißt \emph{extremal} zu dem Funktional $I(\gamma)$, falls
für alle Variationen $\vec{\eta}(t)$ gilt,
\begin{align*}
\frac{\partial}{\partial\alpha}I(\gamma'(\alpha))\bigg|_{\alpha=0} = 0.\fishhere
\end{align*}
\end{defnn}

Wir wollen nun untersuchen, für welche $\gamma$, die Wirkung maximal wird,
\begin{align*}
\frac{\partial}{\partial\alpha} I(\gamma'(\alpha))
&= \int_{t_1}^{t_2}\dt \frac{\partial}{\partial\alpha}
L(q^i + \alpha \eta^i, \dot{q}^i + \alpha\dot{\eta}^i,t)
\bigg|_{\alpha=0}\\
&= \int_{t_1}^{t_2}\dt \sum_i
\bigg[
\frac{\partial L}{\partial q^i} L(q^i + \alpha \eta^i, \dot{q}^i +
\alpha\dot{\eta}^i,t)\eta^i \\
 &\qquad + \frac{\partial}{\partial \dot{q}^i}L(q^i +
\alpha \eta^i, \dot{q}^i + \alpha\dot{\eta}^i,t)\dot{\eta}^i
\bigg]_{\alpha=0}\\
&= \int_{t_1}^{t_2}\dt \sum_i
\left[
\frac{\partial }{\partial q^i} L(q^i, \dot{q}^i, t)\eta^i +
\frac{\partial}{\partial \dot{q}^i}L(q^i, \dot{q}^i, t)\dot{\eta}^i
\right]\\
&\overset{\text{\tiny part.int.}}{=}
\int_{t_1}^{t_2}\dt \sum_i \left[
\frac{\partial}{\partial q^i} - \frac{\diffd}{\dt}\frac{\partial}{\partial
\dot{q}^i}\right]L\eta^i + \underbrace{\sum_i \frac{\partial}{\partial
\dot{q}^i}L\eta^i\bigg|_{t_1}^{t_2}}_{=0\;\text{\tiny nach Vor.}}\\
&= \int\limits_{t_1}^{t_2}
\sum_i \left[
\frac{\partial}{\partial q^i}L(q^i, \dot{q}^i, t) -
\frac{\diffd}{\dt}\frac{\partial}{\partial \dot{q}^i} 
L(q^i, \dot{q}^i, t)
\right]\eta^i \overset{!}{=} 0
\end{align*}
Der Ausdruck verschwindet genau dann für jede Variation $\eta$, wenn der
Integrand bereits verschwindet.
\begin{proof}
Angenommen das Integral verschwindet für einen Integrand, der an einem Punkt
$x$ $\neq 0$ ist, dann folgt Aufgrund der Stetigkeit, dass der Integrand in
einer Umgebung von $x$ nicht verschwindet. Wählt man nun $\eta$ so, dass es nur
in dieser Umgebung ungleich Null ist, ist das Integral nicht
null.\dipper\qedhere
\end{proof}

Wir haben somit folgenden Satz bewiesen.

\begin{propn}
Eine notwendige und hinreichende Bedingung, damit die Kurve $q^i(t)$ ein
extremal zum Funktional $I(\gamma)$ ist, ist das Erfüllen der
Euler-Lagrange-Gleichungen
\begin{align*}
\frac{\diffd}{\dt}\frac{\partial}{\partial \dot{q}^i}L(q^i,\dot{q}^i,t) -
\frac{\partial}{\partial q^i}L(q^i,\dot{q}^i,t) = 0.\fishhere
\end{align*}
\end{propn}

\begin{bemn}
Die Funktionalableitung hat die Form,
\begin{align*}
\frac{\delta I}{\delta q^i(t)} = \frac{\partial L}{\partial q^i} -
\frac{\diffd}{\dt}\frac{\partial L}{\partial \dot{q}^i}.\maphere
\end{align*}
\end{bemn}

\subsection{Hamilton's Prinzip der kleinsten Wirkung}

\begin{propn}[Hamilton's Prinzip der kleinsten Wirkung]
Die Bewegung eines Systems von der Zeit $t_1$ nach $t_2$ ist so, dass die
Wirkung
\begin{align*}
S:= \int_{t_1}^{t_2} \dt L
\end{align*}
mit dem Lagrange $L=T-V$ extremal wird.\fishhere
\end{propn}
\begin{proof}
Die Variationsrechnung liefert, dass ein Extremum der Wirkung die
Euler-Lagrange-Gleichungen erfüllt und diese sind äquivalent zu den Newtonschen
Bewegungsgleichungen.\qedhere
\end{proof}

\begin{bemn}
Die Dimension der Wirkung ist,
\begin{align*}
[\text{Wirkung}] = [\text{Energie}]\cdot[\text{Zeit}] =
[\text{Impuls}]\cdot[\text{Länge}].
\end{align*}
Somit entspricht die Einheit der Wirkung der von $h$ bzw $\hbar$.\maphere
\end{bemn}

Durch die Variationsrechnung haben wir die lokalen Bewegungsgleichungen durch
ein globales Konzept der extremalen Wirkung ersetzt. Dadurch ergeben sich sehr
markante Vorteile.
\begin{itemize}[label=\labelitem]
  \item Der Lagrange besteht aus den experimentell zugänglichen Größen $T$
  kinetische Energie und $V$ potentielle Energie.
  \item Das Variationsprinzip ist invariant unter Koordinatentransformation.
  Man erhält sehr einfach die Bewegungsgleichungen in krumlinigen Koordinaten. 
  \item Die Größen $q^i$ können beliebige verallgemeinerte Koordinaten sein 
 und beliebige Enheiten annehmen.
  \item Symmetrien und Erhaltungsgrößen sind im Lagrange Formalismus sehr
  elegant zur formulieren.
\end{itemize}

\begin{bsp}
Betrachte ein Teilchen in der Ebene im Zentralpotential $V(r)$.
\begin{figure}[!htbp]
  \centering
\begin{pspicture}(0,-1.2)(3.45,1.6)
\psline{->}(0.2453125,-0.87457985)(3.2453125,-0.87457985)
\psline{->}(0.4453125,-1.0745798)(0.4453125,1.3254201)
\psline{->}(0.44,-0.8804687)(1.5,0.19953126)
\psline[linecolor=yellow]{->}(1.56,0.25953126)(2.3,0.35953125)
\psarc(0.6,-0.9){0.6}{2.29061}{57.380756}
\psdots[linecolor=darkblue](1.56,0.25953126)

\rput(0.9,-0.65){\color{gdarkgray}$\ph$}
\rput(1.06,0.10953125){\color{gdarkgray}$\vec{r}$}
\rput(2.2867188,0.58953124){\color{gdarkgray}$\vec{v}$}
\rput(3.3054688,-1.0304687){\color{gdarkgray}$x$}
\rput(0.2878125,1.4495312){\color{gdarkgray}$y$}
\end{pspicture}
  \caption{Teilchen in der Ebene.}
\end{figure}

In karthesischen Koordinaten ist
\begin{align*}
&T = \frac{m}{2}(\dot{x}^2+\dot{y}^2)\\
&V=V(\sqrt{x^2+y^2})
\end{align*}
und die verallgeminerten Koordinaten haben die Form $q^1 = x, q^2 = y$.


In Polarkoordinaten ist $q^1 = r\cos \ph, q^2 = r\sin \ph$.
Die kinetische Energie hat hier die Form,
\begin{align*}
T&=\frac{m}{2}\left(\left(\dot{r}\cos\ph - r\sin\ph\right)^2
+ \left(\dot{r}\sin\ph + r\cos\ph\right)^2\right)\\
&= \frac{m}{2}\left(
\dot{r}^2 \cos^2\ph + r^2\sin^2\ph \dot{\ph} + \dot{r}^2\sin^2\ph
+ r^2\cos^2\ph\dot{\ph}^2 \right)\\
&= \frac{m}{2}\left(\dot{r}^2 + r^2\dot{\ph}^2\right).
\end{align*}
Der Lagrange ist also,
\begin{align*}
L(r,\ph,\dot{r},\dot{\ph}) =\frac{m}{2}\left(\dot{r}^2+r^2\dot{\ph}^2\right) -
V(r).
\end{align*}
Die Euler-Lagrange-Gleichung hat nun die Form,
\begin{align*}
&\frac{\diffd}{\dt}\frac{\partial L}{\partial \dot{r}} - \frac{\partial
L}{\partial r}
= m\ddot{r} - mr\dot{\ph}^2 + V'(r) = 0.\\
&\frac{\diffd}{\dt}\frac{\partial L}{\partial \dot{\ph}} - \frac{\partial
L}{\partial \ph}
= \frac{\diffd}{\dt}mr^2\dot{\ph}  = 0 \Rightarrow mr^2\dot{\ph} = \const.
\end{align*}
Wir erhalten so sofort die Drehimpulserhaltung,
\begin{align*}
L_z = m(\dot{x}y-x\dot{y}) = mr^2\dot{\ph} = \const.
\end{align*}
\end{bsp}

\begin{bemn}[Bemerkungen zu den Größen.]
\begin{itemize}[label=\labelitem]
  \item $q^i$: Verallgemeinerte Koordinaten.
  \item $p_i = \frac{\partial L}{\partial \dot{q}^i}$: Verallgemeinerte
  Impulse.\footnote{Auch generalisierter, kanonischer, kanonisch
  konjugierter oder konjugierter Impuls.}
  \item $\frac{\partial L}{\partial q^i}$: Verallgemeinerte Kräfte.
\end{itemize}
Die verallgemeinerten Größen können unter Koordinatentransformation ihre
Dimension ändern, die $q^i$ beispielsweise müssen also in beliebigen
krummlinigen Koordinaten nicht mehr die Einheit einer Länge haben.\maphere
\end{bemn}

\begin{defnn}
Eine Koordinate $q^i$ heißt \emph{zyklisch}, wenn $L$ nicht von $q^i$ abhängt,
\begin{align*}
\frac{\partial L}{\partial q^i} = 0.\fishhere
\end{align*}
\end{defnn}

\begin{propn}
Für zyklische Koordinaten ist der dazugehörige kanonische Impuls
 eine Erhaltungsgröße.
\begin{align*}
\frac{\partial L}{\partial q^i} = 0 \Rightarrow \frac{\diffd p_i}{\dt} =
0 \Rightarrow p_i=\frac{\partial L}{\partial \dot{q}^i} = \const.\fishhere
\end{align*}
\end{propn}

\begin{bsp}
Betrachte erneut ein Teilchen in der Ebene im Zentralpotential. Der Lagrange
hat in Polarkoordinaten die Form
\begin{align*}
L(r,\ph,\dot{r},\dot{\ph}) = \frac{m}{2}\left(\dot{r}^2 + r^2\dot{\ph}^2
\right)-V(r).
\end{align*}
Offensichtlich ist $\ph$ zyklisch, wir erhalten so die Drehimpulserhaltung
\begin{align*}
L = p_\ph = \frac{\partial L}{\partial\dot{\ph}} = mr^2\dot{\ph} =
\const.\bsphere
\end{align*}
\end{bsp}

\begin{bemn}
Unterscheiden sich zwei Lagrange Funktionen $L$ und $L'$ nur durch eine totale
Ableitung,
\begin{align*}
L' = L + \frac{\diffd}{\dt}F,
\end{align*}
so führt dies auf dieselben Bewegungsgleichungen.\maphere
\end{bemn}
\begin{proof}
Dies ergibt sich direkt, wenn wir die Wirkung betrachten,
\begin{align*}
S'(\gamma) = 
\int\limits_{t_1}^{t_2} \dt L' =
\int\limits_{t_1}^{t_2} \dt \left(L + \frac{\diffd}{\dt} F\right) =
\left(\int\limits_{t_1}^{t_2} \dt L\right) + F\big|_{t_1}^{t_2}. 
\end{align*}
Nun hängt $F$ nicht von $\gamma$ ab und daher ist ein zu $L$ extremales
$\gamma$ auch extremal zu $L'$.\qedhere
\end{proof}

\begin{bsp}
Potentiale lassen sich nicht eindeutig definieren, denn zwei Potentiale, die
sich durch eine Konstante unterscheiden
\begin{align*}
V(q^i),\qquad V'(q^i) = V(q^i) + V_0,
\end{align*}
führen zu zwei Lagrange Gleichungen, die sich lediglich durch eine totale
Ableitung unterscheiden,
\begin{align*}
L' - L = V_0 = \frac{\diffd}{\dt}(t\cdot V_0).\bsphere
\end{align*}
\end{bsp}

\subsection{Lorentz-Kraft}
Aus Experimenten wissen wir, dass ein geladenes Teilchen im elektrischen Feld
$\vec{E}$ und magnetischem Feld $\vec{B}$ die Lorentz-Kraft erfährt,
\begin{align*}
\vec{F} = q\cdot\vec{E} + \frac{q}{c}\vec{v}\times\vec{B}.
\end{align*}
Nun lassen sich diese Felder über Potentiale definieren,
\begin{align*}
&\vec{E} = -\nabla \phi - \frac{1}{c}\frac{\partial}{\partial
t}\vec{A},\\& \vec{B} = \rot\vec{A},
\end{align*} 
mit einem Skalarpotential $\phi$ und einem Vektorpotential $\vec{A}$. $\vec{A}$
und $\phi$ sind nicht eindeutig und insbesondere nicht experimentell zugänglich;
lediglich das vom Potential erzeugte Feld lässt sich messen.

Lagrange Funktion und Euler-Lagrange-Gleichung haben die Form,
\begin{align*}
&L(\vec{r},\dvec{r},t) = \frac{m}{2}\dvec{r}^2 - q\left[\phi(\vec{r},t) -
\frac{\dvec{r}}{c}\vec{A}(\vec{r},t) \right],\\
&m\ddvec{r} = \vec{F}_L = q\vec{E} + \frac{q}{c}\vec{v}\times\vec{B}.
\end{align*}
Der kanonische Impuls hat nun die Form,
\begin{align*}
\vec{p} = \frac{\partial L}{\partial \dvec{r}} = m\dvec{r} +
\frac{q}{c}\vec{A}(\vec{r},t).
\end{align*}
\begin{bemn}[Übungsaufgabe:]
Zeige, dass die Euler-Lagrange-Gleichungen die Lorentzkraft ergeben.
\end{bemn}

Da die Potentiale $\phi$ und $\vec{A}$ nicht eindeutig sind, lassen sich
Transformationen definieren unter denen $\vec{E}$- und $\vec{B}$-Feld
invariant sind:
\begin{align*}
&\vec{A}\mapsto \vec{A}' = \vec{A} + \nabla \lambda(\vec{r},t),\\
&\phi\mapsto \phi' = \phi - \frac{1}{c}\partial_t \lambda(\vec{r},t).
\end{align*}
Diese Transformationen werden \emph{Eichtransformationen} genannt. Wendet man
eine Eichtransformation an, unterscheidet sich der Lagrange wieder nur um eine
totale Ableitung,
\begin{align*}
L' &= \frac{m}{2}\dvec{r}^2 - q\left[\phi - \frac{1}{c}\partial_t \lambda -
\frac{\dvec{r}}{c}\left(\vec{A} + \nabla\lambda\right)\right]
= L + \frac{q}{c}\left[\partial_t \lambda + \dvec{r}\nabla\lambda \right] \\ 
&=L+ \frac{q}{c}\frac{\diffd}{\dt}\lambda.
\end{align*}

\subsection{Noether Theorem}

Wir kennen bereits eine ganz spezielle Symmetrie, die zyklischen Koordinaten,
und haben gesehen, dass diese uns stets eine Erhaltungsgröße liefert. Wir
wollen nun einen allgemeinen Zusammenhangen zwischen Symmetrien und
Erhaltungssätzen aufstellen.

Betrachte eine kontinuierliche Schar von Koordinatentransformationen,
\begin{align*}
h_s^i:\; &q^i \mapsto \tilde{q}^i = h_s^i(q^j),\\
&\dot{q}^i \mapsto \dot{\tilde{q}}^i = \sum_j \frac{\partial h_s^i}{\partial
q^j}\dot{q}^j\\
h_0^i:\; &q^i \mapsto q^i.
\end{align*}

\begin{bsp}
Translationen haben so die Form,
\begin{align*}
&q^i \mapsto \tilde{q}^i + s\cdot a^i,\\
&\dot{q}^i \mapsto \dot{\tilde{q}}^i  = \dot{q}^i.\bsphere
\end{align*}
\end{bsp}
\begin{bsp}
Eine Rotation um die $z$-Achse ist gegeben durch,
\begin{align*}
&\vec{r} \mapsto \vec{r}' = R(s)\vec{r} =
\begin{pmatrix}
\cos s & \sin s & 0\\
-\sin s & \cos s & 0\\
0 & 0 & 1
\end{pmatrix}
\begin{pmatrix}
q^1 \\ q^2 \\ q^3 
\end{pmatrix},\\
&\dvec{r}\mapsto \dvec{r}' = R(s)\dvec{r}.
\end{align*}
Für $s=0$ ist $R=\Id$.\bsphere
\end{bsp}

Die Schar von Koordinatentransformationen ist eine \emph{Symmetrie des Systems},
wenn sie den Lagrange nicht verändert, d.h. wenn
\begin{align*}
&L(\tilde{q}^i,\dot{\tilde{q}}^i,t) = L(q^i,\dot{q}^i,t), \forall s,\\
\Leftrightarrow\; & \frac{\diffd}{\ds}L(\tilde{q}^i,\dot{\tilde{q}}^i,t) = 0.
\end{align*}

\begin{propn}[Noether Theorem]
Für eine Symmetrie $h_s^i$ gibt es eine Erhaltungsgröße der Form,
\begin{align*}
I(q^i,\dot{q}^i) = \sum_i \frac{\partial L}{\partial \dot{q}^i}
\frac{\diffd h_s^i(q^i)}{\ds} \bigg|_{s=0}.\fishhere
\end{align*}
\end{propn}

\begin{bsp}
$L(r,\dot{r},\ph,\dot{\ph})$ und $\ph$ ist zyklisch.
\begin{align*}
\left.
\begin{aligned}
&h_s^i: \ph\mapsto \ph+s,\\
&\frac{\diffd h_s}{\diffd s} = 1
\end{aligned}
\right\}\Rightarrow \frac{\partial L}{\partial \dot{\ph}} =
I(q^i,\dot{q}^i).\bsphere
\end{align*}
\end{bsp}

\begin{proof}[Beweis des Noether Theorems.]
Sei $q^i(t)$ eine Lösung der Bewegungsgleichung, dann ist $\tilde{q}^i(t) =
h_s^i(q^i(t))$ ebenfalls eine Lösung der Bewegungsgleichung.
\begin{align*}
0 &= \frac{\diffd}{\ds}L(\tilde{q}^i,\dot{\tilde{q}}^i,t)
= \sum_i \left[\frac{\partial L}{\partial q^i}\frac{\diffd}{\ds} \tilde{q}^i +
\frac{\partial L}{\partial \dot{q}^i}\frac{\diffd}{\ds}\dot{\tilde{q}}^i
\right]\\ &= \sum_i \left[\left(\frac{\diffd}{\dt}\frac{\partial L}{\partial
\dot{q}^i}\right)\frac{\diffd}{\ds}\tilde{q}^i + \frac{\partial
L}{\partial \dot{q}^i}\frac{\diffd}{\dt}\left(\frac{\diffd}{\ds}\tilde{q}^i
\right) \right]\\
&= \sum_i \frac{\diffd}{\dt} \left[\frac{\partial L}{\partial
\dot{q}^i}\frac{\diffd}{\ds}\tilde{q}^i \right].
\end{align*}
Also ist $\sum_i\dfrac{\partial L}{\partial
\dot{q}^i}\dfrac{\diffd}{\ds}\tilde{q}^i$ eine Erhaltungsgröße für alle $s$, also auch für $s=0$ und damit gilt,
\begin{align*}
\left.\frac{\diffd}{\dt} \sum_i\left[\frac{\partial L}{\partial
\dot{q}^i}\frac{\diffd}{\ds}\tilde{q}^i \right]\right|_{s=0} = 0.\qedhere
\end{align*}
\end{proof}

\begin{bsp}
%\begin{enumerate}[label=\arabic{*}.)]
\textit{Translationssymmetrie in einem System von $N$ Teilchen.}
\begin{align*}
&\vec{r}_i \mapsto \vec{r}_i + s\vec{a},\\
&\dvec{r}_i \mapsto \dvec{r}_i',
\end{align*}
wobei $\vec{a}$ ein beliebiger aber von $i$ unabhängiger Translationsvektor
ist, d.h. das ganze Teilchensystem wird um $s\vec{a}$ verschoben. 
\begin{align*}
&\frac{\diffd}{\ds} h_s^i = \frac{\diffd}{\ds} \left(\vec{r}_i + s\vec{a}\right)
= \vec{a}.\\
&\frac{\partial L}{\partial \dvec{r}_i} = \vec{p}_i. 
\end{align*}
Wir erhalten somit die Erhaltungsgröße,
\begin{align*}
I(\vec{r}_i,\dvec{r}_i) = \sum_i \left(\vec{p}_i\cdot \vec{a}\right) =
\left(\sum_i \vec{p}_i\right)\cdot\vec{a} = \vec{p}_\tot\cdot\vec{a}.
\end{align*}
Liegt eine Translationssymmetrie entlang allen Koordinatenachsen vor, so	 gilt
\begin{align*}
\vec{p}_\tot = \sum_i \vec{p}_i = \const.
\end{align*}
Im Schwerefeld mit Gravitation in $z$-Richtung und Translationssymmetrie in $x$-
und $y$-Richtung wären nur $\vec{p}_x$ und $\vec{p}_y$ erhalten.\bsphere
\end{bsp}
\begin{bsp}
\textit{Rotationssymmetrie.} Für kleine Winkel können wir die Rotation um
die Achse $\vec{n}$ schreiben als,
\begin{align*}
\vec{r}_i \mapsto \vec{r}_i + s\vec{n}\times\vec{r}_i,\\
\dvec{r}_i \mapsto \dvec{r}_i + s\vec{n}\times\dvec{r}_i.
\end{align*}
\begin{align*}
&\frac{\diffd}{\ds} h_s^i = \frac{\diffd}{\ds}\left(\vec{r}_i +
s\vec{n}\times\vec{r}_i\right) = \vec{n}\times\vec{r}_i,\\
& I(\vec{r}_i,\dvec{r}_i) = \sum_i \vec{p}_i \cdot \vec{n}\times \vec{r}_i =
\sum_i \vec{n}\cdot \vec{r}_i\times\vec{p}_i = \vec{n}\sum_i \vec{L}_i =
\vec{n}\cdot \vec{L}_\tot.
\end{align*}
Bei Rotationssymmetrie entlang aller Koordinatenachsen ist der
Gesamtdrehimpuls $\vec{L}_\tot$ erhalten.
\end{bsp}
\begin{bemn}[Übungsaufgabe:]
Welcher Teil des Drehimpulses ist nicht erhalten, wenn die
Rotationssymmetrie nur um die $z$-Achse gegeben ist?
\end{bemn}
\begin{bsp}
\textit{Galilei Boost.}
\begin{align*}
&\vec{r}_i \mapsto \vec{r}_i + s\vec{v}t\\
&\dvec{r}_i \mapsto \dvec{r}_i + s\vec{v}
\end{align*}
Die Lagrangefunktion ist nicht invariant unter dieser Operation, sondern ändert
sich um eine totale Ableitung,
\begin{align*}
L\mapsto L'=L + \sum_i\left[\frac{m_i}{2}s^2\vec{v}^2 + m_i\dvec{r} \vec{v}s
\right]= L + \frac{\diffd}{\dt}
\left(\underbrace{\sum_i \frac{m_i}{2}s^2\vec{v}^2t + m_i
s\vec{r}\vec{v}}_{F(q^i,\dot{q}^i)}\right).
\end{align*}
Jedoch gilt $\dfrac{\diffd}{\ds}L = 0$ und damit auch
\begin{align*}
0 = \dfrac{\diffd}{\ds}\left[L' - \frac{\diffd}{\dt}F\right]
= \dfrac{\diffd}{\dt}\left[\sum_i\frac{\partial L}{\partial
\dot{q}^i}\frac{\diffd h_s^i}{\ds} - \frac{\diffd}{\ds}F\right].
\end{align*}
Die Erhaltunsgröße hat somit die Form
\begin{align*}
I(q^i,\dot{q}^i) = \sum_i \frac{\partial L}{\partial q^i}\frac{\diffd
h_s^i(q^i)}{\ds} - \frac{\diffd}{\ds} F(q^i,\dot{q}^i)\bigg|_{s=0}.
\end{align*}
Damit folgt der Schwerpunktssatz
\begin{align*}
\sum_i \vec{p}_i\vec{v}t - \sum_i m_i\vec{r}_i\vec{v} =
\underbrace{\left(\sum_i \vec{p}_i t - \sum_i
m\vec{r}_i\right)}_{\text{Erhaltungsgröße}}\vec{v} = \const.\bsphere
\end{align*}
%\end{enumerate}
\end{bsp}

\subsection{Energieerhaltung}
Unser System ist homogen in der Zeit, d.h. kein Zeitpunkt ist ausgezeichnet.
Betrachten wir also eine Translation der Zeit
\begin{align*}
&t\mapsto t' = t+a,
\end{align*}
so bleibt die Lagrangefunktion unverändert und ist daher nicht explizit von
der Zeit abhängig,
\begin{align*}
&L = L' \quad \frac{\partial L}{\partial t} = 0.
\end{align*}
Sie ist natürlich implizit von der Zeit abhängig, da $q^i$ und $\dot{q}^i$ in
$L(q^i(t),\dot{q}^i(t))$ von der Zeit abhängen. Die totale Ableitung
$\frac{\diffd}{\dt} L$ muss daher auch nicht verschwinden, sondern es gilt
\begin{align*}
\frac{\diffd}{\dt} L(q^{i},\dot{q}^i) &= \sum_i \left[\frac{\partial L}{\partial
q^{i}}\frac{\diffd}{\dt}q^{i} + \frac{\partial L}{\partial
\dot{q}^{i}}\frac{\diffd}{\dt}\dot{q}^{i} \right] + \frac{\partial L}{\partial
t}\\
&= \sum_i \left[\left(\frac{\diffd}{\dt}\frac{\partial
L}{\partial \dot{q}^i}\right)\frac{\diffd}{\dt}q^{i} + \frac{\partial
L}{\partial \dot{q}^{i}}\frac{\diffd}{\dt}\dot{q}^{i} \right]\\
&= \frac{\diffd}{\dt}\sum_i \frac{\partial L}{\partial\dot{q}^i}\dot{q}^i,\\
\Rightarrow\; & \frac{\diffd}{\dt}\underbrace{\left[\sum_i \frac{\partial
L}{\partial \dot{q}^i} \dot{q}^i - L\right]}_{:=H}=0.
\end{align*}
Die Energie,
\begin{align*}
H = \sum_i \vec{p}_i \dot{q}^i - L(q^i,\dot{q}^i),\qquad \vec{p}_i :=
\frac{\partial L}{\partial \dot{q}^i}
\end{align*}
ist eine Erhaltungsgröße, die aus der Translationsinvarianz der Zeit folgt.

\begin{bemn}
Für geschwindigkeitsunabhängige Kräfte im Potential $V$ gilt,
\begin{align*}
&\frac{\partial L}{\partial \dot{q}^i} = \frac{\partial T}{\partial
\dot{q}^i},\\
&H = \sum_i \frac{\partial L}{\partial \dot{q}^i}\dot{q}^i - (T-V)  =
T+V.\maphere
\end{align*}
\end{bemn}

Wir haben somit für jede der 10 Symmetrien der Galilei-Gruppe eine
Erhaltungsgröße gefunden.


\newpage
\section{Zweikörper Zentralkraft Problem}
%TODO: Bild
Als Anwendung des Lagrange Formalismus untersuchen wir zwei Körper mit Massen
$m_1$ und $m_2$ in einem rotationssymmetrischen Wechselwirkungspotential $V$.

Das Zweikörperproblem ist exakt lösbar und von besonderem Interesse, da es die
Bewegung der Erde um die Sonne mit hinreichender Genauigkeit beschreibt.
Später werden wir so auch die \emph{Keplerschen Gesetze} herleiten, zunächst
wollen wir uns jedoch allgemein mit dem Problem befassen.

Zur Lösung des Zweikörperproblems werden wir die entwickelten Erhaltungsgrößen
verwenden, durch die die exakte Lösung sehr einfach zu bestimmen ist. Die
Lagrangefunktion hat hier die Form,
\begin{align*}
L(\vec{r}_1,\vec{r}_2,\dvec{r}_1,\dvec{r}_2) = \frac{m_1}{2}\dvec{r}_1^2 +
\frac{m_2}{2}\dvec{r}_2^2 - V(\abs{\vec{r}_1-\vec{r}_2}).
\end{align*}
Da unsere Vektoren 3-dimensional sind, erhalten wir einen 6-dimensionalen
Konfigurationsraum, d.h. ein System von 6 gekoppelte Differentialgleichungen.
Mit Hilfe der Symmetrien können wir jedoch die Freiheitsgrade so reduzieren,
dass das Problem exakt lösbar wird.

\subsection{Reduktion auf 1-Teilchen Problem}
Dank der Impulserhaltung und dem Schwerpunktsatz wissen wir, dass der
Schwerpunkt einer trivialen Bewegung folgt. Dies können wir durch das
Einführen von Schwerpunkts- $\vec{R}$ und Relativkoordinaten $\vec{r}$
ausnutzen,
\begin{align*}
&\vec{R} = \frac{m_1\vec{r}_1+m_2\vec{r}_2}{m_1+m_2},\\
&\vec{r} = \vec{r}_2-\vec{r}_1.
\end{align*}
Die bisherigen Koordinaten haben dann die Form,
\begin{align*}
&\vec{r}_1 = \vec{R} - \frac{m_2}{m_1+m_2}\vec{r},\\
&\dvec{r}_1 = \dvec{R} - \frac{m_2}{m_1+m_2}\dvec{r},\\
&\vec{r}_2 = \vec{R} + \frac{m_1}{m_1+m_2}\vec{r},\\
&\dvec{r}_2 = \dvec{R} + \frac{m_1}{m_1+m_2}\dvec{r}.
\end{align*}
Dadurch erhalten wir für die kinetische Energie den Ausdruck,
\begin{align*}
T &= \frac{m_1}{2}\left(\dvec{R} - \frac{m_2}{m_1+m_2}\dvec{r}\right)^2
+ \frac{m_2}{2}\left(\dvec{R} + \frac{m_1}{m_1+m_2}\dvec{r}\right)^2\\
&= \frac{1}{2}\left[(m_1+m_2)\dvec{R}^2 + \frac{m_1m_2^2}{(m_1+m_2)^2}\dvec{r}^2
+ \frac{m_2m_1^2}{(m_1+m_2)^2}\dvec{r}^2\right]\\
&= \frac{1}{2}\left[(m_1+m_2)\dvec{R}^2 +
\frac{m_1m_2}{m_1+m_2}\dvec{r}^2\right]\\
&= \frac{1}{2}\left[(m_1+m_2)\dvec{R}^2 +
\mu\dvec{r}^2\right],
\end{align*}
wobei wir $\mu=\dfrac{m_1m_2}{m_1+m_2}$ als \emph{reduzierte Masse} einführen.
Die Lagrangefunktion vereinfacht sich somit zu,
\begin{align*}
L(\vec{R},\vec{r},\dvec{R},\dvec{r}) = \frac{1}{2}\left[(m_1+m_2)\dvec{R}^2 +
\mu\dvec{r}^2\right] - V(\abs{\vec{r}}).
\end{align*}
Offensichtlich ist $\vec{R}$ eine zyklische Koordinate und es gilt,
\begin{align*}
\vec{p} = \frac{\partial L}{\partial \vec{R}} = (m_1+m_2)\dvec{R} =\const =
(m_1+m_2)\vec{v}.
\end{align*}
Daraus ergibt sich sofort der Schwerpunktssatz,
\begin{align*}
\vec{R}(t) = \vec{R}_0 + \vec{v}t.
\end{align*}
Es genügt daher nur noch die Relativkoordinate zu betrachten. Wir erhalten ein
effektives Einteilchenproblem mit Koordinate $\vec{r}$ und effektiver Masse
$\mu$. Das Problem hat sich um 3 Freiheitsgrade reduziert.

Im Folgenden betrachten wir somit den Lagrange 
\begin{align*}
L(\vec{r},\dvec{r}) = \frac{1}{2}\mu\dvec{r}^2 - V(\abs{\vec{r}}).
\end{align*}
\begin{bemn}[Bemerkungen.]
Betrachten wir den Grenzafll $m_2>>m_1$, dann gilt
\begin{align*}
&\mu = \frac{m_1m_2}{m_1+m_2}\approx \frac{m_1m_2}{m_2} = m_1,\\
&\vec{R} = \frac{m_1\vec{r}_1+m_2\vec{r}_2}{m_1+m_2} \approx 
\frac{m_1}{m_2}\vec{r}_1 + \vec{r}_2\approx \vec{r}_2.
\end{align*}
Dies entspricht dem System Erde-Sonne, bei dem der Schwerpunkt in der Sonne
liegt und die Erde um die Sonne kreist.

Ist hingegen $m_1=m_2=m$, so gilt
\begin{align*}
&\mu = \frac{m_1m_2}{m_1+m_2} = \frac{m^2}{2m} = \frac{1}{2}m,\\
&\vec{R} = \frac{m_1\vec{r}_1+m_2\vec{r}_2}{m_1+m_2} =
\frac{1}{2}\left(\vec{r}_1+\vec{r}_2\right).
\end{align*}
Der Schwerpunkt liegt im Mittelpunkt der Verbindungsstrecke der Massen und die
Massen kreisen darum.\maphere
\end{bemn}

\subsection{Integration der Bewegungsgleichung}
Wir wollen die Bewegungsgleichung nun exakt lösen. Dazu müssen wir das 
Potential $V$, das als rotationssymmetrisch vorausgesetzt ist, etwas genauer
studieren. Betrachten eine Rotation im Raum,
\begin{align*}
&\vec{r}\mapsto \vec{M}\vec{r},\qquad \vec{M}\in\mathrm{SO}(3),\\
&\abs{\vec{r}} = \abs{\vec{M}\vec{r}} =
\sqrt{\lin{\vec{M}\vec{r},\vec{M}\vec{r}}} =
\sqrt{\lin{\vec{M}\vec{M}\vec{r},\vec{r}}}
= \sqrt{\lin{\vec{r},\vec{r}}} = \abs{r}.
\end{align*}
Die Länge des Vektors $\vec{r}$ ist also invariant unter Rotation. Aufgrund der
Rotationssymmetrie des Potentials ist der Drehimpuls
\begin{align*}
\vec{L} = \vec{r}\times\vec{p},
\end{align*}
eine Erhaltungsgröße mit $\abs{\vec{L}}\equiv l = \const$. Es gilt daher
\begin{align*}
\lin{\vec{r},\vec{L}} = \lin{\vec{r},\vec{r}\times\vec{p}} =
\lin{\vec{p},\vec{r}\times\vec{r}} = 0.
\end{align*}
Aufgrund der Konstanz von $\vec{L}$ steht somit $\vec{r}$ stets senkrecht auf
$\vec{L}$ und daher verläuft die Kurve $t\mapsto \vec{r}(t)$ in einer Ebene.
\begin{bemn}
Falls $l=0$ bricht die obige Argumentation zusammen. In diesem Fall ist die
Bewegung jedoch geradlinig und findet daher ebenfalls in einer Ebene
statt.\maphere
\end{bemn}

\begin{figure}[!htbp]
  \centering
\begin{pspicture}(0,-1.07)(2.4,1.07)
\psline(0.0,-0.37)(1.6,1.05)
\psline(0.0,-0.37)(0.8,-1.05)
\psline(0.8,-1.05)(2.38,0.37)
\psline(1.6,1.05)(2.38,0.37)

%\psline{->}(1.06,0.01)(1.7,0.21)
\psarc(1.08,0.07){0.2}{5}{153.43495}
\psdots[dotsize=0.02](1.1,0.13)
\psline{->}(1.06,0.01)(0.34,0.69)

\psbezier[linecolor=darkblue]{->}(1.0658889,0.02411111)(1.4258889,0.2641111)(1.7058889,-0.2758889)(1.9658889,0.4441111)
\psarc[linewidth=0.02]{->}(0.4558889,0.5741111){0.17}{201.0375}{127.874985}

\psdots[dotsize=0.1](1.06,0.01)

\rput(0.2025,0.9){\color{gdarkgray}$\vec{L}$}
\rput(1.72,0.44){\color{gdarkgray}$\vec{r}$}
\end{pspicture}
  \caption{Wahl der Bewegungsebene.}
\end{figure}

Nun kann diese Ebene beliebig im Raum liegen, wir können aber ein
Koordinatensystem so wählen, dass die $z$-Achse entlang der Drehimpulsachse
verläuft. Die Bewegung findet dann in der $x,y$-Ebene statt. Es
bietet sich nun die Wahl von Polarkoordianten an, wodurch der Lagrange die Form
annimmt,
\begin{align*}
L(r,\ph,\dot{r},\dot{\ph}) = \frac{1}{2}\mu\left(\dot{r}^2
+r^2\dot{\ph}^2\right) - V(r).
\end{align*}
Die Koordinate $\ph$ ist zyklisch und die Erhaltungsgröße
\begin{align*}
p_\ph = \frac{\partial L}{\partial \dot{\ph}} = \mu r^2\dot{\ph} \equiv l.
\end{align*}
Die Drehimpulserhaltung eliminiert somit 2 Freiheitsgrade. Die Richtung des
Drehimpulses schränkt die Bewegung auf eine Ebene ein ($r,\ph$), während der
Betrag des Drehimpulses $\ph$ eliminiert. Es bleibt ein eindimensionales
Problem zu lösen.
\begin{bemn}
Das 2. Keplersche Gesetz ist äquivalent zur Drehimpulserhaltung,
%TODO: Skizze Keppler,
\begin{align*}
&\dA = \frac{1}{2}r r\dph = \frac{1}{2}r^2\dph\\
&\frac{\dA}{\dt} = \frac{1}{2}r^2\dot{\ph} =\frac{l}{2\mu} = \const.\maphere
\end{align*}
\end{bemn}
Die Bewegungsgleichung für die Koordinate $r$ reduziert sich zu
\begin{align*}
&\frac{\diffd}{\dt}\frac{\partial L}{\partial \dot{r}} = \frac{\partial
L}{\partial r}\\
\Leftrightarrow & \mu\ddot{r} = \mu r \dot{\ph}^2 - V'(r).
\end{align*}
Mit $\dot{\ph} = \frac{l}{\mu r^2}$ können wir die $\ph$ Abhängigkeit
eliminieren und erhalten somit,
\begin{align*}
\mu\ddot{r} = \mu r \frac{l^2}{\mu^2r^4} - V'(r) = \frac{l^2}{\mu r^3} - V'(r)
= - \frac{\diffd}{\dr}\left[\frac{1}{2\mu}\frac{l^2}{r^2} +
V(r)\right] = -U'(r),
\end{align*}
wobei wir $U(r) = \frac{1}{2\mu}\frac{l^2}{r^2} + V(r)$ als \emph{effektives
Potential} bezeichnen.

Die Gleichung verhält sich nun, wie ein eindimensionales Teilchen im Potential
$U(r)$ mit der Energie,
\begin{align*}
E = \frac{\mu}{2}\left(\dot{r}^2 + \dot{\ph}^2r^2 \right) + V(r)
= \frac{\mu}{2}\dot{r}^2 + \underbrace{\frac{1}{2\mu}\frac{l^2}{r^2}  +
V(r)}_{=U(r)}.
\end{align*}

\begin{figure}[!htbp]
  \centering
\begin{pspicture}(0,-2.4)(4,2.7)
\psline{->}(0.37,-1.6)(0.37,1.8)
\psline{->}(0.17645577,0.6)(3.8,0.6)
\psbezier[linecolor=yellow](1.0564557,-1.590038)(1.0564557,-1.1100379)(1.8564558,0.48996207)(3.6764557,0.5099621)
\psbezier[linecolor=darkblue](0.79645574,1.4699621)(0.79645574,0.66996205)(0.83645576,-0.5500379)(1.2764558,-0.5500379)(1.7164558,-0.5500379)(2.0564559,0.46996206)(3.6964557,0.5099621)
\psdots(1.2764558,-0.47003794)


\rput{90}(0.2,1.7){\small\color{gdarkgray}Streuzustände}
\rput(0.2,-0.9){\rput{-90}{\small\color{gdarkgray}Gebundene Zustände}}

\rput(1.6,1.2){\color{gdarkgray}$U(r)=\frac{1}{r^2}$}
\rput[l](1.4,-1.4){\color{gdarkgray} $V(r)=\frac{1}{r^\alpha},\;\alpha <2$}

\end{pspicture}

  \caption{Teilchenpotential und effektives Potential mit Gleichgewichtslage.}
\end{figure}

Eine formale Lösung erhalten wir aus der Energieerhaltung,
\begin{align*}
\dot{r}^2 = \frac{2}{\mu}\left(E - U(r) \right),
\end{align*}
wobei wir lediglich den auslaufenden Teil ($\dot{r}>0$) betrachten.
\begin{align*}
&\dot{r} = \sqrt{\frac{2}{\mu}}\sqrt{E-U(r)},\\
\Rightarrow & \int\dt \frac{\dot{r}}{\sqrt{\frac{2}{\mu}(E-U(r))}} = \int\dr
\sqrt{\frac{\mu}{2}}\frac{1}{\sqrt{E-U(r)}} = \int dt = t-t_0.
\end{align*}
Die formale Lösung ist somit,
\begin{align*}
&t-t_0 = \underbrace{\int\limits_{r_0}^r \dr
\sqrt{\frac{\mu}{2}}\frac{1}{\sqrt{E-U(r)}}}_{:=F(r)}\\
\Rightarrow & r(t) = F^{-1}(t-t_0).
\end{align*}
Das Integral lässt sich jedoch nur in wenigen Fällen analytisch lösen. Wir sind
meist aber nicht an der Bewegungsgleichung selbst, sondern nur an der Form
der Bahnkurve interessiert und erhalten für die Bahntrajektorien
$\ph(r), r(\ph)$,
\begin{align*}
\frac{\dph}{\dr} = \frac{\dph}{\dt}\frac{\dt}{\dr} = \frac{l}{\mu
r^2}\frac{1}{\dot{r}} = \frac{l}{\mu
r^2}\sqrt{\frac{\mu}{2}}\frac{1}{\sqrt{E-U(r)}}
\end{align*}
\begin{align*}
\ph-\ph_0 =  \int\dr \frac{l}{\mu
r^2}\sqrt{\frac{\mu}{2}}\frac{1}{\sqrt{E-U(r)}} = \frac{l}{\sqrt{2\mu}}\int\dr
\frac{1}{r^2}\frac{1}{\sqrt{E-U(r)}}.
\end{align*}

Qualitativ lässt sich durch die Analyse des effektiven Potentials bereits viel
über die Bahn aussagen.
\begin{itemize}[label=\labelitem]
  \item $V(r)$ dominiert über $\frac{1}{r^2}$ für $r\to\infty$,
  \item $\frac{1}{r^2}$ dominiert über $V(r)$ für $r\to 0$.
\end{itemize}

 Für $E = E_\text{min}$ erhalten wir eine Kreisbahn, während sich für $E<0$ i.A.
 Rosettenbahnen mit Preiheldrehung ergeben.

\begin{figure}[H]
  \centering
\begin{pspicture}(-0.2,-3)(5.2,3.4)

\psline{->}(1.24,0.44)(1.26,3.12)
\psline{->}(1.06,2.42)(4.86,2.42)
\psbezier[linewidth=0.04](1.68,3.28)(1.68,2.48)(1.72,1.26)(2.16,1.26)(2.6,1.26)(2.94,2.28)(4.58,2.32)

\psline[linestyle=dashed,dash=0.06cm 0.04cm,linewidth=0.02cm]%
	   (1.06,1.96)(4.86,1.96)
\psline[linestyle=dashed,dash=0.06cm 0.04cm,linewidth=0.02cm]%
	   (1.06,1.26)(4.86,1.26)

\psline[linestyle=dotted,dotsep=0.06cm](2.18,2.42)(2.16,-0.32)
\psline[linestyle=dotted,dotsep=0.06cm](2.76,2.42)(2.74,-1.51)
\psline[linestyle=dotted,dotsep=0.06cm](1.78,2.42)(1.78,-1.52)

\pscircle[linecolor=yellow](1.27,-1.51){0.49}
\pscircle[linecolor=yellow](1.27,-1.51){1.42}
\psbezier[linecolor=darkblue](2.74,-1.16)(2.74,-0.18)(1.9,0.02)(1.12,-0.78)(0.34,-1.58)(0.88,-3.28)(1.6,-2.02)(2.32,-0.76)(0.68,0.78)(0.0,-0.7)

\psdots(1.27,-1.51)

\rput(2.5,3.03){\color{gdarkgray}$U(r)=\frac{1}{r^2}$}
\rput(4.88,2.27){\color{gdarkgray}$\vec{r}$}
\rput[l](0.4,1.96){\color{gdarkgray}$E$}
\rput[l](0.4,1.26){\color{gdarkgray}$E_\text{min}$}
\rput(3.0,2.55){\color{gdarkgray}\tiny$r_\text{max}$}
\rput(2.4,2.55){\color{gdarkgray}\tiny$r_0$}
\rput(2.0,2.55){\color{gdarkgray}\tiny$r_\text{min}$}
\end{pspicture} 

  \caption{Potential und Trajektorie.}
\end{figure}
Für die Oszillation erhält man im Allgemeinen keine rationalen Winkel $\ph =
\frac{2\pi n}{m}$, d.h. im Allgemeinen sind die Trajektorien nicht geschlossen.
Man kann analytisch beweisen, dass es genau zwei Potentiale gibt für die die
Trajektorien geschlossen sind, nämlich $V(r) = r^2$ und $V(r) = \frac{1}{r}$.


\subsection{Gravitationspotential}

Das Potential hat die Form
\begin{align*}
V(r) = -\frac{k}{r},
\end{align*}
wobei $k= G m_1 m_2$ im Fall der Gravitation oder $k = q_1 q_2$ im Fall der
elektrostatischen Wechselwirkung. Das effektive Potential hat die Form
\begin{align*}
U(r) = \frac{l^2}{2\mu r^2} - \frac{k}{r},
\end{align*}
und für die Bahntrajektorie gilt
\begin{align*}
\ph - \ph_0 = \frac{l}{\sqrt{2\mu}}\int\dr \frac{1}{r^2}\frac{1}{\sqrt{E-U(r)}}
= \frac{l}{\sqrt{2\mu}}\int\frac{\dr}{r^2}\frac{1}{\sqrt{E-\frac{l^2}{2\mu r^2}
+ \frac{k}{r}}}.
\end{align*}
Substituieren wir $x = \frac{1}{r}$, dann gilt für die Differentiale $\dx =
-\frac{1}{r^2}\dr$,
\begin{align*}
\ph - \ph_0 = -\frac{l}{\sqrt{2\mu}}\int\dx \frac{1}{\sqrt{E-\frac{l^2}{2\mu}x^2 +
kx}}
= -\int\dx \frac{1}{\sqrt{\frac{2E\mu}{l^2}-x^2 +
\frac{2\mu k}{l^2}x}}
\end{align*}
Ergänzen wir quadratisch mit $-\left(x-\frac{\mu k}{l^2}\right)^2 = -x^2 +
\frac{2\mu k}{l^2}x - \frac{\mu^2 k^2}{l^4}$, erhalten wir
\begin{align*}
\ph - \ph_0 &= -\int\dx \frac{1}{\sqrt{\underbrace{\frac{2E\mu}{l^2} +
\frac{\mu^2 k^2}{l^4}}_{=a^2} +\left(x-\underbrace{\frac{\mu
k}{l^2}}_{=b}\right)^2}}\\ 
&= - \int\dx \frac{1}{\sqrt{a^2-(x-b)^2}}= - \int\dx \frac{1}{a}\frac{1}{\sqrt{1-\left(\frac{x}{a}-\frac{b}{a}\right)^2}}
\\ &= - \int\dy \frac{1}{\sqrt{1-y^2}}= \arccos(y) = \arccos\left(\frac{x}{a}-\frac{b}{a} \right)
\\ &= \arccos\left(\frac{\frac{x}{b} - 1}{\frac{a}{b}}\right)
= \arccos\left(\frac{\frac{x}{\frac{\mu k}{l^2}} -
1}{\frac{\sqrt{\frac{2E\mu}{l^2} + \frac{\mu^2 k^2}{l^4}}}{\frac{\mu
k}{l^2}}}\right)\\ 
&= \arccos\left(\frac{\frac{x}{\frac{\mu k}{l^2}} - 1}{\sqrt{1 +
\frac{2El^2}{\mu k^2}}}\right)
= \arccos\left(\frac{\frac{l^2x}{\mu k} - 1}{\sqrt{1+\frac{2El^2}{\mu
k^2}}}\right) \\ &=
\arccos\left(\frac{\frac{l^2}{\mu k}\frac{1}{r} -1}{\sqrt{1+\frac{2El^2}{\mu
k^2}}} \right).
\end{align*}
Die Bahntrajektorie ist somit gegeben durch,
\begin{align*}
&\frac{p}{r} = 1 + \ep\cos(\ph-\ph_0),\\
&p = \frac{l^2}{\mu k},\qquad \ep = \sqrt{1+\frac{2 E l^2}{\mu k^2}}, 
\end{align*}
was einem Kegelschnitt entspricht.

\begin{figure}[!htbp]
  \centering
  
\begin{pspicture}(0,-0.8)(3.8621874,1.42)
\psline{->}(0.18,-0.6)(0.2,1.28)
\psline{->}(0.0,0.58)(3.8,0.58)

\psbezier[linewidth=0.04](0.58,1.4)(0.58,0.58)(0.66,-0.58)(1.1,-0.58)(1.54,-0.58)(1.88,0.44)(3.52,0.48)

\psline[linestyle=dashed,dash=0.06cm 0.04cm,linewidth=0.02cm](0.0,0.24)(3.8,0.24)
\psline[linestyle=dotted,dotsep=0.06cm](0.66,0.24)(0.66,0.66)
\psline[linestyle=dotted,dotsep=0.06cm](2.34,0.22)(2.34,0.66)

\rput(3.76,0.81){\color{gdarkgray}$r$}
\rput(0.9,0.99){\color{gdarkgray}$\frac{p}{1+\ep}$}
\rput(2.3,0.99){\color{gdarkgray}$\frac{p}{1-\ep}$}
\end{pspicture} 

  \caption{Trajektorie des Teilchens im effektiven Potential.}
\end{figure}
Der Radialteil der Trajektorie oszilliert zwischen $\frac{p}{1+\ep}$ und
$\frac{p}{1-\ep}$.
Wir wollen nun untersuchen wie eine Änderung von $E$ auf $\ep$ wirkt.

\begin{tabular}[h]{lll}
$E > 0 \Rightarrow \ep > 1$ & Hyperbel\\
$E = 0 \Rightarrow \ep = 1$ & Parabel\\
$E < 0 \Rightarrow \ep < 1$ & Ellipse\\
$E=E_\text{\tiny min} \Rightarrow \ep = 0$ & Kreisbahn 
\end{tabular} 

\begin{figure}[!htbp]
  \centering
  
\begin{pspicture}(-0.2,-1.3)(4.2,1.2)

\psellipse[linewidth=0.04,dimen=outer](2.12,0)(1.58,0.8)

\psdots[linecolor=yellow](1.22125,0)
\psdots[linecolor=darkblue](0.56125,0)
\psdots[linecolor=darkblue](3.68125,0)

\psline(0.64125,0)(1.12125,0)
\psline(1.32125,0)(3.60125,0)

\psbezier(3.04125,0.055625)(3.2679167,0.695625)(3.56125,-0.024375)(3.78125,0.555625)
\psbezier(0.10125,-0.624375)(0.32791665,0.015625)(0.62125,-0.704375)(0.84125,-0.124375)

\rput(3.9257812,0.885625){\color{gdarkgray}$\frac{p}{1-\ep}$}
\rput(0.08578125,-1){\color{gdarkgray}$\frac{p}{1+\ep}$}
\end{pspicture}

\caption{Ellipsenbahn der Erde.}
\end{figure}

\subsubsection{Repitition: Kegelschnitte}
%TODO: Skizze Kegelschnitt
Der Kegel kann durch
\begin{align*}
z = \sqrt{x^2+y^2} = r
\end{align*}
parametrisiert werden, die Ebene durch
\begin{align*}
z = z_0 - \lambda x.
\end{align*}

Die Schnittkurve der beiden Flächen ist gegeben durch,
\begin{align*}
&z_0-\lambda x = \sqrt{x^2+y^2}=r\\ 
\Leftrightarrow &z_0 -\lambda r\cos\ph = r\\
\Leftrightarrow &\frac{z_0}{r}  = 1-\lambda\cos\ph.
\end{align*}
und dies ist genau die Form der Trajektorie, die wir bereits berechnet haben.
%TODO: Bildchen Kegelschnitte

Man kann die Schnittkurve auch als \emph{quadratische Form} auffassen, wodurch
wir die Ellipsengleichung erhalten,
\begin{align*}
r^2 = (x^2+y^2) = (p-\ep x)^2 \Rightarrow \frac{\left(x+x_0\right)^2}{a^2} +
\frac{y^2}{b^2} = 1.
\end{align*}
Die \emph{große Halbachse} ist hierbei durch
\begin{align*}
a = \frac{p}{1-\ep^2},
\end{align*}
die \emph{kleine Halbachse} durch
\begin{align*}
b = \frac{p}{\sqrt{1-\ep^2}},
\end{align*}
und der \emph{Brennpunkt} durch
\begin{align*}
x_0 = \frac{\ep p}{1-\ep^2},
\end{align*}
gegeben.

\subsubsection{Keplerschen Gesetze}

Wir sind nun in der Lage mithilfe unserer Lösung des Zweikörper
Zentralkraftproblems die drei Keplerschen Gesetze zu beweisen.
\begin{propn}[1. Keplersches Gesetz]
Die Planeten bewegen sich auf Ellipsen mit der Sonne im Brennpunkt.\fishhere
\end{propn}
Dieses Gesetz gilt, wenn die Masse der Planeten klein im Vergleich zur Masse
der Sonne ist. Kepler hatte in diesem Fall Glück, da unser Sonnensystem gerade
so beschaffen ist.
\begin{proof}
Ist die Masse des Planeten klein im Vergleich zur Sonne, dann liegt der
Brennpunkt näherungsweise in der Sonne.\qedhere
\end{proof}

\begin{propn}[2. Keplersche Gesetz]
Eine von der Sonne zu einem Planeten gezogene Strecke überstreicht in gleichen
Zeiträumen gleiche Flächen.\fishhere
\end{propn}
\begin{proof}
Wir haben bereits gezeigt, dass dieses Gesetz äquivalent zur Drehimpulserhaltung
ist.\qedhere
\end{proof}

\begin{propn}[3. Keplersche Gesetz] Betrachtet man die Bahn eines Planetens
mit großer Halbachse $a$ und der Umlaufzeit $T$, dann ist 
$\dfrac{T^2}{a^3}$ unabhängig vom Planeten.\fishhere
\end{propn}
Auch dieses Gesetzt gilt nur für Planeten mit einer im Vergleich zur Sonne
kleinen Masse.
\begin{proof}
Wir verwenden $\frac{\dA}{\dt} = \frac{l}{2\mu}$. Die Fläche einer Ellipse ist
\begin{align*}
&A = \frac{\dA}{\dt}T = \pi a b = T\frac{l}{2\mu}\\
\Rightarrow & T = \frac{2\mu}{l}\pi a \frac{p}{\sqrt{1-\ep^2}}
= \frac{2\mu}{l}\pi a\frac{\frac{l^2}{\mu k}}{\sqrt{1-1+\frac{2El^2}{\mu k^2}}}
= 2\pi a\sqrt{\frac{\mu}{2E}}.
\end{align*}
Mit $a = \dfrac{p}{1-\ep^2} = \dfrac{l^2}{\mu k}\dfrac{1}{\frac{2El^2}{\mu k^2}}
= \dfrac{k}{2E}$ erhalten wir $\dfrac{1}{2E} = \dfrac{a}{k}$, wobei
\begin{align*}
k = G M_s m_e,\qquad \mu = \frac{M_s m_e}{M_s+m_e}\approx m_e.
\end{align*}
Einsetzen ergibt nun,
\begin{align*}
&T = 2\pi a^{\frac{3}{2}}\frac{1}{\sqrt{G M_S}},\\
\Leftrightarrow & T^2 = \frac{4\pi^2}{G M_s}a^3,
\end{align*}
und der Vorfaktor ist für das Sonnensystem konstant.\qedhere
\end{proof}

Interessant ist, dass für $E\le 0$ die Bahnkurven im Zentralpotential stets
geschlossen sind. Verantwortlich dafür ist eine weitere
Erhaltungsgröße, der \emph{Laplace-Runge-Lenz Vektor}.

\subsubsection{Laplace-Runge-Lenz Vektor}

Das Coulomb Potential $V(\abs{\vec{r}}) = -\frac{k}{\abs{\vec{r}}}$ zeichnet
sich durch eine weitere Erhaltungsgröße aus, den Laplace-Runge-Lenz Vektor
\begin{align*}
\vec{A} = \vec{p}\times\vec{L} - \mu k\frac{\vec{r}}{\abs{\vec{r}}}.
\end{align*}
\begin{proof}
Die Erhaltung des Vektors sieht man wie folgt ein,
\begin{align*}
\frac{\diffd}{\dt} \vec{A} &= \dot{\vec{p}}\times\vec{L} - \mu k\left( 
\frac{\dvec{r}}{\abs{\vec{r}}} -
\frac{\vec{r}}{\abs{\vec{r}}^2}\frac{\diffd}{\dt}\abs{\vec{r}} \right)\\
&=
-\frac{\mu k}{\abs{\vec{r}}^3}\vec{r}\times (\vec{r}\times\dvec{r}) - \frac{\mu
k}{\abs{\vec{r}}^3}\left(\abs{\vec{r}}^2\dvec{r} -
\vec{r}(\vec{r}\dvec{r})\right) = 0.\qedhere
\end{align*}
\end{proof}

Dieser Vektor steht senkrecht zum Drehimpuls $L$
\begin{align*}
\vec{A}\cdot \vec{L} = 0.
\end{align*}
Zudem zeigt er in Richtung des Perihelion. Konsequenz seiner Erhaltung ist,
dass das Perihelion sich nicht dreht.

\subsection{Streuexperimente}

Streuexperimente sind die Schlüsselversuche der Kern- und
Elementarteilchenphysik. Hierbei werden leichte Teilchen auf schwere ruhende
Teilchen geschossen und dadurch abgelenkt. Die Ablenkung kann Aufschluss über
die Struktur der Materie geben.
\begin{figure}[!htbp]
  \centering
  
\begin{pspicture}(-0.2,-1.4)(5.8,1.5)

\psline(0.0,-0.6416429)(3.92,-0.6416429)
\psdots[linecolor=yellow](4.04,-0.6416429)

\psline(4.14,-0.6416429)(5.36,-0.6416429)
\psline(4.1,-0.54164296)(5.1,0.45835707)

\psarc(4.2,-0.6416429){0.5}{0.0}{60.945396}

\psline(5.38,0.35835707)(4.66,1.0783571)
\psbezier[linecolor=darkblue]{->}(0.2,-0.061642937)(1.28,-0.061642937)(2.34,-0.061642937)(2.8,-0.061642937)(3.26,-0.061642937)(3.68,-0.08164294)(4.02,0.21835706)(4.36,0.51835704)(4.54,0.69835705)(4.66,0.8383571)
\psline{<->}(0.86,-0.5816429)(0.86,-0.10164294)

\psbezier(4.58,-0.56164294)(4.86,-0.88164294)(4.16,-0.6416429)(4.4,-1.0216429)
\psbezier(0.6,-0.50164294)(0.56,-1.0216429)(1.02,-0.50164294)(1.04,-1.0616429)
\psdots(0.08,-0.061642937)

\rput(0.58421874,-0.31164294){\color{gdarkgray}$b$}
\rput(0.51453125,0.34835705){\color{gdarkgray}$E=\frac{mv^2}{2}$}
\rput(4.4085937,-0.47164294){\color{gdarkgray}$\th$}

\rput{-45.56502}(0.95505434,3.9956415){\color{gdarkgray}\rput(5.2279687,0.8483571){Detektor}}
\rput(1.0198437,-1.211643){\color{gdarkgray}Stoßparameter}
\rput(4.465,-1.211643){\color{gdarkgray}Streuwinkel}
\end{pspicture} 

\caption{Streuexperiment.}
\end{figure}

Rutherford untersuchte 1909, wie $\alpha$-Teilchen an einer Goldfolie gestreut
werden und entdeckte dabei, dass der Kerndurchmesser um ca. 5 Größenordnungen
kleiner sein muss als der Atomdurchmesser. Viele sehen dies als die
Geburtsstunde der Kernphysik.

Im Folgenden untersuchen wir die Streuung von 2 Teilchen mit Zentralkraft. Wir
arbeiten wieder im Schwerpunktsystem mit der Relativkoordinate $\vec{r}$ als
Variable. Dabei vernachlässigen wir, dass das schwere Teilchen durch die
Streuung kinetische Energie erhält, für sehr schwere Teilchen ist dies eine
gute Näherung.

\begin{bemn}
Wir betrachten hier nur elastische Stöße, d.h. die Energie ist erhalten.\maphere 
\end{bemn}

\begin{defnn}
Der \emph{Stoßparameter} $b$ ist der Abstand zwischen Einfallsachse und
Streuzentrum,  $\th$ der \emph{Streuwinkel}.\fishhere
\end{defnn}

Betrachte einen einfallenden Teilchenstrahl mit Fluss
$F\entspr\frac{\text{Anzahl Teilchen}}{cm^2\ s}$. Messe die Zahl
der gestreuten Teilchen $\dN$ im Raumwinkel $\dOmega=
\sin\th\dth\dph$. Wir können keine diskreten Winkel messen, da der Detektor
eine endliche Ausdehnung und Auflösung hat. Es wird daher stets ein
Winkelbereich gemessen, wobei wir daran interessiert sind, diesen möglichst
klein zu halten, um genaue Ergebnisse zu erhalten.

Dies führt zur Definition des \emph{differentiellen Wirkungsquerschnitts},
\begin{align*}
\frac{\dsigma}{\dOmega} = \frac{1}{F}\frac{\dN}{\dOmega}.
\end{align*}
Der totale Streuquerschnitt ist dann gegeben durch,
\begin{align*}
&\sigma = \int \dOmega \left(\frac{\dsigma}{\dOmega}\right)
:= \int\limits_{0}^{2\pi}\dph\int\limits_{0}^\pi \dth \sin\th
\frac{\dsigma}{\dOmega},\\
&\left[\sigma\right] = \text{ Fläche},\quad F\sigma = \text{ Anzahl der
gestreuten Teilchen}
\end{align*}
\begin{bsp}
Im Fall einer Kugel, die von Teilchen umströmt wird, ist $\sigma = \pi
R^2$.\bsphere
\begin{figure}[!htbp]
  \centering
\begin{pspicture}(0,-0.6)(4.6,0.6)
\pscircle(2.21,-0.02){0.39}
\psline[linewidth=0.02]{->}(0.0,0.19)(1.82,0.19)
\psline[linewidth=0.02]{->}(0.0,-0.01)(1.76,-0.01)
\psline[linewidth=0.02]{->}(0.0,-0.21)(1.82,-0.21)
\psline[linewidth=0.02]{->}(0.0,-0.39)(1.96,-0.39)
\psline[linewidth=0.02]{->}(0.0,0.37)(1.96,0.37)
\psline[linewidth=0.02]{->}(0.0,-0.57)(3.04,-0.55)
\psline[linewidth=0.02]{->}(0.0,0.57)(3.04,0.57)
\psarc[linestyle=none,fillstyle=solid,fillcolor=glightgray](2.21,-0.02){0.37}{87.70939}{275.90613}
\psbezier(2.28,-0.05)(2.82,0.47)(3.16,-0.49)(3.68,0.17)

\rput(3.755,0.4){\color{gdarkgray}$\sigma=\pi R^2$}
\end{pspicture} 

\caption{Umströmte Kugel.}
\end{figure}
\end{bsp}

Für rotationssymmetrische Potentiale hängt $\frac{\dsigma}{\dOmega}$ nur vom
Winkel $\th$ und der Energie der einfallenden Teilchen ab. Für jedes Tupel aus
kinetischer Energie $E_\text{\tiny kin}$ und Stoßparameter $b$ können wir die
Trajektorie berechnen und den Wert $\th$ bestimmen. Wir haben damit das inverse
Problem $\th(E,b)$ gelöst, welches uns auf $b(E,\th)$ führt.

Aus der Energieerhaltung folgt, dass der Betrag der Geschwindigkeit für
einfallendes und ausfallendes Teilchen gleich ist,
\begin{align*}
v = \sqrt{\frac{2E}{m}}.
\end{align*}

Der Drehimpuls $l=mbv$ ist ebenfalls erhalten, das ausfallende Teilchen hat
somit auch den Stoßparameter $b$. Haben wir nun eine Lösung $\th(E,b)$, so
können wir eine kleine Winkeländerung $\dth$ durch eine kleine Änderung des
Stoßparameters $\db$ ausdrücken
\begin{align*}
\dth = \abs{\frac{\dth}{\db}}\db.
\end{align*}

\begin{figure}[!htbp]
  \centering
\begin{pspicture}(0,-0.5)(5.38,1.08)
\psline(0.0,-0.42)(3.92,-0.42)
\psdots(4.04,-0.42)
\psline(4.14,-0.42)(5.36,-0.42)
\psline(4.1,-0.32)(5.1,0.68)
\psarc(4.04,-0.42){0.64}{0.0}{178.31532}
\psbezier[linecolor=darkblue]{->}(0.2,0.16)(1.28,0.16)(2.34,0.16)(2.8,0.16)(3.26,0.16)(3.68,0.14)(4.02,0.44)(4.36,0.74)(4.54,0.92)(4.66,1.06)
\psline{<->}(0.86,-0.36)(0.86,0.12)
\psdots(0.08,0.16)
\psline[linestyle=dotted,dotsep=0.06cm](3.94,-0.3)(3.44,0.38)

\rput(4.4085937,-0.25){\color{gdarkgray}$\th$}
\rput(0.58421874,-0.09){\color{gdarkgray}$b$}

\rput(0.30671874,0.57){\color{gdarkgray}$\vec{v}$}

\rput(4.0285935,-0.01){\color{gdarkgray}$\ph$}

\rput(3.6485937,-0.25){\color{gdarkgray}$\ph$}
\end{pspicture} 

\caption{Streuexperiment.}
\end{figure}

Für die gestreuten Teilchen gilt $\dN = b \db\dph F$, also erhalten wir
\begin{align*}
\frac{\dsigma}{\dOmega} &= \frac{1}{F}\frac{\dN}{\dOmega}
= \frac{1}{F}\frac{b\db\dph F}{\sin\th\dth\dph} =
\frac{b\db}{\sin\th\abs{\frac{\dth}{\db}}\db}\\
&= \frac{1}{\sin\th}b\abs{\frac{\diffd}{\dth}b(\th,E)}.\tag{*}
\end{align*}

\subsubsection{Rutherfordstreuung}

Wir betrachten das Potential $\dfrac{1}{r}$ im repulsiven Fall
\begin{align*}
V(r) = -\frac{k}{r},\qquad k = -e e',
\end{align*}
wobei $e,e'$ die Ladung von Streuteilchen und Target sind.

Verwenden wir nun unsere Lösung
\begin{align*}
\frac{1}{r} = \frac{mee'}{l^2}\left(\ep\cos\ph - 1\right),
\end{align*}
so erhalten wir für $r\to\infty$,
\begin{align*}
&0 = \ep\cos\ph - 1
\Leftrightarrow \cos \ph = \frac{1}{\ep}
\overset{\pi = 2\ph+\th}{\Rightarrow} \sin\left(\frac{\th}{2}\right) =
\frac{1}{\ep}\\
&\cot\left(\frac{\th}{2}\right) =
\frac{\cos\left(\frac{\th}{2}\right)}{\sin\left(\frac{\th}{2}\right)}
= \frac{\sqrt{1-\frac{1}{\ep^2}}}{\frac{1}{\ep}} = \sqrt{\ep^2-1}
\end{align*}
Mit $\ep = \sqrt{1+\left(\dfrac{2Eb}{ee'}\right)^2}$ ergibt sich somit,
\begin{align*}
b(\th,E) = \cot\left(\frac{\th}{2}\right) \frac{ee'}{2\abs{E}}.
\end{align*}
Setzen wir dies in (*) ein, ergibt sich
\begin{align*}
\frac{\dsigma}{\dOmega} &= 
\frac{\cot\left(\frac{\th}{2}\right)}{\sin\th}\frac{ee'}{2\abs{E}} \abs{\frac{\diffd}{\dth}
\left(\cot\left(\frac{\th}{2}\right) \frac{ee'}{2\abs{E}}\right)}\\
&=
\abs{\frac{ee'}{2\abs{E}}\cot\left(\frac{\th}{2}\right)\frac{ee'}{2\abs{E}}
\frac{1}{2}\frac{1}{\sin^2\left(\frac{\th}{2}\right)}\frac{1}{\sin\th}}\\
&=
\frac{1}{4}\left(\frac{ee'}{2E}\right)^2\frac{1}{\abs{\sin\left(\frac{\th}{2}\right)}^4}.
\end{align*}
Dies ist die berühmte Rutherfordsche-Streuformel mit der
$\dfrac{1}{\sin\left(\frac{\th}{2}\right)^4}$-Abhängigkeit.


\newpage
\section{Systeme mit Zwangsbedingungen}

In praktischen Anwendungen treten in vielen Fällen Nebenbedingungen auf, die
den Konfigurationsraum des Problems einschränken. Die Idee ist nun, die
Lagrangefunktion so zu transformieren, dass sie sich unter diesen
Nebenbedingungen vereinfacht.

\begin{bsp}
Ein Pendel bestehe aus einer festen Stange der Länge $l$, an der eine Masse $m$
angebracht ist. Die feste Länge $l$ der Stange ergibt hier die
Zwangsbedingung.\bsphere
\end{bsp}
\begin{bsp}
Ein Zylinder, der sich in der schiefen Ebene befindet, erfährt die
Zwangsbedingungen dadurch, dass die Bewegung nur auf der Ebene stattfindet und
durch die Rollbedingung $\ds=R\dph$.\bsphere
\end{bsp}
\begin{figure}[!htbp]
  \centering
\begin{pspicture}(0,-0.88)(1.89,0.7)
\psline(0.856875,0.37)(0.176875,-0.53)
\psline[linestyle=dotted,dotsep=0.06cm](0.856875,0.37)(0.856875,-0.73)
\psdots(0.856875,0.35)
\psdots[linecolor=darkblue,dotstyle=square*](0.156875,-0.55)

\psarc(0.846875,0.32){0.51}{231.78897}{271.97495}
\psbezier(0.756875,0.01)(1.096875,0.37)(1.276875,-0.23)(1.536875,0.37)

\rput(0.13453124,-0.8){\color{gdarkgray}$m$}
\rput(0.4578125,0.14){\color{gdarkgray}$l$}
\rput(1.7214062,0.58){\color{gdarkgray}$\ph$}
\end{pspicture} \qquad
\begin{pspicture}(0.7,-0.5)(3.7,1.4)
\psline{->}(0.84,-0.4)(0.84,1.2471875)
\psline{->}(0.84,-0.4)(3.32,-0.4)
\psline(0.84,-0.4)(3.4,0.6271875)

\pscircle[linecolor=darkblue](2.75,0.8171875){0.40}
\psline[linestyle=dotted,dotsep=0.06cm]{->}(2.76,0.8271875)(2.76,-0.2728125)
\psline(2.76,0.8271875)(2.44,0.5871875)

\psdots(2.76,0.8271875)

\psarc(0.87,-0.3828125){0.87}{0.0}{22}

\psbezier(2.7,0.5071875)(3.1,1.1671875)(3.32,0.3671875)(3.54,0.9071875)

\rput(2.5185938,0.8771875){\color{gdarkgray}$R$}
\rput(3.0896876,0.0771875){\color{gdarkgray}$\vec{F}_g$}
\rput(1.513125,-0.25){\color{gdarkgray}$\alpha$}
\rput(3.5245314,1.1971875){\color{gdarkgray}$\ph$}
\end{pspicture} 

\caption{Pendel und Zylinder auf schiefer Ebene.}
\end{figure}

\subsubsection{Holonome Zwangsbedingungen}

Es gibt verschiedene Arten von Zwangsbedingungen. Die einfachste und häufigste
Art sind die \emph{holonomen Zwangsbedingungen}. Hier lässt sich die
Zwangsbedingung durch einen Satz Gleichungen, die die Koordinaten $q^i$
verknüpfen, darstellen.
\begin{align*}
f_j(q^1,\ldots,q^{3N},t) = 0,\qquad j = 1,\ldots,k.
\end{align*}
Bei $k$ Nebenbedingungen ist die Zahl der Freiheitsgrade auf $3N-k$ reduziert.
\begin{bsp}
Im Fall unseres Pendels ist
\begin{align*}
f_1(x,y) = x^2+y^2 - l^2 = 0.
\end{align*}
In Polarkoordinaten ist dadurch $r=l$ konstant.\bsphere
\end{bsp}
\begin{bsp}
Ein Teilchen gleitet reibungsfrei auf einer Oberfläche,
%TODO: Skizze
\begin{align*}
f(\vec{r}) = 0,
\end{align*}
wobei $f$ die Oberfläche beschreibt.\bsphere
\end{bsp}
\begin{bsp}
Zwei Teilchen, die durch einen starren Stab verbunden sind.
\begin{align*}
f_1(\vec{r}_1,\vec{r}_2) = \abs{\vec{r}_1-\vec{r_2}} - l = 0.
\end{align*}
\begin{figure}[!htbp]
  \centering
\begin{pspicture}(-0.1,-1)(2.52,0.9028125)
\psline{->}(0.02,-0.86)(0.02,0.7771875)
\psline{->}(0.02,-0.86)(2.5,-0.86)
\psdots(0.7,0.5171875)
\psdots(1.86,-0.1628125)
\psline[linecolor=darkblue](0.78,0.4771875)(1.78,-0.1228125)
\psline{->}(0.02,-0.86)(0.66,0.4371875)
\psline{->}(0.02,-0.86)(1.76,-0.2228125)

\rput(0.28,0.1471875){\color{gdarkgray}$\vec{r}_1$}
\rput(1.18,-0.5928125){\color{gdarkgray}$\vec{r}_2$}
\rput(0.65765625,0.8071875){\color{gdarkgray}$m_1$}
\rput(2.2,-0.2128125){\color{gdarkgray}$m_2$}
\rput(1.4009376,0.4471875){\color{gdarkgray}$l$}
\end{pspicture}

\caption{Starr verbundene Teilchen.}
\end{figure}

Eine Anwendung davon finden wir, wenn wir $O_2$-Moleküle bei tiefen
Temperaturen betrachten. Falls $k_BT << E_\text{\tiny Vibration}$, bleiben dem
Molekül nur noch die Translationsfreiheitsgrade.\bsphere
\end{bsp}

\begin{bemn}
Holonome Zwangsbedingungen definieren eine Hyperfläche im Konfigurationsraum
mit der Dimension $f=3N-k$.

Im Fall des Pendels wäre die Hyperfläche ein Kreis mit Radius $l$, die
verallgemeinerte Koordinate auf der Hyperfläche der Winkel $\ph$.\maphere
\end{bemn}

\subsubsection{Nichtholome Zwangsbedingungen}

Holonome Zwangsbedingungen sind sehr gutartig, da sie Probleme durch die
Reduktion von Freiheitsgraden vereinfachen. Es gibt jedoch auch
\emph{nichtholonome Zwangsbedinungen}, das sind Zwangsbedingungen, die sich
nicht auf die Form,
\begin{align*}
f_i(\vec{r},t) = 0,
\end{align*}
bringen lassen. In vielen Fällen verkomplizieren sie das Problem zusätzlich. Es
kann passieren, dass ein analytisch lösbares System durch nichtholonome
Zwangsbedingungen nicht mehr analytisch lösbar wird.

\begin{bsp}
Das Teilchen in der Box hat die Zwangsbedingungen,
\begin{align*}
&0\le x\le a,\\
&0\le y\le b.
\end{align*}
Die Nebenbedingungen liegen hier in Form einer Ungleichung vor; sie reduzieren
die Anzahl an Freiheitsgraden nicht.\bsphere

\begin{figure}[!htbp]
  \centering
\begin{pspicture}(0,-1.0117188)(2.7809374,0.99171877)
\psline{->}(0.28,-0.69)(0.28,0.9717187)
\psline{->}(0.26,-0.69)(2.7609375,-0.69)
\psline{->}(0.28,-0.69)(1.5609375,0.15171875)
\psdots[linecolor=darkblue](1.6209375,0.19171876)
\psline[linestyle=dotted,dotsep=0.06cm](0.28,0.63)(2.28,0.63)
\psline[linestyle=dotted,dotsep=0.06cm](2.28,-0.69)(2.28,0.63)
\psdots[dotstyle=square*](2.28,0.63)

\rput(2.2740624,-0.85828125){\color{gdarkgray}$a$}
\rput(0.08515625,0.6617187){\color{gdarkgray}$b$}
\rput(0.8409375,-0.03828125){\color{gdarkgray}$\vec{r}$}
\end{pspicture} 

\caption{Teilchen in der Box.}
\end{figure}
\end{bsp}

\begin{bsp}
Rollen eines Rades in der Ebene, mit dem Berührpunkt $(x,y)$, dem Winkel $\th$
zur Orientierung der Achse zur $x$-Richtung und dem Rollwinkel $\ph$.

\begin{figure}[!htbp]
  \centering
\begin{pspicture}(0,-1.02)(2.14,1.02)
\psline{->}(0.0,-0.18)(0.76,0.6)
\psline{->}(0.0,-0.18)(1.6,-1.0)
\psbezier[linecolor=darkblue]{->}(0.68,0.2)(1.44,0.48)(0.92,-0.92)(2.12,-0.72)
\psline{->}(0.0,-0.18)(0.98,0.18)
\psbezier(1.1883721,0.6150961)(1.26,0.23019223)(0.9595349,0.0)(0.88976747,0.40851015)(0.82,0.8170203)(1.1167442,1.0)(1.1883721,0.6150961)
\psline(1.02,0.5)(1.2,0.4)

\rput(1.7585938,0.53){\color{gdarkgray}$R$}
\psbezier[linewidth=0.02cm](1.1,0.5)(1.48,0.78)(1.26,0.18)(1.6,0.54)
\end{pspicture} 
\caption{Rollen eines Raeds in der Ebene.}
\end{figure}

Die Rollbedingung ist hier $\abs{\vec{v}} = R\dot{\ph}$, was auf
\begin{align*}
&\dot{x} = R\dot{\ph}\cos\th,\\
&\dot{y} = R\dot{\ph}\sin\th,
\end{align*}
führt. Man kann diese Nebenbedingungen nicht auf holonome Form bringen, da
$\th$ und $\ph$ vom Weg abhängen.\bsphere
\end{bsp}

Abgesehen von holonom und nicht-holonom gibt es noch weitere Möglichkeiten,
Zwangsbedingungen zu charakterisieren. \emph{Skleronome} (``starre'')
Zwangsbedingungen,
\begin{align*}
f_j(q^1,\ldots,q^{3N}) = 0,
\end{align*}
hängen nicht explizit von der Zeit ab. \emph{Rheonome} (``fließende'')
Zwangsbedingungen,
\begin{align*}
f_j(q^1,\ldots,q^{3N},t) = 0,
\end{align*}
hängen hingegen explizit von der Zeit ab.

\begin{bsp}
Eine Schiefe Ebene mit zeitabhängigem Neigungswinkel.\bsphere
\end{bsp}

\subsection{Dynamik eines Systems mit holonomen Zwangsbedingungen}

Im Folgenden wollen wir das Vorgehen zum Lösen von Problemen mit
Zwangsbedingungen erarbeiten. Auf Beweise soll dabei zunächst verzichtet
werden; die Korrektheit der Methode werden wir zum Schluss elegant mit dem
d'Almbertschen Prinzip zeigen.

\begin{enumerate}[label=\arabic{*}.)]
  \item Bestimme den Konfigurationsraum des Systems und führe verallgemeinerte
  Koordinaten ein, welche die Zwangsbedingungen erfüllen,
\begin{align*}
q^i,\qquad i = 1,\ldots,f.
\end{align*}
Die natürlichen Koordinaten lassen sich dann mit Hilfe der verallgemeinerten
ausdrücken,
\begin{align*}
\vec{r}(q^1,\ldots,q^f).
\end{align*}
\item Drücke die kinetische Energie in diesen verallgemeinerten Koordinaten aus,
\begin{align*}
T = \sum_i \frac{m_i}{2}\dvec{r}_i^2 \longrightarrow T =
\frac{1}{2}\sum\limits_{i\neq j}^f A_{ij}(q^k)\dot{q}^i\dot{q}^j.
\end{align*}
Verfahre analog mit dem Potential,
\begin{align*}
V(\vec{r}_1,\ldots,\vec{r}_N) \longrightarrow V(q^1,\ldots,q^f).
\end{align*}
\item Wir erhalten so die Lagrangefunktion in den verallgemeinerten Koordinaten,
\begin{align*}
L(q^i,\dot{q}^i,t) = T-V.
\end{align*}
Die Bewegungsgleichungen folgen anschließend aus den Euler-Lagrange-Gleichungen,
\begin{align*}
\frac{\diffd}{\dt}\frac{\partial L}{\partial \dot{q}^i} - \frac{\partial
L}{\partial q^i} = 0,\qquad i=1,\ldots,f.
\end{align*}
\end{enumerate}

\begin{bsp}
Wir wollen nun das Problem unseres Pendels unter Zwangsbedingungen lösen.
%TODO: Skizze
Sei $m$ die Masse und $\vec{r} = \begin{pmatrix}x\\y\end{pmatrix}$ die
Koordiante, sowie
\begin{align*}
&x^2+y^2 = l^2\\
&z = 0
\end{align*}
die Zwangsbedingung.
\begin{enumerate}[label=\arabic{*}.)]
  \item Durch die Einführung von Polarkoordinaten wird die Zwangsbedingung zu
  $r=l$. Wir erhalten damit
\begin{align*}
&x = l\sin\ph,\\
&y = -l\cos\ph,\\
&z = 0
\end{align*} 
wobei $\ph$ die verallgemeinerte Koordinate des Problems ist.
\item Um die kinetische Energie zu bestimmen, benötigen wir die Ableitungen,
\begin{align*}
&\dot{x} = l\dot{\ph}\cos\ph,\\
&\dot{y} = l\dot{\ph}\sin\ph.
\end{align*}
Die kinetische Energie hat somit die Form,
\begin{align*}
T = \frac{1}{2}m\left(\dot{x}^2+\dot{y}^2\right) = \frac{1}{2}m l^2\dot{\ph}^2.
\end{align*}
Das Potential ist gegeben durch,
\begin{align*}
V = mgy = -mgl\cos\ph.
\end{align*}
\item Die Lagrangefunktion hat damit in verallgemeinerten Koordinaten die
Gestalt,
\begin{align*}
L(\ph,\dot{\ph}) = \frac{1}{2}ml^2\dot{\ph}^2+mgl\cos\ph.
\end{align*}
Die Bewegungsgleichungen erhalten wir somit durch
\begin{align*}
&\ddot{\ph} = -\frac{g}{l}\sin\ph.
\end{align*}
Für $\dot{\ph}=0$ und $\ph=0,\pi,2\pi$ erhalten wir stationäre Lösungen.
Für $\dot{\ph}$ klein ergibt sich in $\ph=0,2\pi,4\pi,\ldots$ eine harmonische
Schwingung, während das Gleichgewicht in $\pi,3\pi,\ldots$ instabil ist.\bsphere
%TODO: Phasendiagramm.
\end{enumerate}
\end{bsp}

\begin{bemn}
In einem System mit holonomen Zwangsbedingungen müssen zusätzliche Kräfte
wirken, die die Teilchen auf die Hyperfläche zwängen. Die Newtonschen Gleichungen haben die
Form,
\begin{align*}
m_i \ddvec{r}_i = -\nabla_i V + \vec{Z}_i,
\end{align*}
wobei $\vec{Z}_i$ die zusätzliche Kraft beschreibt.\maphere
\end{bemn}

\begin{defnn}
Die Größen,
\begin{align*}
\vec{Z}_i = m_i\ddvec{r}_i - \vec{F}_i = m_i\ddvec{r}_i + \nabla_i V,\qquad i =
1,\ldots,N
\end{align*}
heißen \emph{Zwangskräfte}. Diese Kräfte sind apriori unbekannt, können aber
aus der Lösung des Problems berechnet werden.\fishhere
\end{defnn}

\begin{bsp}
%TODO: Pendel Ruhelage
Durchquert das Pendel die Ruhelage, so ist
\begin{align*}
\ddvec{r} = 0 \Rightarrow \vec{Z} = -\vec{F}_g.\bsphere
\end{align*}
\end{bsp}

\begin{defnn}
Eine \emph{virtuelle Verrückung} $\delta \vec{r}_i$ oder $\delta q^i$ ist eine
infinitesimale Änderung der Lagekoordinaten, welche mit den Zwangsbedingungen
verträglich ist.\fishhere
\end{defnn}
%TODO: Bild Virtuelle Verrückung

\subsubsection{Prinzip von d'Alembert}
\begin{propn}[Prinzip von d'Alembert]
In einem System mit Zwangsbedingungen erfüllt die Bewegungsgleichung,
\begin{align*}
\sum\limits_{i=1}^N \left(m_i\ddvec{r}_i - \vec{F}_i \right)\delta\vec{r}_i = 0,
\end{align*}
für alle virtuellen Verrückungen $\delta\vec{r}_i$.\fishhere
\end{propn}

Diese Aussage ist äquivalent dazu, dass Zwangskräfte unter einer virtuellen
Verrückung keine Arbeit leisten,
\begin{align*}
\delta A = \vec{Z}_i \cdot \delta\vec{r}_i = \left(m_i\ddvec{r}_i -
\vec{F}_i\right) \delta\vec{r}_i = 0.
\end{align*}
In einigen Lehrbüchern wird das Prinzip von d'Alembert als Axiom festgelegt,
für alle Standardprobleme der klassischen Mechanik leisten die Zwangskräfte
jedoch keine Arbeit.
%\begin{bsp}
%TODO: Zwangskraft = Normalkraft
%\end{bsp}

\begin{bemn}
Im statischen Fall reduziert sich das Prinzip von d'Almebert auf eine Bedingung
für die Gleichgewichtslage,
\begin{align*}
\sum_i \vec{F}_i\delta\vec{r}_i = 0.\maphere
\end{align*}
\end{bemn}

\begin{bsp}
Betrachten wir einen Flaschenzug.
%TODO: Flaschenzug
Es gilt $\delta h_1 = -2 \delta h_2$ und
\begin{align*}
&\vec{F}_1 \delta\vec{h}_1 + \vec{F}_2\delta\vec{h}_2 = 0\\
\Rightarrow & m_1 g \delta h_1 - m_2 g \frac{\delta h_1}{2} = 0 \Leftrightarrow
 \delta h_1 g \left(m_1 - \frac{m_2}{2}\right) = 0.
 \end{align*}
Für ein Gleichgewicht muss gelten
\begin{align*}
&\frac{m_2}{m_1} = 2.\bsphere
\end{align*}
\end{bsp}

Wir wollen nun zeigen, dass aus dem Prinzip von d'Alembert das Hamiltonsche
Variationsprinzip folgt und somit unser Lösungsschema für holonome
Zwangsbedingungen äquivalent zum Prinzip von d'Alembert ist.

\begin{proof}[Beweis der Äquivalenz.]
Wir führen dazu geeignete Koordinaten ein, die die holonomen Zwangsbedingungen
erfüllen,
\begin{align*}
&\vec{r}_i(q^1,\ldots,q^f,t).
\end{align*}
Eine virtuelle Verrückung hat somit die Form,
\begin{align*}
&\delta\vec{r}_i = \sum\limits_{j=1}^f \frac{\partial \vec{r}_i}{\partial
q^j}\delta q^j.
\end{align*}
Betrachten wir nun das Prinzip von d'Alembert
\begin{align*}
0 = \sum_i \left(m_i \ddvec{r}_i - \vec{F}_i\right)\delta\vec{r}_i,
\end{align*}
so erhalten wir für die Summanden:
\begin{itemize}
  \item 
\begin{align*}
\sum_i \vec{F}_i \delta \vec{r}_i = \sum_{i,j} \vec{F}_i \frac{\partial
\vec{r}_i}{\partial q^j}\delta q^j
= \sum_j \underbrace{\left(\sum_i \vec{F}_i \frac{\partial\vec{r}_i}{\partial
q^j}\right)}_{:=\vec{Q}_j}\delta q^j.
\end{align*}
Wobei wir $\vec{Q}_j$ als verallgemeinerte Kraft bezeichnen. Es gilt
\begin{align*}
\vec{Q}_j = \sum_i \vec{F}_i \frac{\partial\vec{r}_i}{\partial q^j}
= - \sum_i \nabla_i V \frac{\partial\vec{r}_i}{\partial q^j}
= -\frac{\diffd}{\diffd q^j}V(q^1,\ldots,q^f).
\end{align*}
\item
\begin{align*}
\sum_i m_i\ddvec{r}_i \delta\vec{r}_i  &= \sum_i\sum_j m_i\ddvec{r}_i 
\frac{\partial \vec{r}_i}{\partial q^j}\delta q^j = 
\sum_j \left(\sum_i m_i\ddvec{r}_i 
\frac{\partial \vec{r}_i}{\partial q^j}\right)\delta q^j\\
&= \sum_j \delta q^j \left[\sum_i \frac{\diffd}{\dt}\left( m_i\dvec{r}_i 
\frac{\partial \vec{r}_i}{\partial q^j}\right) -
m_i\dvec{r}_i\frac{\diffd}{\dt}\frac{\partial \vec{r}_i}{\partial q^j}\right]
\end{align*}
Nun ist
\begin{align*}
\frac{\partial
\dvec{r}_i}{\partial\dot{q}^j} = \frac{\partial}{\partial \dot{q}^j} \left[
\sum\limits_k \frac{\partial \vec{r}_i}{\partial q^k}\dot{q}^k +
\frac{\partial}{\partial t}\vec{r}_i \right] = \frac{\partial
\vec{r}_i}{\partial q^j}
\end{align*}
und daher gilt,
\begin{align*}
\sum_i m_i\ddvec{r}_i \delta\vec{r}_i = \sum_j \delta q^j \left[\sum_i
\frac{\diffd}{\dt}\underbrace{\left( m_i\dvec{r}_i \frac{\partial
\dvec{r}_i}{\partial \dot{q}^j}\right)}_{=\frac{\partial T}{\partial \dot{q}^j}}
- \underbrace{m_i\dvec{r}_i \frac{\partial \dvec{r}_i}{\partial
q^j}}_{\frac{\partial T}{\partial q^j}}\right]
\end{align*}
\end{itemize}
Zusammengefasst gilt also,
\begin{align*}
0 = \sum_j \left[\frac{\diffd}{\dt}\frac{\partial T}{\partial
\dot{q}^j}-\frac{\partial T}{\partial q^j} + \frac{\partial V}{\partial
q^j} \right]\delta q^j,
\end{align*}
und da das Potential nicht geschwindigkeitsabhängig ist, erhalten wir
\begin{align*}
0 = \sum_j \left[\frac{\diffd}{\dt}\frac{\partial L}{\partial
\dot{q}^j}-\frac{\partial L}{\partial q^j}\right]\delta q^j,
\end{align*}
für alle Variationen $\delta q^j$. Da die Variationen unabhängig sind, ist
somit der Ausdruck in der Klammer Null und wir erhalten die
Euler-Lagrange-Gleichungen.\qedhere
\end{proof}

Das Prinzip vn d'Alembert ist also äquivalent zum Hamiltonischen Prinzip
angewandt auf die Lagrange Funktion mit verallgemeinerten Koordianten.

\begin{bemn}
Ein freies Teilchen, das sich unter Zwangsbedingungen auf einer Fläche bewegt,
bewegt sich auf einer Geodäte, d.h. entlang der kürzesten Verbindungen zwischen
zwei Punkten.\maphere
\end{bemn}
\begin{bsp}
Ein käftefreies Teilchen das sich durch Zwangsbedingungen auf einer
Kugeloberfläche bewegt, bewegt sich entlang eines Großkreises.
\begin{proof}
Für ein allgemeines kräftefreies Teilchen ist der Lagrange
\begin{align*}
L = T = \frac{1}{2}m\dvec{r}^2.
\end{align*}
Die Euler-Lagrange-Gleichungen haben die Form,
\begin{align*}
\frac{\diffd}{\dt}\frac{\partial T}{\partial \dot{q}^i} - \frac{\partial
T}{\partial q^i}  = 0.
\end{align*}
Für eine Geodäte ist der Ausdruck,
\begin{align*}
\int\dt \sqrt{\dvec{r}^2} = \int\dt \sqrt{\frac{2}{m}T}
\end{align*}
extremal. Ableiten dieses Ausdrucks ergibt,
\begin{align*}
\frac{1}{2}\frac{\diffd}{\dt}\frac{\partial T}{\partial
\dot{q}^i}\frac{1}{\sqrt{T}} - \frac{1}{2}\frac{\partial T}{\partial
q^i}\frac{1}{\sqrt{T}} = 0.
\end{align*}
Die Energie ist erhalten, also ist $T=\const$ und wir erhalten die
Euler-Lagrange-Gleichung,
\begin{align*}
\frac{\diffd}{\dt}\frac{\partial T}{\partial \dot{q}^i} - \frac{\partial
T}{\partial q^i}  = 0.\qedhere\bsphere
\end{align*}
\end{proof}
\end{bsp}

\subsection{Mathematische Struktur}

Wir wollen noch kurz auf die mathematische Struktur der Probleme mit holonomen
Zwangsbedingungen eingehen.

Die Zwangsbedingungen
\begin{align*}
f_i(\vec{r}_1,\ldots,\vec{r}_N) = 0,\qquad i = 1,\ldots,k,
\end{align*}
definieren eine \emph{Mannigfaltigkeit} $M\subseteq\R^{3N}$, durch $M =
\bigcap_i f_i^{-1}(0)$ der Dimension $3N-k$.

Die verallgeminerten Koordinaten $q^i$, $i=1,\ldots,f$ stellen \emph{Karten} zu
dieser Mannigfaltigkeit $M$ dar.
%TODO: Bild Karte

An jedem Punkt $p$ der Mannigfaltigkeit haben wir eine Tangentialfläche $TM_p$.
Diese bildet im Gegensatz zur Mannigfaltigkeit einen Vektorraum. Die
Tangentialvektoren sind alle möglichen $\dot{q}^i$ in $p$, d.h. $TM_p$ ist
die Menge der Geschwindigkeitsvektoren der Kurven in $M$ durch $p$.

Für eine Mannigfaltigkeit, die sich in den $\R^n$ einbetten lässt, kann ein
Skalarprodukt definiert werden. Eine solche Mannigfaltigkeit heißt
\emph{Riemannsche Mannigfaltigkeit}. Durch holonome Zwangsbedingungen werden
also immer Riemannsche Mannigfaltigkeiten definiert. Das Skalarprodukt auf $M$ ist die
\emph{kinetische Energie} gegeben durch,
\begin{align*}
T: TM_p\times TM_p \to \R,\quad \dot{q}^i,\dot{q}^j \mapsto
\lin{\dot{q}^i,\dot{q}^j} = \sum\limits_{i,j} a_{ij}(q^k)\dot{q}^i\dot{q}^j,
\end{align*}
wobei $a_{ij}(q^k)$ einen metrischen Tensor bezeichnet.
Die \emph{potentielle Energie} ist eine rellwerte Funktion auf $M$,
\begin{align*}
V: M\to \R,\quad q^i \mapsto V(q^i).
\end{align*}
Die Vereinigung aller Tangentialräume an $M$ heißt \emph{Tangentialbündel},
\begin{align*}
TM = \bigcup\limits_{p\in M} TM_p.
\end{align*}
Die Lagrangefunktion ist somit eine skalare Funktion auf dem Tangentialbündel,
\begin{align*}
L : TM\to\R,\quad \dot{q}^i,q^i  \mapsto L(\dot{q}^i,q^i).
\end{align*} 
Die verallgemeinerten Impulse
\begin{align*}
p_j = \frac{\partial L}{\partial \dot{q}^j}
\end{align*}
sind Elemente aus dem \emph{Dualraum} von $TM_p$, d.h. eine Abbildung von
$TM_p$ in die reellen Zahlen,
\begin{align*}
p_i : TM_p \to \R,\quad q^i\mapsto \sum_i p_i q^i,\qquad p_i\in TM_p^*.
\end{align*}
Analog ist auch $\frac{\partial L}{\partial q^i}$ ein Element aus dem Dualraum.

Eine virtuelle Verrückung $\delta q^i$ ist ein Vektor im Tangentialraum, d.h.
ein Geschwindigkeitsvektor einer Kurve $q^i(t)$ am Punkt $p$.

Betrachten wir nun eine Variablentransformation,
\begin{align*}
q^i \mapsto q^i(\xi^1, \ldots, \xi^f),
\end{align*}
so transformieren die $\dot{q}^i$ \emph{kontravariant}
\begin{align*}
\dot{q}^i = \sum_j \frac{\partial q^i}{\partial \xi^j}\dot{\xi}^j,
\end{align*}
und die verallgemeinerten Impulse \emph{kovariant}
\begin{align*}
p_i = \frac{\partial L}{\partial \dot{q}^i} = \sum_j \frac{\partial
\eta_j}{\partial \dot{q}^i}\frac{\partial L}{\partial \eta^j}.
\end{align*}
Beim Prinzip von d'Alembert muss beachtet werden, dass 
Vektoren und Impulse unterschiedlich transformieren.
Die Euler-Lagrange-Gleichung ist jedoch vollständig kontravariant. Beim
Übergang von Geschwindigkeiten zu Impulsen müssen wir den metrischen Tensor
berücksichtigen,
\begin{align*}
p_i = \sum\limits_j a_ij(q^i)\dot{q}^j.
\end{align*}

\begin{bemn}
Ein kräftefreies Teilchen, das sich auf einer Hyperfläche bewegt, folgt einer
Geodäte. Die Distanz wird dabei von der durch das Skalarprodukt (kinetische
Energie) induzierten Metrik gemessen.\maphere
\end{bemn}

\begin{bsp}
Die Bewegung auf einer rotationssymmetrischen Fläche in Zylinderkoordinaten
($r,\ph,z$) wird durch die holonome Zwangsbedingung $z=f(r)$ beschrieben. Die
kinetische Energie ist
\begin{align*}
T = \frac{m}{2}\left[\dot{r}^2 + r^2\dot{\ph}^2 + (f'(r)\dot{r})^2\right] =
L(r,\ph,\dot{r},\dot{\ph}).
\end{align*}
Die Drehimpulserhaltung liefert $p_\ph = r^2\dot{\ph} = \const =
r\abs{v}\sin\alpha$. Die Energieerhaltung $T=\frac{1}{2}\abs{v}^2 = \const
\Rightarrow \abs{v}  = \const$.
\begin{align*}
\Rightarrow &r \sin\alpha = \const\\
\Rightarrow &r = \frac{p_\rho}{\abs{v} \sin\alpha} > \const. 
\end{align*}
Die Bewegung ist auf ein Band eingeschränkt.\bsphere
%TODO: Bild des Bandes
\end{bsp}

\newpage
\section{Spezielle Relativitätstheorie}

Bei der Beschreibung der Elektrodynamik mittels den Maxwell-Gleichungen ergibt
sich ein Problem mit der Klassischen Mechanik, da die Maxwell-Gleichungen nicht
forminvariant unter Galilei-Transformationen sind. Aus der Sicht der
Klassischen Mechanik kann daher die Maxwell Theorie nicht korrekt sein.

Auf der anderen Seite wissen wir aus dem Experiment, dass sich
elektromagnetische Wellen mit Lichtgeschwindigkeit ausbreiten,
\begin{align*}
c = 299792458 \mathrm{m/s}.
\end{align*}
Bei Abstrahlung in Ruhe breiten sie sich kugelförmig aus. Bewegt sich die
Quelle, ist nach der Klassischen Mechanik die Ausbreitung nicht mehr
kugelförmig, es kommt zum Dopplereffekt.
\begin{figure}[!htbp]
\centering
\begin{pspicture}(-0.1,-1.2)(2.3,1.2)
\psline{->}(0.74,-0.49046874)(0.74,0.9895313)
\psline{->}(0.62,-0.37046874)(2.08,-0.37046874)
\pscircle[linecolor=yellow](0.74,-0.39046875){0.24}
\pscircle[linecolor=yellow](0.74,-0.39046875){0.5}
\pscircle[linecolor=yellow](0.74,-0.39046875){0.74}

\rput(0.5878125,1.0795312){\color{gdarkgray}$y$}
\rput(2.0854688,-0.54046875){\color{gdarkgray}$x$}
\end{pspicture}
\begin{pspicture}(-0.1,-1.24)(4.1,1.24)
\psline{->}(0.2378125,-1.0)(0.2378125,0.94)
\psline{->}(0.0778125,-0.88)(3.7178125,-0.88)

\psdots[linecolor=darkblue](2.0378125,0.0)
\psbezier[linecolor=yellow](1.5178125,0.0)(1.5178125,-0.88)(3.0378125,-0.66)(3.0378125,0.0)(3.0378125,0.66)(1.5178125,0.88)(1.5178125,0.0)
\psbezier[linecolor=yellow](1.6978126,0.0)(1.6978126,-0.58)(2.7178125,-0.46)(2.7378125,0.0)(2.7578125,0.46)(1.6978126,0.58)(1.6978126,0.0)
\psbezier[linecolor=yellow](1.2978125,0.0)(1.3178124,-1.22)(3.3378124,-0.78)(3.3378124,0.0)(3.3378124,0.78)(1.2778125,1.22)(1.2978125,0.0)
\psline[linecolor=darkblue]{->}(1.1778125,-0.04)(0.3578125,-0.04)
\psline[linecolor=darkblue]{->}(3.4378126,-0.04)(3.9178126,-0.04)

\rput(0.085625,0.95){\color{gdarkgray}$y$}
\rput(3.6032813,-1.05){\color{gdarkgray}$x$}
\rput(0.7603125,0.23){\color{gdarkgray}$v+c$}
\rput(3.7,0.23){\color{gdarkgray}$v-c$}
\end{pspicture} 
\caption{Wellenausbreitung im ruhenden und bewegten Bezugssystem.}
\end{figure}

Da elektromagnetische Wellen eine endliche Ausbreitungsgeschwindigkeit haben,
müsste so die Lichtgeschwindigkeit vom Bezugssystem abhängen. Insbesondere
würde ein absolutes Ruhesystem existieren, in dem Licht sich mit $c_0$
ausbreitet. Es wurden zahlreiche Experimente durchgeführt, um die
Relativbewegung des Laborsystems zu diesem absoluten Bezugssystem
nachzuweisen, darunter auch das berühmte Michelson-Morley Experiment, welches
schließlich das experimentelle Ergebnis lieferte, dass kein absolutes
Bezugssystem existiert und die Lichtgeschwindigkeit vom Bezugssystem unabhängig
ist. Die Klassische Mechanik bricht also bei Lichtgeschwindigkeit zusammen.
 
Einstein löste diesen Widerspruch, indem er davon ausging, dass sowohl die
Klassische Mechanik als auch die Maxwell Theorie korrekt sind, jedoch eine
experiementell nicht verifizierbare Annahme, die absolute Gleichzeitigkeit
aufgab. In der Klassischen Mechanik gingen wir bisher von einer absoluten
Zeitskala aus, die in allen relativ zueinander bewegten Koordinatensystem verwendet
werden kann. Die Aussage, zwei Ereignisse finden gleichzeitig statt, hat so
absolute Bedeutung. Einstein forderte nun, dass jedem Bezugssystem eine
eigene von anderen Bezugssystem unabhängige Zeitskala zuzuordnen ist.

\begin{bsp}
Geht man von absoluter Gleichzeitigkeit aus, nehmen ruhender und bewegter
Beobachter dieselben Ereignisse als ``gleichzeitig'' wahr (Siehe
Abb. \ref{figure:5:AbsGleichzeitigkeit}).

Dies führt jedoch dazu, dass ruhender und bewegter Beobachter eine
unterschiedliche Lichtgeschwindigkeit wahrnehmen ($v_1\neq v_2$), was
im Wiederspruch zur Konstanz der Lichtgeschwindigkeit steht.

\begin{figure}[H]
\centering
\begin{pspicture}(0.2,-1.4)(5.6,2.035)
%\psgrid
\psline{->}(0.3,-0.495)(4.48,-0.475)
\psline{->}(2.14,-0.575)(2.14,1.305)
\psline[linecolor=yellow](0.32,1.025)(2.12,-0.475)
\psline[linecolor=yellow](3.94,1.005)(2.12,-0.475)
\psline[linecolor=yellow]{<-}(3.44,0.585)(2.12,-0.475)

\psbezier(2.2,-0.595)(2.34,-1.095)(2.94,-0.715)(3.08,-1.095)

\psdots[linecolor=darkblue,dotsize=0.2](2.14,-0.495)
\psline[linecolor=darkblue]{->}(1.16,-1.995)(3.4,1.425)
\psbezier(3.48,1.485)(3.66,1.805)(4.04,1.325)(4.44,1.665)

\psline[dotsep=0.06cm]{<->}(0.84,0.665)(2.82,0.665)
\psline[dotsep=0.06cm]{<->}(2.96,0.665)(3.48,0.665)

\psbezier(3.58,0.625)(3.72,0.125)(4.32,0.505)(4.46,0.125)

\rput(0.76671875,0.175){\small\color{gdarkgray}Licht}
\rput(3.73,-1.285){\small\color{gdarkgray}ruhender Beobachter}
\rput(4.1,1.855){\small\color{gdarkgray}bewegter Beobachter}
\rput(4.3,-0.0050){\small\color{gdarkgray}Ereignis}

\rput(1.8909374,0.855){\small\color{gdarkgray}$v_2$}
\rput(3.2964063,0.855){\small\color{gdarkgray}$v_1$}

\rput(1.9685937,1.415){\small\color{gdarkgray}$t$}
\rput(4.445469,-0.645){\small\color{gdarkgray}$x$}
\end{pspicture} 
\begin{pspicture}(0.2,-1.4)(5.6,2.035)
%\psgrid
\psline{->}(0.3,-0.495)(4.48,-0.475)
\psline{->}(2.14,-0.575)(2.14,1.305)
\psline[linecolor=yellow](0.32,1.025)(2.12,-0.475)
\psline[linecolor=yellow](3.94,1.005)(2.12,-0.475)
\psline[linecolor=yellow]{<-}(3.44,0.585)(2.12,-0.475)

\psbezier(2.2,-0.595)(2.34,-1.095)(2.94,-0.715)(3.08,-1.095)

\psdots[linecolor=darkblue,dotsize=0.2](2.14,-0.495)
\psline[linecolor=darkblue]{->}(1.16,-1.995)(3.4,1.425)
\psbezier(3.48,1.485)(3.66,1.805)(4.04,1.325)(4.44,1.665)

\psline[dotsep=0.06cm]{<->}(2.42,-0.015)(1.94,-0.255)
\psline[dotsep=0.06cm]{<->}(2.94,0.245)(2.52,0.0050)

%\psbezier(3.58,0.625)(3.72,0.125)(4.32,0.505)(4.46,0.125)

\psbezier(3.0325,0.165)(3.2725,-0.195)(3.3325,0.185)(3.6125,-0.095)

%\psbezier(0.5125,1.325)(0.6325,0.865)(0.8925,1.2154762)(0.9725,0.865)
\psbezier(0.86,1.525)(0.9,1.125)(1.26,1.305)(1.28,0.945)

\psline{<->}(0.8325,0.725)(3.5725,0.725)

\rput(0.76671875,0.175){\small\color{gdarkgray}Licht}
\rput(3.73,-1.285){\small\color{gdarkgray}ruhender Beobachter}
\rput(4.1,1.855){\small\color{gdarkgray}bewegter Beobachter}
\rput(4.6,-0.205){\small\color{gdarkgray}Ereignis bew.}

\rput(1.9309375,0.095){\small\color{gdarkgray}$v_2$}
\rput(2.9764063,0.435){\small\color{gdarkgray}$v_1$}

\rput(1.9685937,1.415){\small\color{gdarkgray}$t$}
\rput(4.445469,-0.645){\small\color{gdarkgray}$x$}

\rput(1.2,1.795){\small\color{gdarkgray}Ereignis ruh.}

\end{pspicture} 

\caption{Gleichzeitige Ereignisse in der KM (links) und der SRT (rechts).}
\label{figure:5:AbsGleichzeitigkeit}
\end{figure}

Gibt man die absolute Gleichzeitigkeit auf und fordert dagegen, dass die
Lichtgeschwindigkeit eine universelle konstante für alle Bezugssysteme ist,
nehmen ruhender und bewegter Beobachter unterschiedliche
Ereignisse als ``gleichzeitig'' wahr. Ein bewegter
Beobachter hat eine ``verschobene Idee'' davon, was Gleichzeitigkeit
bedeutet.\bsphere
\end{bsp}

\addtocounter{subsection}{-1}
\subsection{Mathematische Formulierung}
Einstein baut seine ``Spezielle Relativitätstheorie'' auf zwei Axiomen auf.
\begin{enumerate}[label=(\roman{*})]
  \item\label{prop:5:SRT:1} \textit{Spezielles Relativitätsprinzip.}\\
Alle Naturgesetze haben in allen Inertialsystemen die selbe Form.
  \item\label{prop:5:SRT:2} \textit{Konstanz der Lichtgeschwindigkeit.}\\
$c$ ist eine universelle Konstante für alle Inertialsysteme.
\end{enumerate}

Zur mathematischen Formulierung dieser Theorie bietet es sich an, \emph{4-er
Ortsvektoren} einzuführen,
\begin{align*}
\rvec{x}^\mu =
\begin{pmatrix}
ct\\\vec{x}              
\end{pmatrix} = 
\begin{pmatrix}
x^0 \\ x^1 \\ x^2 \\ x^3
\end{pmatrix},\quad \text{d.h. } x_0 = ct,
\end{align*}
die ein Ereignis, welches in einem System $K$ stattfindet, durch Angabe von
Zeit und Ort des Geschehens charakterisieren. Indem man die Zeit $t$ mit
der Lichtgeschwindigkeit multipliziert, haben alle Komponenten des 4-er Vektors
die Dimension einer Länge. Die 4-er Vektoren sind Elemente des
\emph{Minkowski-Raums}, ein vierdimensionaler Vektorraum, der nicht mehr mit
der euklidischen Metrik versehen ist. Wir werden uns später damit ausführlicher
befassen.

\begin{bemn}[Bemerkung zur Notation.]
Für einen 4-er Vektor verwenden wir das Symbol $\rvec{x}^\mu$ bzw.
$\rvec{x}$. Die $\mu$-te Komponente des 4-er Vektors bezeichnen wir mit
$x^\mu$, wobei $\mu=0,\ldots,3$.

Die \emph{Einstein'sche Summennotation} erlaubt eine kompakte Schreibweise für
Summen,
\begin{align*}
&x_\mu x^\mu\equiv\sum_\mu x_\mu x^\mu,\\
&A_{\mu\nu} x^\mu x^\nu\equiv\sum_{\mu,\nu} A_{\mu\nu} x^\mu x^\nu. 
\end{align*}
Wann immer der gleiche Index oben ${}^\mu$ und unten ${}_\mu$ in einem Ausdruck
auftritt, wird darüber summiert.\maphere
\end{bemn}
Es lassen sich nun Transformationen definieren, die ein Inertialsystem
bezüglich dieser Darstellung in ein anderes überführen. Homogenität und
Isotropie des Raumes verlangen, dass diese Transformationen affine Abbildungen
sind
\begin{align*}
\rvec{x}' = \Lambda \rvec{x} + \rvec{a},
\end{align*}
wobei $\Lambda$ eine $4\times 4$-Matrix und $\rvec{a}$ einen konstanten 4-er
Vektor beschreibt.

Zur Beschreibung dieser Transformationen genügt es, sich zunächst auf einen
Boost eines Inertialsystems $K'$, dessen Ursprung sich mit konstanter
Geschwindigkeit entlang der $x$-Achse relativ zu $K$ bewegt, zu beschränken
($\rvec{a}=0$). Die Transformation hat daher die Form,
\begin{align*}
&\begin{rcases}
{x^0}' = a(v)x^0 + b(v)x^1\\
{x^1}' = d(v)x^1 + e(v)x^0\\
{x^2}' = x^2\\
{x^3}' = x^3
\end{rcases}
\Rightarrow\Lambda =
\begin{pmatrix}
a & b & 0 & 0\\
e & d & 0 & 0\\
0 & 0 & 1 & 0\\
0 & 0 & 0 & 1
\end{pmatrix}.
\end{align*}
Homogenität und Isotropie des Raumes erfordern, dass die Koeffizienten
lediglich von $v$ abhängen können. Kehren wir die Richtung der $x$-Achse in $K$
und $K'$ sowie $v$ um,
\begin{align*}
&{x^0}' = a(-v)x^0 - b(-v)x^1,\\
&-{x^1}' = -d(-v)x^1 + e(-v)x^0,\\
&{x^2}' = x^2,\\
&{x^3}' = x^3.
\end{align*}
 erzwingt die Isotropie, dass sich die
Transformationsformel nicht ändert, d.h.
\begin{align*}
&a(-v) = a(v),\qquad b(v) = -b(-v),\\
&d(v) = d(-v),\qquad e(-v) = -e(v).
\end{align*}
Betrachten wir nun den Nullvektor im $K'$ zum Zeitpunkt $t$, ergibt sich
\begin{align*}
{x^1}' = 0 = d(v)vt + e(v)ct \Rightarrow \frac{e(v)}{d(v)} = -\frac{v}{c} \equiv
-\beta.
\end{align*}
Wir sehen, dass die Koeffizienten der Matrix $\Lambda$ bereits durch die Art
der Bewegung eingeschränkt werden.

Da es unabhängig ist, ob sich $K'$ mit der Geschwindigkeit $v$ zu $K$ oder $K$
sich mit der Geschwindigkeit $-v$ zu $K'$ bewegt, hat die Rücktransformation
die Form,
\begin{align*}
&x^0 = a(-v){x^0}' + b(-v){x^1}' = a(v){x^0}' - b(v){x^1}'\\
&x^1 = d(-v){x^1}' + e(-v){x^0}' = d(v){x^1}' - e(v){x^0}'\\
&x^2 = {x^2}'\\
&x^3 = {x^3}'\\
\Rightarrow\;&\Lambda^{-1} =
\begin{pmatrix}
a & -b & 0 & 0\\
-e & d & 0 & 0\\
0 & 0 & 1 & 0\\
0 & 0 & 0 & 1
\end{pmatrix}.
\end{align*}
$\Lambda$ und $\Lambda^{-1}$ sind invers zueinander, d.h.
$\Lambda^{-1} \Lambda = \Id$,
woraus sich folgendes Gleichungssystem ergibt,
\begin{align*}
&a^2 -be = 1, && -ab + bd = 0,\\
&-ae + de = 0, && -be +d^2 = 1.
\end{align*}
Wählen wir $\kappa\in\R$ als freien Parameter, so besitzt das System eine
eindeutige Lösung. Setzen wir
\begin{align*}
b = -a\beta \kappa^2,
\end{align*}
ergibt sich für die Transformationen die Form
\begin{align*}
&{x^0}' = \gamma x^0 - \gamma\beta\kappa^2x^1\\
&{x^1}' = -\gamma\beta\kappa^2 x^0 + \gamma x^1\\
&{x^2}' = x^2\\
&{x^3}' = x^3
\end{align*}
wobei $\beta = \dfrac{v}{c}$, $\gamma=\dfrac{1}{\sqrt{1-\kappa^2\beta^2}}$.
\begin{bemn}
Für $\kappa = 0$ ist $\gamma=1$ und wir erhalten die
Galilei-Transformation für einen boost.

Prinzip \ref{prop:5:SRT:1} kann also auch auf die Gallilei-Transformationen
führen. Diese sind jedoch nicht mit \ref{prop:5:SRT:2} verträglich. Wählt man
$\kappa = 1$, so sind \ref{prop:5:SRT:1} und \ref{prop:5:SRT:2} erfüllt. Die
Größe $\frac{c}{\kappa}$ spielt die Rolle einer maximalen
Signalgeschwindigkeit. Für $\kappa=0$ werden alle Wechselwirkungen instantan,
für $\kappa=1$ ist die Ausbreitungsgeschwindigkeit die
Lichtgeschwindigkeit.\maphere
\end{bemn}

Wir erhalten für $\kappa=1$ die \emph{Lorentz-Transformationen}, welche die
Lichtgeschwindigkeit bei Transformation zwischen Inertialsystemen invariant
lassen. Die Lorenz-Transformation für einen Boost in $x$-Richtung ist gegeben
durch,
\begin{align*}
\begin{rcases}
{x^0}' = \gamma x^0 - \gamma\beta x^1\\
{x^1}' = -\gamma\beta x^0 + \gamma x^1\\
{x^2}' = x^2\\
{x^3}' = x^3
\end{rcases}
\Rightarrow \Lambda = 
\begin{pmatrix}
\gamma & -\gamma\beta & 0 & 0\\
-\gamma\beta & \gamma & 0 & 0\\
0 & 0 & 1 & 0\\
0 & 0 & 0 & 1
\end{pmatrix}.
\end{align*}
\begin{figure}[!htbp]
\begin{pspicture}(-0.1,-1.151875)(3.0834374,1.151875)

\psline[linecolor=darkblue]{->}(0.3,-0.85)(1.0078125,-0.44703126)

\psline{->}(0.3078125,-0.971875)(0.3078125,0.968125)
\psline{->}(0.1478125,-0.851875)(3.0078125,-0.851875)
\psline{->}(1.0278125,-0.451875)(1.4878125,0.928125)
\psline{->}(0.9478125,-0.391875)(2.4878125,-0.151875)

\rput(0.11203125,0.998125){\color{gdarkgray}$ct$}
\rput(2.9532812,-1.001875){\color{gdarkgray}$x$}
\rput(1.2365625,0.978125){\color{gdarkgray}$ct'$}
\rput(2.5703125,-0.321875){\color{gdarkgray}$x'$}

\rput(1.5,-0.65703124){\color{gdarkgray}$(ct,vt)$}
\end{pspicture} 
\caption{Lorentztransformation im Minkowski-Raum}
\end{figure}

\begin{bemn}
Die Lorentz-Transformation lässt die Größe
\begin{align*}
(x^0)^2 - \vec{x}^2 = ct^2-(x^1)^2-(x^2)^2-(x^3)^2 = g_{\mu\nu}x^\nu x^\mu
\end{align*} 
unverändert. $g_{\mu\nu}$ bezeichnet hier den metrischen Tensor,
\begin{align*}
(g_{\mu\nu}) =
\begin{pmatrix}
1 & 0& 0 & 0\\
0 & -1 & 0 & 0\\
0 & 0 & -1 & 0\\
0 & 0 & 0 & -1
\end{pmatrix}.
\end{align*}
\begin{proof}
Wir betrachten nur den Spezialfall des $x$-Richtung Boost, die übrigen folgen
analog.
\begin{align*}
(x'^0)^2 - \vec{x}^{'2} &=
(\gamma x^0 - \gamma\beta x^2)^2 - (-\gamma\beta x^0 + \gamma x^1) - (x^2)^2 +
(x^3)^2 \\
&= \gamma^2(x^0)^2 + \gamma^2\beta^2(x^1)^2 - \gamma^2\beta^2(x^0)^2 -
\gamma^2(x^1)^2 - (x^2)^2 - (x^3)^2\\
&= (x^0)^2\underbrace{\gamma^2(1-\beta^2)}_{=1} +
(x^1)^2\underbrace{\gamma^2(\beta^2)-1}_{=-1} - (x^2)^2 - (x^3)^2\\
&= (x^0)^2 - (x^1)^2 - (x^2)^2 - (x^3)^2\qedhere\maphere
\end{align*}
\end{proof}
\end{bemn}

\subsubsection{Die Lichtgeschwindigkeit als universelle Konstante}

Wir wollen nun zeigen, dass die Lichtgeschwindigkeit $c$ unabhängig vom
Inertialsystem ist.
\begin{proof}
Bewege sich ein Teilchen in einem Ineratialsystem $K$ konstant mit
Geschwindigkeit $c$ in Richtung $\vec{n}$ ($\vec{n}$ Einheitsvektor). Die
Bewegungsgleichung ist dann gegeben durch,
\begin{align*}
&\vec{x}(t) = ct\vec{n},\qquad \abs{\dvec{x}} = c.
\end{align*} 
Transformieren wir nun in Ineratialsystem $K'$, das sich relativ zu $K$ mit
Geschwindigkeit $v$ bewegt, so erhalten wir die Gleichungen
\begin{align*}
&ct' = \gamma (x^0) -\gamma\beta (x^1) = \gamma ct - \gamma\beta ctn^1 = \gamma
ct(1-\beta n^1),\\
&(x^1)' = -\gamma\beta (x^0) + \gamma (x^1) = -\gamma\beta ct + \gamma ct n^1 =
\gamma ct(n^1-\beta) = ct'\frac{n^1-\beta}{1-\beta n^1},\\
&(x^2)' = (x^2) = ctn^2 = \frac{1}{\gamma}\frac{ct'}{1-\beta n^1}n^2,\\
&(x^3)' = (x^3) = ctn^3 = \frac{1}{\gamma}\frac{ct'}{1-\beta n^1}n^3.
\end{align*}
Um die Geschwindigkeit des Teilchens in $K'$ zu erhalten, müssen wir die
Bewegungsgleichung in $K'$ aufstellen,
\begin{align*}
\vec{x}'(t') = ct'
\begin{pmatrix}
\frac{n^1-\beta}{1-\beta n^1}\\
\frac{1}{\gamma}\frac{n^2}{1-\beta n^1}\\
\frac{1}{\gamma}\frac{n^3}{1-\beta n^1}
\end{pmatrix}
= ct'\vec{n}'
\end{align*}
Nun ist $\abs{\dvec{x}'} = c\abs{\vec{n}'}$. Wir müssen also zeigen $\abs{n}' =
1$. Dazu betrachten wir den Spezialfall
\begin{align*}
\vec{n} = 
\begin{pmatrix}
1 \\ 0\\ 0
\end{pmatrix}
\Rightarrow
\vec{n}' =
\begin{pmatrix}
\frac{1-\beta}{1-\beta}\\
\frac{1}{\gamma}0\\
\frac{1}{\gamma}0
\end{pmatrix}
= \begin{pmatrix}
  1 \\ 0\\ 0
  \end{pmatrix}.\qedhere
\end{align*}
\end{proof}

% Gilt
% \begin{align*}
% &(x^0)' = \gamma x^0 - \gamma\beta x^1 = \const,\\
% \Leftrightarrow & ct - \beta x^1 = \const,
% \end{align*}
% so treten die Ereignisse gleichzeitig ein. Ist
% \begin{align*}
% &(x^1)' = -\gamma\beta x^0 + \gamma x^1 = \const,\\
% \Leftrightarrow & ct = \const + \beta x^1,
% \end{align*}
% so finden die Ereignisse am gleichen Ort statt. 

\subsection{Einfache Folgerungen aus der SRT}

Die Lorenztransformationen zeigen, dass es nicht möglich ist, sich schneller
als das Licht zu bewegen.

\begin{defnn}
Sei $\rvec{x}^\nu$ ein $4$-er Vektor. Wir bezeichnen Ergeignisse abhängig vom
Skalarpodukt als
\begin{align*}
&g_{\mu\nu}x^\mu x^\nu > 0 \qquad: \text{\emph{zeitartig}},\\
&g_{\mu\nu}x^\mu x^\nu = 0 \qquad: \text{\emph{lichtartig}},\\
&g_{\mu\nu}x^\mu x^\nu < 0 \qquad: \text{\emph{raumartig}}.\fishhere
\end{align*}
\end{defnn}

\begin{figure}[!htbp]
  \centering
\begin{pspicture}(0,-2.2939062)(4.8934374,2.2939062)
\psline[linecolor=yellow](0.0,-2.1560917)(3.9798,1.8239063)
\psline[linecolor=yellow](3.9798,-2.1560917)(0.0,1.8239063)
\psline{->}(0.0,-0.15609375)(4.0,-0.15609375)
\psline{->}(1.98,-2.2160938)(1.98,1.9239062)

\psbezier{->}(3.9,-1.2760937)(3.46,-1.6960938)(3.68,-0.99609375)(3.24,-1.2560937)

\psdots[dotstyle=x](3.26,-0.57609373)
\psdots[dotstyle=x](3.12,-2.0360937)
\psdots[linecolor=darkblue,dotsize=0.2](1.98,-0.15609375)

%\psbezier[linecolor=darkblue]{<-}(2.74,1.9439063)(3.2,1.4039062)(3.0,0.86390626)(2.46,0.80390626)(1.92,0.74390626)(2.2457285,0.2946278)(1.98,-0.17609376)(1.7142714,-0.6468153)(1.94,-1.1760937)(1.48,-1.2560937)(1.02,-1.3360938)(1.12,-1.6760937)(1.74,-2.0160937)

\psbezier[linecolor=darkblue]{<-}(2.745889,1.92)(2.9258888,1.22)(2.705889,1.08)(2.46,0.8058889)(2.214111,0.5317778)(2.2457285,0.29661044)(1.98,-0.17411113)(1.7142715,-0.6448327)(1.9058889,-0.82)(1.4858888,-1.28)(1.0658889,-1.74)(0.4858889,-1.84)(0.44588888,-2.22)

\rput(3.665625,0.25390625){\small\color{gdarkgray}raumartig}

\rput(0.923125,1.7476562){\small\color{gdarkgray}zeitartig}

\rput(4.231406,-1.0660938){\small\color{gdarkgray}lichtartig}

\rput(1.6065625,-0.00609375){\small\color{gdarkgray}A}

\rput(3.4871874,-0.66609377){\small\color{gdarkgray}B}

\rput(3.3271875,-2.1260939){\small\color{gdarkgray}C}

\rput(3.2885938,2.1139061){\small\color{gdarkgray}Teilchen mit $v<c$}
\end{pspicture}

\caption{Lichtkegel mit ruhendem Punkt $A$, raumartigen Punkt $B$ und
zeitartigem Punkt $C$.}
\end{figure}

Als Beobachter im Punkt $A$ zur Zeit $t=0$ können wir nur zeitartige
Ereignisse, die sich im unteren Lichtkegel befinden, wahrnehmen bzw. von ihnen
beeinflusst werden. Zeitartige Ereignisse, die sich im oberen Lichtkegel
befinden, können beeinflusst werden. Sind zwei Ereignisse raumartig getrennt,
können sie sich gegenseitig weder wahrnehmen noch beeinflussen.
% Lichtartige
%Ereignisse können von Beobachtern, die sich mit Lichtgeschwindigkeit bewegen,
%wahrgenommen bzw. beeinflusst werden.

Gilt in einem Bezugssystem für zwei zeitartige Ereignisse $A$ und $C$, dass $C$
vor $A$ stattfindet, d.h. $x_A^0 > x_C^0$, so gilt dies in jedem Bezugssystem.
Um dies einzusehen betrachte $\rvec{x}^\mu = \rvec{x}_A^\mu - \rvec{x}_C^\mu$,
% =\begin{pmatrix} x_A^0-x_C^0\\\vec{x}_A -\vec{x}_C\end{pmatrix}$, 
wobei $\rvec{x}_\mu\rvec{x}^\mu  = (x_A^0-x_C^0)^2-(\vec{x}_A-\vec{x}_C)^2 > 0$.
Angenommen es gibt ein Bezugssystem mit $x_A^0=x_C^0$, dann ist
$\rvec{x}_\mu\rvec{x}^\mu < 0$, das Skalarprodukt ist jedoch invariant unter
Lorentz-Transformationen, d.h. es kann kein solches Bezugssystem geben.
Es gibt also eine absolute Vergangenheit und Zukunft für zeitartige Ereignisse.
Für raumartige Ereignisse ist die zeitliche Folge jedoch relativ und nicht
absolut. Raumartig getrennte Ereignisse können also nicht kausal verknüpft sein.

Die Lichtgeschwindigkeit $c$ ist daher die maximale Geschwindigkeit für
Informationen und Teilchen.

\subsubsection{Relativität der Gleichzeitigkeit}

Für einen ruhenden Beobachter finden zwei Ereignisse $A$ und $B$ gleichzeitig
statt, wenn
\begin{align*}
x_A^0 = x_B^0.
\end{align*}
Für einen gleichförmig mit der Geschwindigkeit $v$ bewegten Beobachter finden
sie gleichzeitig statt, wenn
\begin{align*}
{x_A^0}' = {x_B^{0}}' = c\tau.
\end{align*}
Für den ruhenden Beobachter entspricht dies den Ereignissen
\begin{align*}
&c\tau = \gamma x^0 - \beta\gamma x^1\\
&x^0 = \frac{c\tau + \beta\gamma x^1}{\gamma} = \frac{c\tau}{\gamma} +
\beta x^1,
\end{align*}
die auf einer Geraden mit Steigung $\beta$ liegen.

\begin{figure}[!htbp]
  \centering
\begin{pspicture}(0.8,-1.011875)(4.5,2.131875)
\psarc(1.39,-0.641875){1.39}{69.17911}{88.60282}
\rput{-90.0}(2.111875,0.708125){\psarc(1.41,-0.701875){1.39}{91.39718}{110.82089}}
\psline[linecolor=darkblue]{->}(1.42,-0.671875)(2.24,1.668125)
\psline[linecolor=darkblue]{->}(1.42,-0.671875)(3.76,0.148125)
\psline{->}(1.42,-0.831875)(1.42,1.668125)
\psline{->}(1.25,-0.661875)(3.75,-0.661875)

\rput(1.593125,0.538125){\color{gdarkgray}$\alpha$}
\rput(2.553125,-0.441875){\color{gdarkgray}$\alpha$}
\rput(1.22125,1.798125){\color{gdarkgray}$x^0$}
\rput(3.925625,-0.681875){\color{gdarkgray}$x^1$}
\rput(4.0925,0.278125){\color{gdarkgray}${x^1}'$}
\rput(2.4725,1.958125){\color{gdarkgray}${x^0}'$}
\rput(2.011875,1.578125){\color{gdarkgray}$K'$}
\rput(1.2685938,-0.861875){\color{gdarkgray}$K$}

\rput(2.6265626,0.458125){\color{gdarkgray}$A$}
\rput(3.5471876,0.858125){\color{gdarkgray}$B$}
\rput(2.9471874,1.058125){\color{gdarkgray}$C$}

\psline[linecolor=yellow](1.42,0.048125)(3.96,1.328125)
\psdots[dotstyle=x](2.64,0.668125)
\psdots[dotstyle=x](3.36,1.028125)
\psdots[dotstyle=x](3.0,0.848125)

\end{pspicture} 

\caption{Ereignisse $A$ und $B$ mit Detektor $C$.}
\end{figure}

Die allgemeine Definition von Gleichzeitigkeit lautet daher,
\begin{defnn}
Zwei Ereignisse $A$ und $B$ heißen \emph{gleichzeitig}, wenn von $A$ und $B$
zur gleichen Zeit ausgesendtes Licht bei einem in der Mitte ruhenden Detektor
zur gleichen Zeit eintrifft.\fishhere
\end{defnn}

\subsubsection{Längenkontraktion}

Betrachte einen Stab mit der Länge $L_0$, der im Bezugsystem $K'$ ruht. Das
Bezugsystem $K'$ bewege sich relativ zu $K$ mit der Geschwindigkeit $v$.

\begin{figure}[!htbp]
  \centering
\begin{pspicture}(-0.1,-2.1)(5.625,2.061875)
\psline{->}(0.3553125,-1.841875)(0.3553125,0.658125)
\psline{->}(0.1853125,-1.671875)(2.6853125,-1.671875)
\psline{->}(2.8353126,-0.741875)(2.8353126,1.758125)
\psline{->}(2.6653125,-0.571875)(5.1653123,-0.571875)

\rput(0.1565625,0.788125){\color{gdarkgray}$x^0$}
\rput(2.8609376,-1.691875){\color{gdarkgray}$x^1$}
\rput(0.20390625,-1.871875){\color{gdarkgray}$K$}

\rput(2.6678126,1.888125){\color{gdarkgray}${x^0}'$}
\rput(5.3878126,-0.591875){\color{gdarkgray}${x^1}'$}
\rput(2.6271875,-0.771875){\color{gdarkgray}$K'$}

\psline(0.8753125,-1.681875)(1.3953125,-0.581875)
\psline(1.7753125,-1.681875)(2.2953124,-0.581875)

\psframe[linecolor=darkblue,fillstyle=solid,fillcolor=darkblue](4.9,-0.521875)(3.1553125,-0.601875)
\psframe[linecolor=darkblue,fillstyle=solid,fillcolor=darkblue](2.1153126,-0.901875)(1.2353125,-0.981875)
\rput(0.8609375,-1.891875){\color{gdarkgray}$x_L^1$}
\rput(1.7809376,-1.911875){\color{gdarkgray}$x_R^1$}
\rput(3.1809375,-0.891875){\color{gdarkgray}${x_L^1}'$}
\rput(4.85,-0.891875){\color{gdarkgray}${x_R^1}'$}

\psline{->}(3.2153125,1.298125)(4.0153127,1.298125)

\rput(3.5620313,1.488125){\color{gdarkgray}$v$}
\end{pspicture} 

\caption{Längenkontraktion eines Stabes.}
\end{figure}

Um die Länge $L=\left(x_R^1- x_L^1\right)$ in $K$ zu messen, müssen linkes und
rechtes Ende zur gleichen Zeit lokalisiert werden ($x_R^0-x_L^0=0$),
\begin{align*}
L_0 &= {x_R^1}'-{x_L^1}' = 
\gamma \left(x_R^1-\beta x_R^0 \right)
- \gamma \left(x_L^1-\beta x_L^0 \right)
= \gamma\left(x_R^1- x_L^1\right)=\gamma L\\
\Rightarrow L &=\sqrt{1-\frac{v^2}{c^2}}L_0 < L_0.
\end{align*}
Die Länge des Stabes ist also in jedem relativ zum Ruhesystem bewegten System
\textit{kleiner}. Diesen Effekt nennt man \emph{Lorentzkontraktion}.

\subsubsection{Zeitdilatation}

Betrachte eine Uhr, die im Koordinatenursprung von $K$ ruht. Ein Zeitintervall
$\tau=t_A-t_B$, das wir im Ruhesystem $K$ ablesen, heißt \emph{Eigenzeit}. Lesen
wir nun zusätzlich die Zeit in einem Bezugssystem $K'$ ab, das sich relativ zu $K$
mit der Geschwindigkeit $v$ bewegt.

\begin{figure}[H]
  \centering
\begin{pspicture}(-0.1,-1.6)(3.18375,1.5017188)
\psline{->}(0.4878125,-1.3017187)(0.4878125,1.1982813)
\psline{->}(0.3178125,-1.1317188)(2.8178124,-1.1317188)

\psline[linecolor=yellow]{->}(0.4878125,-1.1417187)(2.0078125,0.93828124)

\psline(1.6078125,0.37828124)(1.6078125,-1.1217188)
\psline(1.6278125,0.39828125)(0.4878125,0.39828125)

\psdots[linecolor=darkblue](0.4878125,-1.1417187)
\psdots[linecolor=darkblue](1.6078125,0.37828124)

\rput(0.20203125,0.40828124){\color{gdarkgray}$c\Delta t$}
\rput(0.2890625,1.3282813){\color{gdarkgray}$x^0$}

\rput(2.9934375,-1.1517187){\color{gdarkgray}$x^1$}
\rput(1.625625,-1.3517188){\color{gdarkgray}$x_B^1$}

\rput(2.3,0.96828127){\color{gdarkgray}${x^0}'$}
\rput(0.334375,-1.3317188){\color{gdarkgray}$A$}
\rput(1.855,0.22828124){\color{gdarkgray}$B$}

\end{pspicture} 

\caption{Zeitdilatation einer bewegten Uhr.}
\end{figure}

Mit den bekannten Transformationsformeln ergibt sich in $K'$,
\begin{align*}
&{x_A^0}'-{x_B^0}' = \gamma \left({x_A^0} + \beta {x_A^1}\right) -
\gamma\left({x_B^0} - \beta {x_B^1}\right) =
\gamma\left({x_B^0}-{x_A^0}\right)\\
\Rightarrow &\tau' =  \gamma \tau > \tau.
\end{align*}
Von $K'$ aus betrachtet vergeht,
bis auf der Uhr das Zeitintervall $\tau$ verstrichen ist, in $K'$ tatsächlich \textit{mehr Zeit}.
Eine bewegte Uhr geht also um den konstanten Faktor
\begin{align*}
\frac{1}{\sqrt{1 - \frac{v^2}{c^2}}}
\end{align*}
``langsamer'' als in ihrem Ruhesystem.

\begin{bemn}
Das Differential der Eigenzeit ist
\begin{align*}
\dtau = \frac{1}{\gamma}\dt = \sqrt{1-\frac{v^2}{c^2}}\dt.
\end{align*}
Messen wir im Laborsystem die Zeitdifferenz $t_2-t_1$, erhalten wir für 
eine allgemeine Bewegung die Zeitdifferenz der bewegten Uhr in ihrem
Ruhesystem
\begin{align*}
\tau_2 - \tau_1 = \int\limits_{t_1}^{t_2} \dt\sqrt{1-\frac{v^2}{c^2}}.\maphere
\end{align*}
\end{bemn}

\begin{bsp}
\textit{Die Lichtuhr}. Zwischen zwei in $K'$ ruhenenden Spiegeln mit Abstand $l$
wird ein Lichtstrahl hin- und hergesandt. Immer wenn der Lichtstrahl auf einem
der Spiegel registriert wird, wird die Lichtuhr um eine Zeiteinheit weitergestellt.

\begin{figure}[!htbp]
  \centering
\begin{pspicture}(0,-0.643)(4.279568,0.622)
\psline(0.0,-0.198)(1.7805682,-0.198)
\psline(0.0,0.603)(1.7805682,0.603)

\psline(2.48,-0.198)(4.260568,-0.198)
\psline(2.48,0.603)(4.260568,0.603)

\psline[linecolor=yellow]{<->}(0.9,-0.078)(0.9,0.522)
\psline[linecolor=yellow]{<-}(3.4,-0.138)(2.88,0.542)
\psline[linecolor=yellow]{->}(3.36,-0.11)(3.88,0.562)

\psline(3.38,-0.138)(3.38,-0.318)
\psline(2.88,-0.138)(2.88,-0.318)

\rput(0.1009375,0.232){\color{gdarkgray}$l$}
\rput(2.5809374,0.232){\color{gdarkgray}$l$}
\rput(3.10875,-0.488){\color{gdarkgray}$v\Delta t$}
\end{pspicture} 

\caption{Lichtuhr in ruhendem (links) und bewegtem (link) Zustand.}
\end{figure}

Im Ruhesystem $K'$ benötigt der Lichtstrahl, um von einem Spiegel zum anderen zu
gelangen, die Zeit $\Delta\tau' = l/c$. Nun bewege sich $K'$ relativ zu $K$ mit
der Geschwindigkeit $v$ entlang der Spiegelachse. Aus der Sicht von $K$ hat das
Licht zwischen zwei Spiegeln eine größere Distanz zurückzulegen, die
Lichtgeschwindigkeit ist aber in allen Inertialsystem gleich, d.h. in $K$
vergeht zwischen zwei Spiegeln die Zeit
\begin{align*}
&\Delta\tau =  \frac{\sqrt{l^2 + v^2\Delta\tau^2}}{c}\\
&\Delta\tau^2 = \frac{l^2 + v^2\Delta\tau^2}{c^2} = {\Delta\tau'}^2 +
\frac{v^2}{c^2}\Delta\tau^2\\
&\Delta\tau = \frac{1}{\sqrt{1-\frac{v^2}{c^2}}}\Delta\tau' > \Delta\tau'.
\end{align*}
Liest man in $K$ ab, geht die in $K'$ ruhende Uhr langsamer, als eine in $K$
ruhende Uhr, da mehr Zeit vergeht, bis die Lichtuhr um ein Zeitintervall
weitergestellt wird.

Andersherum würde ein in $K'$ ruhender Beobachter aufgrund der selben
Überlegung sehen, dass eine zu ihm bewegte Lichtuhr in $K$ langsamer geht, als
seine in $K'$ ruhende. Dies ist die Symmetrie der Zeitdilatation, jeder misst, dass die
Uhr des anderen langsamer geht.\bsphere
\end{bsp}

\begin{bemn}
Betrachten wir die Trajektorie eines bewegten Teilchens, so muss diese so
geformt sein, dass sie sich in jedem Punkt innerhalb des von dort ausgehenden
Lichtkegels befindet.
\begin{figure}[!htbp]
  \centering
\begin{pspicture}(0,-1.2941111)(4.034111,1.3)
\psline[linecolor=yellow](2.005889,-1.08)(3.9998,0.9058889)
\psline[linecolor=yellow](1.9858888,-1.08)(0.0,0.9058889)
\psline{->}(0.02,-1.0741111)(4.02,-1.0741111)
\psline{->}(1.9858888,-1.28)(2.0,1.0058888)

\psline(2.5647683,0.2)(2.3458889,0.52)
\psline(2.5816052,0.2)(2.785889,0.52)
\psline(1.1647682,0.44)(0.9458889,0.76)
\psline(1.1816051,0.44)(1.3858889,0.76)

\psbezier[linecolor=darkblue]{->}(1.9858888,-1.08)(2.3858888,-0.48)(2.185889,-0.36)(2.485889,0.04)(2.785889,0.44)(2.485889,0.92)(2.8858888,1.28)
\psbezier[linecolor=darkblue](1.5058889,-0.28)(1.3658888,0.28)(1.2858889,-0.14)(1.1658889,0.44)(1.0458889,1.02)(1.4858888,0.2)(1.745889,0.82)

\psellipse(2.5647683,0.53103274)(0.22600006,0.06896515)
\psellipse(1.1647683,0.7710327)(0.22399993,0.06896515)

\psdots(2.5698888,0.2)
\psdots(1.1738889,0.44)
\end{pspicture} 
\caption{Trajektore innerhalb und außerhalb der zulässigen Geschwindigkeit.}
\end{figure}
\end{bemn}

\begin{bsp}
\textit{Der Myon-Zerfall.}
In ca. 10km Höhe werden aufgrund der Wechselwirkung von Protonen aus dem
Weltraum mit unserer Atmosphäre Myonen gebildet. Sie haben eine sehr hohe
Geschwindigkeit $v\sim 0.998c\Rightarrow\gamma\approx 16$. Die Teilchen sind
nicht stabil und haben eine Lebensdauer von wenigen $\mathrm{\mu s}$. Mögliche
Myon-Zerfälle sind,
\begin{align*}
&\mu^+ \longrightarrow e^+ + \nu_e  + \overline{\nu}_\mu,\\
&\mu^- \longrightarrow e^- + \nu_\mu  + \overline{\nu}_e.
\end{align*}
Im Ruhesystem der Myonen kann der Zerfall beschrieben werden durch,
\begin{align*}
N(t) = N_0 e^{-\frac{t}{\tau}},\qquad \tau = 2.2\mathrm{\mu s}\quad
\text{mittlere Lebensdauer}.
\end{align*}

Rechnet man klassisch, benötigen die Myonen bis zum Erreichen der Erde die Zeit,
\begin{align*}
t_{\text{Flug}} = \frac{10\mathrm{km}}{0.998c} \approx 30 \mathrm{\mu s}.
\end{align*}
Es sind also kaum Myonen auf der Erde zu erwarten, man kann jedoch eine sehr
große Anzahl von eintreffenden Myonen messen.

Aufgrund der hohen Geschwindigkeit der Myonen kommen relativistische Effekte
zu tragen. Wir müssen die Eigenzeit der Myonen betrachten,
\begin{align*}
\tau_{\text{Flug}} = \frac{1}{\gamma}t_{\text{Flug}} \approx
\frac{1}{16}30\mathrm{\mu s} = 1.9\mathrm{\mu s}.
\end{align*}
Dies ist einer von vielen experimentellen Befunden, der die Korrektheit der SRT
bestätigt.\bsphere
\end{bsp}

\subsubsection{Zwillingsparadoxon}

Betrachtet man die Zwillingsbrüder $A$ und $B$. $A$ befinde sich in Ruhe,
während $B$ eine längere Reise mit $v\approx c$ macht. Trifft $B$ wieder auf der
Erde ein, ist er viel jünger als $A$.
\begin{figure}[H]
  \centering
\begin{pspicture}(0,-1.3717188)(4.2418265,1.3717188)
\psline[linecolor=yellow](2.005889,-1.0317187)(3.9998,0.95417017)
\psline[linecolor=yellow](1.9858888,-1.0317187)(0.0,0.95417017)
\psline{->}(0.02,-1.0258299)(4.02,-1.0258299)
\psline{->}(1.9858888,-1.2317188)(1.9858888,1.0541701)
\psbezier[linecolor=darkblue](1.9858888,-1.0117188)(2.305889,-0.31171876)(2.305889,-0.07171875)(1.9858888,0.50828123)
\psdots(1.9858888,0.48828125)
\psdots(1.9858888,-1.0117188)

\rput(2.2530763,0.49828124){\color{gdarkgray}$B$}
\rput(1.8124514,-1.2217188){\color{gdarkgray}$A$}
\rput(1.8671389,1.1982813){\color{gdarkgray}$x^0$}
\rput(4.0515137,-1.2017188){\color{gdarkgray}$x^1$}
\end{pspicture} 
\caption{Reise von $B$ im Minkowski-Raum.}
\end{figure}
Man kann dies mit der Zeitdilatation erklären,
\begin{align*}
\tau_2 - \tau_1 = \int\limits_{t_1}^{t_2}
\underbrace{\sqrt{1-\frac{v^2}{c^2}}}_{<1}\dt < t_2-t_1.
\end{align*}
Die für $B$ vergangene Zeit ist kleiner als die für $A$ vergangene.
\begin{bsp}
Betrachten wir den exotischen Fall der instantanen Beschleunigung.

\begin{figure}[!htbp]
  \centering
\begin{pspicture}(0,-1.3717188)(4.2418265,1.3717188)
\psline[linecolor=yellow](2.005889,-1.0317187)(3.9998,0.95417017)
\psline[linecolor=yellow](1.9858888,-1.0317187)(0.0,0.95417017)
\psline{->}(0.02,-1.0258299)(4.02,-1.0258299)
\psline{->}(1.9858888,-1.2317188)(1.9858888,1.0541701)
\psline[linecolor=darkblue](1.9858888,-1.0117188)(2.4058888,-0.15171875)(1.9858888,0.48828125)
\psdots(1.9858888,0.48828125)
\psdots(1.9858888,-1.0117188)

\rput(2.2530763,0.49828124){\color{gdarkgray}$B$}
\rput(1.8124514,-1.2217188){\color{gdarkgray}$A$}
\rput(1.8671389,1.1982813){\color{gdarkgray}$x^0$}
\rput(4.0515137,-1.2017188){\color{gdarkgray}$x^1$}
\end{pspicture}
\caption{Reise mit instantaner Beschleunigung.}
\end{figure}

Sei $v=0.8c$ und $t_2-t_1=10\text{ Jahre}$, so ergibt sich für die
Zeitdilatation,
\begin{align*}
\tau_2 -\tau_1 = 6\text{ Jahre}.
\end{align*}
Dies scheint paradox, da die Situation auf den ersten Blick Symmetrisch
erscheint, d.h. aus der Sicht von $B$ müsste ebenfalls weniger Zeit vergangen sein, als aus der
Sicht von $A$. Die Situation kann jedoch nicht symmetrisch sein, denn während
$A$ sich stets im selben Inertialsystem befindet, wechselt $B$ das
Inertialsystem. Es handelt sich hier also um kein Paradoxon.

Aus der Sicht von $B$ ist die in der Raumzeit zurückzulegende Strecke
``Lorentz-kontrahiert'', d.h. er benötigt weniger Zeit als $A$.\bsphere
\end{bsp}
\begin{bsp}
Um Auszuschließen, dass die Beschleunigungsphase von $B$ den entscheidenden
Unterschied macht, sollen nun $A$ und $B$ eine Reise machen und dabei identisch
beschleunigen, $B$ soll nur länger mit der hohen Geschwindigkeit fliegen.
$A$ komme nach $10$ Jahren auf die Erde zurück, $B$ nach $30$.

Für $A$ ist zwischen Abflug und $B$s Ankunft die Zeit
\begin{align*}
(\text{Beschleunigungsbeitrag}) + \frac{1}{\gamma}\left(10 \text{
Jahre}\right) + 20\text{ Jahre}
\end{align*}
vergangen. Für $B$ ist zwischen Abflug und Ankunft die Zeit
\begin{align*}
(\text{Beschleunigungs Beitrag}) + \frac{1}{\gamma}\left(30 \text{
Jahre}\right)
\end{align*}
vergangen. Für die Differenz ergibt sich,
\begin{align*}
\Delta T_A - \Delta T_B &=  \frac{1}{\gamma}\left(10 \text{
Jahre}\right) + 20\text{ Jahre} - \frac{1}{\gamma}\left(30 \text{
Jahre}\right) \\ 
&= \left(1-\frac{1}{\gamma}\right)\underbrace{\left(20\text{
Jahre}\right)}_{\text{Reisezeit diff.}}.
\end{align*}
Bei $v=0.8c$ beträgt die Differenz $\left(1-\frac{3}{5}\right)20\text{ Jahre} =
8\text{ Jahre}$.\bsphere
\end{bsp}

\subsection{Lorentztransformation}

In der Speziellen Relativitätstheorie werden die Galilei Transformationen, die
in der Klassischen Mechanik ein Inertialsystem in ein anderes überführen,
durch Poincaretransformationen ersetzt. Diese Transformationen bilden ebenfalls
eine Gruppe, die \emph{Poincare-Gruppe}, die die Galilei Gruppe ersetzt. Eine
Poincaretransformation hat die Form
\begin{align*}
\rvec{x}' = \Lambda\rvec{x}+\rvec{a},
\end{align*}
d.h. sie setzt sich aus einer Lorentztransformation $\rvec{\Lambda}$ und einer
Translation $\rvec{a}$ zusammen. Dabei ist die Lorentztransformation eine
lineare Abbildung
\begin{align*}
{x^\mu}' = \Lambda_\nu^\mu x^\nu,
\end{align*}
die dadurch charakterisiert ist, dass sie die Minkovski Metrik
\begin{align*}
g=g_{\alpha\beta} = g^\mu\nu = \begin{pmatrix}
                               1 & \\
                               & -1 \\
                               && -1 \\
                               &&& -1
                               \end{pmatrix}
\end{align*}
invariant lässt, d.h. $\lin{\Lambda \rvec{x},\Lambda \rvec{x}}_g =
\lin{\rvec{x},\rvec{x}}_g$.

Somit ist auch der Betrag des $4$-er Vektors $\rvec{x}^\nu$ invariant,
\begin{align*}
g_{\mu\nu}x^\mu x^\nu = (x^0)^2 - (\vec{x})^2 = \const.
\end{align*}
\begin{proof}[Beweis der Invarianz.]
Mit $y^\mu = \Lambda_\nu^\mu x^\nu$ erhalten wir,
\begin{align*}
g_{\mu\nu}y^\mu y^\nu = g_{\mu\nu}\Lambda_{\alpha}^\mu x^\alpha
\Lambda_\beta^\nu x^\beta = \left(\Lambda_\alpha^\mu
g_{\mu\nu}\Lambda_\beta^\nu\right)x^\alpha x^\beta = g_{\alpha\beta}x^\alpha
x^\beta.\qedhere
\end{align*}
\end{proof}

Für jeden 4-er Vektor $\rvec{x}^\mu$ ist daher die Größe
\begin{align*}
g_{\mu\nu}x^\mu x^\nu = {x^0}^2 - \vec{x}^2 = \const 
\end{align*}
eine feste Größe in allen Inertialsystemen. Insbesondere wird der Lichtkegel mit
\begin{align*}
g_{\mu\nu}x^\mu x^\nu = 0
\end{align*}
auf sich selbst abgebildet.

Eine spezielle Lorentztransformation haben wir bereits kennen gelernt, den
Boost in $x$-Richtung,
\begin{align*}
\Lambda_\mu^\nu = \begin{pmatrix}
                  \gamma & -\gamma\beta\\
                  -\gamma\beta & \gamma\\
                  && 1\\
                  &&& 1
                  \end{pmatrix}
= \begin{pmatrix}
\cosh\psi & -\sinh\psi\\
-\sinh\psi & \cosh\psi\\
&& 1\\
&&& 1
\end{pmatrix}
\end{align*}
mit der \emph{Rapidität} $\psi$ und den Relationen
\begin{align*}
\tanh \psi = \frac{v}{c},\quad \gamma =
\frac{1}{\sqrt{1-\frac{v^2}{c^2}}},\quad \beta = \frac{v}{c}.
\end{align*}
Rotationen haben die Form,
\begin{align*}
\Lambda_\mu^\nu = \begin{pmatrix}
                  1 & 0 & 0 & 0\\
                  0 & \\
                  0 & & \vec{R}\\
                  0 
                  \end{pmatrix}
\end{align*}
mit einer $3\times 3$-Rotationsmatrix $\vec{R}$.

Analog zur klassischen Mechanik gibt es auch die zwei diskreten
Transformationen Raumspiegelung und Zeitumkehr
\begin{align*}
P =
\begin{pmatrix}
1 &\\
& -1\\
&& -1\\
&&& -1
\end{pmatrix},\qquad 
T=\begin{pmatrix}
-1 &\\
& 1\\
&& 1\\
&&& 1
\end{pmatrix}.
\end{align*}
\begin{bemn}
Man kann zeigen, dass sich jede Lorentztransformation als Produkt dieser
einfachen Transformationen schreiben lässt.\maphere
\end{bemn}

\subsubsection{Addition von Geschwindigkeiten}

In der Speziellen Relativitätstheorie ist die Geschwindigkeitsaddition nicht
mehr linear wie in der Newtonschen Mechanik.

\begin{figure}[!htbp]
  \centering

\begin{pspicture}(-0.1,-1.491875)(3.5175,1.491875)
\psline{->}(0.3478125,-1.331875)(0.3478125,0.968125)
\psline{->}(0.2478125,-1.191875)(2.4878125,-1.191875)
\psline{->}(1.3878125,-0.471875)(1.3878125,1.128125)
\psline{->}(1.2678125,-0.371875)(3.1078124,-0.371875)
\psline{->}(2.3278124,0.288125)(2.3278124,1.248125)
\psline{->}(2.2478125,0.408125)(3.2878125,0.408125)
\psline[linecolor=darkblue]{->}(0.3478125,-1.191875)(1.3478125,-0.431875)
\psline[linecolor=darkblue]{->}(1.3878125,-0.371875)(2.3078125,0.348125)

\rput(1.71875,0.198125){\color{gdarkgray}$v_2$}
\rput(0.7242187,-0.621875){\color{gdarkgray}$v_1$}
\rput(2.4332812,-1.341875){\color{gdarkgray}$x$}
\rput(0.11203125,1.018125){\color{gdarkgray}$ct$}
\rput(3.0903125,-0.541875){\color{gdarkgray}$x'$}
\rput(1.1765625,1.198125){\color{gdarkgray}$ct'$}
\rput(3.3303125,0.258125){\color{gdarkgray}$x''$}
\rput(2.1965625,1.318125){\color{gdarkgray}$ct''$}
\end{pspicture} 
\caption{Relativbewegung von drei Koordinatensystemen.}
\end{figure}


 Um die genaue Form herzuleiten, betrachten wir die
Intertialsysteme $K$, $K'$ und $K''$ wobei sich $K'$ mit $v_1$ relativ zu $K$
und $K''$ mit $v_2$ relativ zu $K'$ bewegt.
Die Transformation von $K$ nach $K''$ erhalten wir, indem wir die einzelnen
Transformationen hintereinanderausführen,
\begin{align*}
\Lambda(v_2)\Lambda(v_1) &= 
\begin{pmatrix}
\cosh\chi_2 & -\sinh\chi_2\\
-\sinh\chi_2 & \cosh\chi_2\\
&& 1\\
&&& 1
\end{pmatrix}\\
&\quad \cdot
\begin{pmatrix}
\cosh\chi_1 & -\sinh\chi_1\\
-\sinh\chi_1 & \cosh\chi_1\\
&& 1\\
&&& 1
\end{pmatrix}\\
% &=
% \begin{pmatrix}
% \tiny\cosh\chi_2 \cosh\chi_1 + \sinh\chi_1\sinh\chi_2 & -\sinh\chi_1\cosh\chi_2
% - \sinh\chi_2\cosh\chi_1\\
% \cosh\chi_2 \cosh\chi_1 + \sinh\chi_1\sinh\chi_2 &-\sinh\chi_1\cosh\chi_2 -
% \sinh\chi_2\cosh\chi_1\\
% && 1\\ 
% &&& 1
% \end{pmatrix}\\
&=
\begin{pmatrix}
\cosh(\chi_1+\chi_2) & -\sinh(\chi_1+\chi_2)\\
-\sinh(\chi_1+\chi_2) & \cosh(\chi_1+\chi_2)\\
&& 1\\
&&& 1
\end{pmatrix}
\end{align*}
Es werden also nicht die Geschwindigkeiten sondern die Rapiditäten addiert,
\begin{align*}
\chi_3 = \chi_1+\chi_2\Rightarrow v_3 = c\tanh \chi_3 =
\frac{v_1+v_2}{1+\frac{v_1v_2}{c^2}}.
\end{align*}

\begin{bemn}
$v_3$ ist stets kleiner als die Lichtgeschwindigkeit $c$.\maphere
\end{bemn}

\subsection{Energie und Impuls}

Das Differential
\begin{align*}
\ds^2 = c^2\dt^2 - \diffd\vec{x}^2 =
c^2\left(1-\frac{\vec{v}^2}{c^2}\right)\dt^2 \equiv c^2\dtau^2
\end{align*}
ist lorentzinvariant und somit ist die Eigenzeit
\begin{align*}
\dtau = \frac{\ds}{c} = \sqrt{1-\frac{v^2}{c^2}}\dt
\end{align*}
eine lorentzinvariante skalare Größe.

Somit erhalten wir aus dem $4$-er Vektor $\rvec{x} = \begin{pmatrix}ct \\
\vec{x}\end{pmatrix}$ einen neuen Vektor,
\begin{align*}
&\rvec{u} = \frac{\diffd}{\dtau}\rvec{x} = \gamma \frac{\diffd}{\dt}\rvec{x} =
\gamma\begin{pmatrix}c \\ \vec{v}\end{pmatrix} = \begin{pmatrix} \gamma c \\
\gamma\vec{v}\end{pmatrix},\\
&\rvec{x} \mapsto \Lambda\rvec{x},\\
&\rvec{u} \mapsto \Lambda \rvec{u} = \Lambda \frac{\diffd}{\dtau}\rvec{x} =
\Lambda \frac{\diffd}{\dtau}\Lambda^{-1}\Lambda \rvec{x} =
\frac{\diffd}{\dtau}\Lambda\rvec{x}.
\end{align*}
\begin{bemn}[Bemerkungen.]
\begin{enumerate}[label=\arabic{*}.)]
  \item $\rvec{u}g\rvec{u} = \gamma^2c^2 - \gamma^2\vec{v}^2 =
c^2\gamma^2\left(1-\frac{v^2}{c^2}\right) = c^2$, d.h. 
 die Minkowski Metrik ist eine Konstante für alle Bewegungen.
  \item  Die Geschwindigkeit $\vec{v}$ ist kein Anteil eines 4-er
  Vektors.\maphere
\end{enumerate}
\end{bemn}

Wir definieren den $4$-er Impuls
\begin{align*}
&\rvec{p} = m\rvec{u} = m\frac{\diffd}{\dtau}\rvec{x},\\
&\rvec{p}^\mu = \begin{pmatrix}mc\gamma \\ m\vec{v}\gamma\end{pmatrix},\quad
\rvec{p}_\mu = g_{\mu\nu}\rvec{p}^\nu \begin{pmatrix}mc\gamma \\
-m\vec{v}\gamma\end{pmatrix}
\end{align*}
In der modernen Physik sieht man davon ab, die Masse geschwindigkeitsabhängig
zu definieren, sondern fasst sie als skalare lorentzinvariante Größe auf.
Es ist oft viel geschickter so vorzugehen, auch wenn in Deutschen Lehrbüchern -
historisch bedingt - oftmals von einer geschwindigkeitsabhängigen Masse
ausgegangen wird.

Mit dem so definierten Impuls erhalten wir,
\begin{align*}
p^\mu p_\mu =g_{\mu\nu} p^\mu p^\nu = m^2c^2.
\end{align*}
Für ein abgeschlossenes System ist daher $\rvec{p}^\mu$ eine
Erhaltungsgröße. Die Zeitkomponente von $\rvec{p}$ übernimmt die Rolle einer Energie,
\begin{align*}
\rvec{p}  = \begin{pmatrix}\frac{E}{c}\\ \vec{p}\end{pmatrix}.
\end{align*}
Für kleine Geschwindigkeiten $\frac{v}{c} << c$ erhalten wir so 
\begin{align*}
E &= mc^2\gamma = mc^2\left(\frac{1}{\sqrt{1-\frac{v^2}{c^2}}}\right)
= mc^2 \left( 1+ \frac{v^2}{2c^2} + \frac{3}{8}\frac{v^4}{c^4} + \ldots\right)\\
&\approx mc^2 + \frac{1}{2}mv^2. 
\end{align*}
\begin{bemn}[Bemerkungen.]
\begin{enumerate}[label=\arabic{*}.)]
  \item Der 4-er Impuls eines Elementarteilchens ist eine kovariante Größe
  $\rvec{p}^\mu$ und seine Masse der Skalar $p^\mu p_\mu = c^2m^2$.
  \item Für Photonen gilt $m=0$ und daher $\frac{E}{c} = \abs{\vec{p}}$, was
  auf die Dispersion führt $\frac{\omega}{c} = \abs{\vec{k}}$.\maphere
\end{enumerate}
\end{bemn}
\begin{bsp}
\textit{Elektron-Positron-Kollision}.
\begin{align*}
e^- + e^+ \longrightarrow 2\gamma.
\end{align*}
Es gilt die relativistische Energie und Impulserhaltung,
\begin{align*}
\rvec{p}_A + \rvec{p}_B = \rvec{p}_C + \rvec{p}_D.
\end{align*} 
Wir gehen davon aus, dass sich $e^-$ und $e^+$ in Ruhe befinden.
\begin{align*}
\Rightarrow 2mc^2 = E_c+ E_D
\end{align*}
Nach dem Stoß muss der Gesamtimpuls Null sein,
\begin{align*}
\Rightarrow &\rvec{p}_C = -\rvec{p}_D\\
\Rightarrow &E_C = E_D = mc^2.
\end{align*}
Wir sehen somit, dass die relativistische Energie-Impulserhaltung die Erzeugung
und die Annihilation von Teilchen ermöglicht.\bsphere
\end{bsp}
\begin{bsp}
Die Sonne strahlt Energie in Form von Photonen ab. Durch diesen Energieverlust
verliert sie auch stetig an Masse.\bsphere
\end{bsp}

\subsubsection{Weitere 4-er Vektoren}

Nachdem wir $4$-er Vektoren für Ort, Geschwindigkeit und Impuls eingeführt
haben, wollen wir prüfen, welche Größen sich außerdem so darstellen lassen.
Betrachten wir zunächst die Beschleunigung,
\begin{align*}
\rvec{a} &= \frac{\diffd}{\dtau}\rvec{u} = \frac{\diffd^2}{\dtau^2}\rvec{x} =
\gamma\frac{\diffd}{\dt}\gamma\begin{pmatrix}c\\ \vec{v}\end{pmatrix}
= \gamma^2 \begin{pmatrix}0\\ \vec{a}\end{pmatrix} +
\gamma\left(\frac{-2\vec{v}\vec{a}}{c^2}\right)\gamma^3\left(-\frac{1}{2}\right)
\begin{pmatrix}c \\ \vec{v}\end{pmatrix}\\
&= \gamma^2 \begin{pmatrix}0\\\vec{a}\end{pmatrix} + \gamma^4
\frac{\vec{v}\vec{a}}{c^2}\begin{pmatrix}c\\\vec{v}\end{pmatrix}.
\end{align*}
\begin{bemn}
Beschleunigung und Impuls stehen senkrecht aufeinander
\begin{align*}
a^\mu p_\mu =
\frac{1}{2}\frac{\diffd}{\dtau}p^\mu p_\mu = 0.\maphere
\end{align*}
\end{bemn}
Der Vorteil der Darstellung durch $4$er-Vektoren ist, dass eine solche Größe
``trivial'' unter Lorentztransformationen transformiert. D.h. sie transformiert
so wie $\rvec{x}$ oder $\rvec{p}$,
\begin{align*}
{A^\mu}' = \Lambda_\nu^\mu A^\nu.
\end{align*}
Wir wollen nun weitere Größen betrachten, die sich durch $4$er-Vektoren
darstellen lassen:
\begin{itemize}[label=\labelitem]
\item Wellenvektor des Lichts,
\begin{align*}
&\rvec{k}^\mu = 
\begin{pmatrix}\frac{\omega}{c}\\\vec{k}\end{pmatrix},\quad\rvec{k}_\mu = 
\begin{pmatrix}\frac{\omega}{c}\\-\vec{k}\end{pmatrix}.
\end{align*}
Somit ist auch die Phase
\begin{align*}
\Rightarrow & e^{ix^\mu k_\mu} = e^{i\left(\omega t -
\vec{k}\vec{x}\right)}
\end{align*}
eine skalare Größe. $\omega t - \vec{k}\vec{x}$ ist hierbei lorentzinvariant.
\item Elektromagnetisches Potential mit dem Skalarpotential $\phi$ und dem
Vektorpotential $\vec{A}$,
\begin{align*}
\rvec{A}^\mu = \begin{pmatrix}\phi\\\vec{A}\end{pmatrix}.
\end{align*}
\item Ableitungsoperatoren
\begin{align*}
&\rvec{\partial_\mu} = %TODO: Bold \partial?
\begin{pmatrix}
\frac{\partial_t}{c}\\
\vec{\nabla}
\end{pmatrix},\\
&\rvec{\partial^\mu} = 
\begin{pmatrix}
\frac{\partial_t}{c}\\
-\vec{\nabla}
\end{pmatrix}.
\end{align*}
\item D'Alembertoperator (Wellengleichung $\square E = 0$)
\begin{align*}
\rvec{\square} = \partial_\mu\partial^\mu = g_{\mu\nu}\partial^\mu\partial^\nu
= \frac{\partial_t^2}{c^2}-\Delta. %TODO: Bold \square?
\end{align*}
\item Stromdichte mit Ladungsstromdichte $\rho$ und Teilchenstromdichte
$\vec{j}$,
\begin{align*}
\rvec{j}^\mu =
\gamma\begin{pmatrix} \rho\\ \vec{j}\end{pmatrix}.
\end{align*}
\item Für das $E$- und $B$-Feld erhalten wir den elektromagnetischen
Feldtensor, einen Tensor 2. Stufe,
\begin{align*}
\rvec{F}^{\mu\nu} 
= \rvec{\partial^\mu} \rvec{A}^\nu - \rvec{\partial^\nu} \rvec{A}^\mu.
\end{align*}
In karthesischen Koordinaten mit Minkowski-Metrik ist die Matrixdarstellung des
Tensors gegeben durch
\begin{align*}
\rvec{F}^{\mu\nu}  = 
\begin{pmatrix}
0 & -E_x & -E_y & -E_z\\
E_x & 0 & -B_z & B_y\\
E_y & B_z & 0 & -B_x\\
E_z & -B_y & B_x & 0
\end{pmatrix}.
\end{align*}
Wenden wir eine Lorentztransformation an, so ergibt sich
\begin{align*}
(F^{\mu\nu})' &= (\partial^\mu)'(A^\nu)' - (\partial^\nu)'(A^\mu)'
= \Lambda_\alpha^\mu \Lambda_\beta^\mu \left[\partial^\alpha A^\beta -
\partial^\beta A^\alpha\right] \\ &= \Lambda_\alpha^\mu\Lambda_\beta^\nu
F^{\alpha\beta}.
\end{align*}
Die Maxwellgleichungen haben nun die Form,
\begin{align*}
&\rvec{\partial_\mu} \rvec{F}^{\mu\nu} = -\frac{4\pi}{c}\rvec{j}^\nu,\\
&\rvec{\partial^\alpha} \rvec{F}^{\mu\nu} + \rvec{\partial^\mu}
\rvec{F}^{\nu\alpha} + \rvec{\partial^\nu} \rvec{F}^{\alpha\mu} = 0,
\end{align*}
sie sind lorentzinvariant.
\end{itemize}

\subsection{Relativistischer Dopplereffekt}

Betrachte ein Inertialsystem $K'$, das sich mit der Geschwindigkeit $v$ in
$z$-Richtung relativ zu $K$ bewegt.

\begin{figure}[H]
  \centering
\begin{pspicture}(0,-1.501875)(3.92,1.541875)
\psline[linecolor=darkblue]{->}(0.44,-1.061875)(2.12,-0.261875)
\psline{->}(0.44,-1.201875)(0.44,1.098125)
\psline{->}(0.34,-1.061875)(2.58,-1.061875)
\psline{->}(2.18,-0.301875)(2.18,1.298125)
\psline{->}(2.06,-0.201875)(3.9,-0.201875)
\pscircle[linecolor=yellow](0.43,-1.071875){0.43}
\pscircle[linecolor=yellow](0.43,-1.071875){0.27}
\pscircle[linecolor=yellow](0.43,-1.071875){0.14}

\rput(1.5764062,-0.271875){\color{gdarkgray}$v$}
%\rput(2.5254688,-1.211875){\color{gdarkgray}$v_1$}
\rput(0.20859376,1.148125){\color{gdarkgray}$K$}
\rput(1.971875,1.368125){\color{gdarkgray}$K'$}

\end{pspicture} 
\caption{Transformation der Lichtausbreitung.}
\end{figure}

Der Wellenvektor $\rvec{k}^\mu = \begin{pmatrix}\omega/c\\\vec{k}\end{pmatrix}$
transformiert trivial unter Lorentztransformationen, wir erhalten daher für ${\rvec{k}^\mu}'$,
\begin{align*}
%&{k^\mu}' = \Lambda_\nu^\mu k^\nu,\\
&k_x' = k_x,\quad
k_y' = k_y,\quad
k_z' = \gamma\left(k_z - \beta\frac{\omega}{c}\right),\\
&\omega' = \gamma\left(\omega - vk_z\right)
%= \gamma\omega\left(1-\frac{v}{c}\cos\th\right),
\end{align*}
Stellen wir $\vec{k}$ und $\vec{k}'$ in Kugelkoordinaten dar mit $\omega/c =
\abs{\vec{k}}$ und $\omega'/c = \abs{\vec{k}'}$, so folgt
\begin{align*}
\omega' = \gamma\omega\left(1-v/c\cos\th\right)
\end{align*}
und dies ist gerade der relativistische Dopplereffekt.

Für eine Ausbreitung entlang des boost, d.h. $\th = 0$ ergibt sich,
\begin{align*}
&\omega' = \omega\sqrt{\frac{1-\beta}{1+\beta}},\qquad\lambda' =
\lambda\sqrt{\frac{1+\beta}{1-\beta}}.
\end{align*}
% 
% \begin{bsp}
% Lichtquelle im bewegten Bezugssystem.


% Wenden wir einen Lorentzboost in $z$-Richtung an,
% \begin{align*}
% &{k^\mu}' = \Lambda_\nu^\mu k^\nu,\\
% &k_x' = k_x,\quad
% k_y' = k_y,\quad
% k_z' = \gamma\left(k_z - \beta\frac{\omega}{c}\right),\\
% &\omega' = \gamma\left(\omega - vk_z\right)
% = \gamma\omega\left(1-\frac{v}{c}\cos\th\right),
% \end{align*}
% so erhalten wir gerade den relativistischen Dopplereffekt.\bsphere
% \end{bsp}

\subsection{Relativistische Kraftgesetze}

Für $v\to 0$ gilt das Newtonsche Gesetz
\begin{align*}
\vec{F} = m\frac{\diffd^2\vec{r}}{\dt^2}.
\end{align*}
Wir können dies auch als $4$er-Vektor schreiben,
\begin{align*}
&m\frac{\diffd^2 }{\dtau^2} \rvec{x}^\alpha= \rvec{f}^\alpha \Leftrightarrow
\frac{\diffd}{\dtau} \rvec{p}^\alpha= \rvec{f}^\alpha,
\end{align*}
wobei die Kraft im Ruhesystem zwangsläufig die Form haben muss,
\begin{align*}
&\rvec{f}_0 = \begin{pmatrix}0\\\vec{F}\end{pmatrix},
\end{align*}
mit der nicht relativistischen Kraft $\vec{F}$ für $\vec{v} = 0$.

Die Idee ist nun, in das Ruhesystem des Teilchens ($t=\tau$) zu transformieren,
dort das Newtonsche Gesetz zu verwenden und anschließend zurückzutransformieren.
\begin{align*}
\frac{\diffd^2}{\dtau^2}\begin{pmatrix}ct\\\vec{x}\end{pmatrix}
= \frac{\diffd^2}{\dt^2}\begin{pmatrix}ct\\\vec{r}\end{pmatrix}
= \begin{pmatrix}0\\\frac{\diffd^2}{\dt^2}\vec{r}\end{pmatrix}
\end{align*}
Im Laborsystem wirkt somit die $4$er-Kraft,
\begin{align*}
&f^\alpha = \Lambda_\beta^\alpha(-\vec{v})f_0^\beta,\qquad
\rvec{f} = \rvec{\Lambda}\rvec{f_0},
\end{align*}
wobei hier die Rücktransformation ins Laborsystem $\Lambda_\beta^\alpha$ von
$-\vec{v}$ abhängt.
\begin{align*}
\Rightarrow
\rvec{f} = \begin{pmatrix}f^0\\\vec{f}\end{pmatrix}
= 
\begin{pmatrix}
\gamma\frac{\vec{v}}{c}\vec{F}\\
\vec{F} + (\gamma-1)\frac{\vec{v}\vec{F}}{v^2}\vec{v}
\end{pmatrix}.
\end{align*}
$f_0$ beschreibt hier die Änderung der Energie.
\begin{bsp}
Wir können nun die relativistische Kraftgleichung der Elektrodynamik
angeben,
\begin{align*}
&\frac{\diffd}{\dtau}\rvec{p}^\alpha =
\frac{e}{c}\rvec{F}^{\alpha\beta}\rvec{u}_\beta,\\
&\frac{\diffd}{\dtau}\rvec{p}_\alpha =
\frac{e}{c}\rvec{F}_{\alpha\beta}\rvec{u}^\beta.
\end{align*}
In Komponenten geschrieben,
\begin{align*}
&\frac{\diffd}{\dt}(\gamma m\vec{v}) = e\left(\vec{E} + \frac{1}{c}
\vec{v}\times\vec{B}\right), && \text{Lorentzkraft},\\
&\frac{\diffd}{\dt}(\gamma mc^2) = e\vec{E}\vec{v}, &&\text{geleistete
Abreit}.
\end{align*}
\begin{proof}[Beweis der Darstellung.]
Im Ruhesystem gilt die Elektrostatik
\begin{align*}
\vec{F} = e\vec{E},\qquad \vec{E}\text{ elektrisches Feld}.
\end{align*}
Mit dem elektromagnetischen Feldtensor im Ruhesystem
\begin{align*}
&\rvec{F}_0^{\alpha\beta} = 
\begin{pmatrix}
0 & -E_x & -E_y & -E_z\\
E_x & 0 & -B_z & B_y\\
E_y & B_z & 0 & -B_x\\
E_z & -B_y & B_x & 0
\end{pmatrix},
\end{align*}
folgt das relativistische Kraftgesetz,
\begin{align*}
\rvec{f}_0^\alpha =
\begin{pmatrix}
0\\\vec{F}
\end{pmatrix}
= \frac{e}{c}F_0^{\alpha\beta} u_\beta,
\end{align*}
wobei im Ruhesystem,
\begin{align*}
&\rvec{u}_\beta = \rvec{u}_\beta^0 = g_{\beta\nu}\rvec{u}_0^\nu =
g_{\beta\nu}\frac{\diffd}{\dtau}\rvec{x}_0^\nu =
\begin{pmatrix}c\\\vec{0}\end{pmatrix}.
\end{align*}
Rücktransformation ins Laborsystem ergibt
\begin{align*}
f^\alpha =
\Lambda_\beta^\alpha f_0^\beta = \frac{e}{c}\Lambda_\beta^\alpha F_0^{\beta\nu}
g_{\nu\mu} u_0^\mu = 
\frac{e}{c}\underbrace{\Lambda_\beta^\alpha
F_0^{\beta\nu}\Lambda_\nu^\gamma}_{F^{\alpha\gamma}} g_{\gamma\delta}
\underbrace{\Lambda_{\mu}^\delta u_0^\mu}_{u^\delta}
= \frac{e}{c}F^{\alpha\gamma}u_\gamma,
\end{align*}
wobei $F^{\alpha\gamma}$ den Feldtensor im Laborsystem und $u^\delta$ die
Geschwindigkeit im Laborsystem beschreiben,
\begin{align*}
\rvec{u} = 
\begin{pmatrix}
c\gamma\\ -\gamma\vec{v}
\end{pmatrix},\qquad \rvec{f}
=
\frac{e}{c}
\begin{pmatrix}
c\gamma \vec{E}\vec{v}\\
c\gamma\vec{E} + \gamma \vec{v}\times\vec{B}
\end{pmatrix}
= \gamma\frac{\diffd}{\dt}m u^\alpha
= \gamma\frac{\diffd}{\dt}
\begin{pmatrix}
\frac{E}{c}\\ m\gamma\vec{v}
\end{pmatrix}.\qedhere\bsphere
\end{align*}
\end{proof}
\end{bsp}

\subsection{Variationsprinzip}
Um den Lagrangeformalismus auch im relativistischen Fall anwenden zu können,
ist es geschickt, die Trajektorien nicht mit $t$ sondern einem beliebigen
Parameter $\lambda$ zu parametrisieren. Man erhält so die relativistische
Wirkung
\begin{align*}
S_{\text{rel}} = \int \dlambda\ L,
\end{align*}
und den Lagrange
\begin{align*}
L\left(\rvec{x}^\mu(\lambda), \frac{\diffd \rvec{x}^\mu}{\dlambda} \right)
= \underbrace{-mc\sqrt{g_{\mu\nu}\frac{\diffd
x^\mu}{\dlambda}\frac{\diffd x^\nu}{\dlambda}}}_{\text{neue rel.
Energie}} - \underbrace{\frac{e}{c}A_\mu (x^\nu(\lambda))\frac{\diffd x^\mu}{\dlambda}}_{\text{Wechsel zwischen
Feldern}}.
\end{align*}
Die Wirkung ist lorentzinvariant und unabhängig von der Parametrisierung
$\lambda$.

Die Variationsrechnung liefert nun
\begin{align*}
\frac{\diffd}{\dlambda}\left(\frac{\partial L}{\partial \frac{\diffd
x^\mu}{\dlambda}}\right) - \frac{\partial L}{\partial x^\mu} = 0,
\end{align*}
wobei
\begin{align*}
&\frac{\partial L}{\partial \frac{\diffd
x^\mu}{\dlambda}} = 
- 
\frac{1}{2}\underbrace{\frac{mc}{\sqrt{g_{\mu\nu}\ldots}}}_{c}2\frac{\diffd}{\dlambda}x_\mu
- \frac{e}{c}A_\mu = - m\frac{\diffd x_\mu}{\dlambda} - \frac{e}{c} A_\mu\\
&\frac{\partial L}{\partial x^\mu} = 
-\frac{e}{c}\partial_\mu A_\nu \frac{\diffd}{\dlambda}x^\nu\\
&\frac{\diffd}{\dlambda}\left(\frac{\partial L}{\partial \frac{\diffd
x^\mu}{\dlambda}}\right) =
- m\frac{\diffd}{\dlambda} \frac{\diffd}{\dlambda} x_\mu -
\frac{c}{c}\partial_\nu A_\mu \frac{\diffd}{\dlambda}x^\nu.
\end{align*}
Einsetzen in die Euler-Lagrange-Gleichungen ergibt,
\begin{align*}
m\frac{\diffd^2}{\dlambda^2}x_\mu = \frac{e}{c}\frac{\diffd}{\dlambda}x^\nu
\underbrace{\left[\partial_\mu A_\nu - \partial_\nu A_\mu\right]}_{F_{\mu\nu}}.
\end{align*}
Setzte $\lambda=\tau$ der Eigenzeit, so folgt
\begin{align*}
\frac{\diffd}{\dtau} \rvec{p}_\mu = \frac{e}{c} \rvec{F}_{\mu\nu}\rvec{u}^\nu.
\end{align*}
\begin{bemn}
Für $\lambda=t$ und kleine Geschwindigkeiten $v<<c$ erhalten wir die klassische
Lagrangefunktion und Wirkung
\begin{align*}
&S_{\text{rel}} \longrightarrow \int\dt L\\
&L\longrightarrow - \underbrace{mc^2}_{\text{irrelevante Konstante}} +
\underbrace{\frac{m}{2}v^2}_{T} - \underbrace{e\rho}_{V} +
\underbrace{\frac{e}{c}\vec{v}\vec{A}}_{\text{Anteil für Lorentzkraft}}.\maphere
\end{align*}
\end{bemn}

\newpage
\section{Starre Körper}

\subsection{Bewegung im beschleunigten Bezugssystem und Scheinkräfte}

Ein Punktteilchen wird in einem Inertialsystem vollständig durch den Lagrange
\begin{align*}
L(\vec{q},\dvec{q}) = T-V
\end{align*}
beschrieben und folgt der Newtonschen Gleichung,
\begin{align*}
m\ddvec{q} = -\nabla V = \vec{F}.
\end{align*}

In einem beschleunigten Bezugssystem treten zusätzliche Terme auf, die wir als
\emph{Scheinkräfte} bezeichnen.

Im Inertialsystem $K$ habe das Teilchen die Koordinate $\vec{q}$, im bewegten
System $K'$ die Koordinate $\vec{Q}$. Der Übergang von $K$ in $K'$ findet durch
Translationen und Rotationen statt,
\begin{align*}
\vec{q} = B\vec{Q} + \vec{r}(t),
\end{align*}
wobei $B\in\SO(3)$ eine Rotationsmatrix und $\vec{r}(t)$ den Translationsvektor
beschreibt.

\begin{figure}[htbp]
\centering
\begin{pspicture}(0,-1.215)(3.125,1.25)
\psline{->}(0.3,-1.21)(0.3,1.09)
\psline{->}(0.1,-1.01)(3.12,-1.01)
\psline{->}(1.98,0.01)(2.46,0.93)
\psline{->}(1.94,0.19)(2.94,-0.33)
\psline{->}(0.3,-1.01)(2.0,0.11)
\psdots(1.76,0.87)
\psline[linecolor=yellow]{->}(2.04,0.15)(1.82,0.77)
\psline[linecolor=darkblue]{->}(0.3,-1.01)(1.7,0.75)

\rput(1.52,-0.4){\color{gdarkgray}$r$}
\rput(0.99,0.295){\color{gdarkgray}$q$}
\rput(2.08,0.655){\color{gdarkgray}$Q$}
\rput(0.14,1.075){\color{gdarkgray}$K$}
\rput(2.62,0.895){\color{gdarkgray}$K'$}
\end{pspicture} 
\caption{Intertialsystem $K$ und bewegtes System $K'$}
\end{figure}

Die Geschwindigkeit $\vec{v} = \frac{\diffd}{\dt}\vec{q}$ lässt
sich somit schreiben als
\begin{align*}
\dvec{q} = \dot{B}\vec{Q} + B\dvec{Q} + \dvec{r}(t).
\end{align*}
Wir betrachten zunächst zwei Spezialfälle:
\begin{enumerate}[label=\arabic{*})]
  \item \textit{Translationsbewegung}.
\begin{align*}
&B = \Id,\\
\Rightarrow & \dvec{q} = \dvec{Q} + \dvec{r}(t).
\end{align*}
Der Lagrange hat daher im bewegten System die Form,
\begin{align*}
L(\vec{Q},\dvec{Q}) &= \frac{m}{2}\left(\dvec{Q}+\dvec{r}\right)^2 - V =
\frac{m}{2}\dvec{Q}^2 + m\dvec{Q}\dvec{r} + \frac{m}{2}\dvec{r}^2-V\\
&= \frac{m}{2}\dvec{Q}^2 - m\vec{Q}\ddvec{r} - V +
\underbrace{\frac{m}{2}\dvec{r}^2 +
\frac{\diffd}{\dt}\left(m\vec{Q}\dvec{r}\right)}_{(*)}.
\end{align*}
(*) ist für die Bewegung irrelevant, da sie den Lagrange nur um eine Konstante
und eine totale Ableitung ändert.
\begin{align*}
L(\vec{Q},\dvec{Q}) = \frac{m}{2}\dvec{Q}^2 -
\left(V+m\vec{Q}\vec{a}_s(t)\right),
\end{align*}
wobei $\vec{a}_s(t)$ die Beschleunigung des bewegten Systems beschreibt.
Mittels Euler-Lagrange-Gleichungen erhalten wir so,
\begin{align*}
m\ddvec{Q} = -\frac{\partial V}{\partial \vec{Q}} -
\underbrace{m\vec{a}_s(t)}_{\text{Scheinkraft durch beschl. Bewegung des
Systems}}
\end{align*}
Ein Beobachter im bewegten Bezugssystem nimmt die Scheinkraft als Kraft wahr,
die zusätzlich auf den Kröper wirkt.
\item \textit{Rotationen}.
\begin{align*}
&\vec{r} = 0,\\
\Rightarrow & \dvec{q} = \dot{B}\vec{Q} + B\dvec{Q}.
\end{align*}
Für beliebige Rotationen lässt sich $\dot{B}\vec{Q}$ schreiben als
\begin{align*}
\dot{B}\vec{Q} = \dot{B}B^{-1}\vec{q} = \vec{\omega}\times\vec{q},\tag{**}
\end{align*}
wobei $\vec{\omega}$ die Winkelgeschwindigkeit darstellt.
\begin{bemn}
$\vec{\omega}$ ist die Winkelgeschwindigkeit im Ruhesystem,
$\vec{\Omega}=B^{-1}\vec{\omega}$ ist die Winkelgeschwindigkeit im
beschleunigten System,
\begin{align*}
B^{-1}\dot{B}\vec{Q} = B^{-1}(\vec{\omega}\times \vec{q}) =
(B^{-1}\vec{\omega}\times B^{-1}\vec{q}) = \vec{\Omega}\times\vec{Q}.\maphere
\end{align*}
\end{bemn}
\begin{proof}[Beweis der Darstellung (**).]
Es gilt $\dot{B}\vec{Q} = \dot{B}B^{-1}\vec{q}$. $B$ ist eine Rotation, d.h.
\begin{align*}
B B^\top = B^\top B = \Id \Leftrightarrow B^\top = B^{-1}.
\end{align*}
Die Ableitung der Identität verschwindet, d.h.
\begin{align*}
&\frac{\diffd}{\dt}(B B^{-1}) = \dot{B}B^\top  + B\dot{B}^\top = 0,\\
\Leftrightarrow & \dot{B}B^\top  + \left(\dot{B}B^{-1}\right)^\top = 0,
\end{align*}
und daher ist $\dot{B} B^\top$ eine schiefe (antisymmetrische) Matrix. Sie hat
die Form,
\begin{align*}
&\dot{B}B^\top = 
\begin{pmatrix}
0 & -\omega_3 & \omega_2\\
\omega_3 & 0 & -\omega_1\\
-\omega_2 & \omega_1 & 0
\end{pmatrix},\qquad \omega_1,\omega_2,\omega_3\in\R.\\
&\dot{B}B^{-1}\vec{q} = 
\begin{pmatrix}
\omega_2 q_3 - \omega_3 q_2\\
\omega_3 q_1 - \omega_1 q_3\\
\omega_1 q_2 - \omega_2 q_1
\end{pmatrix}
= \vec{\omega}\times \vec{q},\qquad \vec{\omega} =
\begin{pmatrix}\omega_1\\\omega_2\\\omega_3\end{pmatrix}.\qedhere
\end{align*}
\end{proof}
Für die Geschwindigkeit erhalten wir also,
\begin{align*}
\dvec{q} = \vec{\omega}\times\vec{q} + B\dvec{Q}.
\end{align*}
Damit können wir die kinetische Energie angeben,
\begin{align*}
T &= \frac{1}{2}m\left(\vec{\omega}\times \vec{q} + B\dvec{Q} \right)^2
= \frac{m}{2} \left(\left(B\vec{\Omega}\times B\vec{Q}\right) + B\dvec{Q}
\right)^2\\
&= \frac{m}{2} \left(B\left(\vec{\Omega}\times\vec{Q} +
\dvec{Q} \right)\right)^2 = 
\frac{m}{2} \lin{B\left(\vec{\Omega}\times\vec{Q} +
\dvec{Q} \right),B\left(\vec{\Omega}\times\vec{Q} +
\dvec{Q} \right)}
\\
&= \frac{m}{2} \left(\vec{\Omega}\times\vec{Q} + \dvec{Q}\right)^2
= \frac{m}{2} \dvec{Q}^2 + m\dvec{Q}\left(\vec{\Omega}\times\vec{Q}\right) +
\frac{m}{2}\left(\vec{\Omega}\times\vec{Q}\right)^2.
\end{align*}
Einsetzen in die Lagrangefunktion ergibt,
\begin{align*}
L(\vec{Q},\dvec{Q}) = \frac{m}{2}\dvec{Q}^2 + m\dvec{Q}\left(\vec{\Omega}\times\vec{Q}\right) +
\frac{m}{2}\left(\vec{\Omega}\times\vec{Q}\right)^2 - V,
\end{align*}
die Euler-Lagrange-Gleichungen liefern,
\begin{align*}
m\ddvec{Q} = -\frac{\partial V}{\partial \vec{Q}} +
\underbrace{m\vec{\Omega}\times \left(
\vec{Q}\times\vec{\Omega}\right)}_{\text{Zentrifugalkraft}} +
\underbrace{2m\dvec{Q}\times\vec{\Omega}}_{\text{Corioliskraft}} +
m\vec{Q}\times\dvec{\Omega}.
\end{align*}
\end{enumerate}

\begin{bsp}
Die durch die Erddrehung ``wirkende'' Zentrifugalkraft ändert das
Gravitationsfeld der Erde.
\begin{figure}[htbp]
\centering
\begin{pspicture}(0,-1.51)(6.76,1.51)
\pscircle[linecolor=darkblue](1.5,-0.01){0.9}
\psline{->}(1.5,1.49)(1.5,0.97)
\psline{->}(3.0,-0.01)(2.46,-0.01)
\psline{->}(1.5,-1.49)(1.5,-0.97)
\psline{->}(0.0,-0.01)(0.52,-0.01)
\psline{->}(2.62,1.11)(2.2,0.69)
\psline{->}(2.62,-1.13)(2.2,-0.71)
\psline{->}(0.4,-1.11)(0.82,-0.69)
\psline{->}(0.38,1.09)(0.8,0.67)

\pscircle[linecolor=darkblue](5.4,-0.01){0.9}
\psline{->}(5.4,1.49)(5.4,0.97)
\psline{->}(6.74,-0.01)(6.36,-0.01)
\psline{->}(5.4,-1.49)(5.4,-0.97)
\psline{->}(4.04,-0.01)(4.42,-0.01)
\psline{->}(6.4,0.99)(6.1,0.69)
\psline{->}(6.4,-0.99)(6.1,-0.71)
\psline{->}(4.38,-0.97)(4.72,-0.69)
\psline{->}(4.38,0.99)(4.7,0.67)
\end{pspicture} 
\caption{Wirkung der Zentrifugalkraft auf die Erdanziehung (Übertrieben
dargestllt)}
\end{figure}

Der Einfluss ist jedoch sehr klein. Die Verzerrung, die entsteht, da die Erde
keine perfekte Kugel ist und eine inhomogene Dichteverteilung hat, ist viel
größer.\bsphere
\end{bsp}
\begin{bsp}
Das Foucault'sche Pendel ist ein Nachweis für die Corioliskraft.\bsphere
\end{bsp}

\subsection{Starre Körper}
Ein starrer Körper ist eine Idealisierung eines festen Körpers, wobei alle
Abstände zwischen den Teilchen fixiert sind,
\begin{align*}
\abs{\vec{q}_i - \vec{q}_j} = q_{ij}  = \const. 
\end{align*}
Dies ist ein Spezialfall einer holonomen Zwangsbedingung.

\begin{bemn}
Die Bewegung des starren Körpers ist ausgezeichnet durch die Lage des Körpers
(Rotation um einen Punkt) und die Bewegung dieses Punktes. Die
Konfigurationsmannigfaltigkeit des Körpers ist
\begin{align*}
\R^3\times \SO(3),
\end{align*}
eine 6-dimensionale Hyperfläche.

Wir führen daher zwei Koordinatensysteme ein. Das Laborsystem $(K,\vec{q})$ und
das körperfeste System $(K',\vec{Q})$ im Ursprung $\vec{r}(t)$.\maphere
\begin{figure}
\centering
\begin{pspicture}(0,-1.215)(3.125,1.25)
\psline{->}(0.3,-1.21)(0.3,1.09)
\psline{->}(0.1,-1.01)(3.12,-1.01)
\psline{->}(1.98,0.01)(2.46,0.93)
\psline{->}(1.94,0.19)(2.94,-0.33)
\psline{->}(0.3,-1.01)(2.0,0.11)
\psdots(1.76,0.87)
\psline[linecolor=yellow]{->}(2.04,0.15)(1.82,0.77)
\psline[linecolor=darkblue]{->}(0.3,-1.01)(1.7,0.75)

\rput(1.52,-0.4){\color{gdarkgray}$r$}
\rput(0.99,0.295){\color{gdarkgray}$q$}
\rput(2.08,0.655){\color{gdarkgray}$Q$}
\rput(0.14,1.075){\color{gdarkgray}$K$}
\rput(2.62,0.895){\color{gdarkgray}$K'$}
\end{pspicture} 
\caption{Laborsystem und körperfestes System.}
\end{figure}
\end{bemn} 

\subsubsection{Erhaltungsgrößen}

Für einen freien starren Körper bewegt sich der Schwerpunkt gleichförmig. Die
Bewegung des freien Körpers um den Schwerpunkt hat den Drehimpuls und die
Energie als Erhaltungsgröße.

\begin{tabular}[h]{l|l|ll}
 & Laborsystem & Körperfestsystem\\\hline
 Position des Teilchens $i$ & $\vec{q}_i$ & $\vec{Q}_i$ & $\vec{q}_i =
 B\vec{Q}_i +\vec{r}$\\ Winkelgeschwindigkeit & $\vec{\omega}$ & $\vec{\Omega}$ & $\vec{\omega} =
 B\vec{\Omega}$\\
 Drehimpuls & $\vec{m}$ & $\vec{M}$ & $\vec{m} = B\vec{M}$
\end{tabular}

Die Geschwindigkeit ist gegeben durch,
\begin{align*}
\vec{v}_i = \vec{\omega}\times \left(\vec{q}_i - \vec{r}\right) + \dvec{r}.
\end{align*}
\begin{proof}[Beweis der Geschwindigkeitsdarstellung.]
$\vec{v}_i = \dot{B}\vec{Q}_i + \dvec{r} + B\dvec{Q}_i$. Im starren Körper sind
jedoch alle Teilchen fixiert, d.h. $B\dvec{Q}_i = 0$.\qedhere
\end{proof}
Einsetzen in die kinetische Energie ergibt,
\begin{align*}
T = \sum_i \frac{m_i}{2}\vec{v}_i^2 = 
\underbrace{\sum_i \frac{m_i}{2}\dvec{r}^2}_{\frac{1}{2}M\dvec{r}^2} +
\sum_i m_i \dvec{r} \left(\vec{\omega}\times \left(\vec{q}_i
-\vec{r} \right) \right) + \frac{1}{2}\sum_i m_i
\left[\vec{\omega}\times \left(\vec{q}_i - \vec{r}\right) \right]^2.
\end{align*}
Betrachten wir die $2$-Fälle,
\begin{enumerate}[label=(\roman{*})]
  \item Bewegung im Schwerpunkt, dann ist $\sum_i m_i(\vec{q}_i -\vec{r}) = 0$,
  \item Bewegung mit $\vec{r}$ fixiert, dann ist $\dvec{r} = 0$.
\end{enumerate}
Somit lässt sich die kinetische Energie schreiben als,
\begin{align*}
T = \underbrace{\frac{1}{2}M\dvec{r}^2}_{\text{Schwerpunktsenergie}}+
\underbrace{\frac{1}{2}\sum_i m_i \left(\vec{\Omega}\times \vec{Q}_i
\right)^2}_{\text{Rotationsenergie}}.
\end{align*}
Die Rotationsenergie lässt sich schreiben als,
\begin{align*}
T_R &= \frac{1}{2}\sum_i m_i \left[\vec{\Omega}^2\vec{Q}_i^2 -
\left(\vec{\Omega}\vec{Q}_i\right)^2 \right] = \frac{1}{2}
\sum\limits_{j,k=1}^3 \Omega_j\Omega_k \underbrace{\sum_i m_i\left[ \vec{Q}_i^2
\delta^{jk} - Q_i^k Q_i^j \right]}_{=I^{jk}}\\
&= \frac{1}{2}\sum\limits_{j,k=1}^3 \Omega_i \Omega_k I^{jk}
= \frac{1}{2}\vec{\Omega} I \vec{\Omega}.
\end{align*}
Die Elemente des Trägheitstensors $I$ haben die Form,
\begin{align*}
I^{jk} = \sum\limits_{i}m_i \left[\delta^{jk} Q_i^2 - Q_i^jQ_i^k \right]
= \int\dr^3 \rho(\vec{r}) \left[\vec{r}^2 - r^jr^k\right],
\end{align*}
wobei $\rho(\vec{r})$ die Massendichte des starren Körper bezeichnet,
\begin{align*}
\rho(\vec{r}) = \sum_ i m_i \delta(\vec{r}-\vec{Q}_i).
\end{align*}
\begin{bemn}
Die totale Masse ist gegeben durch,
\begin{align*}
M  = \int\dr^3 \rho(\vec{r}),
\end{align*}
der Schwerpunkt durch,
\begin{align*}
\vec{r}_s = \frac{1}{M}\int\dr^3 \rho(\vec{r})\vec{r}.\maphere
\end{align*}
\end{bemn}

Der Trägheitstensor ist symmetrisch, $I^{jk} = I^{kj}$, und positiv. Er kann
durch geeignete Wahl des Körperfesten Systems auf Diagonalgestalt gebracht
werden,
\begin{align*}
I = \begin{pmatrix}
    I_1 & 0 & 0\\
    0 & I_2 & 0\\
    0 & 0 & I_3
    \end{pmatrix}.
\end{align*}
$I_1,I_2,I_3$ heißen \emph{Hauptträgheitsmomente}, die zugehörigen
Basisvektoren $\vec{e}_k$ \emph{Hauptträgheitsachsen}. Die Rotationsenergie
nimmt dann folgende Form an
\begin{align*}
T_R = \frac{1}{2}\sum\limits_{i=1}^3 I_i \Omega_i^2.
\end{align*}

\begin{pspicture}(0,-1.56)(3.92,1.56)
\psframe(1.98,-0.22)(0.2,-1.0)
\psline{->}(1.74,0.28)(1.74,1.26)
\psline{->}(2.56,-0.06)(3.48,-0.06)
\psline{->}(1.04,-0.56)(0.38,-1.2)

\psline(0.2,-0.22)(1.5,0.74)(3.12,0.74)
\psline(1.98,-0.22)(3.12,0.74)(3.12,0.08)(1.98,-1.0)

\rput(1.98,1.365){\color{gdarkgray}$\vec{e}_3$}
\rput(3.71,-0.035){\color{gdarkgray}$\vec{e}_2$}
\rput(0.29,-1.395){\color{gdarkgray}$\vec{e}_1$}

\rput(1.08,-1.135){\color{gdarkgray}$a$}
\rput(0.09,-0.555){\color{gdarkgray}$b$}
\rput(2.8,-0.395){\color{gdarkgray}$c$}
\end{pspicture} 

\begin{bemn}[Bemerkungen.]
\begin{enumerate}[label=\arabic{*}.)]
  \item 
$I_1+I_2\ge I_3$, für alle Permutationen.
\begin{proof}
Verwende dazu die Definition des Trägheitsmoments,
\begin{align*}
I_1 + I_2 &= \int\dr^3 \rho(\vec{r})\left[2\vec{r}^2 - r_1^2 -r_2^2\right]
= \int\dr^3 \rho(\vec{r})\left[r_1^2 + r_2^2 + \underbrace{2r_3^2}_{\ge
0}\right]\\
& \ge \int\dr^3 \rho(\vec{r})\left[r_1^2 + r_2^2 \right]
= I_3.\qedhere\maphere
\end{align*}
\end{proof}
\item $I_1 = I_2 = I_3$, Kugelkreisel,\\
$I_1=I_2\neq I_3$, symmetrischer Kreisel,\\
$I_1\neq I_2\neq I_3$, unsymmetrischer Kreisel.
\end{enumerate}
\end{bemn}

\begin{bsp}
Homogener Quader,
\begin{align*}
&I_1 = \frac{1}{12}m(b^2+c^2),\\
&I_2 = \frac{1}{12}m(a^2+c^2),\\
&I_3 = \frac{1}{12}m(a^2+b^2).\bsphere
\end{align*}
\end{bsp}
\begin{bsp}
Homogene Kugel
\begin{align*}
I=\frac{2}{5}mR^2.\bsphere
\end{align*}
\end{bsp}
\begin{bsp}
Homogener Zylinder (Vollzylinder der Länge $l$)
\begin{align*}
&I_1 = \frac{1}{2}mR^2,\\
&I_2=I_3 = \frac{1}{4}mr^2 + \frac{1}{12}ml^2.\bsphere
\end{align*}
\end{bsp}

\begin{propn}[Satz von Steiner]
Es sei $I^{jk}$ der Trägheitstensor relativ zum Schwerpunkt. So gilt für den
Trägheitstensor $\tilde{I}^{jk}$ bezügl eines um $\vec{a}$ veschobenen Punkts
\begin{align*}
\tilde{I}^{jk} = I^{jk} + M\left[\delta^{jk} a^2 - a^j a^k\right].\fishhere
\end{align*}
\end{propn}
\begin{proof}
Der Beweis ist eine leichte Übung.\qedhere
\end{proof}

\subsubsection{Drehimpuls}

Der Drehimpuls bezüglich eines körperfesten Punktes $O$ ist gegeben durch,
\begin{align*}
\vec{m} = \sum_i \vec{q}_i \times \vec{v}_i m_i.
\end{align*}
$O$ ist fixiert im Laborsystem. Setzen wir $\vec{v}_i =
\vec{\omega}\times\vec{q}_i$, so erhalten wir
\begin{align*}
\vec{m} = \sum_i m_i \vec{q}_i \times \left( \vec{\omega}\times\vec{q}_i\right).
\end{align*}
Im bewegten System ist
\begin{align*}
\vec{M} = B^{-1}\vec{m} = \sum_i m_i \vec{Q}_i \times \left(\vec{\Omega}\times
\vec{Q}_i\right) = \sum_i m_i \left[\vec{\Omega}\left(\vec{Q}_i \right)^2 -
\vec{Q}_i \left(\vec{\Omega}\vec{Q}_i\right)\right] = I\vec{\Omega},\tag{*}
\end{align*}
denn
\begin{align*}
M^j = \sum_i \sum_k \left[\delta^{jk} \left(\vec{Q}_i\right)^2 - Q_i^j Q_i^k
\right]m_i \Omega_k  = \sum_k I^{jk}\Omega_k.
\end{align*}
\begin{bemn}[Bemerkungen.]
\begin{enumerate}[label=\arabic{*}.)]
  \item $T_R = \frac{1}{2}\vec{M}\vec{Q}$.
  \item Der Drehimpuls $\vec{M}$ ist im Allgemeinen nicht parallel zur
Drehachse $\vec{\Omega}$.
\item Für ein freies System ist der Drehimpuls im Laborsystem erhalten
$\vec{m}=\const.$ Der Drehimpuls im rotierenden System ist aber nicht
zwangsläufig erhalten $\vec{M}\neq\const$.\maphere
\end{enumerate}
\end{bemn}

Falls ein Körper nur rotiert, so ist es optimal den Ursprung des bewegten
Koordinatensystems in den Schwerpunkt zu setzen.
% Für einen bewegten Körper mit Ursprung des rotierenden Systems im Schwerpunkt
% gilt für den totalen Drehimpuls,
\begin{align*}
&\vec{q}_i  = B\vec{Q}_i +\vec{r}_s,\\
&\vec{v}_i = B\left(\vec{\Omega}\times \vec{Q}_i\right)+\dvec{r}_s,\\
&\vec{J} = \sum_i m_i(\vec{Q}_i + \vec{R}_s) \times
\left[\vec{\Omega}\times \vec{Q}_i + \vec{v}_s\right].
\end{align*}
Da wir uns im Schwerpunkt befinden, verschwinden die Kreuzterme und es gilt,
\begin{align*}
\vec{J} = \underbrace{M_\tot \vec{R}_s \times \vec{v}_s}_{\text{Bahndrehimpuls}}
+
\underbrace{\vec{I}\vec{Q}}_{\vec{M}}.
\end{align*}
$\vec{M}$ ist der innere Drehimpuls (Spin). Der totale Drehimpuls im
Laborsystem ist nun
\begin{align*}
\vec{j} = B\vec{J} = \underbrace{M_\tot \vec{r}_s \times
\vec{v}_s}_{\text{Bahndrehimpuls}} +
\underbrace{\vec{m}}_{\text{Spin}}.
\end{align*}
$J$ bzw. $j$ lässt sich daher aufteilen in die äußere Bewegung (z.b. Kreisbahn)
und die innere Bewegung (Eigendrehimpuls).

Unter Einfluss einer äußeren Kraft ändert sich der Drehimpuls eines Körpers mit
seinem Drehmoment (fixiere Punkt $O$ des Körpers),
\begin{align*}
\frac{\diffd}{\dt}\vec{m} &= \frac{\diffd}{\dt}\sum_i
m_i\vec{q}_i\times\dvec{q}_i
= \sum_i m_i \vec{q}_i\times\ddvec{q}_i = \sum_i \vec{q}_i\times \vec{F_i} =
\vec{n}.
\end{align*}
$\vec{n}$ ist das realtive Drehmoment zum Körperfesten Punkt $O$.

Für das mitrotierende System bedeutet dies,
\begin{align*}
\frac{\diffd}{\dt}\vec{M} = \frac{\diffd}{\dt}B^{-1}\vec{m}
= B^{-1}\vec{n} + \dot{B}^{-1}B\vec{M} = \vec{N} + \vec{M}\times\vec{\Omega}.
\end{align*}

Dies lässt sich in der \emph{Euler-Gleichung} zusammenfassen
\begin{align*}
\frac{\diffd}{\dt}\vec{M} = \vec{N} + \vec{M}\times\vec{\Omega},
\end{align*}
d.h. auch in Abwesenheit äußerer Kräfte $\vec{N}=0$ ist der Drehimpuls im
rotierenden System nicht erhalten.

Die Lösung des freien Kreisels lässt sich einfach geometrisch interpretieren.
Wir haben zwei Erhaltungsgrößen
\begin{align*}
&E = \frac{M_1^2}{I_1} + \frac{M_2^2}{I_2} + \frac{M_3^2}{I_3} = \const.\\
&M^2 = M_1^2+M_2^2 + M_3^2 = \const.
\end{align*}
Die Schnittmenge beschreibt Kurven auf einem Ellipsoiden.
\begin{itemize}[label=\labelitem]
  \item 6 stationäre Lösungen entlang der Hauptachsen.
  \item 4 stabile Lösungen entlang $\vec{e}_1$ und $\vec{e}_3$ ($I_1<I_2<I_3$).
  \item 2 instabile Lösungen entlang der mittleren Hauptachse.
\end{itemize}

\subsection{Lagrange's top}

Wir betrachten nun den symmetrischen Kreisel im Schwerepotential.

In diesem System sind zwei Größen erhalten:
\begin{itemize}[label=\labelitem]
  \item Die Energie $E = T+V$.
  \item Der Drehimpuls $m_z$ entlang der $z$-Achse.
\end{itemize}

Für einen symmetrischen Kreisel $(I_1=I_2\neq I_3)$ ist das Problem analytisch
lösbar. Wähle dazu geeignete Koordinaten $\ph,\psi,\th$.
%TODO: Skizze Kreisel

Das Laborsystem sei $K$ mit $\vec{e}_x$, $\vec{e}_y$, $\vec{e}_z$, das
körperfeste System sei $K'$ mit $\vec{e}_1$, $\vec{e}_2$, $\vec{e}_3$ entlang
der Trägheitsachsen des Kreisels. Es gilt
\begin{align*}
\vec{e}_N = \vec{e}_z\times \vec{e}_3.
\end{align*}
Um $K$ in $K'$ zu transformieren verwenden wir die 3 Transformationen,
\begin{align*}
&\ph: \text{Rotation um } \vec{e}_z: \vec{e}_x\mapsto \vec{e}_N\\
&\th: \text{Rotation um } \vec{e}_N: \vec{e}_z\mapsto \vec{e}_3\\
&\psi: \text{Rotation um } \vec{e}_3: \vec{e}_N\mapsto \vec{e}_1
\end{align*}
Weiter gilt,
\begin{align*}
&\vec{e}_N = \cos\psi \vec{e}_1 - \sin\psi \vec{e}_2\\
&\vec{e}_z = \cos\ph \vec{e}_3 + \cos\psi \sin\th \vec{e}_2 + \sin\psi \sin\th
\vec{e}_1
\end{align*}
Wir haben jetzt ein geeignetes Koordinatensystem. Der nächste Schritt ist die
Lagrangefunktion in diesen Koordinaten aufzustellen.

Potentielle Energie,
\begin{align*}
U = m g l \cos\th,
\end{align*}
wobei $l$ die Position des Schwerpunkts bezeichnet.

Um die Kinetische Energie zu bestimmen, verwenden wir den Hilfsvektor
\begin{align*}
\vec{\Omega} &= \dot{\th}\vec{e}_N + \dot{\psi}\vec{e}_3 + \dot{\ph}\vec{e}_z =
\Omega_1 \vec{e}_1 + \Omega_2\vec{e}_2 + \Omega_3\vec{e}_3\\
&= \vec{e}_1\left(\dot{\th}\cos\psi + \dot{\ph}\sin\psi\sin\th \right)
+ \vec{e}_2 \left( -\dot{\th}\sin\psi + \dot{\ph}\cos\psi\sin\th \right)
+ \vec{e}_3 \left(\dot{\psi} + \dot{\ph}\cos\th\right)
\end{align*}

Für $K'$ folgt somit:
\begin{align*}
&\Omega_1 = \dot{\th}\cos\psi + \dot{\ph}\sin\psi\sin\th,\\
&\Omega_2 = -\dot{\th}\sin\psi + \dot{\ph}\cos\psi\sin\th,\\
&\Omega_3 = \dot{\psi} + \dot{\ph}\cos\th.
\end{align*}

Einsetzen in die kinetische Energie ergibt,
\begin{align*}
T = \frac{1}{2}\left( I_1\Omega_1^2 + I_2 \Omega_2^2 + I_3\Omega_3^2 \right)
= \frac{I_1}{2}\left( \dot{\th}^2 + \dot{\ph}^2\sin^2\th \right) +
\frac{I_3}{2}\left( \dot{\psi} + \dot{\ph}\cos\th\right)^2.
\end{align*}

Die Lagrangefunktion ist nun gegeben durch,
\begin{align*}
L &= L(\ph,\th,\psi,\dot{\ph},\dot{\th},\dot{\psi}) = T-U \\ &=
\frac{I_1}{2}\left( \dot{\th}^2 + \dot{\ph}^2\sin^2\th \right) +
\frac{I_3}{2}\left( \dot{\psi} + \dot{\ph}\cos\th\right)^2
-mgl\cos\th
\end{align*}
Der Aufwand der durch die Einführung von Eulerkoordinaten entstand zahlt sich
nun dadurch aus, dass in der Lagrangefunktion zwei zyklische Koordinaten
auftreten, nämlich $\ph$ und $\psi$.

Die Erhaltungsgrößen erhalten wir durch Differenzieren,
\begin{align*}
&\frac{\partial L}{\partial\dot{\ph}} = \dot{\ph}\left(I_1 \sin^2\th + I_3
\cos^2\th  \right) + \dot{\psi}I_3 \cos\th = M_z = \const,\\
&\frac{\partial L}{\partial\dot{\psi}} = \dot{\ph}I_3\cos\th + \dot{\psi}I_3 =
M_3 = \const\\
\Rightarrow&\dot{\ph} = \frac{M_z - M_3\cos\th}{I_1 \sin^2 \th}\\
&\dot{\psi} = \frac{M_3}{I_3} - \cos\th \frac{M_z - M_3\cos\th}{I_1\sin^2\th}
\end{align*}
Dies können wir nun verwenden um die $\dot{\ph}$ und $\dot{\psi}$ Abhängigkeit
des Lagranges zu eliminieren.

Verbleibende Gleichung für die Inklination $\th$ folgt aus der
Energieerhaltung
\begin{align*}
\underbrace{E - \frac{M_3^2}{2I_3}}_{E'} = \frac{I_1}{2}\dot{\th}^2 +
\underbrace{\frac{\left(M_z - M_3\cos\th\right)^2}{2I\sin^2\th} +
mgl\cos\th}_{U_{\text{eff}}(\th)}.
\end{align*}
Somit haben wir nur noch ein 1-dimensionales Problem. Ein solches ist formal
exakt lösbar.

\subsubsection{Qualitative Diskussion}

Wir sind jetzt nicht daran interessiert das Problem exakt zu lösen, da dies
sehr aufwändig ist und nicht viel zum Verständnis beiträgt.  Jedoch wollen wir
das System qualitativ betrachten.

Einführen einer neuen Kooridnaten
\begin{align*}
&u = \cos\th, \qquad -1\le u\le 1,
\end{align*}
und den Abkürzungen
\begin{align*}
&a \equiv  \frac{M_2}{I_1},\quad b  \equiv \frac{M_3}{I_1},
\quad\alpha \equiv \frac{2E'}{I_1},\quad
\beta \equiv \frac{2mgl}{I_1} \ge 0.
\end{align*}
Wir erhalten somit für
\begin{align*}
\dot{u}^2 = f(u) = \left(\alpha-\beta u \right)(1-u^2)-(a-bu)^2
\end{align*}
ein Polynom 3. Grades in $u$.
\begin{align*}
\dot{\ph} = \frac{a-bu}{1-u^2}.
\end{align*}

\begin{figure}[htbp]
  \centering
\begin{pspicture}(-1,-1)(4,3)
 \psaxes[labels=none,ticks=none]{->}%
 	(0,0)(-0.5,-0.5)(3.5,2.5)%
 	[\color{gdarkgray}$u$,-90]%
 	[\color{gdarkgray}$f(u)$,0]
 	
 \psxTick(1){\color{gdarkgray}u_1}
 \psxTick(2){\color{gdarkgray}u_2}
 
\psplot[linewidth=1.2pt,%
	     linecolor=darkblue,%
	     algebraic=true]%
	     {0.5}{3.5}{(x-1)*(x-2)*(x-3)}

%\rput(2,-0.4){\color{gdarkgray}$f(u)$}
\end{pspicture} 
  \caption{Phasendiagramm für $u$.}
\end{figure}

Die Bewegung findet zwischen $u_1$ und $u_2$ statt. D.h. die Inklination $\th$
oszilliert zwischen den Winkeln $\th_1$ und $\th_2$ periodisch. Diese Bewegung
wird \emph{Nutation} genannt.

Der azimutale Winkel $\ph$ folgt aus
\begin{align*}
\dot{\ph} = \frac{a-bu}{1-u^2}.
\end{align*}
Folgende 3 Fälle sind möglich

\begin{figure}[htbp]
\centering
\begin{pspicture}(0,-1.7)(2.24,1.3)
\pscircle(0.9,-0.36){0.9}
\psellipse[linestyle=dashed](0.9,0.34)(0.44,0.2)
\psline{->}(0.9,0.42)(0.9,1.02)
\psbezier(0.18,0.18)(0.18,-0.46)(1.62,-0.46)(1.62,0.16)
\psbezier(0.04,-0.12)(0.04,-0.98)(1.76,-0.98)(1.76,-0.12)

\psbezier[linecolor=darkblue]{->}(0.1,0.04)(0.14,-0.14)(0.09427359,-0.3594825)(0.16,-0.46)(0.22572641,-0.5605175)(0.32,-0.14)(0.44,-0.2)(0.56,-0.26)(0.54,-0.72)(0.68,-0.74)(0.82,-0.76)(0.94,-0.3)(1.04,-0.3)(1.14,-0.3)(1.18,-0.42)(1.26,-0.54)

\rput(0.9,1.205){\color{gdarkgray}$z$}
\rput(0.57,0.705){\color{gdarkgray}$\th'$}
\rput(1.77,0.365){\color{gdarkgray}$\th_1$}
\rput(2.03,-0.255){\color{gdarkgray}$\th_2$}

\rput(0.9,-1.5){\color{gdarkgray}$(I)$}
\end{pspicture}
\begin{pspicture}(0,-1.7)(2.24,1.3)
\pscircle(0.9,-0.36){0.9}
\psellipse[linestyle=dashed](0.9,0.34)(0.44,0.2)
\psline{->}(0.9,0.42)(0.9,1.02)
\psbezier(0.18,0.18)(0.18,-0.46)(1.62,-0.46)(1.62,0.16)
\psbezier(0.04,-0.12)(0.04,-0.98)(1.76,-0.98)(1.76,-0.12)

\psbezier[linecolor=darkblue]{->}(0.1,0.04)(0.04,-0.14)(0.06,-0.38)(0.16,-0.46)(0.26,-0.54)(0.64,-0.28)(0.44,-0.2)(0.24,-0.12)(0.42,-0.72)(0.68,-0.74)(0.94,-0.76)(1.28,-0.26)(1.04,-0.3)(0.8,-0.34)(1.28,-0.98)(1.38,-0.48)

\rput(0.9,1.205){\color{gdarkgray}$z$}
\rput(0.57,0.705){\color{gdarkgray}$\th'$}
\rput(1.77,0.365){\color{gdarkgray}$\th_1$}
\rput(2.03,-0.255){\color{gdarkgray}$\th_2$}

\rput(0.9,-1.5){\color{gdarkgray}$(II)$}
\end{pspicture} 
\begin{pspicture}(0,-1.7)(2.24,1.3)
\pscircle(0.9,-0.36){0.9}
\psellipse[linestyle=dashed](0.9,0.34)(0.44,0.2)
\psline{->}(0.9,0.42)(0.9,1.02)
\psbezier(0.18,0.18)(0.18,-0.46)(1.62,-0.46)(1.62,0.16)
\psbezier(0.04,-0.12)(0.04,-0.98)(1.76,-0.98)(1.76,-0.12)

\psbezier[linecolor=darkblue]{->}(0.1,0.04)(0.04,-0.14)(0.06,-0.38)(0.16,-0.46)(0.26,-0.54)(0.44,-0.22)(0.44,-0.22)(0.44,-0.22)(0.46,-0.76)(0.68,-0.74)(0.9,-0.72)(0.98,-0.32)(0.98,-0.32)(0.98,-0.32)(1.22,-1.1)(1.48,-0.38)

\rput(0.9,1.205){\color{gdarkgray}$z$}
\rput(0.57,0.705){\color{gdarkgray}$\th'$}
\rput(1.77,0.365){\color{gdarkgray}$\th_1$}
\rput(2.03,-0.255){\color{gdarkgray}$\th_2$}

\rput(0.9,-1.5){\color{gdarkgray}$(III)$}
\end{pspicture}
\caption{Bewegung des Kreisels.}
\end{figure}

Falls $a=bu$ außerhalb von $(u_1,u_2)$ liegt, steigt
$\ph$ monoton an (Fall I).

Falls $u=\frac{a}{b}$ im Intervall $(u_1,u_2)$, so hat $\dot{\ph}$ einen  
Vorzeichenwechsel, $\ph$ bewegt sich vor und zurück (Fall II).

Fall III folgt, wenn wir den Kreisel mit $\dot{\ph} = 0$ fallen lassen.

Die azimutale Bewegung wird \emph{Präzession} genannt.

\subsubsection{Vorestellung der $SO(3)$}

Die $SU(2)$, die \emph{spezielle unitäre Gruppe}, das sind die unitären
$2\times 2$-Matritzen mit Determinante $1$, ist eine Kugel im 4-dimensionalen
Raum mit Radius 1. Identifizieren wir jeweils zwei  Punkte der $SU(2)$, so
erhalten wir die $SO(3)$.

Kehren wir nun zum Kreis zurück.
Durch die Einführung der Eulerwinkel wird jede Rotation im $\R^3$ durch drei
Rotationen um die jeweiligen Achsen darstellbar.
\begin{align*}
T = \Omega_1^2 I_1 + \Omega_2^2 I_2 + \Omega_3^2 I_3,\\
\vec{\Omega} = \dot{\th}\vec{e}_N + \dot{\psi}\vec{e}_3 + \dot{\ph}\vec{e}_2.
\end{align*}
Für $I_1 =I_2$ ist das Problem exakt lösbar,
\begin{align*}
&T = \frac{1}{2}I_1 \left(\dot{\th}^2 + \dot{\ph}^2\sin^2\th\right) +
\frac{I_3}{2}\left(\dot{\psi} + \dot{\ph}\cos\th \right)^2,\\
&V = mgl\cos\th.
\end{align*}
$\ph$ und $\psi$ sind zyklische Koordinaten. Wir haben daher 2 Erhaltungsgrößen.

Unter Einfluss von äußeren Kräften ändert sich der Drehimpuls eines Körpers mit
seinem Drehmoment.
\begin{align*}
\frac{\diffd }{\dt}\vec{m}
&= \frac{\diffd}{\dt} \sum_i m_i \vec{q}_i \times \dvec{q}_i
= \frac{\diffd}{\dt}
\sum_i m_i \dvec{q}_i \times \dvec{q}_i + \vec{q}_i \times \ddvec{q}_i\\
&= \sum_i \vec{q}_i \times m\ddvec{q}_i = \sum_i \vec{q}_i \times \vec{F}_i
\equiv \vec{n},
\end{align*}
wobei wir $\vec{n}$ als \emph{Drehmoment} bezeichnen.

\begin{bsp}
Kreiselkräfte beim Fahrradfahren. Um eine Rechtskurve zu fahren, drückt man den
Lenker rechts nach vorne, wodurch das Rad nach rechts ausweicht.\bsphere
\end{bsp}

Für das mitrotierende System ergibt sich daher,
\begin{align*}
\frac{\diffd}{\dt} \vec{M} &= \frac{\diffd}{\dt}\left(B^{-1}\vec{m} \right)
= \left(\frac{\diffd}{\dt}B^{-1}\right)\vec{m} +
B^{-1}\frac{\diffd}{\dt}\vec{m} =\left(\frac{\diffd}{\dt}B^{-1}\right)B\vec{M}
+ B^{-1}\vec{n} \\ &= \vec{M}\times \vec{\Omega} + \vec{N},
\end{align*}
d.h. auch ohne äußere Kräfte ist der Drehimpuls des rotierenden Systems im
Allgeinen nicht erhalten. Eine Ausnahme bilden Rotationen um die
Hauptträgheitsachsen.

Die Lösung des freien Kreisels lässt sich einfach geometrisch interpretieren.
Wir haben zwei Erhaltungsgrößen,
\begin{align*}
E = \frac{1}{2}\vec{\Omega}\cdot \underbrace{I\vec{\Omega}}_{\vec{M}} =
\frac{1}{2}\vec{M}I^{-1}\vec{M} = \frac{M_1^2}{I_1} + \vec{M_2^2}{I_2} +
\vec{M_3^2}{I_3} = \const,
\end{align*}
dies beschreibt uns einen Ellipsoid,
\begin{align*}
\vec{M}^2 = \left(B^{-1}\vec{m}\right)\left(B^{-1}\vec{m}\right) =
\underbrace{B^{-1}B^{-1}}_{\Id}\left(\vec{m}\cdot\vec{m}\right) = \vec{m}^2
= M_1^2 + M_2^2 + M_3^2 = \const,
\end{align*}
dies beschreibt eine Kugel. Die Schnittmenge dieses Ellipsoids mit der Kugel
schränkt die Bewegung von $\vec{M}$ auf Linien ein.
%TODO: Flußbild
Rotationen um $\vec{e}_1$ und $\vec{e}_3$ sind stabile Lösungen. Insgesamt gibt
es 6 stationäre Lösungen. Davon sind 4 stabil und 2 instabil.

\newpage
\section{Hamilton'sche Dynamik}

Ein Teilchen im Potential ist beschrieben durch,
\begin{align*}
L(q,\dot{q}) = \frac{m}{2}\dot{q}^2 - V(q).
\end{align*}
Hierbei sind $q$ und $\dot{q}$ nicht unabhängig von einander. Die
Euler-Lagrange-Gleichungen, die eine Beziehung zwischen ihnen herstellen, sind
Differentialgleichungen 2. Ordnung.

Wir wollen nun zu neuen Koordinaten $q$, $p$ übergehen, so dass wir eine
Differentialgleichung 1. Ordnung erhalten.

Der Hamilton'sche Impuls $p=\frac{\partial L}{\partial \dot{q}}$ erfüllt die
Relationen
\begin{align*}
&\dot{q} = \frac{p}{m},\\
&\dot{p} = m\ddot{q} = -\frac{\partial V}{\partial q}.
\end{align*}
Wir wollen diese Relationen durch eine neue Funktion $H$ beschreiben,
\begin{align*}
&\dot{q} = \frac{p}{m} = \frac{\partial H}{\partial p},\\
&\dot{p} = -\frac{\partial V}{\partial q} = -\frac{\partial H}{\partial q}.
\end{align*}
Diese Funktion heißt \emph{Hamiltonfunktion} und ist gegeben durch,
\begin{align*}
H(q,p) = \frac{p^2}{2m} + V(q).
\end{align*}
Die Gleichungen
\begin{align*}
&\dot{q} = \frac{p}{m} = \frac{\partial H}{\partial p},\\
&\dot{p} = -\frac{\partial V}{\partial q} = -\frac{\partial H}{\partial q},
\end{align*}
heißen \emph{Hamilton'sche Bewegungsgleichungen}, $(q,p)$ sind harmonische
Variablen.

\subsection{Legendre Transformation}

Wir betrachten die Funktion $f(x)$ mit Variable $x$ und definieren uns eine
neue Funktion $g(y)$, die \emph{Legendre Transformierte} von $f(x)$, mit der
neuen Variablen $y$ mittels,
\begin{align*}
&y := \frac{\diffd f}{\dx},\\
&g(y) := [xy-f(x)](y).
\end{align*}
Um die ``alte" Variable $x$ durch einen von $y$ abhängigen Ausdruck ersetzen zu
können, muss die Gleichung $y = \frac{\diffd f}{\dx}$ nach $x$ aufgelöst
werden $x(y)$. Dies ist nur dann eindeutig, falls $\frac{\diffd^2
f}{\dx^2}\neq 0$ für alle $x$.

\begin{bsp}
Sei $f(x) = x^2$, dann ist $y=\frac{\diffd f}{\dx} = 2x\Rightarrow
x=\frac{1}{2}y$. Die Legendre-Transformierte $g(y)=xy-f(x)$ ist gegeben durch,
\begin{align*}
g(y) = \frac{y^2}{2} - x^2 = \frac{y^2}{2} - \frac{y^2}{4} =
\frac{y^2}{4}.\bsphere
\end{align*}
\end{bsp}

\begin{bemn}[Geometrische Deutung.]
Die Legendre-Transformation beschreibt eine Funktion $f$ in eindeutiger
Weise, indem sie jeder Steigung $y=\frac{\df}{\dx}$ den $y$-Achsen-Abschnitt der dort
anliegenden Tangente zuordnet. Die Funktion $f$ wird somit vollständig durch ihre
Einhüllenden charakterisiert.\maphere
%TODO: Skizze
\end{bemn}

\begin{propn}[Theorem]
Die Legendre-Transformation von $g(y)$ ergibt wieder die Funktion $f(x)$, falls
$g(y)$ die Legendre-Transformation von $f(x)$ ist.\fishhere
\end{propn}
\begin{proof}
Betrachte dazu,
\begin{align*}
z &= \frac{\diffd g}{\dy} = \frac{\diffd}{\dy}\left[ xy-f(x) \right](y)
= x \frac{\diffd }{\dy}y + y \frac{\diffd}{\dy}x  - \frac{\diffd}{\dy}f(x)
\\ &= x + y\frac{\diffd}{\dy}x - \frac{\diffd f(x)}{\dx}\frac{\dx}{\dy}
= x + \frac{\diffd f}{\dx}\frac{\dx}{\dy} - \frac{\diffd
f}{\dx}\frac{\dx}{\dy}.
\end{align*}
Die Legendre-Transformation von $g(y)$ ist
\begin{align*}
h(z) = zy - g(y) = zy-\left[xy-f(x)\right](y) = 
xy - \left[xy-f(x)\right] = f(x).\qedhere
\end{align*}
\end{proof}

\subsection{Hamilton-Funktion und Hamiltonsche Bewegungsgleichung}

Wir erhalten die Hamiltonfunktion indem wir eine Legendre-Transformation auf
die Lagrangefunktion anwenden,
\begin{align*}
L(q^\alpha,\dot{q}^\alpha,t).
\end{align*}
Dabei sei $q^\alpha$ fest und unsere neue Variable der kanonischer Impuls,
\begin{align*}
p^\alpha \equiv \frac{\partial L}{\partial \dot{q}^\alpha}.
\end{align*}
Die Legendre-Transformation hat dann die Form,
\begin{align*}
H(q^\alpha, p_\alpha,t) = \left[\sum_\alpha p_\alpha \dot{q}^\alpha -
L\right]_{q^\alpha,p_\alpha,t}.
\end{align*}
Um die Bewegungsgleichungen zu erhalten, betrachte das Differential,
\begin{align*}
\diffd H &= \sum_\alpha\left[ \frac{\partial H}{\partial p_\alpha}\ddp_\alpha +
\frac{\partial H}{\partial q^\alpha} \ddq^\alpha\right] + \frac{\partial
H}{\partial t}\dt \overset{!}{=} \diffd\left[\sum_\alpha p_\alpha\dot{q}^\alpha - L 
\right]\\
&= \sum_\alpha \left[\dot{q}^\alpha\ddp_\alpha +
p_\alpha\ddotq^\alpha  - \frac{\partial L}{\partial q^\alpha}\ddq^\alpha -
\frac{\partial L}{\partial \dot{q}_\alpha} \diffd\dot{q}^\alpha  \right] +
\frac{\partial L}{\partial t}\dt.
\end{align*}
Somit folgt für die Bewegungsgleichungen,
\begin{align*}
&\frac{\partial H}{\partial p_\alpha} = \dot{q}^\alpha,\\
&\frac{\partial H}{\partial q^\alpha} = -\dot{p}_\alpha.
\end{align*}
Die Hamiltonfunktion und die Hamilton'schen Bewegungsgleichungen sind
äquivalent zur Lagrangefunktion und den Euler-Lagrange-Gleichungen.
\begin{bemn}[Alternative Herleitung.]
Obwohl das totale Differential $\diffd H$ sehr elegant auf die Hamiltonschen
Bewegungsgleichungen führt, wollen wir noch eine alternative Herleitung
betrachten.
\begin{align*}
\frac{\partial H}{\partial p_\alpha} &= \frac{\partial}{\partial p_\alpha}\left[
\sum_\beta \dot{q}^\beta p_\beta -L \right] = \dot{q}^\alpha + \sum_\beta
p_\beta \frac{\partial\dot{q}^\beta}{\partial p_\alpha} - \frac{\partial
L}{\partial p_\alpha}
\\ &= \dot{q}^\alpha + \sum_\beta \frac{\partial \dot{q}^\beta}{\partial
p_\alpha}\left[p_\beta - \frac{\partial L}{\partial \dot{q}^\beta}\right] =
\dot{q}^\alpha.
\end{align*}
Die Rechnung für $\frac{\partial H}{\partial q^\alpha}$ funktioniert
analog.\maphere
\end{bemn}

\begin{bsp}
%\begin{enumerate}[label=\arabic{*}.)]
%  \item
 Sei $L(q,\dot{q}) = \frac{m}{2}\dot{q}^2 - V(q)$. Wir wollen die
  Hamiltonfunktion nun mit Hilfe der Legendretransformation berechnen,
\begin{align*}
p &= \frac{\partial L}{\partial \dot{q}} = m\dot{q} \Rightarrow \dot{q} =
\frac{p}{m}.\\
H(q,p) &= \dot{q}p - L = p\left(\frac{p}{m}\right) -
\frac{m}{2}\left(\frac{p}{m}\right)^2 + V(q) = \frac{p^2}{2m} + V(q).
\end{align*}
Die Hamiltonschen Bewegungsgleichungen lauten also,
\begin{align*}
&\dot{q} = \frac{\partial H}{\partial q} = \frac{p}{m},\\
&\dot{p} = -\frac{\partial H}{\partial q} = - V'(q) =
\frac{\diffd}{\dt}(m\dot{q}) = m\ddot{q}.\bsphere
\end{align*}
\end{bsp}
\begin{bsp}
%\item
 Wir betrachten die Lagrangefunktion in Polarkoordinaten,
\begin{align*}
L(r,\ph,\dot{r},\dot{\ph}) = \frac{m}{2}\left(\dot{r}^2 + r^2 \dot{\ph}^2
\right) - V(r).
\end{align*}
Die Impulse haben nun die Form,
\begin{align*}
&p_r = \frac{\partial L}{\partial \dot{r}} = m\dot{r} \Rightarrow \dot{r} =
\frac{p_r}{m},\\
&p_\ph = \frac{\partial L}{\partial \dot{\ph}} = mr^2\dot{\ph} \Rightarrow
\dot{\ph} = \frac{p_\ph}{mr^2}.
\end{align*}
Die Hamiltonfunktion ist daher
\begin{align*}
H(r,\ph,p_r,p_\ph) &= p_\ph \dot{\ph} + p_r \dot{r} - L(r,\ph,\dot{r},\dot{\ph})
\\ &= \frac{p_\ph^2}{mr^2} + \frac{p_r^2}{m} -
\frac{m}{2}\left(\frac{p_r}{m}\right)^2 -
\frac{m}{2}r^2\left(\frac{p_\ph}{mr^2}\right)^2 + V(r)
\\ &= \frac{1}{2}\frac{p_\ph^2}{mr^2} + \frac{1}{2}\frac{p_r^2}{m} + V(r).
\end{align*}
$\ph$ ist zyklisch, d.h. $p_\ph$ ist erhalten. Im Lagrange müssten wir noch
$\dot{\ph}$ durch $p_\ph$ ausdrücken, in der Hamiltonfunktion ist dies nicht
mehr notwendig.\bsphere
\end{bsp}
\begin{bsp}
%\item
Teilchen im äußeren EM-Feld.
\begin{align*}
&L(\vec{x},\dvec{x}) = \frac{m}{2}\dvec{x}^2 - e\left(\phi(\vec{x}) -
\frac{\dvec{x}}{c}\vec{A} \right)\\
&\vec{p}_x = m\dvec{x} + \frac{e}{c}\vec{A}
\end{align*}
Die Hamiltonfunktion hat nun die Form,
\begin{align*}
H(\vec{x},\vec{p}_x) &= \vec{p}\dvec{x} - L =
\frac{\vec{p}}{m}\left(\vec{p}-\frac{e}{c}\vec{A}(x) \right) -
\frac{1}{2m}\left(\vec{p}-\frac{e}{c}\vec{A}(x) \right)^2 \\ &\quad+ e\phi(x) 
- \frac{e}{cm}\left(\vec{p}-\frac{e}{c}\vec{A}(x) \right)\vec{A}(x)\\
&= \left(\vec{p}-\frac{e}{c}\vec{A}(x)
\right)\left[\frac{\vec{p}}{m}-\frac{1}{2m}\left(\vec{p}-\frac{e}{c}\vec{A}(x)
\right) - \frac{e}{cm}\vec{A}(x)\right] \\ &\quad+ e\phi(x)\\
&= \left(\vec{p}-\frac{e}{c}\vec{A}(x) \right)\left[\frac{\vec{p}}{2m} -
\frac{e}{2mc}\vec{A}(x) \right] + e\phi(x)\\
&= \frac{1}{2m}\left(\vec{p}-\frac{e}{c}\vec{A}(x) \right)^2 +
e\phi(x).\bsphere
\end{align*}
%\end{enumerate}
\end{bsp}

\begin{bemn}
Falls das Problem mit verallgemeinerten Koordinaten zeitunabhängig ist, so gilt,
\begin{align*}
L = T-V \Rightarrow H=T+V
\end{align*} 
und der Wert von $H$ ist die Energie, denn
\begin{align*}
H= \underbrace{\sum_\alpha \dot{p}^\alpha p_\alpha}_{2T} - \underbrace{L}_{T-V}
= T+V.
\end{align*}
Die allgemeine Form der kinetischen Energie ist,
\begin{align*}
T = \sum_{\alpha,\beta} g_{\alpha\beta} \dot{q}^\alpha\dot{q}^\beta.\maphere
\end{align*}
\end{bemn}

\subsection{Poisson-Klammern}
Wir betrachten eine Messgröße $F(q,p,t)$ und berechnen ihre Zeitableitung.
\begin{bsp}
$L_z = xp_y - yp_x$, Drehimpuls in $z$-Richtung.\bsphere
\end{bsp}
\begin{align*}
\frac{\diffd F}{\dt} &= \frac{\partial F}{\partial t} + \sum_\alpha \left[
\frac{\partial F}{\partial q^\alpha}\frac{\diffd q^\alpha}{\dt} +
\frac{\partial F}{\partial p_\alpha} \frac{\diffd p_\alpha}{\dt}
\right] =
\frac{\partial F}{\partial t} + \sum_\alpha \left[
\frac{\partial F}{\partial q^\alpha}\frac{\partial H}{\partial p_\alpha} -
\frac{\partial F}{\partial p_\alpha} \frac{\partial H}{\partial q^\alpha}
\right]\\
&\equiv \frac{\partial F}{\partial t} + \setd{F,H}.
\end{align*}
Dabei haben wir die \emph{Poissonklammer} eingeführt, die definiert ist als
\begin{align*}
\setd{A,B} := \sum_\alpha \left[
\frac{\partial A}{\partial q^\alpha}\frac{\partial B}{\partial p_\alpha} -
\frac{\partial A}{\partial p_\alpha} \frac{\partial B}{\partial q^\alpha}
\right].
\end{align*}
\begin{bemn}[Bemerkungen.]
\begin{enumerate}[label=\arabic{*}.)]
  \item 
Falls $F$ nicht explizit von der Zeit abhängig ist, d.h. $\frac{\partial
F}{\partial t} = 0$, ist $F(q,p)$ genau dann eine Erhaltungsgröße, wenn
$\setd{F,H} = 0$.
\item
Falls $H$ nicht explizit von der Zeit abhängig ist, ist $H$ erhalten,
\begin{align*}
\setd{H,H} = \sum_\alpha \left[
\frac{\partial H}{\partial q^\alpha}\frac{\partial H}{\partial p_\alpha} -
\frac{\partial H}{\partial p_\alpha} \frac{\partial H}{\partial q^\alpha}
\right].\maphere
\end{align*}
\end{enumerate}
\end{bemn}

\begin{bemn}[Spezialfälle.]
Sei $F(q^\alpha,p^\alpha)=q^\alpha$ oder $p_\alpha$, so ist $\frac{\partial
F}{\partial t} = 0$ und,
\begin{align*}
\dot{q}^\alpha &= \setd{q^\alpha,H} = \sum\limits_\beta \left[ \frac{\partial
q^\alpha}{\partial q^\beta}\frac{\partial H}{\partial p^\beta} - \frac{\partial
q^\alpha}{\partial p_\beta}\frac{\partial H}{\partial q^\beta} \right]
= \frac{\partial H}{\partial p^\alpha},\\
\dot{p}_\alpha &= \setd{p_\alpha, H} = \sum_\beta \left[ 
\frac{\partial p_\alpha}{\partial q^\beta}\frac{\partial H}{\partial p_\beta} -
\frac{\partial p_\alpha}{\partial p_\beta}\frac{\partial H}{\partial q^\beta}
\right]
= -\frac{\partial H}{\partial q^\alpha}.\maphere
\end{align*}
\end{bemn}

\begin{propn}[Eigenschaften]
Die Poisson-Klammer ist ein bilinearer Differentialoperator mit den
Eigenschaften,
\begin{enumerate}[label=(\roman{*})]
  \item $\setd{A,B+C}=\setd{A,B}+\setd{A,C}$,\qquad (\textit{Linearität}),
  \item $\setd{A,B} = -\setd{B,A} \Rightarrow \setd{A,A} = 0$,\qquad
  (\textit{Antisymmetrie}),
  \item $\setd{A,c} = 0$ für $c\in\R$,
  \item $\setd{A,\setd{B,C}} + \setd{B,\setd{C,A}} + \setd{C,\setd{A,B}} =
  0$,\qquad (\textit{Jacobi Identität}).\fishhere
\end{enumerate}
\end{propn}

\begin{bemn}
$\setd{q^\alpha,p_\beta} = \delta_{\alpha\beta}$.
\begin{proof}
$\sum_\gamma \underbrace{\frac{\partial q^\alpha}{\partial
q^\gamma}}_{\delta_{\alpha\gamma}}
\underbrace{\frac{\partial p_\beta}{p_\gamma}}_{\delta_{\beta\gamma}} -
\underbrace{\frac{\partial q^\alpha}{\partial
p_\gamma}}_{0}\underbrace{\frac{\partial p^\beta}{\partial
q^\gamma}}_{0}$.\qedhere\maphere
\end{proof}
\end{bemn}

\subsubsection{Symplektische Notation}

Die Hamiltonschen Bewegungsgleichungen lassen sich in einer Gleichung in
folgender Form zusammenfassen,
\begin{align*}
\begin{pmatrix}
\dot{q}\\\dot{p}
\end{pmatrix}
=
\underbrace{\begin{pmatrix}
0 & 1\\
-1 & 0
\end{pmatrix}}_{\ep_{ij}}
\begin{pmatrix}
\frac{\partial H}{\partial q}\\
\frac{\partial H}{\partial p}
\end{pmatrix},
\end{align*}
wobei $\ep_{ij}$ den \emph{antisymmetrischen Tensor} bezeichnet.

Wir definieren nun einen neuen Vektor,
\begin{align*}
&\sym{x} = 
\begin{pmatrix}
q^1\\ p^1 \\ \vdots \\ q^f\\p^f
\end{pmatrix}
=
\begin{pmatrix}
x^1\\\vdots\\ x^{2f}
\end{pmatrix},\\
&\frac{\partial x^i}{\partial t}
= 
\sum_j \ep_{ij} \frac{\partial H}{\partial x_j},
\end{align*}
mit dem total antisymmetrischen Tensor
\begin{align*}
\ep =
\begin{pmatrix}
\begin{matrix}
0 & 1\\
-1 & 0
\end{matrix}\\
& \ddots\\
&& \begin{matrix}
0 & 1\\
-1 & 0
\end{matrix}
\end{pmatrix}.
\end{align*}
Zur Abkürzung führen wir folgende Notation ein,
\begin{align*}
&\frac{\diffd \sym{x}}{\dt} = \ep\frac{\partial H}{\partial \sym{x}} =
\ep\nabla H(\sym{x})\\
& \setd{A,B} = \sum_{i,j} \frac{\partial A}{\partial x^i} \ep_{ij}
\frac{\partial B}{\partial x^j} = \frac{\partial A}{\partial
\sym{x}}\ep\frac{\partial B}{\partial \sym{x}}.
\end{align*}

\subsection{Extremalprinzip}

Analog zum Lagrange existiert ein Extremalprinzip zu dem die Hamiltonschen
Bewegungsgleichungen äquivalent sind.

Für die Lagrangefunktion impliziert eine Variation von $q(t)$ auch eine
Variation von $\dot{q}(t)$. Die Euler-Lagrange-Gleichungen sind das Extremal
der Wirkung
\begin{align*}
S=\int\limits_{t_1}^{t_2}\dt L(q(t),\dot{q}(t),t).
\end{align*}

Für die Hamiltonfunktion sind nun $q(t)$ und $p(t)$ voneinander unabhängige
Größen. Übertragen wir die Wirkung für die Lagrangefunktion auf den Hamilton,
ergibt sich,
\begin{align*}
S = \int\limits_{t_1}^{t_2} \dt \underbrace{\sum_\alpha p_\alpha \dot{q}^\alpha
- H(q^\alpha,p_\alpha,t)}_{L'},
\end{align*}
wobei $L'$ hier für den Zahlenwert der Lagrangefunktion zum Zeitpunkt $t$
steht.

Anwendung der Euler-Lagrange-Gleichungen auf die unabhängigen Größen
$q^\alpha$ und $p^\beta$ ergibt,
\begin{align*}
&\frac{\diffd}{\dt}\frac{\partial L'}{\partial \dot{q}^\alpha} - \frac{\partial
L}{\partial q^\alpha} = 0,\\
&\frac{\diffd}{\dt}\frac{\partial L'}{\partial \dot{p}^\alpha} - \frac{\partial
L}{\partial p^\alpha} = 0.
\end{align*}
Einsetzen des Ausdrucks für $L'$ führt wieder auf die Hamiltonschen
Bewegungsgleichungen,
\begin{align*}
&\frac{\partial L'}{\partial \dot{q}^\alpha} = p_\alpha\\
\Rightarrow &\frac{\diffd}{\dt}p_\alpha = \dot{p}_\alpha = \frac{\partial
L'}{\partial q^\alpha} = -\frac{\partial H}{\partial q^\alpha}\\
&\frac{\partial L'}{\partial \dot{p}_\alpha} = 0\\
\Rightarrow &0 = \frac{\partial L'}{\partial p_\alpha} = \dot{q}^\alpha -
\frac{\partial H}{\partial p_\alpha} \Rightarrow \dot{q}^\alpha =
\frac{\partial H}{\partial p_\alpha}.
\end{align*}
\begin{bemn}
Die Wirkung entlang einer physikalischen Bahn mit $H=E$ ist gegeben durch,
\begin{align*}
S = \int\limits_{t_1}^{t_2} \dt \left[\sum_\alpha p_\alpha\dot{q}^\alpha -H
\right] = 
\sum_\alpha \int_\gamma p_\alpha \ddq^\alpha - E(t_2-t_1).\maphere
\end{align*}
\end{bemn}
\begin{figure}[htbp]
\centering
\begin{pspicture}(0,-1.41)(3.3,1.41)
\psline{->}(0.22,-1.19)(0.24,1.27)
\psline{->}(0.04,-1.01)(3.2,-1.01)
\psbezier[linecolor=darkblue]{**-**}(0.74,-0.49)(1.0719048,0.45040816)(2.1,-0.59)(2.38,0.47)

\rput(3.17,-1.185){\color{gdarkgray}$q^\alpha$}
\rput(0.05,1.255){\color{gdarkgray}$t$}
\rput(0.68,-0.645){\color{gdarkgray}$t_1$}
\rput(2.56,0.555){\color{gdarkgray}$t_2$}
\end{pspicture}

\caption{Wirkung entlang einer Trajektorie}
\end{figure}

\subsection{Kanonische Transformationen}
 
Wir wollen nun untersuchen, welche Transformationen die Hamiltonfunktion
invariant lasssen. Für die Newtonschen Gleichungen waren das lediglich die
Galilei-Transformationen. Der Lagrangeformalismus hingegen ist invariant unter
beliebigen Transformationen, d.h.
 \begin{align*}
 &Q^\alpha = f(q^\beta,t)\\
 &\dot{Q}^\alpha = \frac{\diffd}{\dt}f(q^\beta,t)
 \end{align*}
für jede Transformation $f$ im Konfigurationsraum. Beispielsweise ließen sich
durch das Einführen von Kugelkoordinaten zahlreiche Probleme leicht auflösen bzw. reduzieren.

 Im Hamiltonformalismus sind $q^\alpha$ und $p^\alpha$ von einander unabhängig,
 d.h. Transformationen können $p^\alpha$ ändern während sie $q^\alpha$ invariant
 lassen und umgekehrt.
 %Im Lagrangeformalismus waren durch die feste
 %Verknüpfung von $q$ und $\dot{q}$ beliebige Transformationen zugelassen. 

Die Klasse von Transformationen im Phasenraum ($(q,p)$-Raum), die die
Hamiltonschen Bewegungsgleichungen
invariant lassen heißen \emph{kanonische Transformationen}.

Wir wollen uns zunächst auf die wichtigste Untergruppe dieser Transformationen
\begin{align*}
(q^\alpha,p^\alpha)\mapsto (Q^\alpha,P^\alpha)
\end{align*}
beschränken, die die folgenden Eigenschaften erfüllen:
\begin{enumerate}[label=(\roman{*})]
  \item zeitunabhängig,
  \item kontinuierlich in einem Parameter $s$.
\end{enumerate}

Wir betrachten eine Messgröße $F$, die nicht explizit von der Zeit abhängt,
entlang einer Bewegungskurve. Aufgrund der Zeitunabhängigkeit der
Transformation ist $\frac{\diffd F}{\dt}$ invariant,
\begin{align*}
\frac{\diffd F}{\dt} = \setd{F,H}_{q,p} = \setd{F,H}_{Q,P},
\end{align*}
d.h. die Poissonklammer in neuen und alten Koordinaten ist identisch.
Da dies für alle Hamiltonfunktionen gelten muss, folgt allgemein
\begin{align*}
&\setd{A,B}_{q,p} = \setd{A,B}_{Q,P},\\
&\sum_\alpha \frac{\partial A}{\partial q^\alpha}\frac{\partial B}{\partial
p_\alpha} - \frac{\partial A}{\partial p_\alpha}\frac{\partial B}{\partial
q^\alpha}
=
\sum_\alpha \frac{\partial A}{\partial Q^\alpha}\frac{\partial B}{\partial
P_\alpha} - \frac{\partial A}{\partial P_\alpha}\frac{\partial B}{\partial
Q^\alpha},
\end{align*}
bzw. in symplektischer Schreibweise,
\begin{align*}
\sum_{i,j} \frac{\partial A}{\partial \sym{x}_i}\ep_{ij}\frac{\partial
B}{\partial \sym{x}_j}
=
\sum_{i,j} \frac{\partial A}{\partial \sym{y}_i}\ep_{ij}\frac{\partial
B}{\partial \sym{y}_j},
\end{align*}
wobei
\begin{align*}
\sym{x} =
\begin{pmatrix}
q^1\\p^1\\\vdots \\ q^f \\ p^f
\end{pmatrix},\qquad
\sym{y} =
\begin{pmatrix}
Q^1\\P^1\\\vdots \\ Q^f \\ P^f
\end{pmatrix}.
\end{align*}
\begin{bemn}[Spezialfall.]
$A=P_\alpha$, $B=Q^\beta$ $\Rightarrow$ $\setd{P_\alpha,Q^\beta}_{q,p} =
\setd{P_\alpha, Q^\beta}_{Q,P} = \delta_{\alpha\beta}$.

\begin{bsp}
Sei $Q^\beta = \lambda q^\beta$, so folgt $P_\alpha =
\frac{1}{\lambda}p_\alpha$, denn $\setd{Q^\alpha,P_\alpha} =
\delta_{\alpha\beta}$.\bsphere
\end{bsp}

Seien $\sym{y}$ die neuen und $\sym{x}$ die alten Koordinaten in symplektischer
Notation, so ergibt sich,
\begin{align*}
\setd{y_i,y_j}_x = \ep_{ij} = \sum_{l,m} \frac{\partial y_i}{\partial
x_k}\ep_{lm} \frac{\partial y_j}{\partial x_m} = \ep_{ij}
\end{align*}
Bezeichne $M_{il}:=\frac{\partial y_i}{\partial x_l}$ die Funktionalmatrix der
kanonischen Transformation, so gilt
\begin{align*}
M\ep M^\top = \ep.\maphere
\end{align*}
\end{bemn}

\begin{defnn}
Die \emph{Symplektische Gruppe}
\begin{align*}
Sp(f,\R) := \setdef{M \in \R^{2f\times 2f}}{M\ep M^\top =\ep}.
\end{align*}
ist die Gruppe der $2f\times 2f$-Matritzen mit $M\ep M^\top = \ep$.\fishhere
\end{defnn}

Entwickeln wir $\sym{y}$ in $s$, so ergibt sich,
\begin{align*}
y_i = x_i + s\cdot v_i + o(s^2),
\end{align*}
mit $\sym{v}_i(x) = \frac{\partial y_i}{\partial s}\bigg|_{s=0}$.
Differentiation der Entwicklung nach $x_k$ ergibt,
\begin{align*}
M_{ik}  = \frac{\partial y_i}{\partial x_k} = \delta_{ik} + s
\underbrace{\frac{\partial v_i}{\partial x_k}}_{=:m_{ik}},
\end{align*}
d.h. wir können die Funktionalmatrix $M$ in $s$ entwickeln,
\begin{align*}
M = \Id + s\ m + o(s^2).
\end{align*}
Einsetzen in die Bedingung für eine kanonische Transformation ergibt,
\begin{align*}
\ep &= M\ep M^\top = \left(\Id + s\ m + o(s^2)\right)\ep\left(\Id + s\ m +
o(s^2) \right) \\ &= \ep + s\left[m\ep + \ep m^\top \right] + o(s^2)\\
\Rightarrow & m\ep + \ep m^\top = 0
\end{align*}
Multiplikation von rechts und links mit $\ep$ unter verwendung von $\ep^2 =\Id$
und $\ep^\top = -\ep$ ergibt,
\begin{align*}
\ep m \ep^2 + \ep^2 m^\top\ep = 0
\Leftrightarrow
\ep m -m^\top e^\top  = 0
\Leftrightarrow
\ep m - (\ep m)^\top = 0.\tag{*}
\end{align*}
Setze nun
\begin{align*}
\sum_j e_{ij} m_{jk} = \sum_j e_{ij}\frac{\partial v_j}{\partial x_k}
= \frac{\partial}{\partial x_k}\underbrace{\left(\sum_j \ep_{ij}v_j
\right)}_{=:g_i},
\end{align*}
so nimmt (*) die Form an,
\begin{align*}
\frac{\partial g_i}{\partial x_k} - \frac{\partial g_k}{\partial x_i} = 0.
\end{align*}
In drei Dimensionen wäre dies äquivalent damit, dass $\rot g = 0$. Hier haben
wir die höherdimensionale Verallgemeinerung. Analog zum dreidimensionalen Fall
existiert daher ein Skalarfeld $G$ so, dass
\begin{align*}
g_i = -\nabla_i G = - \frac{\partial G}{\partial x_i}(\sym{x}). 
\end{align*}
$G$ heißt \emph{Erzeugende} Funktion für die kanonische Transformation (bzw. den
kanonischen Fluss).

\subsubsection{Zusammenfassung}

\begin{enumerate}[label=\arabic{*}.)]
  \item $v_i = \frac{\diffd y_i}{\ds}\bigg|_{s=0} = -\sum_j \ep_{ij}g_j =
  \sum_j \ep \frac{\partial G}{\partial y_j}$.
  \item Zu jeder kanonischen Transformation $g$ existiert eine erzeugende
  Funktion $G$.
  \item Jede Funktion $G$ erzeugt eine kanonische Transformation.

Insbesondere erzeugt die Hamiltonfunktion eine kanonische Transformation. Diese
ist die Transformation im Koordinatenraum, die zu gegebenen Anfangsbedingungen
die Hamiltonschen Bewegungsgleichungen liefert.
\end{enumerate}

\begin{bsp}
\textit{Erzeugende für eine Drehung um die $z$-Achse}. Die Drehung ist
charakterisiert durch,
\begin{align*}
\vec{y} = \vec{x} + s\vec{v}\times\vec{x}.
\end{align*}
Dabei ist,
\begin{align*}
&\frac{\diffd y_1}{\ds} = -x_2\\
&\frac{\diffd y_2}{\ds} = x_1\\
&\frac{\diffd y_3}{\ds} = 0.
\end{align*}
D.h. $G=L_3 = x_1p_2 - x_2p_1$, die $z$-Komponente des Drehimpulses, ist
Erzeugende für die Rotation.

Ananlog sieht man, dass der Impuls die Erzeugende für die Translation im Raum
ist.\bsphere
\end{bsp}

\subsubsection{Symmetrien und Erhaltungssätze}

Eine kanonische Transformation $g$ ist eine \emph{Symmetrie}, wenn sie die
Hamiltonfunktion nicht ändert,
\begin{align*}
0 = \frac{\diffd H}{\ds}(\sym{y}) = \sum_i \frac{\partial H}{\partial
y_i}\frac{\partial y_i}{\partial s} =
\sum_{i,j} \frac{\partial H}{\partial y_i}\ep_{ij}\frac{\partial
G}{\partial y_j} = \setd{H,G}.
\end{align*}
D.h. $g$ ist genau dann eine Symmetrie, wenn die Poissonklammer von $H$ und
ihrer Erzeugenden $G$ verschwindet.

Die Erzeugende einer Symmetrie ist daher stets eine Erhaltungsgröße, denn
\begin{align*}
\frac{\diffd G}{\dt} = \setd{G,H} = 0.
\end{align*}

\subsection{Endliche kanonische Transformationen}

Die Transformation lässt die Bewegungsgleichungen
\begin{align*}
&\dot{q}^\alpha = \frac{\partial H}{\partial p_\alpha},\quad
\dot{p}_\alpha = -\frac{\partial H}{\partial q^\alpha}
\end{align*}
invariant. In neuen Koordinaten,
\begin{align*}
&\dot{Q}^\alpha = \frac{\partial H'}{\partial P_\alpha},\quad
\dot{P}_\alpha = -\frac{\partial H'}{\partial Q^\alpha}.
\end{align*}
Falls $H$ zeitabhängig ist, erhalten wir eine neue Hamiltonfunktion $H'$.

Die Hamiltonschen Bewegungsgleichungen sind äquivalent zum Hamiltonschen
Prinzip der kleinsten Wirkung,
\begin{align*}
&\delta\left(\sum_\alpha p_\alpha\dot{q}^\alpha - H\right) = 0,\\
&\delta\left(\sum_\alpha P_\alpha\dot{Q}^\alpha - H'\right) = 0,
\end{align*}
d.h. $\sum_\alpha p_\alpha\dot{q}^\alpha - H$ und $\sum_\alpha
P_\alpha\dot{Q}^\alpha - H'$ dürfen sich lediglich um eine totale
Zeitableitung unterscheiden. Dies ergibt eine neue Funktion,
\begin{align*}
&\frac{\diffd F}{\dt} = \sum_\alpha p_\alpha \dot{q}^\alpha -
P_\alpha\dot{Q}^\alpha - (H-H'),\\
&\diffd F = \sum_\alpha p_\alpha \ddq^\alpha -
P_\alpha\dQ^\alpha - (H-H')\dt.
\end{align*}

Wir nehmen an, dass $F(q,Q,t)$ die Form,
\begin{align*}
\diffd F = \sum_\alpha p_\alpha \ddq^\alpha -
P_\alpha\dQ^\alpha - \frac{\partial F}{\partial t}\dt
\end{align*}
hat. Dadurch ergeben sich die Bedingungen,
\begin{align*}
\frac{\partial F}{\partial q^\alpha} = p_\alpha,\quad
-\frac{\partial F}{\partial Q^\alpha} = P_\alpha,\quad
\frac{\partial F}{\partial t} = H-H'.
\end{align*}
Unter diesen Bedingungen erzeugt $F$ eine kanonische Transformation.

\begin{bsp}
Betrachte den Harmonischen Oszillator mit,
\begin{align*}
H=\frac{p^2}{2m} + k\frac{q^2}{2},\quad \omega^2=\frac{k}{m}.
\end{align*}
Durch kühne Überlegung erhält man die Erzeugende,
\begin{align*}
F = \frac{m\omega q^2}{2}\cot(Q),
\end{align*}
dann ist
\begin{align*}
&p = \frac{\partial F}{\partial q} = m\omega q \cot Q,\\
&P = -\frac{\partial F}{\partial Q} = -\frac{m\omega q^2}{2}\frac{1}{\sin^2 Q}.
\end{align*}
Auflösen der Gleichung ergibt,
\begin{align*}
&q = \sqrt{\frac{2P}{m\omega}}\sin Q\\
&p = \sqrt{2Pm\omega}\cos Q.
\end{align*}
Die neue Hamiltonfunktion hat die Form,
\begin{align*}
H' &= H + \underbrace{\frac{\partial F}{\partial t}}_{=0}
= \frac{2 Pm\omega \cos^2 Q}{2m} + \omega^2 m\frac{2P}{\omega}\frac{\sin^2
Q}{2}\\ &= P\omega \cos^2 Q + P\omega \sin^2 Q = P\omega.
\end{align*}
Nach der kanonischen Transformation ist $Q$ eine zyklische Koordinate,
\begin{align*}
&Q = \omega t + \ph,\\
&E = \omega P \Rightarrow P = \frac{E}{\omega}.\bsphere
\end{align*}
\end{bsp}

\subsection{Hamilton-Jacobi Differentialgleichungen}

Wir wollen nun untersuchen, welche kanonische Transformation $F_2$ zur Folge
hat, dass
\begin{align*}
&H'(Q^\alpha, P_\alpha) = \const = 0,\\
&\dot{Q}^\alpha = \frac{\partial H'}{\partial p_\alpha} = 0\\
&\dot{P}_\alpha = -\frac{\partial H'}{\partial q^\alpha} = 0.
\end{align*}
Wir erhalten so eine Differentialgleichung, deren Lösung die gesuchte kanonische
Transformation ist.

Die Erzeugende dieser Transformation ist die Wirkung,
\begin{align*}
S = \int\limits_{t_1}^{t_2}\dt L.
\end{align*}

Betrachte nun zu festem Anfangspunkt $q^\alpha$, $S(q^\alpha,t)$ als Funktion
des Endpunkts für \textit{physikalische Bahnen}. Für die Ortsableitung ergibt
sich,
\begin{align*}
\frac{\partial S}{\partial q^\alpha} &= \frac{\partial}{\partial
q^\alpha}\int\limits_{t_1}^{t} \dt L = 
\int\limits_{t_1}^t \ds \frac{\partial L}{\partial q^\alpha}
+ \sum_\beta \frac{\partial L}{\partial \dot{q}^\beta}\frac{\partial
\dot{q}^\beta}{\partial q^\alpha}
\\ &= \underbrace{\int\limits_{t_1}^t \ds \left[\frac{\partial L}{\partial
q^\alpha} - \frac{\diffd}{\dt}\frac{\partial L}{\partial q^\alpha}\right]}_{=0}
+ \frac{\partial L}{\partial \dot{q}^\alpha}\bigg|_{t_1}^t = p_\alpha.
\end{align*}
Betrachte nun $q^\alpha$ als fixiert und differenziere nach $t$,
\begin{align*}
&\frac{\diffd S}{\dt} =\frac{\diffd }{\dt}\int\limits_{t_1}^t \dt L = L
= \frac{\partial S}{\partial t} + \sum_\alpha \frac{\partial S}{\partial
q^\alpha}\frac{\partial q^\alpha}{\partial \dot{q}^\alpha}\\
\Rightarrow & \frac{\partial S}{\partial t} = \frac{\diffd S}{\dt} -
\sum_\alpha p_\alpha\dot{q}^\alpha =- H.
\end{align*}

Die Wirkungsfunktion $S(q^\alpha,t)$ erfüllt die folgenden
Differentialgleichungen,
\begin{align*}
&\frac{\partial S}{\partial t} = -H, \quad
\frac{\partial S}{\partial q^\alpha} = p_\alpha\\
&\diffd S = \sum_\alpha p_\alpha \ddq^\alpha - H\dt
\end{align*}
Damit folgt die \emph{Hamilton-Jacobi Differentialgleichung},
\begin{align*}
\frac{\partial S}{\partial t} = -H(q^\alpha,p_\alpha,t) =
-H(q^\alpha,\frac{\partial S}{\partial q^\alpha},t).
\end{align*}
$S(q^\alpha,t)$ ist somit durch eine partielle Differentialgleichung 1. Ordnung
in $f+1$ Variablen bestimmt. Für partielle Differentialgleichungen existieren
weitreichende Lösungsverfahren und -sätze, mit denen man sich in der
Quantenelektrodynamik ausführlich beschäftigt.

Die Lösung ist durch $f+1$ Integrationskonstanten bestimmt. Eine
Integrationkonstante ist trivial,
\begin{align*}
S(q^\alpha,t,a^\alpha),\quad \alpha = 1,\ldots,f.
\end{align*}
Die Lösung dieser Differentialgleichung können wir als Erzeugende einer
kanonische Transformation mit $p_\alpha = a^\alpha$ auffassen,
\begin{align*}
&F_2(q^\alpha,p_\alpha) = S(q^\alpha, t, a^\alpha).\\
&p_\alpha = \frac{\partial S}{\partial q^\alpha}\\
&Q^\alpha = \frac{\partial S}{\partial p_\alpha}\\
&H' = H+\frac{\partial S}{\partial t} = 0.
\end{align*}

\begin{bsp}
Wir betrachten erneut den harmonischen Oszillator,
\begin{align*}
H = \frac{p^2}{2m} + m\omega^2q^2,
\end{align*}
und lösen die Hamilton-Jacobi Differentialgleichungen.

Separation führt auf,
\begin{align*}
S(q^\alpha,t) = S_0(q^\alpha) + S_1(t).
\end{align*}
Für den zeitabhängigen Teil ist
\begin{align*}
&\frac{\partial S_1}{\partial t} =
-\frac{1}{2m}\left(\frac{\partial S_0(q)}{\partial q} \right)^2 - m\omega^2
q^2\\
\Rightarrow &
\frac{\partial S_1}{\partial t} = -a^1 
\Rightarrow  
\frac{1}{2m}\left(\frac{\partial S_0(q)}{\partial q} \right)^2 + m\omega^2 q^2
= a^1
\end{align*}
Für den Ortsanteil erhalten wir so,
\begin{align*}
\frac{\partial S_0}{\partial q} = \sqrt{2m}\sqrt{a^1 -m\omega^2 q^2}
\end{align*}
dies kann man integrieren,
\begin{align*}
S_0(q) = \int \ddq \sqrt{2m}\sqrt{a^1 - m\omega^2 q^2}. 
\end{align*}
Die Lösung hat daher die Form,
\begin{align*}
S(q,t) = \int \ddq \sqrt{2m}\sqrt{a^1 - m\omega^2 q^2} + a^1 t.
\end{align*}
Man kann hiervon eine analytische Lösung berechnen. Betrachten wir die neuen
Koordinaten,
\begin{align*}
&P_\alpha = a^1,\quad Q^\alpha = \frac{\partial S}{\partial a^1}
= \int\ddq \sqrt{\frac{m}{2}} \frac{1}{\sqrt{a^1 - m\omega^2 q^2}} - t\\
\Rightarrow &
Q^\alpha + t =
\frac{1}{\omega}\arcsin\left(q\sqrt{\frac{m\omega^2}{2a^1}}\right).
\end{align*}
In den neuen Koordinaten $(Q^\alpha,P_\alpha)$ verschwindet die
Hamiltonfunktion $H'=0$, d.h. $P=\const$, $Q=\const$ und daher
\begin{align*}
&q = \sqrt{\frac{2P}{m\omega^2}}\sin\left(\omega(t+Q) \right)\\
&p = \ldots
\end{align*}
D.h. in den neuen Koordinaten ist der Impuls $P$ die Amplitude und $Q$ die
Phase des Oszillators, beide Konstanten.\bsphere
\end{bsp}

\end{document}